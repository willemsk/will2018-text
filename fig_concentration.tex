\begin{figure}[htbp]
\centering
\begin{subfigure}[t]{8.2cm}
  \centering
  \caption{}\vspace{-3mm}\label{fig:concentration_pore_average_vs_concentration}
  \includegraphics[scale=1]{figures/concentration/concentration_pore_average_vs_concentration}
\end{subfigure}
\begin{minipage}[t]{8.2cm}
\begin{subfigure}[t]{8.2cm}
	\centering
	\caption{}\vspace{-3mm}\label{fig:concentration_contours}
	\includegraphics[scale=1]{figures/concentration/concentration_contours}
\end{subfigure}
\begin{subfigure}[t]{8.2cm}
  \centering
  \caption{}\vspace{-3mm}\label{fig:concentration_radial_profiles}
  \includegraphics[scale=1]{figures/concentration/concentration_radial_profiles}
\end{subfigure}
\end{minipage}

% caption
\caption
[\textbf{Ion concentration distribution inside ClyA.}]
{
\textbf{Ion concentration distribution inside ClyA.}
(\subref{fig:concentration_pore_average_vs_concentration})
Relative \Na\ and \Cl\ concentrations averaged over the entire pore volume 
($\left<\ci/\cbulk\right>_\text{pore}$) as a function of the reservoir salt concentration ($\cbulk$). ClyA's 
negatively charged interior results in the enhancement and the depletion of respectively \Na\ and \Cl\ ions 
inside the pore, particularly at lower $\cbulk$. Both effects diminish with increasing $\cbulk$, and bulk 
values ($\left<\ci/\cbulk\right>_\text{pore}\approx1$) are observed for both ions at $\cbulk\approx0.5$~M. 
Interestingly, for $\cbulk<0.5$~M, the \Cl\ concentration depends strongly on the bias voltage $\vbias$, i.e. 
near bulk like conditions ($\left<\ci/\cbulk\right>_\text{pore}\approx 1$) and virtual depletion 
($\left<\ci/\cbulk\right>_\text{pore}\approx 0.1$) at respectively $+150$ and $-150$~mV for $\cbulk=0.05$~M.
(\subref{fig:concentration_contours})
Cross-section contour plots of the \Na\ and \Cl\  concentrations inside the pore relative to the bulk 
value of $0.15$~M at bias voltages $-100$ and $+100$~mV. These plots reveal the local concentration changes 
near charged residues, particularly in the highly negatively charged constriction. While conditions inside 
the lumen of the pore are generally close to bulk, a strong depletion of \Cl\  can occur under high 
negative bias voltages.
(\subref{fig:concentration_radial_profiles})
Relative \Na\ and \Cl\  concentration profiles (to $\cbulk=0.15$~M) along the radius of the pore at the 
middle of the constriction ($z=-0.3$~nm) and the lumen ($z=5$~nm) for $-100$ and $+100$~mV, showing the 
formation of the electrical double layer.
}

\label{fig:concentration}

\end{figure}
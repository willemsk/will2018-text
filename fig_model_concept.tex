\begin{figure}[htbp]
  
	\centering
	\begin{minipage}[t]{3.5cm}
		\begin{subfigure}[t]{3.5cm}
			\centering
			\caption{}\label{fig:clya_atomistic_side}
			\includegraphics[width=3.5cm]{figures/concept/clya_side}
		\end{subfigure}
		\begin{subfigure}[t]{3.5cm}
			\centering
			\caption{}\label{fig:clya_atomistic_top}
			\includegraphics[width=3.5cm]{figures/concept/clya_top}
		\end{subfigure}
	\end{minipage}
  \hspace{1cm}
  \begin{minipage}[t]{8.25cm}
    \begin{subfigure}[t]{4.5cm}
      \centering
      \caption{}\label{fig:model_geometry}
      \includegraphics[width=4.5cm]{figures/concept/model_geometry}
    \end{subfigure}
    \hspace{-0.8cm}
    \begin{minipage}[t]{4.2cm}
      \begin{subfigure}[t]{4.2cm}
        \centering
        \caption{}\label{fig:model_geometry_zoom}
        \includegraphics[width=4.2cm]{figures/concept/model_geometry_zoom}
      \end{subfigure}
      \vspace{0.5cm}
      \begin{minipage}[t]{4.5cm}
        \hspace{1cm}
        \begin{subfigure}[t]{1.2cm}
          \centering
          \caption{}\label{fig:model_geometry_vs_wedge}
          \vspace{-3mm}
          \includegraphics[width=1.2cm]{figures/concept/model_geometry_vs_wedge}
        \end{subfigure}
        \begin{subfigure}[t]{2cm}
          \centering
          \caption{}\label{fig:model_charge_density}
          \vspace{-3mm}
          \includegraphics[width=2cm]{figures/concept/model_charge_density}
        \end{subfigure}
      \end{minipage}
    \end{minipage}
  \end{minipage}


\caption[All-atom and 2D-axisymmetric models of ClyA.]
{
\textbf{All-atom and 2D-axisymmetric models of ClyA.} (\subref{fig:clya_atomistic_side}) Cross-sectional and 
(\subref{fig:clya_atomistic_top}) top views of an all-atom model of the dodecameric nanopore ClyA-AS embedded 
in a lipid bilayer.\cite{soskine2013}
(\subref{fig:model_geometry}) 2D-axisymmetric model of ClyA (grey) embedded in a lipid bilayer (green) and 
surrounded by a spherical water reservoir (blue). The former two are represented by solid dielectric blocks a 
relative permittivity of $\permittivity_p=20$ and $\permittivity_m=3.2$, respectively. A bias voltage 
($\varpotential_\textrm{bias}$) is applied from the \textit{trans} reservoir boundary (red), while the 
\textit{cis} side (green) is kept grounded. Both boundaries are set up to mimic an infinite reservoir of ions 
and water. Note that the in the ePNP-NS model, the fluid viscosity $\viscosity$, the ion diffusion 
coefficients $\diffusion_{i}$ and the ion mobilities $\mobility_{i}$ are a function of  the local average ion 
concentration $\avionconc$ and distance from the nanopore wall $\walldistance$, while the fluid's relative 
permittivity and $\permittivity_w$ and its density $\density$ are dependent only on $\avionconc$.
(\subref{fig:model_geometry_zoom}) Zoom-in of the nanopore geometry and lipid bilayer. Their respective 
boundaries in contact with the fluid reservoir are set to no-slip (zero fluid velocity) and no-flux 
(impermeable to ions).
(\subref{fig:model_geometry_vs_wedge}) The 2D-axisymmetric geometry was derived directly from the all-atom 
model by computing the average inner and outer radii along the longitudinal axis of the pore, and hence 
closely follows the outline of a \ang{30} wedge out of the homology model. 
(\subref{fig:model_charge_density}) The fixed space charge density map of ClyA-AS ($\scdpore$), obtained by 
Gaussian projection of each atom's partial charge onto a 2D plane (see methods for details).
}

\label{fig:model_concept}

\end{figure}
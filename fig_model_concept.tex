\begin{figure*}[!bt]

	\centering
	\begin{minipage}[t]{5cm}
		\begin{subfigure}[t]{5cm}
			\centering
			\caption{}\label{fig:clya_side}
      \vspace{-5mm}
			\includegraphics[scale=1]{figures/concept/clya_side}
		\end{subfigure}
   	\begin{subfigure}[t]{5cm}
      \centering
      \caption{}\label{fig:clya_top}
      \vspace{-5mm}
      \includegraphics[scale=1]{figures/concept/clya_top}
    \end{subfigure}
	\end{minipage}
  \hspace{0.5cm}
  \begin{minipage}[t]{11.5cm}
    \begin{minipage}[t]{5.5cm}
      \begin{minipage}[t]{5.5cm}
        \begin{subfigure}[t]{1.6cm}
          \centering
          \caption{}\label{fig:model_geometry_vs_wedge}
          \vspace{-3mm}
          \includegraphics[scale=1]{figures/concept/model_geometry_vs_wedge}
        \end{subfigure}
        \begin{subfigure}[t]{2.5cm}
          \centering
          \caption{}\label{fig:model_charge_density}
          \vspace{-3mm}
          \includegraphics[scale=1]{figures/concept/model_charge_density}
        \end{subfigure}
      \end{minipage}
      \begin{subfigure}[t]{5.5cm}
        \centering
        \vspace{0.5cm}
        \caption{}\label{fig:model_geometry_zoom}
        \vspace{-1cm}
        \includegraphics[scale=1]{figures/concept/model_geometry_zoom}
        %\vspace{0.25cm}
      \end{subfigure}
    \end{minipage}
    \hspace{-0.8cm}
    \begin{subfigure}[t]{5.5cm}
      \centering
      \caption{}\label{fig:model_geometry}
      \includegraphics[scale=1]{figures/concept/model_geometry}
    \end{subfigure}
  \end{minipage}

\caption[All-atom and 2D-axisymmetric models of ClyA.]
{
\textbf{All-atom and 2D-axisymmetric models of ClyA.}
(\subref{fig:clya_side}) Axial cross-sectional and (\subref{fig:clya_top}) top views of the dodecameric
nanopore ClyA-AS\cite{Soskine-2013}, derived through homology modelling from the \textit{E. coli} Cytolysin A
crystal structure (PDBID: 2WCD\cite{Mueller-2009}). Figures were rendered with
VMD.\cite{Humphrey-1996,Stone-1998}
(\subref{fig:model_geometry_vs_wedge}) The 2D-axisymmetric geometry was derived directly from the all-atom
model by computing the average inner and outer radii along the longitudinal axis of the pore, and hence
closely follows the outline of a \ang{30} wedge out of the homology model.
(\subref{fig:model_charge_density}) The fixed space charge density ($\scdpore$) map of ClyA-AS, obtained by
Gaussian projection of each atom's partial charge onto a 2D plane (see methods for details).
(\subref{fig:model_geometry_zoom}+\subref{fig:model_geometry}) The 2D-axisymmetric simulation geometry of ClyA
(grey) embedded in a lipid bilayer (green) and surrounded by a spherical water reservoir (blue). Note that all
electrolyte parameters depend on the local average ion concentration
$\avionconc=\frac{1}{n}\sum_{i}^{n}\concentration_{i}$ and that some are also influenced by the distance from
the nanopore wall $\walldistance$.
}\label{fig:model_concept}
\end{figure*}

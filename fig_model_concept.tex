% create the figure
\begin{figure}[htbp]
	\centering
	\begin{minipage}[t]{5.5cm}
		\begin{subfigure}[t]{5cm}
			\centering
			\caption{}\label{fig:clya_atomistic_side}
			\includegraphics[width=5cm]{figures/concept/clya_side}
		\end{subfigure}
		\begin{subfigure}[t]{5cm}
			\centering
			\caption{}\label{fig:clya_atomistic_top}
			\includegraphics[width=5cm]{figures/concept/clya_top}
		\end{subfigure}
	\end{minipage}
  \vspace{1cm}
  \begin{minipage}[t]{8.25cm}
    \begin{subfigure}[t]{4.5cm}
      \centering
      \caption{}\label{fig:model_geometry}
      \includegraphics[width=4.5cm]{figures/concept/model_geometry}
    \end{subfigure}
    \hspace{-0.8cm}
    \begin{minipage}[t]{4.2cm}
      \begin{subfigure}[t]{4.2cm}
        \centering
        \caption{}\label{fig:model_geometry_zoom}
        \includegraphics[width=4.2cm]{figures/concept/model_geometry_zoom}
      \end{subfigure}
      \begin{minipage}[t]{4.5cm}
        \hspace{0.7cm}
        \begin{subfigure}[t]{1.2cm}
          \centering
          \caption{}\label{fig:model_geometry_vs_wedge}
          \includegraphics[width=1.2cm]{figures/concept/model_geometry_vs_wedge}
        \end{subfigure}
        \begin{subfigure}[t]{2cm}
          \centering
          \caption{}\label{fig:model_charge_density}
          \includegraphics[width=2cm]{figures/concept/model_charge_density}
        \end{subfigure}
      \end{minipage}
    \end{minipage}
  \end{minipage}
%	\begin{subfigure}
%		\centering
%		\caption{}\label{fig:model_concept}
%		\includegraphics[scale=1]{figures/concept/model_concept}
%	\end{subfigure}

% caption
\caption[Geometry, charge distribution and boundary conditions of a 2D-axisymmetric of ClyA.]
{
\textbf{Atomistic model of ClyA-AS.}
(\subref{fig:clya_atomistic_side}) Cross-sectional and (\subref{fig:clya_atomistic_top}) top views of an 
all-atom model of the dodecameric nanopore ClyA-AS embedded in a lipid bilayer.
(a) 2D-axisymmetric model of ClyA (grey) embedded in a lipid bilayer (green) and surrounded by a spherical water reservoir (blue).
The former two are represented by solid dielectric blocks a relative permittivity of $\permittivity_p=20$ and $\permittivity_m=3.2$, respectively.
A bias voltage ($\varpotential_\textrm{bias}$) is applied from the \textit{trans} reservoir boundary, while the \textit{cis} side is kept grounded.
Both boundaries are set up to mimic an endless reservoir of ions and water.
(b) Zoom-in clearly showing the nanopore geometry.
Note that the in the ePNP-NS model, the fluid viscosity $\viscosity$, the ion diffusion coefficients $\diffusion_{i}$ and the ion mobilities $\mobility_{i}$ are 
a function of  the local average ion concentration $\avionconc$ and distance from the nanopore wall $\walldistance$,
while the fluid's relative permittivity and $\permittivity_w$ and its density $\density$ are dependent only on $\avionconc$.
The boundaries of nanopore and lipid bilayer in contact with the reservoir are set to no-slip (zero fluid velocity) and no-flux (impermeable to ions).
(c) The fixed space charge density map of ClyA-AS ($\scdpore$), obtained by Gaussian projection of all charged atoms onto a 2D plane (see methods for details).
}

% label
\label{fig:model_concept}

\end{figure}

%% LONG CAPTION
%\textbf{Geometry, charge distribution and boundary conditions of a 2D-axisymmetric of ClyA.}
%(a) 2D-axisymmetric model of ClyA (grey) embedded in a \SI{2.8}{nm} thick lipid bilayer (green), both represented by solid dielectric blocks with permittivities $\relperm_p$ and $\relperm_m$, respectively.
%They are surrounded by a spherical water reservoir with a radius of \SI{250}{\nano\meter} (blue).
%The \textit{cis} and \textit{trans} boundaries are set-up to mimic an infinite reservoir by fixing their ion concentrations ($\concentration_{i} = \concentration_{i,\textrm{bulk}}$) and
%by allowing for unrestricted fluid flow across them $\sigma_f\normvec=\left[-\pressure\identity + \viscosity\left(\nabla\velocity + \left(\nabla\velocity\right)^{\rm T}\right)\right]\normvec=0$).
%A bias potential ($\potential=\varpotential_\textrm{bulk}$) is applied at \textit{trans} while the \textit{cis} boundary is kept grounded ($\potential=0$).
%(b) Zoom-in clearly showing the nanopore geometry.
%The reservoir's relative permittivity $\relperm_w$ and density $\density$ are only dependent on the local average ion concentration $\avionconc$,
%while its viscosity $\viscosity$, the ion diffusion coeffiecients $\diffusion_{i}$ and ion mobilities $\mobility_{i}$ are a function of both $\avionconc$ and the distance from the nanopore walls $\walldistance$.
%The nanopore and lipid bilayer boundaries  in contact with the reservoir have a no flux ($-\normvec\cdot\vec{\flux_{i}}=0$) and no slip ($\vec{\velocity}=0$) boundary conditions.
%(c) Radially averaged charge density map of ClyA-AS, obtained by summation of all charged atoms onto a 2D plane, where each atom was represented as a 2D-Gaussian with a variance proportional the atom radius and a total integral equal to the net charge (see methods for details).
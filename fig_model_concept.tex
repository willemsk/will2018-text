\begin{figure}[htbp]
  
	\centering
	\begin{minipage}[t]{3.5cm}
		\begin{subfigure}[t]{3.5cm}
			\centering
			\caption{}\label{fig:clya_atomistic_side}
			\includegraphics[width=3.5cm]{figures/concept/clya_side}
		\end{subfigure}
		\begin{subfigure}[t]{3.5cm}
			\centering
			\caption{}\label{fig:clya_atomistic_top}
			\includegraphics[width=3.5cm]{figures/concept/clya_top}
		\end{subfigure}
	\end{minipage}
  \hspace{1cm}
  \begin{minipage}[t]{8.25cm}
    \begin{subfigure}[t]{4.5cm}
      \centering
      \caption{}\label{fig:model_geometry}
      \includegraphics[width=4.5cm]{figures/concept/model_geometry}
    \end{subfigure}
    \hspace{-0.8cm}
    \begin{minipage}[t]{4.2cm}
      \begin{subfigure}[t]{4.2cm}
        \centering
        \caption{}\label{fig:model_geometry_zoom}
        \includegraphics[width=4.2cm]{figures/concept/model_geometry_zoom}
      \end{subfigure}
      \vspace{0.5cm}
      \begin{minipage}[t]{4.5cm}
        \hspace{1cm}
        \begin{subfigure}[t]{1.2cm}
          \centering
          \caption{}\label{fig:model_geometry_vs_wedge}
          \vspace{-3mm}
          \includegraphics[width=1.2cm]{figures/concept/model_geometry_vs_wedge}
        \end{subfigure}
        \begin{subfigure}[t]{2cm}
          \centering
          \caption{}\label{fig:model_charge_density}
          \vspace{-3mm}
          \includegraphics[width=2cm]{figures/concept/model_charge_density}
        \end{subfigure}
      \end{minipage}
    \end{minipage}
  \end{minipage}


\caption[All-atom and 2D-axisymmetric models of ClyA.]
{
\textbf{All-atom and 2D-axisymmetric models of ClyA.} (\subref{fig:clya_atomistic_side}) Cross-sectional and 
(\subref{fig:clya_atomistic_top}) top views of an all-atom model of the dodecameric nanopore ClyA-AS embedded 
in a lipid bilayer.\cite{soskine2013}
(\subref{fig:model_geometry}) and (\subref{fig:model_geometry_zoom}) 2D-axisymmetric model of ClyA (grey) 
embedded in a lipid bilayer (green) and surrounded by a spherical water reservoir (blue). The relative 
permittivity of the latter, $\permittivity_w$, is dependent on the local average ion concentration 
$\avionconc$, while the former two have a fixed permittivity of $\permittivity_p=20$ and 
$\permittivity_m=3.2$, respectively. A bias voltage is applied at the \textit{trans} reservoir boundary 
($\potential = \varpotential_\textrm{bias}$), red), while the \textit{cis} side is kept grounded ($\potential 
= 0$, blue). The fixed concentration ($\concentration_i = \concentration_{i,\text{bulk}}$) and no normal 
stress ($\sigma_f \vec{n} = 0$) boundary conditions serve to mimic an infinite reservoir. Note that the in 
the ePNP-NS model, the viscosity $\viscosity$, the ion diffusion coefficients $\diffusion_{i}$ and the ion 
mobilities $\mobility_{i}$ are also a function of both $\avionconc$ and $\walldistance$, the distance 
from the nanopore wall. The fluid density is solely dependent on $\avionconc$. The walls of the nanopore and 
lipid bilayer in contact with the reservoir have Dirichlet boundary conditions, i.e. no-slip ($\vec{u} = 0$) 
for fluids and and no-flux ($-\vec{n} \cdot \vec{\flux_i} = 0$) for ions.
(\subref{fig:model_geometry_vs_wedge}) The 2D-axisymmetric geometry was derived directly from the all-atom 
model by computing the average inner and outer radii along the longitudinal axis of the pore, and hence 
closely follows the outline of a \ang{30} wedge out of the homology model. 
(\subref{fig:model_charge_density}) The fixed space charge density ($\scdpore$) map of ClyA-AS, obtained by 
Gaussian projection of each atom's partial charge onto a 2D plane (see methods for details).
}

\label{fig:model_concept}

\end{figure}
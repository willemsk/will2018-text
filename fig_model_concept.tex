\begin{figure*}[!t]
  
	\centering
	\begin{minipage}[t]{5cm}
		\begin{subfigure}[t]{5cm}
			\centering
			\caption{}\label{fig:clya_side}
			\includegraphics[width=5cm]{figures/concept/clya_side}
		\end{subfigure}
		\begin{subfigure}[t]{5cm}
			\centering
			\caption{}\label{fig:clya_top}
			\includegraphics[width=5cm]{figures/concept/clya_top}
		\end{subfigure}
	\end{minipage}
  %\hspace{1cm}
  \begin{minipage}[t]{11.5cm}
    \begin{subfigure}[t]{6cm}
      \centering
      \caption{}\label{fig:model_geometry}
      \includegraphics[width=6cm]{figures/concept/model_geometry}
    \end{subfigure}
    \hspace{-0.8cm}
    \begin{minipage}[t]{5.5cm}
      \begin{subfigure}[t]{5.5cm}
        \centering
        \caption{}\label{fig:model_geometry_zoom}
        \includegraphics[width=5.5cm]{figures/concept/model_geometry_zoom}
        %\vspace{0.25cm}
      \end{subfigure}
      \begin{minipage}[t]{5.5cm}
        \hspace{1cm}
        \begin{subfigure}[t]{1.6cm}
          \centering
          \caption{}\label{fig:model_geometry_vs_wedge}
          \vspace{-3mm}
          \includegraphics[width=1.6cm]{figures/concept/model_geometry_vs_wedge}
        \end{subfigure}
        \begin{subfigure}[t]{2.5cm}
          \centering
          \caption{}\label{fig:model_charge_density}
          \vspace{-3mm}
          \includegraphics[width=2.5cm]{figures/concept/model_charge_density}
        \end{subfigure}
      \end{minipage}
    \end{minipage}
  \end{minipage}

\caption[All-atom and 2D-axisymmetric models of ClyA.]
{
\textbf{All-atom and 2D-axisymmetric models of ClyA.}
(\subref{fig:clya_side}) Cross-sectional and (\subref{fig:clya_top}) top views of the dodecameric nanopore 
ClyA-AS\cite{Soskine-2013} embedded in a lipid bilayer, derived through homology modelling from the 
\textit{E. coli} Cytolysin A crystal structure (PDBID: 2WCD)\cite{Mueller-2009}.
(\subref{fig:model_geometry}) The full 2D-axisymmetric simulation geometry and 
(\subref{fig:model_geometry_zoom}) a close-up of the nanopore itself. ClyA (grey) is embedded in a lipid 
bilayer (green) and surrounded by a spherical water reservoir (blue). The relative permittivity of the 
latter, $\permittivity_w$, is dependent on the local average ion concentration $\avionconc$, while the former 
two have a fixed permittivity of $\permittivity_p=20$ and $\permittivity_m=3.2$, respectively. A bias voltage 
is applied at the \trans\ reservoir boundary ($\potential = \vbias$), red), while the \cis\ side is kept 
grounded ($\potential = 0$, blue). The fixed concentration ($\concentration_i = \concentration_{\text{s},i}$) 
and no normal stress ($\sigma_f \vec{n} = 0$) boundary conditions serve to mimic an infinite reservoir. Note 
that the in the ePNP-NS model, the viscosity $\viscosity$, the ion diffusion coefficients $\diffusion_{i}$ 
and the ion mobilities $\mobility_{i}$ are also a function of both $\avionconc$ and $\walldistance$, the 
distance from the nanopore wall. The fluid density is solely dependent on $\avionconc$. The walls of the 
nanopore and lipid bilayer in contact with the reservoir have Dirichlet boundary conditions, i.e. no-slip 
($\vec{u} = 0$) for fluids and and no-flux ($-\vec{n} \cdot \vec{\flux_i} = 0$) for ions.
(\subref{fig:model_geometry_vs_wedge})
The 2D-axisymmetric geometry was derived directly from the all-atom model by computing the average inner and 
outer radii along the longitudinal axis of the pore, and hence closely follows the outline of a \ang{30} 
wedge out of the homology model. 
(\subref{fig:model_charge_density})
The fixed space charge density ($\scdpore$) map of ClyA-AS, obtained by Gaussian projection of each atom's 
partial charge onto a 2D plane (see methods for details).
}\label{fig:model_concept}
\end{figure*}
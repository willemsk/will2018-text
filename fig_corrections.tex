
% create the figure
\begin{figure*}[!b]

\centering

% figure path
\includegraphics[width=\textwidth]{figures/si/fig_corrections}

% caption
\caption
[\textbf{Concentration and positional dependent electrolyte properties.}]
{
\textbf{Concentration and positional dependent electrolyte properties.}
(a)
Dependency of the ion self-diffusion coefficients (top, \ce{Na+}: \graphline{solid}{graphblue} and \ce{Cl-}:
\graphline{solid}{graphred}) and the ion electrophoretic mobilities (bottom, \ce{Na+}:
\graphline{dashed}{graphblue} and \ce{Cl-}: \graphline{dashed}{graphred}) on the bulk \ce{NaCl}
concentration. Lines represent empirical fits to experimental literature data (markers). The number next to
the left and right axes correspond to the values at infinite dilution (\SI{0}{\Molar}) and saturation
(\SI{\approx5}{\Molar}),  respectively.
(b)
Dependency of electrolyte density (top, \graphline{solid}{graphpurple}), viscosity (middle,
\graphline{dashed}{graphpurple}) and relative permittivity (bottom, \graphline{dotted}{graphpurple}) on the
bulk \ce{NaCl} concentration. Lines represent empirical fits to experimental literature data (markers).
(c)
Dependency of the relative ion diffusion coefficient and the mobility (top, \graphline{solid}{graphgreen}) and
the relative viscosity (bottom, \graphline{dashed}{graphgreen}) on the distance from the nanopore wall. The
relative diffusion coefficient (and its mobility) declines sharply when an ion approaches within
\SI{\approx1.0}{\nm} of the protein wall. For water molecules, this phenomenon is modelled as a sharp increase
of the fluid viscosity within \SI{\approx1.0}{\nm} distance from the wall. The values for the diffusion
coefficient were taken directly from ref. \citenum{Simakov-2010}, who used an empirical fit on molecular
dynamics data from ref. \citenum{Makarov-1998}. The inverse relative viscosity data was taken directly from
the molecular dynamics study in ref. \citenum{Pronk-2014}, and fitted with a logistic function after
offsetting for the protein hydrodynamic radius.
}

% label
\label{fig:corrections}

\end{figure*}

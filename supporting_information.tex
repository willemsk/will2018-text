%-------------------------------------------------------------------------------
% PREAMBLE AND DOCUMENT FORMATTING
%-------------------------------------------------------------------------------
\documentclass[journal=ancac3, manuscript=suppinfo, etalmode=truncate,maxauthors=0]{achemso}
\setkeys{acs}{etalmode=truncate,maxauthors=0}

% PACKAGES
\usepackage[utf8]{inputenc}
\usepackage[english]{babel}
\usepackage{csquotes}
\usepackage{amsmath}
\usepackage{amsfonts}
\usepackage{amssymb}
\usepackage{textcomp}
\usepackage{textgreek}

\usepackage[version=4]{mhchem}
\usepackage[alsoload=synchem,separate-uncertainty=true,multi-part-units=single,tight-spacing=true]{siunitx}

\usepackage{float}
\usepackage{graphicx}
\usepackage{xcolor}
\usepackage{tikz}
\usetikzlibrary{shapes}
\usepackage{array, booktabs, tabularx, multirow}

% CAPTION FORMATTING
\usepackage[font=footnotesize,labelfont=bf,labelsep=period]{caption} 							% format single-image captions and table titles
\captionsetup[table]{singlelinecheck=false,font=footnotesize,labelfont=bf}
\usepackage[font=footnotesize,labelfont=bf,labelsep=period]{subcaption} 						% format subfigure captions
%\DeclareCaptionSubType*[alph]{figure}
%\renewcommand\thesubfigure{\thefigure\alph{subfigure}}
\captionsetup[subfigure]{labelfont=bf,textfont=normalfont,labelformat=simple,singlelinecheck=false} 	

% CROSS-REFERENCE FORMATTING
% For use with the cleveref package
% Define the format of Figure, Table, Equation, and Section cross-references in the text
\usepackage{xr-hyper}
\usepackage{hyperref}
\usepackage{cleveref}
\crefname{figure}{Fig.}{Figs.}
\Crefname{figure}{Figure}{Figures}
\crefname{table}{Tab.}{Tabs.}
\Crefname{table}{Table}{Tables}
\crefname{equation}{Eq.}{Eqs.}
\Crefname{equation}{Equation}{Equations}
\crefname{section}{Sec.}{Secs.}
\Crefname{section}{Section}{Sections}


% REFERENCES
\renewcommand*{\bibfont}{\normalfont\small}


%-------------------------------------------------------------------------------
% CUSTOM COMMANDS
%-------------------------------------------------------------------------------


% Shorthands
\newcommand{\todo}[1]{\textbf{\textcolor{orange}{#1}}}
\newcommand{\ahl}{\textalpha HL}
\newcommand{\etal}{\textit{et al.}}
\newcommand{\cis}{\textit{cis}}
\newcommand{\trans}{\textit{trans}}

% Units
\DeclareSIUnit{\molar}{\mole\per\cubic\deci\metre}
\DeclareSIUnit{\Molar}{\textsc{M}}
\newcommand{\mM}{\milli\Molar}
\newcommand{\mV}{\milli\volt}
\newcommand{\mps}{\meter\per\second}
\newcommand{\cnmpnspv}{\cubic\nano\meter\per\nano\second\per\volt}

% Vectors and math stuff
\renewcommand{\vec}[1]{\boldsymbol{#1}}
\newcommand{\rpos}{\vec{r}} % positional vector
\newcommand{\normvec}{\vec{\hat{n}}} % normal vector
\newcommand{\identity}{\vec{\rm I}} % Identity vector
\newcommand{\stdev}{\sigma}
\newcommand{\pav}[2]{\left< #1 \right>_{\text{#2}}} % Pore average
\newcommand{\vc}[2]{#1_{#2}} 
\newcommand{\pd}[2]{\displaystyle\frac{\partial #1}{\partial #2}}
\newcommand{\hydrostresstensor}{\sigma_{ij}}

% Physical constants
\newcommand{\boltzmann}{k_{\rm B}}
\newcommand{\avogadro}{N_{\rm A}}
\newcommand{\temp}{T}
\newcommand{\faraday}{\mathcal{F}}
\newcommand{\echarge}{e}

% Field variables
\newcommand{\vel}{u}
\newcommand{\force}{F}
\newcommand{\potential}{\varphi}			% Electrostatic potential
\newcommand{\varpotential}{V}				% Electrostatic potential
\newcommand{\concentration}{c}				% Ion concentration
\newcommand{\velocity}{\vec{u}}				% Fluid velocity
\newcommand{\pressure}{p}					% Fluid pressure
\newcommand{\efield}{\vec{E}}				% Electrical field
\newcommand{\displacement}{\vec{D}} % Electrical displacement fiedl
\newcommand{\walldistance}{d}				% Wall distance

% Dimensionless variables
\newcommand{\dconc}{\bar{\concentration}}		% Dimensionless concentration
\newcommand{\dwall}{\bar{\walldistance}}		% Dimensionless wall distance

% Material and ion parameters
\newcommand{\permittivity}{\varepsilon}
\newcommand{\absperm}{\permittivity_0}				% Permittivity of vacuum
\newcommand{\relperm}{\varepsilon_r}				% Relative permittivity
\newcommand{\dielectric}{\relperm}				% Relative permittivity
\newcommand{\diffusion}{\mathcal{D}}			% Diffusion coefficient
\newcommand{\mobility}{\mu}						% Electrophoretic mobility
\newcommand{\transportn}{t}						% Transport number
\newcommand{\chargen}{z}						% Ion charge number
\newcommand{\ionsize}{a}						% Ion size for smPNP
\newcommand{\density}{\varrho}						% Fluid density
\newcommand{\viscosity}{\eta}					% Fluid viscosity

\newcommand{\iondiffusion}[1]{\diffusion_{#1}}	% Ion Diffusion coefficient
\newcommand{\ionmobility}[1]{\mobility_{#1}}	% Ion electrophoretic mobility
\newcommand{\iontransportn}[1]{\transportn{#1}}	% Ion transport number
\newcommand{\avionconc}{\langle\concentration\rangle}

\newcommand{\tna}{\transportn_{\ce{Na+}}}

\newcommand{\molarconductivity}{\Lambda}
\newcommand{\specmolarconductivity}{\lambda}

% Derived properties
\newcommand{\scd}{\rho}							% Total charge density
\newcommand{\scdpore}{\scd_{\text{pore}}^f}		% Fixed charge density
\newcommand{\scdion}{\scd_{\text{ion}}}			% Ionic charge density
\newcommand{\flux}{\vec{J}} 							% Ion flux
\newcommand{\volumeforce}{\vec{\force}_{\rm ion}}	% Fluid volume force

\newcommand{\current}{I}
\newcommand{\currentsim}{\current_{\rm sim}}
\newcommand{\currentexp}{\current_{\rm exp}}
\newcommand{\conductance}{G}
\newcommand{\icr}{\alpha}

% Shorthands
\newcommand{\ci}{\concentration_{i}}
\newcommand{\cbulk}{\concentration_\text{s}}
\newcommand{\vbias}{\varpotential_\text{b}}
\newcommand{\Na}{\ce{Na+}}
\newcommand{\Cl}{\ce{Cl-}}
\newcommand{\Qion}{Q_{\rm ion}}
\newcommand{\radpot}{\left<\potential\right>_\text{rad}}
\newcommand{\radenergy}{\left< U_{\text{E},i} \right>_{\text{rad}}}
\newcommand{\deltaEt}{\Delta E_{\text{B},i}}
\newcommand{\pH}[1]{pH~\num{#1}}
\newcommand{\kT}{\boltzmann\temp}
\newcommand{\kTe}{\kT / \echarge}

% Atom properties
\newcommand{\partialcharge}{\delta}
\newcommand{\atomradius}{R}


% Colors
\definecolor{graphgreen}  {rgb}{0.30196078, 0.68627451, 0.29019608}
\definecolor{graphpurple} {rgb}{0.59607843, 0.30588235, 0.63921569}
\definecolor{graphblue}   {rgb}{0.29803921, 0.57254902, 0.98823529}
\definecolor{graphred}    {rgb}{0.91372549, 0.15686275, 0.18823529}

% Line and marker styles
\newcommand{\graphline}[2]{\raisebox{2pt}{\tikz{\draw[-,color=#2,#1,line width=1.5pt](0,0) -- (5mm,0);}}}
\newcommand{\graphmarker}[2]{\raisebox{0.5pt}{\tikz{\node[draw,scale=0.5,#1,fill=none, color=#2](){};}}}

%\newcommand{\graphlinemarker}[3]{\raisebox{0pt}{\tikz{\draw[-,#3,#1,line width = 1.0pt](2.mm,0) #2 (3.5mm,1.5mm);\draw[-,#3,#1,line width = 1.0pt](0.,0.8mm) -- (5.5mm,0.8mm)}}}
%\newcommand{\rectangle}{\raisebox{0pt}{\tikz{\draw[-,black,dotted,line width = 1.0pt](0.,0.8mm) -- (5.5mm,0.8mm);\draw[black,solid,line width = 1.0pt](2.mm,0) circle (3.5mm,1.5mm)}}}
\newcommand{\rectangle}[1]{\raisebox{0pt}{\tikz{\draw[-,black,solid,line width = 1.0pt](0.,0.8mm) -- (5.5mm,0.8mm);\draw[black,solid,line width = 1.0pt](2.mm,0) #1 (3.5mm,1.5mm)}}}



%-------------------------------------------------------------------------------
% TITLE AND AUTHOR LIST
%-------------------------------------------------------------------------------
\title{Modelling of Ion and Water Transport in the Biological Nanopore ClyA}
\newcommand{\afimec}{imec, Kapeldreef 75, B-3001 Leuven, Belgium}
\newcommand{\afkulchem}{KU Leuven, Department of Chemistry, Celestijnenlaan 200F, B-3001 Leuven, Belgium}
\newcommand{\afkulphys}{KU Leuven, Department of Physics and Astronomy, Celestijnenlaan 200D, B-3001 Leuven, Belgium}
\newcommand{\afrug}{University of Groningen, Groningen Biomolecular Sciences \& Biotechnology Institute, 9747 AG, Groningen, The Netherlands}

\author{Kherim Willems}
\affiliation{\afkulchem}
\alsoaffiliation{\afimec}

\author{Dino Rui\'{c}}
\affiliation{\afkulphys}
\alsoaffiliation{\afimec}

\author{Florian Lucas}
\affiliation{\afrug}

%\author{Ujjal Barman}
%\affiliation{\afimec}

%\author{Chang Chen}
%\affiliation{\afimec}

\author{Johan Hofkens}
\affiliation{\afkulchem}

\author{Giovanni Maglia}
\email{g.maglia@rug.nl}
\affiliation{\afrug}

\author{Pol Van Dorpe}
\email{Pol.VanDorpe@imec.be}
\affiliation{\afkulchem}
\alsoaffiliation{\afimec}


%===============================================================================
%
% MAIN DOCUMENT BEGINS
%
%===============================================================================
\begin{document}
	
\maketitle

\section{Extended materials and methods}
\subsection{Mutations of the ClyA-AS variant.}
The mutations needed to convert the wild-type ClyA to the AS variant are listed in \cref{tab:clya_as_mutations}.
\begin{table*}[h]


\renewcommand{\arraystretch}{1.5}
\scriptsize
\caption{Mutations of the ClyA-AS variant compared to the \textit{S. tyhpii} wild-type.}
\centering
\label{tab:clya_as_mutations}
\begin{tabular}{rll}
	\toprule
	Position	& WT	& AS \\
	\midrule
	8			& Lys	& Gln	\\
	15			& Asn	& Ser	\\
	38			& Gln	& Lys	\\
	57			& Ala	& Glu 	\\
	67			& Thr	& Val 	\\
	87			& Cys	& Ala 	\\
	90			& Ala	& Val	\\
	95			& Ala	& Ser 	\\
	99			& Leu	& Gln 	\\
	103	 		& Glu	& Gly	\\
	118			& Lys	& Arg 	\\
	119			& Leu	& Ile	\\
	124			& Ile	& Val	\\
	125			& Thr	& Lys	\\
	136			& Val	& Thr	\\
	166			& Phe	& Tyr	\\
	172			& Lys	& Arg	\\
	185			& Val	& Ile	\\
	212			& Lys	& Asn	\\
	214			& Lys	& Arg	\\
	217			& Ser	& Thr	\\
	224			& Thr	& Ser	\\
	227			& Asn	& Ala	\\
	244			& Thr 	& Ala	\\
	276			& Glu	& Gly	\\
	285			& Cys	& Ser	\\
	290			& Lys	& Gln	\\
	\bottomrule
\end{tabular}
\end{table*}

%K8Q
%N15S
%Q38K
%A57E
%T67V
%C87A
%A90V
%A95S
%L99Q
%E103G
%K118R
%L119I
%I124V
%T125K
%V136T
%F166Y
%K172R
%V185I
%K212N
%K214R
%S217T
%T224S
%N227A
%T244A
%E276G
%C285S
%K290Q


\subsection{Fitting of electrolyte properties}
The parameters of the fitting functions used to interpolate the experimental ion diffusion coefficients, mobilities and transport numbers,
and the electrolyte viscosity, density and relative permittivity are given in \cref{tab:corrections_parameters} and 
the resulting curves are plotted in \cref{fig:corrections}.
Note that since most of these functions are merely empirical fits with no physical meaning,
and that they are solely used to interpolate and represent the experimental data.

Finally, for the concentration dependence of the relative permittivity we made use of the model proposed by Gavish et al.:
\begin{align}
\relperm(\dconc) = \permittivity_{r,0} - \left(\permittivity_{r,0} - \permittivity_{r,ms}\right) L \left( \dfrac{3\alpha}{\permittivity_{r,0} - \permittivity_{r,ms}} \dconc \right)
\end{align}

\begin{table*}[ht]

\sisetup{inter-unit-product=\ensuremath{{}\cdot{}}}

\renewcommand{\arraystretch}{1.5}
\scriptsize
\caption{Overview of the \ce{NaCl} fitting parameters used for interpolation.}
\centering
\label{tab:corrections_parameters}
\begin{tabular}{@{}
				l
				S[table-format=1.3]
				S[table-format=-1.2(2)e-1]
				S[table-format=-1.2(2)e-1]
				S[table-format=-1.2(2)e-1]
				S[table-format=-1.2(2)e-1]
				S[table-format=>1.2]
				l
				@{}}
	\toprule
							& \multicolumn{5}{c}{Fitting parameters}									&		&	\\
	\cmidrule{2-6}
	Property				& $P_{0}$	& $P_{1}$		& $P_{2}$		& $P_{3}$		& $P_{4}$		& $R^{2}$	& References	\\
	\midrule
	$\diffusion^+(c)$		& 1.334		& 2.02(14)e-1	& -3.05(41)e-1	& 2.19(38)e-1	& -3.13(108)e-2	& >0.99		& 
	\citenum{Mills-1989}	\\
	$\diffusion^-(c)$		& 2.032		& 1.49(30)e-1	& -4.94(904)e-2	& 3.40(826)e-2	& 1.43(230)e-2	& >0.99		& 
	\citenum{Mills-1989}	\\
	$\mobility^+(c)$		& 5.192		& 7.91(6)e-1	& -3.53(17)e-1	& 1.46(15)e-1	& 9.23(389)e-3	& >0.99		& 
	\citenum{Bianchi-1989,Currie-1960,Goldsack-1976,DellaMonica-1979}	\\
	$\mobility^-(c)$		& 7.909		& 6.29(6)e-1	& -4.29(17)e-1	& 2.12(14)e-1	& -1.07(37)e-2	& >0.99		& 
	\citenum{Bianchi-1989,Currie-1960,Goldsack-1976,DellaMonica-1979}	\\
	$\transportn^+(c)$		& 0.3963	& 9.38(164)e-2 	& 2.86(324)e-3	& -1.88(652)e-2	& 4.51(275)e-3	& 0.98		& 
	\citenum{Panopoulos-1986,Currie-1960,Smits-1966,Schonert-2013,DellaMonica-1979}	\\
	$\viscosity(c)$			& 0.8904	& 7.56(27)e-3	& 7.77(4)e-2	& 1.19(1)e-2	& 5.95(35)e-4	& >0.99		& 
	\citenum{Hai-Lang-1996}	\\
	$\density(c)$			& 0.997		& 4.06(1)e-2	& -6.39(16)e-4	& 				& 				& >0.99		& 
	\citenum{Hai-Lang-1996}	\\
	$\permittivity(c)$		& 78.15		& 3.08(0)e1		& 1.15(0)e1		& 				& 				&			& 
	\citenum{Gavish-2016}	\\
	$\diffusion^+(d)$		&			& 6.2 			& 0.01			& 				& 				&			& 
	\citenum{Makarov-1998,Simakov-2010,Pederson-2015}	\\
	$\diffusion^-(d)$	 	& 			& 6.2			& 0.01			& 				& 				&			& 
	\citenum{Makarov-1998,Simakov-2010,Pederson-2015}	\\
	$\mobility^+(d)$		& 			& 6.2			& 0.01			& 				& 				&			& 
	\citenum{Makarov-1998,Simakov-2010,Pederson-2015}	\\
	$\mobility^-(d)$		& 			& 6.2			& 0.01			& 				& 				&			& 
	\citenum{Makarov-1998,Simakov-2010,Pederson-2015}	\\
	$\viscosity(d)$			& 			& 3.36(23)		& 1.47(23)e-1 	& 				& 				& 0.97		& 
	\citenum{Pronk-2014}	\\
	\bottomrule
\end{tabular}
\end{table*}


% NaCl transport numbers
% panopoulos1986, currie1960, smits1966, schonert2014, dellamonica1979
% NaCl conductance
% bianchi1989, currie1960, goldsack1976, dellamonica1979

% create the figure
\begin{figure*}[!b]

\centering

% figure path
\includegraphics[width=\textwidth]{figures/si/fig_corrections}

% caption
\caption
[\textbf{Concentration and positional dependent electrolyte properties.}]
{
\textbf{Concentration and positional dependent electrolyte properties.}
(a)
Dependency of the ion self-diffusion coefficients (top, \ce{Na+}: \graphline{solid}{graphblue} and \ce{Cl-}: 
\graphline{solid}{graphred}) and the ion electrophoretic mobilities (bottom, \ce{Na+}: 
\graphline{dashed}{graphblue} and \ce{Cl-}: \graphline{dashed}{graphred}) on the bulk \ce{NaCl} 
concentration. Lines represent empirical fits to experimental literature data (markers). The number next to 
the left and right axes correspond to the values at infinite dilution ($0$~M) and saturation ($\approx5$~M), 
respectively.
(b)
Dependency of electrolyte density (top, \graphline{solid}{graphpurple}), viscosity (middle, 
\graphline{dashed}{graphpurple}) and relative permittivity (bottom, \graphline{dotted}{graphpurple}) on the 
bulk \ce{NaCl} concentration. Lines represent empirical fits to experimental literature data (markers).
(c)
Dependency of the relative ion diffusion coefficient and the mobility (top, \graphline{solid}{graphgreen}) 
and the relative viscosity (bottom, \graphline{dashed}{graphgreen}) on the distance from the nanopore wall. 
The relative diffusion coefficient (and its mobility) declines sharply when an ion approaches within 
$\approx1.0$~nm of the protein wall. For water molecules, this phenomenon is modelled as a sharp increase of 
the fluid viscosity within $\approx1.0$~nm distance from the wall. The values for the diffusion coefficient 
were taken directly from ref. \citenum{Simakov-2010}, who used an empirical fit on molecular dynamics data 
from ref. \citenum{Makarov-1998}. The inverse relative viscosity data was taken directly from the molecular 
dynamics study in ref. \citenum{Pronk-2014}, and fitted with a logistic function after offsetting for the 
protein hydrodynamic radius.
}

% label
\label{fig:corrections}

\end{figure*}


\begin{figure}[!htb]
\centering
\begin{minipage}[t]{8.2cm}
\begin{subfigure}[t]{8.2cm}
	\centering
	\caption{}\vspace{-3mm}\label{fig:ion_charge_density_contours}
	\includegraphics[scale=1]{../figures/charge_density/ion_charge_density_contours}
\end{subfigure}
\begin{subfigure}[t]{8.2cm}
  \centering
  \caption{}\vspace{-3mm}\label{fig:ion_charge_density_radial_profiles}
  \includegraphics[scale=1]{../figures/charge_density/ion_charge_density_radial_profiles}
\end{subfigure}
\begin{subfigure}[t]{8.2cm}
	\centering
	\caption{}\vspace{-3mm}\label{fig:ion_charge_pore_bulk_surface_total_vs_concentration}
	\includegraphics[scale=1]{../figures/charge_density/ion_charge_pore_bulk_surface_total_vs_concentration}
\end{subfigure}
\end{minipage}

% caption
\caption
[\textbf{Ion space charge density distribution inside ClyA.}]
{
\textbf{Ion space charge density distribution inside ClyA.}
(\subref{fig:ion_charge_density_contours}) Cross-section contour plots of the ion space charge density
($\scdion$), expressed as number of elementary charges per \si{\cubic\nano\meter}, at \SI{0}{\mV} applied bias
voltage and for salt concentrations \SIlist{0.005;0.05;0.5;5}{\Molar}.
(\subref{fig:ion_charge_density_radial_profiles}) Radial cross-sections of the $\scdion$ at the center of the
constriction ($z=\SI{-0.3}{\nm}$) and the lumen ($z=\SI{5}{\nm}$) of ClyA. The vertical line represents the
the division between ions in the `bulk' ($d>\SI{0.5}{\nm}$) of the pore and those located near its surface
($d\le\SI{0.5}{\nm}$).
(\subref{fig:ion_charge_pore_bulk_surface_total_vs_concentration}) The average number of ionic charges inside
the pore $\pav{\Qion}{p}$, is distributed between the those close to the pore's surface $\pav{\Qion}{s}$, i.e.
within \SI{0.5}{\nm} of the wall, and those in the `bulk' of the pore's interior $\pav{\Qion}{b}$.
}\label{fig:ion_charge_density}

\end{figure}

\begin{figure*}[!htb]
  \centering
  \begin{minipage}[t]{10.75cm}
    \begin{subfigure}[t]{5.5cm}
      \centering
      \caption{}\vspace{-3mm}\label{fig:pressure_contour}
      \includegraphics[scale=1]{figures/pressure/pressure_contour_150mM_+000mV}
    \end{subfigure}
    \hspace{-5mm}
    \begin{subfigure}[t]{2.5cm}
      \centering
      \caption{}\vspace{-3mm}\label{fig:pressure_radial_averages}
      \includegraphics[scale=1]{figures/pressure/pressure_radial_averages_+000mV}
    \end{subfigure}
  \end{minipage}
\centering

% caption
\caption
[\textbf{Pressure distribution inside ClyA.}]
{
\textbf{Pressure distribution inside ClyA.}
(\subref{fig:pressure_contour}) Contourmap of the pressure at $\cbulk=\SI{0.15}{\Molar}$ and
$\vbias=\SI{0}{\mV}$, showing that the \Na\ concentration `hotspots' near the pore wall result in build-up of
electro-osmotic pressure (\SIrange{5}{30}{\atm}) inside the  confined fluid. Pressure drops of such magnitude
over the course of a few nanometers could potentially exert a significant force on a captured
protein.\cite{Hoogerheide-2014}
(\subref{fig:pressure_radial_averages}) The axial pressure profile and averaged along the the entire radius of
the pore at $\vbias=\SI{0}{\mV}$.
%At concentrations $\ge0.5$~M, a negative pressure develops in the `bulk' of the lumen due to the
}\label{fig:pressure}

\end{figure*}




%-------------------------------------------------------------------------------
% SUPPORTING INFORMATION
%-------------------------------------------------------------------------------
\bibliography{modeling1}

\end{document}

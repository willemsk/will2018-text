%-------------------------------------------------------------------------------
% PREAMBLE AND DOCUMENT FORMATTING
%-------------------------------------------------------------------------------
\documentclass[journal=ancac3, manuscript=suppinfo, etalmode=truncate,maxauthors=0]{achemso}
\setkeys{acs}{etalmode=truncate,maxauthors=0}

% PACKAGES
\usepackage[utf8]{inputenc}
\usepackage[english]{babel}
\usepackage{csquotes}
\usepackage{amsmath}
\usepackage{amsfonts}
\usepackage{amssymb}
\usepackage{textcomp}
\usepackage{textgreek}

\usepackage[version=4]{mhchem}
\usepackage[alsoload=synchem,separate-uncertainty=true,multi-part-units=single,tight-spacing=true]{siunitx}

\usepackage{float}
\usepackage{graphicx}
\usepackage{xcolor}
\usepackage{tikz}
\usetikzlibrary{shapes}
\usepackage{array, booktabs, tabularx, multirow}

% CAPTION FORMATTING
\usepackage[font=footnotesize,labelfont=bf,labelsep=period]{caption} 							% format single-image captions and table titles
\captionsetup[table]{singlelinecheck=false,font=footnotesize,labelfont=bf}
\usepackage[font=footnotesize,labelfont=bf,labelsep=period]{subcaption} 						% format subfigure captions
%\DeclareCaptionSubType*[alph]{figure}
%\renewcommand\thesubfigure{\thefigure\alph{subfigure}}
\captionsetup[subfigure]{labelfont=bf,textfont=normalfont,labelformat=simple,singlelinecheck=false} 	

% CROSS-REFERENCE FORMATTING
% For use with the cleveref package
% Define the format of Figure, Table, Equation, and Section cross-references in the text
\usepackage{xr-hyper}
\usepackage{hyperref}
\usepackage{cleveref}
\crefname{figure}{Fig.}{Figs.}
\Crefname{figure}{Figure}{Figures}
\crefname{table}{Tab.}{Tabs.}
\Crefname{table}{Table}{Tables}
\crefname{equation}{Eq.}{Eqs.}
\Crefname{equation}{Equation}{Equations}
\crefname{section}{Sec.}{Secs.}
\Crefname{section}{Section}{Sections}


% REFERENCES
\renewcommand*{\bibfont}{\normalfont\small}


%-------------------------------------------------------------------------------
% CUSTOM COMMANDS
%-------------------------------------------------------------------------------


% Shorthands
\newcommand{\todo}[1]{\textbf{\textcolor{orange}{#1}}}
\newcommand{\ahl}{\textalpha HL}
\newcommand{\etal}{\textit{et al.}}
\newcommand{\cis}{\textit{cis}}
\newcommand{\trans}{\textit{trans}}

% Referencing
\newcommand{\reffig}[1]{Figure~\ref{#1}}
\newcommand{\refsubfig}[2]{\reffig{#1}#2}
\newcommand{\reftable}[1]{Table~\ref{#1}}
\newcommand{\refeq}[1]{Eq.~\ref{#1}}

% Units
\DeclareSIUnit{\molar}{\mole\per\cubic\deci\metre}
\DeclareSIUnit{\Molar}{\textsc{M}}
\newcommand{\mM}{\milli\Molar}
\newcommand{\mV}{\milli\volt}

% Vectors and math stuff
\renewcommand{\vec}[1]{\boldsymbol{#1}}
\newcommand{\rpos}{\vec{r}} % positional vector
\newcommand{\normvec}{\vec{\hat{n}}} % normal vector
\newcommand{\identity}{\vec{\rm I}} % Identity vector
\newcommand{\stdev}{\sigma}

% Physical constants
\newcommand{\boltzmann}{k_{\rm B}}
\newcommand{\avogadro}{N_{\rm A}}
\newcommand{\temp}{T}
\newcommand{\faraday}{\mathcal{F}}
\newcommand{\echarge}{e}

% Field variables
\newcommand{\potential}{\varphi}			% Electrostatic potential
\newcommand{\varpotential}{V}				% Electrostatic potential
\newcommand{\concentration}{c}				% Ion concentration
\newcommand{\velocity}{\vec{u}}				% Fluid velocity
\newcommand{\pressure}{p}					% Fluid pressure
\newcommand{\efield}{\vec{E}}				% Electrical field
\newcommand{\walldistance}{d}				% Wall distance

% Dimensionless variables
\newcommand{\dconc}{\bar{\concentration}}		% Dimensionless concentration
\newcommand{\dwall}{\bar{\walldistance}}		% Dimensionless wall distance

% Material and ion parameters
\newcommand{\permittivity}{\varepsilon}
\newcommand{\absperm}{\permittivity_0}				% Permittivity of vacuum
\newcommand{\relperm}{\varepsilon_r}				% Relative permittivity
\newcommand{\dielectric}{\relperm}				% Relative permittivity
\newcommand{\diffusion}{\mathcal{D}}			% Diffusion coefficient
\newcommand{\mobility}{\mu}						% Electrophoretic mobility
\newcommand{\transportn}{t}						% Transport number
\newcommand{\chargen}{z}						% Ion charge number
\newcommand{\ionsize}{a}						% Ion size for smPNP
\newcommand{\density}{\varrho}						% Fluid density
\newcommand{\viscosity}{\eta}					% Fluid viscosity

\newcommand{\iondiffusion}[1]{\diffusion_{#1}}	% Ion Diffusion coefficient
\newcommand{\ionmobility}[1]{\mobility_{#1}}	% Ion electrophoretic mobility
\newcommand{\iontransportn}[1]{\transportn{#1}}	% Ion transport number
\newcommand{\avionconc}{\langle\concentration_{i}\rangle}

\newcommand{\tna}{\transportn_{\ce{Na+}}}

\newcommand{\molarconductivity}{\Lambda}
\newcommand{\specmolarconductivity}{\lambda}

% Derived properties
\newcommand{\scd}{\rho}							% Total charge density
\newcommand{\scdpore}{\scd_{\rm pore}^f}		% Fixed charge density
\newcommand{\scdion}{\scd_{\rm ion}}			% Ionic charge density
\newcommand{\flux}{J} 							% Ion flux
\newcommand{\volumeforce}{\vec{F}_{\rm ion}}	% Fluid volume force

\newcommand{\current}{I}
\newcommand{\currentsim}{\current_{\rm sim}}
\newcommand{\currentexp}{\current_{\rm exp}}
\newcommand{\conductance}{G}
\newcommand{\icr}{\alpha}

% Shorthands
\newcommand{\ci}{\concentration_{i}}
\newcommand{\cbulk}{\concentration_\text{s}}
\newcommand{\vbias}{\varpotential_\text{b}}
\newcommand{\Na}{\ce{Na+}}
\newcommand{\Cl}{\ce{Cl-}}
\newcommand{\Qion}{Q_{\rm ion}}
\newcommand{\radpot}{\left<\potential\right>_\text{rad}}

% Atom properties
\newcommand{\partialcharge}{\delta}
\newcommand{\atomradius}{R}

% Chemistry 
\newcommand{\pH}[1]{pH~\num{#1}}


% Colors
\definecolor{graphgreen}  {rgb}{0.30196078, 0.68627451, 0.29019608}
\definecolor{graphpurple} {rgb}{0.59607843, 0.30588235, 0.63921569}
\definecolor{graphblue}   {rgb}{0.29803921, 0.57254902, 0.98823529}
\definecolor{graphred}    {rgb}{0.91372549, 0.15686275, 0.18823529}

% Line and marker styles
\newcommand{\graphline}[2]{\raisebox{2pt}{\tikz{\draw[-,color=#2,#1,line width=1.5pt](0,0) -- (5mm,0);}}}
\newcommand{\graphmarker}[2]{\raisebox{0.5pt}{\tikz{\node[draw,scale=0.5,#1,fill=none, color=#2](){};}}}

%\newcommand{\graphlinemarker}[3]{\raisebox{0pt}{\tikz{\draw[-,#3,#1,line width = 1.0pt](2.mm,0) #2 (3.5mm,1.5mm);\draw[-,#3,#1,line width = 1.0pt](0.,0.8mm) -- (5.5mm,0.8mm)}}}
%\newcommand{\rectangle}{\raisebox{0pt}{\tikz{\draw[-,black,dotted,line width = 1.0pt](0.,0.8mm) -- (5.5mm,0.8mm);\draw[black,solid,line width = 1.0pt](2.mm,0) circle (3.5mm,1.5mm)}}}
\newcommand{\rectangle}[1]{\raisebox{0pt}{\tikz{\draw[-,black,solid,line width = 1.0pt](0.,0.8mm) -- (5.5mm,0.8mm);\draw[black,solid,line width = 1.0pt](2.mm,0) #1 (3.5mm,1.5mm)}}}



%-------------------------------------------------------------------------------
% TITLE AND AUTHOR LIST
%-------------------------------------------------------------------------------
\title{Modelling of Ion and Water Transport in the Biological Nanopore ClyA}
\newcommand{\afimec}{imec, Kapeldreef 75, B-3001 Leuven, Belgium}
\newcommand{\afkulchem}{KU Leuven, Department of Chemistry, Celestijnenlaan 200F, B-3001 Leuven, Belgium}
\newcommand{\afkulphys}{KU Leuven, Department of Physics and Astronomy, Celestijnenlaan 200D, B-3001 Leuven, Belgium}
\newcommand{\afrug}{University of Groningen, Groningen Biomolecular Sciences \& Biotechnology Institute, 9747 AG, Groningen, The Netherlands}

\author{Kherim Willems}
\affiliation{\afkulchem}
\alsoaffiliation{\afimec}

\author{Dino Rui\'{c}}
\affiliation{\afkulphys}
\alsoaffiliation{\afimec}

\author{Florian Lucas}
\affiliation{\afrug}

\author{Ujjal Barman}
\affiliation{\afimec}

%\author{Chang Chen}
%\affiliation{\afimec}

\author{Johan Hofkens}
\affiliation{\afkulchem}

\author{Giovanni Maglia}
\email{g.maglia@rug.nl}
\affiliation{\afrug}

\author{Pol Van Dorpe}
\email{Pol.VanDorpe@imec.be}
\affiliation{\afkulchem}
\alsoaffiliation{\afimec}



%===============================================================================
%
% MAIN DOCUMENT BEGINS
%
%===============================================================================
\begin{document}
	
\maketitle

\section{Extended materials and methods}

\subsection{ClyA-AS homology model}
A full atom model of ClyA-AS\cite{Soskine-2013} was build using the MODELLER (version 9.18) software package 
by introduction of the following point mutations in each of the 12 chains of the wild-type ClyA crystal 
structure (PDBID: 
2WCD\cite{Mueller-2009}):
K8Q, N15S, Q38K, A57G, T67V, C87A, A90V, A95S, L99Q, E103G, K118R, L119I, I124V, T125K, V136T, F166Y, K172R, 
V185I, K212N, K214R, S217T, T224S, N227A, T244A, E276G, C285S, K290Q.
The resulting structure was optimized by annealing first the mutated residues alone, and subsequently also 
the closed atoms around them, using a conjugate gradient (from $150$ to $1000$ to $300$~K with $4$~fs 
timesteps).\cite{Sali-1993}

% are listed in \cref{tab:clya_as_mutations}.
% \begin{table*}[h]


\renewcommand{\arraystretch}{1.5}
\scriptsize
\caption{Mutations of the ClyA-AS variant compared to the \textit{S. tyhpii} wild-type.}
\centering
\label{tab:clya_as_mutations}
\begin{tabular}{rll}
	\toprule
	Position	& WT	& AS \\
	\midrule
	8			& Lys	& Gln	\\
	15			& Asn	& Ser	\\
	38			& Gln	& Lys	\\
	57			& Ala	& Glu 	\\
	67			& Thr	& Val 	\\
	87			& Cys	& Ala 	\\
	90			& Ala	& Val	\\
	95			& Ala	& Ser 	\\
	99			& Leu	& Gln 	\\
	103	 		& Glu	& Gly	\\
	118			& Lys	& Arg 	\\
	119			& Leu	& Ile	\\
	124			& Ile	& Val	\\
	125			& Thr	& Lys	\\
	136			& Val	& Thr	\\
	166			& Phe	& Tyr	\\
	172			& Lys	& Arg	\\
	185			& Val	& Ile	\\
	212			& Lys	& Asn	\\
	214			& Lys	& Arg	\\
	217			& Ser	& Thr	\\
	224			& Thr	& Ser	\\
	227			& Asn	& Ala	\\
	244			& Thr 	& Ala	\\
	276			& Glu	& Gly	\\
	285			& Cys	& Ser	\\
	290			& Lys	& Gln	\\
	\bottomrule
\end{tabular}
\end{table*}

%K8Q
%N15S
%Q38K
%A57E
%T67V
%C87A
%A90V
%A95S
%L99Q
%E103G
%K118R
%L119I
%I124V
%T125K
%V136T
%F166Y
%K172R
%V185I
%K212N
%K214R
%S217T
%T224S
%N227A
%T244A
%E276G
%C285S
%K290Q


\subsection{Fitting of the electrolyte properties}

% create the figure
\begin{figure*}[!b]

\centering

% figure path
\includegraphics[width=\textwidth]{../figures/si/fig_corrections}

% caption
\caption
[\textbf{Concentration and positional dependent electrolyte properties.}]
{
\textbf{Concentration and positional dependent electrolyte properties.}
(a)
Dependency of the ion self-diffusion coefficients (top, \Na\: \graphline{solid}{graphblue} and \Cl\:
\graphline{solid}{graphred}) and the ion electrophoretic mobilities (bottom, \Na\:
\graphline{dashed}{graphblue} and \Cl\: \graphline{dashed}{graphred}) on the bulk \ce{NaCl}
concentration. Lines represent empirical fits to experimental literature data (markers). The number next to
the left and right axes correspond to the values at infinite dilution (\SI{0}{\Molar}) and saturation
(\SI{\approx5}{\Molar}),  respectively.
(b)
Dependency of electrolyte density (top, \graphline{solid}{graphpurple}), viscosity (middle,
\graphline{dashed}{graphpurple}) and relative permittivity (bottom, \graphline{dotted}{graphpurple}) on the
bulk \ce{NaCl} concentration. Lines represent empirical fits to experimental literature data (markers).
(c)
Dependency of the relative ion diffusion coefficient and the mobility (top, \graphline{solid}{graphgreen}) and
the relative viscosity (bottom, \graphline{dashed}{graphgreen}) on the distance from the nanopore wall. The
relative diffusion coefficient (and its mobility) declines sharply when an ion approaches within
\SI{\approx1.0}{\nm} of the protein wall. For water molecules, this phenomenon is modelled as a sharp increase
of the fluid viscosity within \SI{\approx1.0}{\nm} distance from the wall. The values for the diffusion
coefficient were taken directly from ref. \citenum{Simakov-2010}, who used an empirical fit on molecular
dynamics data from ref. \citenum{Makarov-1998}. The inverse relative viscosity data was taken directly from
the molecular dynamics study in ref. \citenum{Pronk-2014}, and fitted with a logistic function after
offsetting for the protein hydrodynamic radius.
}

% label
\label{fig:corrections}

\end{figure*}

The parameters of the fitting functions used to interpolate the experimental ion diffusion coefficients, 
mobilities and transport numbers, and the electrolyte viscosity, density and relative permittivity are given 
in \cref{tab:corrections_parameters} and the resulting curves are plotted in \cref{fig:corrections}. Note 
that since most of these functions are merely empirical fits with no physical meaning, and that they are 
solely used to interpolate and represent the experimental data.

Finally, for the concentration dependence of the relative permittivity we made use of the model proposed by Gavish et al.:
\begin{align}
\relperm(\dconc) = \permittivity_{r,0} - \left(\permittivity_{r,0} - \permittivity_{r,ms}\right) L \left( \dfrac{3\alpha}{\permittivity_{r,0} - \permittivity_{r,ms}} \dconc \right)
\end{align}
\begin{table*}[ht]

\sisetup{inter-unit-product=\ensuremath{{}\cdot{}}}

\renewcommand{\arraystretch}{1.5}
\scriptsize
\caption{Overview of the \ce{NaCl} fitting parameters used for interpolation.}
\centering
\label{tab:corrections_parameters}
\begin{tabular}{@{}
				l
				S[table-format=1.3]
				S[table-format=-1.2(2)e-1]
				S[table-format=-1.2(2)e-1]
				S[table-format=-1.2(2)e-1]
				S[table-format=-1.2(2)e-1]
				S[table-format=>1.2]
				l
				@{}}
	\toprule
							& \multicolumn{5}{c}{Fitting parameters}									&		&	\\
	\cmidrule{2-6}
	Property				& $P_{0}$	& $P_{1}$		& $P_{2}$		& $P_{3}$		& $P_{4}$		& $R^{2}$	& References	\\
	\midrule
	$\diffusion^+(c)$		& 1.334		& 2.02(14)e-1	& -3.05(41)e-1	& 2.19(38)e-1	& -3.13(108)e-2	& >0.99		& \citenum{mills1989}	\\
	$\diffusion^-(c)$		& 2.032		& 1.49(30)e-1	& -4.94(904)e-2	& 3.40(826)e-2	& 1.43(230)e-2	& >0.99		& \citenum{mills1989}	\\
	$\mobility^+(c)$		& 5.192		& 7.91(6)e-1	& -3.53(17)e-1	& 1.46(15)e-1	& 9.23(389)e-3	& >0.99		& \citenum{bianchi1989, currie1960, goldsack1976, dellamonica1979}	\\
	$\mobility^-(c)$		& 7.909		& 6.29(6)e-1	& -4.29(17)e-1	& 2.12(14)e-1	& -1.07(37)e-2	& >0.99		& \citenum{bianchi1989, currie1960, goldsack1976, dellamonica1979}	\\
	$\transportn^+(c)$		& 0.3963	& 9.38(164)e-2 	& 2.86(324)e-3	& -1.88(652)e-2	& 4.51(275)e-3	& 0.98		& \citenum{panopoulos1986,currie1960,smits1966,schonert2014,dellamonica1979}	\\
	$\viscosity(c)$			& 0.8904	& 7.56(27)e-3	& 7.77(4)e-2	& 1.19(1)e-2	& 5.95(35)e-4	& >0.99		& \citenum{hai-lang1996}	\\
	$\density(c)$			& 0.997		& 4.06(1)e-2	& -6.39(16)e-4	& 				& 				& >0.99		& \citenum{hai-lang1996}	\\
	$\permittivity(c)$		& 78.15		& 3.08(0)e1		& 1.15(0)e1		& 				& 				&			& \citenum{gavish2016}	\\
	$\diffusion^+(d)$		&			& 6.2 			& 0.01			& 				& 				&			& \citenum{makarov1998,simakov2010,pederson2015}	\\
	$\diffusion^-(d)$	 	& 			& 6.2			& 0.01			& 				& 				&			& \citenum{makarov1998,simakov2010,pederson2015}	\\
	$\mobility^+(d)$		& 			& 6.2			& 0.01			& 				& 				&			& \citenum{makarov1998,simakov2010,pederson2015}	\\
	$\mobility^-(d)$		& 			& 6.2			& 0.01			& 				& 				&			& \citenum{makarov1998,simakov2010,pederson2015}	\\
	$\viscosity(d)$			& 			& 3.36(23)		& 1.47(23)e-1 	& 				& 				& 0.97		& \citenum{pronk2013}	\\
	\bottomrule
\end{tabular}
\end{table*}


% NaCl transport numbers
% panopoulos1986, currie1960, smits1966, schonert2014, dellamonica1979
% NaCl conductance
% bianchi1989, currie1960, goldsack1976, dellamonica1979

\subsection{Surface integration to compute pore averaged values}
The average pore values for quantity of interest $X$ was computed by
\begin{align}
  \left< X \right>_{\alpha} =
    \displaystyle\frac{\displaystyle\iint_{V_{\alpha}} \beta_{\alpha} X \,dr\,dz}
                      {\displaystyle\iint_{V_{\alpha}} \beta_{\alpha} \,dr\,dz}
\end{align}
where
\begin{equation}
  \alpha=
  \begin{cases}
    \text{p}, & d \ge 0  \text{~nm} \text{, average over the entire pore} \\
    \text{b}, & d > 0.5  \text{~nm} \text{, average over the pore `bulk' }  \\
    \text{s}, & d \le 0.5\text{~nm} \text{, average over the pore `surface' } 
  \end{cases}
\end{equation}
and
\begin{align}
  \beta_{\text{p}} &=
  \begin{cases}
    1, & \text{if}\ -1.85\le z \le 12.25  \text{ and } r \le r_\text{p}(z) \\
    0, & \text{otherwise}
  \end{cases} \\
  \beta_{\text{b}} &=
  \begin{cases}
  1, & \text{if}\ -1.85\le z \le 12.25  \text{ and } r \le r_\text{p}(z) \text{ and } d > 0.5 \\
  0, & \text{otherwise}
  \end{cases} \\
  \beta_{\text{s}} &=
  \begin{cases}
  1, & \text{if}\ -1.85\le z \le 12.25  \text{ and } r \le r_\text{p}(z) \text{ and } d \le 0.5 \\
  0, & \text{otherwise}
  \end{cases}
\end{align}
with $d$ is the distance from the nanopore wall and $r_\text{p}(z)$ is the radius of the pore at height $z$.


\section{Extented results}
\subsection{Voltage dependency of the conductance log-log slopes}
The values of the fitted power-law exponent $\gamma$ for both the simulated (ePNP-NS) and experimental data 
in the low and high concentration regimes can be found in \cref{fig:conductance_loglogslopes}.
\begin{figure*}[!hbt]
  \centering
  \includegraphics[scale=1]{../figures/conductance/conductance_loglogslopes_epnp.pdf}

\caption
[\textbf{Voltage dependency of the fitting parameter $\gamma$}.]
{
\textbf{Voltage dependency of the fitting parameter $\gamma$}
The experimental and simulated (ePNP-NS) ionic conductances were fitted using a power law $G(\cbulk) = a
\cbulk^{\gamma}$ in the low (\SIrange{0.005}{0.15}{\Molar}) and high (\SIrange{0.15}{2.0}{\Molar})
concentration regimes.
}

% label
\label{fig:conductance_loglogslopes}

\end{figure*}


\subsection{Peak values of the radial potential profiles inside ClyA}
The peak values of the radial electrostatic potential $\radpot$ at the \cis\ entry, middle of the lumen and 
the \trans\ constriction for $0.005$, $0.05$, $0.15$, $0.5$ and $5$~M \ce{NaCl} are summarized in 
\cref{tab:radial_potential}.
\begin{table}[!hbt]
  \footnotesize
  \caption[]{Peak radial potential.}\label{tab:radial_potential}
  \centering
  \begin{tabularx}{8cm}{SSSS}
    \toprule
    & \multicolumn{3}{c}{$\radpot$ (mV)} \\
    \cmidrule{2-4}
    & {\cis}      & {lumen}    & {\trans}  \\
    {$\cbulk$ (\si{\Molar})} & {$z\approx\SI{10}{\nm}$} & {$z\approx\SI{5}{\nm}$} & {$z\approx\SI{0}{\nm}$} \\
    \midrule
    0.005          & -80         & -108       & -144   \\
    0.05           & -34         &  -50       &  -86   \\
    0.15           & -19         &  -29       &  -57   \\
    0.5            &  -9.3       &  -14       &  -30   \\
    5              &  -1.9       &   -1.7     &   -4.2 \\
    \bottomrule
  \end{tabularx}
\end{table}


\subsection{Pressure distribution inside ClyA}
The electro-osmotic pressure distribution inside ClyA, as a consequence of the strong local enhancement of 
the ion concentration, is given in \cref{fig:pressure}.
\begin{figure*}[!htb]
  \centering
  \begin{minipage}[t]{10.75cm}
    \begin{subfigure}[t]{5.5cm}
      \centering
      \caption{}\vspace{-3mm}\label{fig:pressure_contour}
      \includegraphics[scale=1]{figures/pressure/pressure_contour_150mM_+000mV}
    \end{subfigure}
    \hspace{-5mm}
    \begin{subfigure}[t]{2.5cm}
      \centering
      \caption{}\vspace{-3mm}\label{fig:pressure_radial_averages}
      \includegraphics[scale=1]{figures/pressure/pressure_radial_averages_+000mV}
    \end{subfigure}
  \end{minipage}
\centering

% caption
\caption
[\textbf{Pressure distribution inside ClyA.}]
{
\textbf{Pressure distribution inside ClyA.}
(\subref{fig:pressure_contour}) Contourmap of the pressure at $\cbulk=\SI{0.15}{\Molar}$ and
$\vbias=\SI{0}{\mV}$, showing that the \Na\ concentration `hotspots' near the pore wall result in build-up of
electro-osmotic pressure (\SIrange{5}{30}{\atm}) inside the  confined fluid. Pressure drops of such magnitude
over the course of a few nanometers could potentially exert a significant force on a captured
protein.\cite{Hoogerheide-2014}
(\subref{fig:pressure_radial_averages}) The axial pressure profile and averaged along the the entire radius of
the pore at $\vbias=\SI{0}{\mV}$.
%At concentrations $\ge0.5$~M, a negative pressure develops in the `bulk' of the lumen due to the
}\label{fig:pressure}

\end{figure*}



%-------------------------------------------------------------------------------
% SUPPORTING INFORMATION
%-------------------------------------------------------------------------------
\bibliography{modeling1}

\end{document}

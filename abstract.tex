Biological nanopores have become valuable tools for the detection and analysis of molecules and the sequencing of nucleic acids at the single-molecule level. However, the physical understanding of the transport of ions and water across small nanopores and the scale the electrophoretic forces inside the nanopore is still superficial. From this perspective, we have built a detailed, 2D-axisymmetroc continuum model of Cytolysin A (ClyA), a biological nanopore utilized for DNA and protein analysis. We used the steady-state Poison-Nernst-Planck (PNP) and Navier-Stokes (NS) equations to simulate the ionic fluxes and water flow through ClyA for a wide range of ionic strengths and bias voltages. To improve the accuracy of our model, we extended the PNP-NS equations to include finite ion-size effects and ion-protein and water-protein interactions. Furthermore, we implemented self-consistent concentration dependencies for the ion diffusion coefficient and mobility on the one hand, and for the solvent viscosity, density and relative permittivity on the other. Our results show that, only when these are corrections enabled, our simplified ClyA model is able to accurately predict the ionic currents over an wide range of ionic strengths. We further report on the influence of both the bias voltage and the bulk ionic strenght on the average ion concentrations inside the pore, the magnitude and direction of the electro-osmotic flow, and the shape of the electrostatic potential landscape inside the nanopore.
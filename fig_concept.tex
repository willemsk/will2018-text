% Extra commands (e.g. for linestyles)

% create the figure
\begin{figure*}[b]

\centering

% figure path
\includegraphics[scale=1]{figures/fig_concept}

% caption
\caption[\textbf{Geometry, charge distribution and boundary conditions of a 2D-axisymmetric of ClyA.}]
{
\textbf{Geometry, charge distribution and boundary conditions of a 2D-axisymmetric of ClyA.}
(a) Cross-sectional (top) and top (bottom) views of an all-atom model of the dodecameric nanopore Cytolysin A (ClyA) embedded in a lipid bilayer (green).
(b) 2D-axisymmetric model of ClyA (grey) embedded in a \SI{2.8}{nm} thick lipid bilayer (green), both represented by solid dielectric blocks with permittivities $\relperm_p$ and $\relperm_m$, respectively.
They are surrounded by a spherical water reservoir with a radius of \SI{250}{nm}.
The water relative permittivity $\relperm_w(c)$, density $\density(c)$ andthe ion diffusion coeffiecient $\diffusion_{i}(c,d)$, ion mobility $\mobility_{i}(c,d)$,  viscosity $\viscosity(c,d)$, are all dependent on the local ion concentration $c$ and some on the distance from the nanopore's walls . The \textit{cis} and \textit{trans} boundaries are set-up to mimic an infite reservoir by fixing the ion concentrations ($\concentration{i} = \concentration{i,\textrm{bulk}}$) and allowing for unrestricted fluid flow across them. A bias potential ($\varpotential=\varpotential_\textrm{bulk}$) is applied at \textit{trans} while the \textit{cis} boundary is kept grounded ($\varpotential=0$).
(c) Zoom-in of the nanopore geometry.
(d) Radially averaged charge density map of ClyA-AS, obtained by summation of all charged atoms onto a 2D plane, where each atom was represented as a 2D-Gaussian with a variance proportional the atom radius and a total integral equal to the net charge (see methods for details).
}

% label
\label{fig:concept}

\end{figure*}

%-------------------------------------------------------------------------------
% PREAMBLE AND DOCUMENT FORMATTING
%-------------------------------------------------------------------------------
\documentclass[journal=ancac3, manuscript=suppinfo, etalmode=truncate,maxauthors=0]{achemso}
\setkeys{acs}{etalmode=truncate,maxauthors=0}

% PACKAGES
\usepackage[utf8]{inputenc}
\usepackage[english]{babel}
\usepackage{csquotes}
\usepackage{amsmath}
\usepackage{amsfonts}
\usepackage{amssymb}
\usepackage{textcomp}
\usepackage{textgreek}

\usepackage[version=4]{mhchem}
\usepackage[alsoload=synchem,separate-uncertainty=true,multi-part-units=single,tight-spacing=true]{siunitx}

\usepackage{float}
\usepackage{graphicx}
\usepackage{xcolor}
\usepackage{tikz}
\usetikzlibrary{shapes}
\usepackage{array, booktabs, tabularx, multirow}

% CAPTION FORMATTING
\usepackage[font=footnotesize,labelfont=bf,labelsep=period]{caption} 							% format single-image captions and table titles
\captionsetup[table]{singlelinecheck=false,font=footnotesize,labelfont=bf}
\usepackage[font=footnotesize,labelfont=bf,labelsep=period]{subcaption} 						% format subfigure captions
%\DeclareCaptionSubType*[alph]{figure}
%\renewcommand\thesubfigure{\thefigure\alph{subfigure}}
\captionsetup[subfigure]{labelfont=bf,textfont=normalfont,labelformat=simple,singlelinecheck=false}

% CROSS-REFERENCE FORMATTING
% For use with the cleveref package
% Define the format of Figure, Table, Equation, and Section cross-references in the text
\usepackage{xr-hyper}
\usepackage{hyperref}
\usepackage{cleveref}
\crefname{figure}{Fig.}{Figs.}
\Crefname{figure}{Figure}{Figures}
\crefname{table}{Tab.}{Tabs.}
\Crefname{table}{Table}{Tables}
\crefname{equation}{Eq.}{Eqs.}
\Crefname{equation}{Equation}{Equations}
\crefname{section}{Sec.}{Secs.}
\Crefname{section}{Section}{Sections}


% REFERENCES
\renewcommand*{\bibfont}{\normalfont\small}


%-------------------------------------------------------------------------------
% CUSTOM COMMANDS
%-------------------------------------------------------------------------------


% Shorthands
\newcommand{\todo}[1]{\textbf{\textcolor{orange}{#1}}}
\newcommand{\ahl}{\textalpha HL}
\newcommand{\etal}{\textit{et al.}}
\newcommand{\cis}{\textit{cis}}
\newcommand{\trans}{\textit{trans}}

% Units
\DeclareSIUnit{\molar}{\mole\per\cubic\deci\metre}
\DeclareSIUnit{\Molar}{\textsc{M}}
\DeclareSIUnit{\atm}{\textsc{atm}}
\newcommand{\mM}{\milli\Molar}
\newcommand{\mV}{\milli\volt}
\newcommand{\um}{\micro\meter}
\newcommand{\nm}{\nano\meter}
\newcommand{\ns}{\nano\second}
\newcommand{\ps}{\pico\second}
\newcommand{\fs}{\femto\second}
\newcommand{\pN}{\pico\newton}
\newcommand{\ec}{\elementarycharge}
\newcommand{\dC}{\degreeCelsius}
\newcommand{\mps}{\meter\per\second}
\newcommand{\cnmpnspv}{\cubic\nano\meter\per\nano\second\per\volt}

% Vectors and math stuff
\renewcommand{\vec}[1]{\boldsymbol{#1}}
\newcommand{\rpos}{\vec{r}} % positional vector
\newcommand{\normvec}{\vec{\hat{n}}} % normal vector
\newcommand{\identity}{\vec{\rm I}} % Identity vector
\newcommand{\stdev}{\sigma}
\newcommand{\pav}[2]{\left< #1 \right>_{\text{#2}}} % Pore average
\newcommand{\vc}[2]{#1_{#2}}
\newcommand{\pd}[2]{\displaystyle\frac{\partial #1}{\partial #2}}
\newcommand{\hydrostresstensor}{\sigma_{ij}}
\def\dsum{\displaystyle\sum}

% Physical constants
\newcommand{\boltzmann}{k_{\rm B}}
\newcommand{\avogadro}{N_{\rm A}}
\newcommand{\temp}{T}
\newcommand{\faraday}{\mathcal{F}}
\newcommand{\echarge}{e}

% Field variables
\newcommand{\vel}{u}                  % Fluid velocity
\newcommand{\force}{F}                % A force
\newcommand{\potential}{\varphi}			% Electrostatic potential
\newcommand{\varpotential}{V}				  % Electrostatic potential
\newcommand{\concentration}{c}				% Ion concentration
\newcommand{\velocity}{\vec{u}}				% Fluid velocity
\newcommand{\pressure}{p}					    % Fluid pressure
\newcommand{\efield}{\vec{E}}         % Electrical field
\newcommand{\displacement}{\vec{D}}   % Electrical displacement field
\newcommand{\walldistance}{d}				  % Wall distance

% Dimensionless variables
\newcommand{\dconc}{\bar{\concentration}}		% Dimensionless concentration
\newcommand{\dwall}{\bar{\walldistance}}		% Dimensionless wall distance

% Material and ion parameters
\newcommand{\permittivity}{\varepsilon}
\newcommand{\absperm}{\permittivity_0}				% Permittivity of vacuum
\newcommand{\relperm}{\permittivity_r}				% Relative permittivity
\newcommand{\dielectric}{\relperm}				    % Relative permittivity
\newcommand{\diffusion}{\mathcal{D}}		     	% Diffusion coefficient
\newcommand{\mobility}{\mu}				         		% Electrophoretic mobility
\newcommand{\transportn}{t}				         		% Transport number
\newcommand{\chargen}{z}				          		% Ion charge number
\newcommand{\ionsize}{a}					           	% Ion size for smPNP
\newcommand{\density}{\varrho}			     			% Fluid density
\newcommand{\viscosity}{\eta}				         	% Fluid viscosity

\newcommand{\iondiffusion}[1]{\diffusion_{#1}}	        % Ion Diffusion coefficient
\newcommand{\ionmobility}[1]{\mobility_{#1}}	          % Ion electrophoretic mobility
\newcommand{\iontransportn}[1]{\transportn{#1}}	        % Ion transport number
\newcommand{\avionconc}{\langle\concentration\rangle}   % Average ion concentration

\newcommand{\tna}{\transportn_{\ce{Na+}}}

\newcommand{\molarconductivity}{\Lambda}
\newcommand{\specmolarconductivity}{\lambda}

% Derived properties
\newcommand{\scd}{\rho}							% Total charge density
\newcommand{\scdpore}{\scd_{\text{pore}}^f}		% Fixed charge density
\newcommand{\scdion}{\scd_{\text{ion}}}			% Ionic charge density
\newcommand{\flux}{\vec{J}} 							% Ion flux
\newcommand{\volumeforce}{\vec{\force}_{\rm ion}}	% Fluid volume force

\newcommand{\current}{I}
\newcommand{\currentsim}{\current_{\rm sim}}
\newcommand{\currentexp}{\current_{\rm exp}}
\newcommand{\conductance}{G}
\newcommand{\icr}{\alpha}

% Shorthands
\newcommand{\ci}{\concentration_{i}}
\newcommand{\cbulk}{\concentration_\text{s}}
\newcommand{\vbias}{\varpotential_\text{b}}
\newcommand{\Na}{\ce{Na+}}
\newcommand{\Cl}{\ce{Cl-}}
\newcommand{\Qion}{Q_{\rm ion}}
\newcommand{\radpot}{\left<\potential\right>_\text{rad}}
\newcommand{\radenergy}{\left< U_{\text{E},i} \right>_{\text{rad}}}
\newcommand{\deltaEt}{\Delta E_{\text{B},i}}
\newcommand{\pH}[1]{pH~\num{#1}}
\newcommand{\kT}{\boltzmann\temp}
\newcommand{\kTe}{\kT / \echarge}

% Atom properties
\newcommand{\partialcharge}{\delta}
\newcommand{\atomradius}{R}


% Colors
\definecolor{graphgreen}  {rgb}{0.30196078, 0.68627451, 0.29019608}
\definecolor{graphpurple} {rgb}{0.59607843, 0.30588235, 0.63921569}
\definecolor{graphblue}   {rgb}{0.29803921, 0.57254902, 0.98823529}
\definecolor{graphred}    {rgb}{0.91372549, 0.15686275, 0.18823529}

% Line and marker styles
\newcommand{\graphline}[2]{\raisebox{2pt}{\tikz{\draw[-,color=#2,#1,line width=1.5pt](0,0) -- (5mm,0);}}}
\newcommand{\graphmarker}[2]{\raisebox{0.5pt}{\tikz{\node[draw,scale=0.5,#1,fill=none, color=#2](){};}}}

%\newcommand{\graphlinemarker}[3]{\raisebox{0pt}{\tikz{\draw[-,#3,#1,line width = 1.0pt](2.mm,0) #2 (3.5mm,1.5mm);\draw[-,#3,#1,line width = 1.0pt](0.,0.8mm) -- (5.5mm,0.8mm)}}}
%\newcommand{\rectangle}{\raisebox{0pt}{\tikz{\draw[-,black,dotted,line width = 1.0pt](0.,0.8mm) -- (5.5mm,0.8mm);\draw[black,solid,line width = 1.0pt](2.mm,0) circle (3.5mm,1.5mm)}}}
\newcommand{\rectangle}[1]{\raisebox{0pt}{\tikz{\draw[-,black,solid,line width = 1.0pt](0.,0.8mm) -- (5.5mm,0.8mm);\draw[black,solid,line width = 1.0pt](2.mm,0) #1 (3.5mm,1.5mm)}}}



%-------------------------------------------------------------------------------
% TITLE AND AUTHOR LIST
%-------------------------------------------------------------------------------
\title{Modeling of Ion and Water Transport in the Biological Nanopore ClyA}
\newcommand{\afimec}{imec, Kapeldreef 75, B-3001 Leuven, Belgium}
\newcommand{\afkulchem}{KU Leuven, Department of Chemistry, Celestijnenlaan 200F, B-3001 Leuven, Belgium}
\newcommand{\afkulphys}{KU Leuven, Department of Physics and Astronomy, Celestijnenlaan 200D, B-3001 Leuven, Belgium}
\newcommand{\afrug}{University of Groningen, Groningen Biomolecular Sciences \& Biotechnology Institute, 9747 AG, Groningen, The Netherlands}

\author{Kherim Willems}
\affiliation{\afkulchem}
\alsoaffiliation{\afimec}

\author{Dino Rui\'{c}}
\affiliation{\afkulphys}
\alsoaffiliation{\afimec}

\author{Florian Lucas}
\affiliation{\afrug}

\author{Ujjal Barman}
\affiliation{\afimec}

%\author{Chang Chen}
%\affiliation{\afimec}

\author{Johan Hofkens}
\affiliation{\afkulchem}

\author{Giovanni Maglia}
\email{g.maglia@rug.nl}
\affiliation{\afrug}

\author{Pol Van Dorpe}
\email{Pol.VanDorpe@imec.be}
\affiliation{\afkulchem}
\alsoaffiliation{\afimec}



%===============================================================================
%
% MAIN DOCUMENT BEGINS
%
%===============================================================================
\begin{document}

\maketitle

\newpage
\section{Extended materials and methods}

\subsection{Fitting of the electrolyte properties}

% create the figure
\begin{figure*}[!b]

\centering

% figure path
\includegraphics[width=\textwidth]{../figures/si/fig_corrections}

% caption
\caption
[\textbf{Concentration and positional dependent electrolyte properties.}]
{
\textbf{Concentration and positional dependent electrolyte properties.}
(a)
Dependency of the ion self-diffusion coefficients (top, \Na\: \graphline{solid}{graphblue} and \Cl\:
\graphline{solid}{graphred}) and the ion electrophoretic mobilities (bottom, \Na\:
\graphline{dashed}{graphblue} and \Cl\: \graphline{dashed}{graphred}) on the bulk \ce{NaCl}
concentration. Lines represent empirical fits to experimental literature data (markers). The number next to
the left and right axes correspond to the values at infinite dilution (\SI{0}{\Molar}) and saturation
(\SI{\approx5}{\Molar}),  respectively.
(b)
Dependency of electrolyte density (top, \graphline{solid}{graphpurple}), viscosity (middle,
\graphline{dashed}{graphpurple}) and relative permittivity (bottom, \graphline{dotted}{graphpurple}) on the
bulk \ce{NaCl} concentration. Lines represent empirical fits to experimental literature data (markers).
(c)
Dependency of the relative ion diffusion coefficient and the mobility (top, \graphline{solid}{graphgreen}) and
the relative viscosity (bottom, \graphline{dashed}{graphgreen}) on the distance from the nanopore wall. The
relative diffusion coefficient (and its mobility) declines sharply when an ion approaches within
\SI{\approx1.0}{\nm} of the protein wall. For water molecules, this phenomenon is modelled as a sharp increase
of the fluid viscosity within \SI{\approx1.0}{\nm} distance from the wall. The values for the diffusion
coefficient were taken directly from ref. \citenum{Simakov-2010}, who used an empirical fit on molecular
dynamics data from ref. \citenum{Makarov-1998}. The inverse relative viscosity data was taken directly from
the molecular dynamics study in ref. \citenum{Pronk-2014}, and fitted with a logistic function after
offsetting for the protein hydrodynamic radius.
}

% label
\label{fig:corrections}

\end{figure*}

The parameters of the fitting functions used to interpolate the experimental ion
diffusion coefficients, mobilities and transport numbers, and the electrolyte
viscosity, density and relative permittivity are given in
\cref{tab:corrections_parameters} and the resulting curves are plotted in
\cref{fig:corrections}. Note that since most of these functions are merely
empirical fits with no physical meaning, and that they are solely used to
interpolate and represent the experimental data.

Finally, for the concentration dependence of the relative permittivity we made
use of the model proposed by Gavish et al.\cite{Gavish-2016}
%
\begin{align}
\relperm(\dconc) = \permittivity_{r,0} - \left(\permittivity_{r,0} - \permittivity_{r,ms}\right) L \left( \dfrac{3\alpha}{\permittivity_{r,0} - \permittivity_{r,ms}} \dconc \right)
\end{align}
%
\begin{table*}[ht]

\sisetup{inter-unit-product=\ensuremath{{}\cdot{}}}

\renewcommand{\arraystretch}{1.5}
\scriptsize
\caption{Overview of the \ce{NaCl} fitting parameters used for interpolation.}
\centering
\label{tab:corrections_parameters}
\begin{tabular}{@{}
				l
				S[table-format=1.3]
				S[table-format=-1.2(2)e-1]
				S[table-format=-1.2(2)e-1]
				S[table-format=-1.2(2)e-1]
				S[table-format=-1.2(2)e-1]
				S[table-format=>1.2]
				l
				@{}}
	\toprule
							& \multicolumn{5}{c}{Fitting parameters}									&		&	\\
	\cmidrule{2-6}
	Property				& $P_{0}$	& $P_{1}$		& $P_{2}$		& $P_{3}$		& $P_{4}$		& $R^{2}$	& References	\\
	\midrule
	$\diffusion^+(c)$		& 1.334		& 2.02(14)e-1	& -3.05(41)e-1	& 2.19(38)e-1	& -3.13(108)e-2	& >0.99		& \citenum{mills1989}	\\
	$\diffusion^-(c)$		& 2.032		& 1.49(30)e-1	& -4.94(904)e-2	& 3.40(826)e-2	& 1.43(230)e-2	& >0.99		& \citenum{mills1989}	\\
	$\mobility^+(c)$		& 5.192		& 7.91(6)e-1	& -3.53(17)e-1	& 1.46(15)e-1	& 9.23(389)e-3	& >0.99		& \citenum{bianchi1989, currie1960, goldsack1976, dellamonica1979}	\\
	$\mobility^-(c)$		& 7.909		& 6.29(6)e-1	& -4.29(17)e-1	& 2.12(14)e-1	& -1.07(37)e-2	& >0.99		& \citenum{bianchi1989, currie1960, goldsack1976, dellamonica1979}	\\
	$\transportn^+(c)$		& 0.3963	& 9.38(164)e-2 	& 2.86(324)e-3	& -1.88(652)e-2	& 4.51(275)e-3	& 0.98		& \citenum{panopoulos1986,currie1960,smits1966,schonert2014,dellamonica1979}	\\
	$\viscosity(c)$			& 0.8904	& 7.56(27)e-3	& 7.77(4)e-2	& 1.19(1)e-2	& 5.95(35)e-4	& >0.99		& \citenum{hai-lang1996}	\\
	$\density(c)$			& 0.997		& 4.06(1)e-2	& -6.39(16)e-4	& 				& 				& >0.99		& \citenum{hai-lang1996}	\\
	$\permittivity(c)$		& 78.15		& 3.08(0)e1		& 1.15(0)e1		& 				& 				&			& \citenum{gavish2016}	\\
	$\diffusion^+(d)$		&			& 6.2 			& 0.01			& 				& 				&			& \citenum{makarov1998,simakov2010,pederson2015}	\\
	$\diffusion^-(d)$	 	& 			& 6.2			& 0.01			& 				& 				&			& \citenum{makarov1998,simakov2010,pederson2015}	\\
	$\mobility^+(d)$		& 			& 6.2			& 0.01			& 				& 				&			& \citenum{makarov1998,simakov2010,pederson2015}	\\
	$\mobility^-(d)$		& 			& 6.2			& 0.01			& 				& 				&			& \citenum{makarov1998,simakov2010,pederson2015}	\\
	$\viscosity(d)$			& 			& 3.36(23)		& 1.47(23)e-1 	& 				& 				& 0.97		& \citenum{pronk2013}	\\
	\bottomrule
\end{tabular}
\end{table*}


% NaCl transport numbers
% panopoulos1986, currie1960, smits1966, schonert2014, dellamonica1979
% NaCl conductance
% bianchi1989, currie1960, goldsack1976, dellamonica1979

\newpage
\subsection{Surface integration to compute pore averaged values}

The average pore values for quantity of interest $X$ was computed by
%
\begin{align}
  \left< X \right>_{\alpha} =
    \displaystyle\frac{\displaystyle\iint_{V_{\alpha}} \beta_{\alpha} X \,dr\,dz}
                      {\displaystyle\iint_{V_{\alpha}} \beta_{\alpha} \,dr\,dz}
  \text{ ,}
\end{align}
%
where
%
\begin{equation}
  \alpha=
  \begin{cases}
    \text{p}, & d \ge 0  \text{~nm} \text{, average over the entire pore} \\
    \text{b}, & d > 0.5  \text{~nm} \text{, average over the pore `bulk' }  \\
    \text{s}, & d \le 0.5\text{~nm} \text{, average over the pore `surface' }
  \end{cases}
\end{equation}
%
and
%
\begin{align}
  \beta_{\text{p}} &=
  \begin{cases}
    1, & \text{if}\ -1.85\le z \le 12.25  \text{ and } r \le r_\text{p}(z) \\
    0, & \text{otherwise}
  \end{cases} \\
  \beta_{\text{b}} &=
  \begin{cases}
  1, & \text{if}\ -1.85\le z \le 12.25  \text{ and } r \le r_\text{p}(z) \text{ and } d > 0.5 \\
  0, & \text{otherwise}
  \end{cases} \\
  \beta_{\text{s}} &=
  \begin{cases}
  1, & \text{if}\ -1.85\le z \le 12.25  \text{ and } r \le r_\text{p}(z) \text{ and } d \le 0.5 \\
  0, & \text{otherwise}
  \end{cases}
\end{align}
%
with $d$ the distance from the nanopore wall and $r_\text{p}(z)$ is the radius of the pore at height $z$.


\newpage
\subsection{Weak forms of the ePNP-NS equations}
To solve partial differential equations with the finite element method, we must
be derive their weak form. This is achieved through multiplication of the
equation with an arbitrary test function and integration over their relevant
domains and boundaries (\cref{fig:domains_an_boundaries}). The full
computational domain of our model ($\Omega$) is subdivided into domains for the
pore ($\Omega_p$), the lipid bilayer ($\Omega_m$) and the electrolyte reservoir
($\Omega_w$). The relevant boundaries of these domains are also indicated, i.e.
the reservoir's exterior edges at the \cis\ $\Gamma_{w,c}$) and \trans\
$\Gamma_{w,t}$) sides, the outer edge of the lipid bilayer ($\Gamma_{m}$) and
the interface of the fluid with the nanopore and the bilayer ($\Gamma_{p+m}$).

The following paragraphs detail the equations that were used in this paper, together with their corresponding
weak forms.

\begin{figure*}[!b]
\centering
\includegraphics[scale=1]{../figures/si/domains_and_boundaries.pdf}
% caption
\caption
[\textbf{Computational domains and boundaries.}]
{
\textbf{Computational domains and boundaries.}
The full computational domain ($\Omega$) of the model is subdivided into various subdomains ($\Omega_x$) and
their limiting boundaries ($\Gamma_x$).
}

% label
\label{fig:domains_an_boundaries}

\end{figure*}


\paragraph{Poisson equation.}
%
The global potential distribution is described by the Poisson equation
(PE)\cite{Lu-2012}
%
\begin{align}
\label{eq:poisson}
\nabla \cdot \displacement = -\left( \scdpore + \scdion \right)
\text{\quad with }
\displacement = \absperm \relperm \nabla \potential
\text{ ,}
\end{align}
%
with $\displacement$ the electrical displacement field, $\potential$ the
electric potential, $\absperm$ the vacuum permittivity
($8.85419\times10^{-12}\text{~F\,m}^{-1}$), and $\relperm$ the local relative
permittivity of the medium. $\scdpore$ and $\scdion$ are the fixed (due to the
pore) and mobile (due to the ions) charge distributions, respectively.

Multiplication of \cref{eq:poisson} with the potential test function $\psi$ and
integration over the entire model $\Omega=\Omega_w+\Omega_p+\Omega_m$ gives
%
\begin{align}
\displaystyle\int_{\Omega}
\left[
  \nabla \cdot \displacement
\right]
\psi \,d\Omega
={}&
- \displaystyle\int_{\Omega} \left[ \scdpore + \scdion \right] \psi \,d\Omega \text{ ,}
\end{align}
%
which, after applying the Gauss divergence theorem, yields the final weak
formulation
%
\begin{align}
\displaystyle\int_{\Omega}
\left[
  \nabla \psi \cdot \displacement
\right]
\,d\Omega
- \displaystyle\int_{\Gamma_\text{PE}}
\left[
  \psi \displacement \cdot \vec{n}
\right]
\,d\Gamma_\text{PE}
={}&
\displaystyle\int_{\Omega} \left[ \psi \scdpore \right] \,d\Omega
+
\displaystyle\int_{\Omega} \left[ \psi \scdion \right] \,d\Omega
\text{ ,}
\end{align}
%
with boundaries $\Gamma_\text{PE}=\Gamma_{w,c}+\Gamma_{w,t}+\Gamma_m$ and
$\vec{n}$ their normal vector. The boundary integrals at $\Gamma_{w,c}$ and
$\Gamma_{w,t}$ are evaluated using the Dirichlet boundary conditions (BCs)
$\potential=0$ and $\potential=\vbias$, respectively. A zero charge BC, $\vec{n}
\cdot \displacement = 0$, is used for the integral at $\Gamma_{m}$.

\paragraph{Size-modified Nernst-Planck equation.}
%
The total ionic flux $\flux_{i}$ of ion $i$ at steady-state is expressed by the
size-modified Nernst-Planck equation (smNPE)\cite{Lu-2012}
%
\begin{equation}
\label{eq:smnp}
\pd{\concentration_{i}}{t} = - \nabla\cdot\flux_{i} = - \nabla \cdot
\left(
  \diffusion_{i}\nabla\concentration_{i}
  + \chargen_{i}\mobility_{i}\concentration_{i}\nabla\potential
  + \vec{\beta_{i}} \concentration_{i}
  - \velocity\concentration_{i}
\right)
\text{\quad with }
\vec{\beta_{i}} =
  \frac{\ionsize_{i}^3/\ionsize_{0}^3 \displaystyle\sum_{j} \ionsize_{j}^3 \nabla \concentration_{j}}
      {1 - \displaystyle\sum_{j} \avogadro \ionsize_{j}^3 \concentration_{j}}
\text{ ,}
\end{equation}
%
with ion diffusion coefficient $\diffusion_{i}$, concentration
$\concentration_{i}$, charge number $\chargen_{i}$, mobility $\mobility_{i}$,
electrostatic potential $\potential$, steric saturation factor $\beta_{i}$ and
fluid velocity $\velocity$. $\avogadro$ is Avogadro's constant
($6.022\times10^{23}\text{~mol}^{-1}$) and $\ionsize_{i}$ and $\ionsize_{0}$ are
the limiting cubic diameters for ions and water, respectively. Using the ion
concentration test function $d_i$, the weak form of \cref{eq:smnp} becomes
%
\begin{align}
% Raw integral
\displaystyle\int_{\Omega_w} \left[ \pd{\concentration_{i}}{t} \right] d_{i} \,d\Omega_w ={}&
\displaystyle\int_{\Omega_w} \left[ - \nabla \cdot \flux_{i} \right] d_{i} \,d\Omega_w \notag \\
% Gauss divergence theorem
\displaystyle\int_{\Omega_w} \left[ d_{i} \pd{\concentration_{i}}{t} \right] \,d\Omega_w ={}&
\displaystyle\int_{\Omega_w} \left[\nabla d_{i} \cdot \flux_{i} \right]\,d\Omega_w
- \displaystyle\int_{\Gamma_\text{NP}}
\left[ d_{i} \flux_{i} \cdot \vec{n} \right]\,d\Gamma_\text{NP} \notag \\
% Fill in fluc components
={}&
% Domain
\displaystyle\int_{\Omega_w}
\left[
  \nabla d_{i} \cdot
  \left(
    \diffusion_{i}\nabla\concentration_{i}
    + \chargen_{i}\mobility_{i}\concentration_{i}\nabla\potential
    + \vec{\beta_{i}} \concentration_{i}
    - \velocity \concentration_{i}
  \right)
\right]
\,d\Omega_w \notag \\
% Boundary
& - \displaystyle\int_{\Gamma_\text{NP}}
\left[
  d_{i}
  \left(
    \diffusion_{i} \nabla \concentration_{i}
    + \chargen_{i} \mobility_{i} \concentration_{i} \nabla \potential
    + \vec{\beta_{i}} \concentration_{i}
    - \velocity \concentration_{i}
  \right)
  \cdot \vec{n}
\right]
\,d\Gamma_\text{NP}
\text{ .}
\end{align}
%
The integrals on the boundaries $\Gamma_\text{NP} =
\Gamma_{w,c}+\Gamma_{w,t}+\Gamma_{p+m}$ are evaluated using the Dirichlet
boundary condition $\concentration_{i} = \cbulk$ for $\Gamma_{w,c}$ and
$\Gamma_{w,t}$, and the no flux boundary condition $\vec{n} \cdot \flux_{i} = 0$
for $\Gamma_{p+m}$

\paragraph{Variable density and viscosity Navier-Stokes equation.}
%
The steady-state, laminar fluid flow of an incompressible fluid with a variable
density and viscosity is given by the system of equations\cite{Axelsson-2015}
%
\begin{align}
\label{eq:ns_density_continuity}
\velocity \cdot \nabla \density ={}& 0 \\
\label{eq:ns_conservation}
\left( \velocity \cdot \nabla \right) \left( \density\velocity \right)
+ \nabla \cdot \hydrostresstensor ={}& \vec{\force} \text{\quad with }
\hydrostresstensor =
  \pressure\identity - \viscosity\left[\nabla\velocity+\left(\nabla\velocity \right)^\mathsf{T}\right]
\\
\label{eq:ns_velocity_continuity}
\nabla \cdot \left( \density\velocity \right) - \velocity \cdot \nabla \density ={}& 0
\text { ,}
\end{align}
%
with fluid velocity $\velocity$, density $\density$, hydrodynamic stress tensor
$\hydrostresstensor$, viscosity $\viscosity$, pressure $\pressure$ and body
force $\vec{\force}$. The pressure test function $q$ is used to derive the weak
forms of \cref{eq:ns_density_continuity}
%
\begin{align}
% Density continuity
\displaystyle\int_{\Omega_w} \left[ \velocity \cdot \nabla \density \right] q \,d\Omega_w ={}&
\displaystyle\int_{\Omega_w} \left[ q \velocity \cdot \nabla \density \right] \,d\Omega_w = 0 \text{ ,}
\end{align}
and \cref{eq:ns_velocity_continuity}
\begin{align}
% Velocity continuity
\displaystyle\int_{\Omega_w}
\left[\nabla \cdot \left( \density \velocity \right) - \velocity \cdot \nabla \density \right] q
\,d\Omega_w ={}&
\displaystyle\int_{\Omega_w}
\left[\nabla \cdot \left( \density \velocity \right) \right] q \,d\Omega_w
-
\displaystyle\int_{\Omega_w}
\left[ q \velocity \cdot \nabla \density \right] \,d\Omega_w \\
={}&
\displaystyle\int_{\Omega_w}
\left[\nabla q \cdot \left( \density \velocity \right) \right] \,d\Omega_w
-
\displaystyle\int_{\Gamma_\text{NS}}
\left[ q \left( \density \velocity \right) \cdot \vec{n} \right] \,d\Gamma_\text{NS}
-
\displaystyle\int_{\Omega_w}
\left[ q \velocity \cdot \nabla \density \right] \,d\Omega_w \notag
\text{ ,}
\end{align}
%
while for \cref{eq:ns_conservation} we use the velocity test function
$\vec{v}=\left[v_r, v_\phi, v_z\right]$
%
\begin{align}
\displaystyle\int_{\Omega_w}
\left[
  \left( \velocity \cdot \nabla \right) \left( \density\velocity \right) + \nabla \cdot \hydrostresstensor
\right]
\cdot \vec{v} \,d\Omega_w
={}&
\displaystyle\int_{\Omega_w} \vec{\force} \cdot \vec{v} \,d\Omega_w \\
% Conservation
\displaystyle\int_{\Omega_w}
\left[
  \left( \velocity \cdot \nabla \right) \left( \density\velocity \right) \cdot\vec{v}
\right]
\,d\Omega_w
-
\displaystyle\int_{\Omega_w}
\left[
\hydrostresstensor \cdot \nabla\vec{v}
\right]
\,d\Omega_w
+
\displaystyle\int_{\Gamma_\text{NS}}
\left[
\vec{v} \cdot
\hydrostresstensor
\cdot\vec{n}
\right]
\,d\Gamma_\text{NS}
={}& \displaystyle\int_{\Omega_w} \vec{\force} \cdot \vec{v} \,d\Omega_w \notag
\text{ ,}
\end{align}
%
with boundaries $\Gamma_\text{NS} = \Gamma_{w,c}+\Gamma_{w,t}+\Gamma_{p+m}$. The
no-slip Dirichlet BC $\velocity = 0$ is applied to $\Gamma_{p+m}$, and the no
normal stress $\hydrostresstensor\vec{n}=0$ is used for $\Gamma_{w,c}$ and
$\Gamma_{w,t}$.

\newpage
\section{Extended results}

\subsection{Peak values of the radial potential profiles inside ClyA}

The peak values of the radial electrostatic potential $\radpot$ at the \cis\
entry, middle of the lumen and the \trans\ constriction for $0.005$, $0.05$,
$0.15$, $0.5$ and $5$~M \ce{NaCl} are summarized in \cref{tab:radial_potential}.
\begin{table}[!hbt]
  \footnotesize
  \caption[]{Peak radial potential.}\label{tab:radial_potential}
  \centering
  \begin{tabularx}{8cm}{SSSS}
    \toprule
    & \multicolumn{3}{c}{$\radpot$ (mV)} \\
    \cmidrule{2-4}
    & {\cis}      & {lumen}    & {\trans}  \\
    {$\cbulk$ (\si{\Molar})} & {$z\approx\SI{10}{\nm}$} & {$z\approx\SI{5}{\nm}$} & {$z\approx\SI{0}{\nm}$} \\
    \midrule
    0.005          & -80         & -108       & -144   \\
    0.05           & -34         &  -50       &  -86   \\
    0.15           & -19         &  -29       &  -57   \\
    0.5            &  -9.3       &  -14       &  -30   \\
    5              &  -1.9       &   -1.7     &   -4.2 \\
    \bottomrule
  \end{tabularx}
\end{table}


\subsection{Pressure distribution inside ClyA}
The electro-osmotic pressure distribution inside ClyA, as a consequence of the
strong local enhancement of the ion concentration, is given in
\cref{fig:pressure}.
\begin{figure*}[!htb]
  \centering
  \begin{minipage}[t]{10.75cm}
    \begin{subfigure}[t]{5.5cm}
      \centering
      \caption{}\vspace{-3mm}\label{fig:pressure_contour}
      \includegraphics[scale=1]{figures/pressure/pressure_contour_150mM_+000mV}
    \end{subfigure}
    \hspace{-5mm}
    \begin{subfigure}[t]{2.5cm}
      \centering
      \caption{}\vspace{-3mm}\label{fig:pressure_radial_averages}
      \includegraphics[scale=1]{figures/pressure/pressure_radial_averages_+000mV}
    \end{subfigure}
  \end{minipage}
\centering

% caption
\caption
[\textbf{Pressure distribution inside ClyA.}]
{
\textbf{Pressure distribution inside ClyA.}
(\subref{fig:pressure_contour}) Contourmap of the pressure at $\cbulk=\SI{0.15}{\Molar}$ and
$\vbias=\SI{0}{\mV}$, showing that the \Na\ concentration `hotspots' near the pore wall result in build-up of
electro-osmotic pressure (\SIrange{5}{30}{\atm}) inside the  confined fluid. Pressure drops of such magnitude
over the course of a few nanometers could potentially exert a significant force on a captured
protein.\cite{Hoogerheide-2014}
(\subref{fig:pressure_radial_averages}) The axial pressure profile and averaged along the the entire radius of
the pore at $\vbias=\SI{0}{\mV}$.
%At concentrations $\ge0.5$~M, a negative pressure develops in the `bulk' of the lumen due to the
}\label{fig:pressure}

\end{figure*}



%-------------------------------------------------------------------------------
% SUPPORTING INFORMATION
%-------------------------------------------------------------------------------
\bibliography{../shared/bibliography}

\end{document}

% Extra commands (e.g. for linestyles)

% create the figure
\begin{figure}[b]

\centering

% figure path
\includegraphics[scale=1]{figures/fig_model_concept}

% caption
\caption[\textbf{Geometry, charge distribution and boundary conditions of a 2D-axisymmetric of ClyA.}]
{
\textbf{Geometry, charge distribution and boundary conditions of a 2D-axisymmetric of ClyA.}
(a) 2D-axisymmetric model of ClyA (grey) embedded in a \SI{2.8}{nm} thick lipid bilayer (green), both represented by solid dielectric blocks with permittivities $\relperm_p$ and $\relperm_m$, respectively.
They are surrounded by a spherical water reservoir with a radius of \SI{250}{\nano\meter} (blue).
The \textit{cis} and \textit{trans} boundaries are set-up to mimic an infinite reservoir by fixing their ion concentrations ($\concentration_{i} = \concentration_{i,\textrm{bulk}}$) and
by allowing for unrestricted fluid flow across them ($\left[-\pressure\identity + \viscosity\left(\nabla\velocity + \left(\nabla\velocity\right)^{\rm T}\right)\right]\normvec=\sigma_f\normvec=0$).
A bias potential ($\varpotential=\varpotential_\textrm{bulk}$) is applied at \textit{trans} while the \textit{cis} boundary is kept grounded ($\varpotential=0$).
(b) Zoom-in clearly showing the nanopore geometry and boundary conditions.
The reservoir's relative permittivity $\relperm_w$ and density $\density$ are only dependent on the local average ion concentration $\avionconc$,
while its viscosity $\viscosity$, the ion diffusion coeffiecients $\diffusion_{i}$ and ion mobilities $\mobility_{i}$ are a function of both $\avionconc$ and the distance from the nanopore walls $\walldistance$.

(c) 
(d) Radially averaged charge density map of ClyA-AS, obtained by summation of all charged atoms onto a 2D plane, where each atom was represented as a 2D-Gaussian with a variance proportional the atom radius and a total integral equal to the net charge (see methods for details).
}

% label
\label{fig:model_concept}

\end{figure}

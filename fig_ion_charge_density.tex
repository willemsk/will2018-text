\begin{figure}[!htb]
\centering
\begin{minipage}[t]{8.2cm}
\begin{subfigure}[t]{8.2cm}
	\centering
	\caption{}\vspace{-3mm}\label{fig:ion_charge_density_contours}
	\includegraphics[scale=1]{figures/charge_density/ion_charge_density_contours}
\end{subfigure}
\begin{subfigure}[t]{8.2cm}
  \centering
  \caption{}\vspace{-3mm}\label{fig:ion_charge_density_radial_profiles}
  \includegraphics[scale=1]{figures/charge_density/ion_charge_density_radial_profiles}
\end{subfigure}
\begin{subfigure}[t]{8.2cm}
	\centering
	\caption{}\vspace{-3mm}\label{fig:ion_charge_pore_bulk_surface_total_vs_concentration}
	\includegraphics[scale=1]{figures/charge_density/ion_charge_pore_bulk_surface_total_vs_concentration}
\end{subfigure}
\end{minipage}

% caption
\caption
[\textbf{Ion space charge density distribution inside ClyA.}]
{
\textbf{Ion space charge density distribution inside ClyA.}
(\subref{fig:ion_charge_density_contours})
Cross-section contour plots of the ion space charge density ($\scdion$), expressed as number of elementary 
charges per \si{\cubic\nano\meter}, for $0.005$, $0.05$, $0.5$ and $5$~M at $0$~mV applied bias voltage, 
showing the loss of electrical double layer (EDL) overlap with increasing ionic strength.
(\subref{fig:ion_charge_density_radial_profiles})
Radial profiles of the space charge density at the center of the constriction ($z=-0.3$~nm) and the lumen 
($z=5$~nm) for the concentrations given in (\subref{fig:ion_charge_density_contours}). The vertical line is 
draw at $0.5$~nm distance from the nanopore wall and represents the separation between ions in the center of 
`bulk' of the pore and those at the surface.
(\subref{fig:ion_charge_pore_bulk_surface_total_vs_concentration})
Plot of the total amount of ionic charge ($Q_{\text{ion}}$) present inside the pore in function of the salt 
concentration. At ionic strengths $>0.1$~M, the amount of surface and `bulk' charges rapidly diverge, with 
the latter increasing and the former diminishing to $0$ at $5$~M.
}

\label{fig:ion_charge_density}

\end{figure}
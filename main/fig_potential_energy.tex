\begin{figure}[!ht]
  \centering
  \begin{subfigure}[t]{8.25cm}
    \centering
    \caption{}\vspace{-5mm}\label{fig:potential_energy_radial_averages}
    \includegraphics[scale=1]{../figures/potential_energy/potential_energy_radial_averages}
  \end{subfigure}
  \begin{subfigure}[t]{8.25cm}
    \centering
    \caption{}\vspace{-3mm}\label{fig:potential_energy_trans_barrier}
    \includegraphics[scale=1]{../figures/potential_energy/potential_energy_trans_barrier}
  \end{subfigure}

\caption
[\textbf{Radially averaged electrostatic energy for single ions.}]
{
\textbf{Radially averaged electrostatic energy for single ions.}
(\subref{fig:potential_energy_radial_averages}) Approximate electrostatic energy landscape for single ions
$\radenergy = \chargen_{i} \echarge \radpot$ as calculated directly from the radial electrostatic potential at
\SIlist{+150;-150}{\mV} applied bias voltages for monovalent cations and anions. The grey arrows indicate the
direction in which the ions must travel in order  to contribute positively to the ionic current.
(\subref{fig:potential_energy_trans_barrier}) Height of the electrostatic energy barrier ($\Delta
E_{\text{B},i}$) at the \trans\ constriction. Note that  $\Delta E_{\text{B},i}$ is much higher for negative
voltages and rises logarithmically at lower  concentrations. The divergence between \SI{+0}{\mV} and
\SI{-0}{\mV} for $\cbulk<\SI{0.3}{\Molar}$ highlights the difference in barrier height when traversing the
pore from \cis\ to \trans\ or vice versa.
}\label{fig:potential_energy}

\end{figure}

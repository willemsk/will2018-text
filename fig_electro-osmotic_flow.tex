% create the figure
\begin{figure*}[htbp]

\centering

% figure path
\includegraphics[scale=1]{figures/fig_electro-osmotic_flow}

% caption
\caption
[\textbf{Concentration and voltage dependency of the electro-osmotic flow inside ClyA.}]
{
(a) Representative contour plots of the electro-osmotic flow velocity under low (\SI{50}{\milli\Molar}) and
high (\SI{500}{\milli\Molar}) salt concentrations at negative (\SI{-150}{\milli\volt}) and positive (\SI{+150}{\milli\volt}) bias potentials.
The arrows indicate the direction of the fluid while the lines show the shape of the velocity field.
(b) Velocity profiles at the centre of the pore for bulk concentrations of \SI{50}{\milli\Molar}
(\SI{+150}{\milli\volt}: \graphline{solid}{graphgreen} and\SI{-150}{\milli\volt}: \graphline{dotted}{graphgreen}) 
and \SI{500}{\milli\Molar} (\SI{+150}{\milli\volt}: \graphline{solid}{graphpurple} and 
\SI{-150}{\milli\volt}: \graphline{dotted}{graphpurple}) at negative and positive bias potentials.
Negative values indicate flow from \textit{cis} to \textit{trans} (i.e. top to bottom) and vice versa.
The velocity from the \textit{cis} side rises gradually to a plateau value in the lumen.
Close to the trans constriction, the fluid velocity increases rapidly to a maximum value,
after which it falls again at the same rate upon exit from the pore.
(c) Total electro-osmotic flow rate ($Q_\textrm{eo}$, \SI{}{\cubic\nano\metre\per\nano\second}) vs. bulk salt concentration for positive (top) and negative (bottom) bias potentials.
In the low concentration regime, $Q_\textrm{eo}$ increases rapidly with concentration to a maximum at approximately \SI{500}{\milli\Molar}, followed by by a gradual decline.
(d) The rectification of the electro-osmotic flow rate ($R^+=Q_\textrm{eo}^+/Q_\textrm{eo}^-$) plotted against the concentration.
$R^+$ shows a maximum at \SI{50}{\milli\Molar}, after which it falls to reach unity at approximately \SI{500}{\milli\Molar}, regardless of the applied bias.
A minimum is then reached at \SI{1000}{\milli\Molar}, followed by again by a gradual approach to unity at higher concentrations.
}

% label
\label{fig:electro-osmotic_flow}

\end{figure*}
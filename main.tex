%-------------------------------------------------------------------------------
% PREAMBLE AND DOCUMENT FORMATTING
%-------------------------------------------------------------------------------
\documentclass[journal=ancac3, manuscript=article, etalmode=truncate,maxauthors=0]{achemso}
	\setkeys{acs}{etalmode=truncate,maxauthors=0}

% PACKAGES
\usepackage[utf8]{inputenc}
\usepackage[english]{babel}
\usepackage{csquotes}
\usepackage{amsmath}
\usepackage{amsfonts}
\usepackage{amssymb}
\usepackage{textcomp}
\usepackage{textgreek}

\usepackage[version=4]{mhchem}
\usepackage[alsoload=synchem]{siunitx}
	\sisetup{separate-uncertainty=true}
	\sisetup{multi-part-units=single}
	\sisetup{tight-spacing=true}
	\sisetup{inter-unit-product=\ensuremath{{}\cdot{}}}
	\sisetup{list-units=single}
	\sisetup{range-units=single}

\usepackage{float}
\usepackage{graphicx}
\usepackage{xcolor}
\usepackage{tikz}
	\usetikzlibrary{shapes,arrows.meta}
\usepackage{pgf}
\usepackage{pgfplots}
\pgfplotsset{compat=1.15}
\usepackage{array, booktabs, tabularx, multirow}
\usepackage{pdfpages}


% CAPTION FORMATTING
\usepackage[font=footnotesize,labelfont=bf,labelsep=period]{caption} 							% format single-image captions and table titles
	\captionsetup[table]{singlelinecheck=false,font=footnotesize,labelfont=bf}
\usepackage[font=footnotesize,labelfont=bf,labelsep=period]{subcaption} 						% format 
%subfigure captions
	%\DeclareCaptionSubType*[alph]{figure}
	%\renewcommand\thesubfigure{\thefigure\alph{subfigure}}
	\captionsetup[subfigure]{labelfont=bf,textfont=normalfont,labelformat=simple,singlelinecheck=false} 	

% CROSS-REFERENCE FORMATTING
% For use with the cleveref package
% Define the format of Figure, Table, Equation, and Section cross-references in the text
\usepackage{xr-hyper}
\usepackage{hyperref}
\usepackage{cleveref}
\creflabelformat{equation}{#2#1#3}
\crefname{figure}{Fig.}{Figs.}
\Crefname{figure}{Figure}{Figures}
\crefname{table}{Tab.}{Tabs.}
\Crefname{table}{Table}{Tables}
\crefname{equation}{Eq.}{Eqs.}
\Crefname{equation}{Equation}{Equations}
\crefname{section}{Sec.}{Secs.}
\Crefname{section}{Section}{Sections}


% TITLE AND ABSTRACT
%\let\oldmaketitle\maketitle
%\let\maketitle\relax

% REFERENCES
\renewcommand*{\bibfont}{\normalfont\small}

% SUPPORTING INFO
\usepackage{xr}
\externaldocument[suppinfo:]{supporting_information}

%-------------------------------------------------------------------------------
% CUSTOM COMMANDS
%-------------------------------------------------------------------------------

% Shorthands
\newcommand{\todo}[1]{\textbf{\textcolor{orange}{#1}}}
\newcommand{\ahl}{\textalpha HL}
\newcommand{\etal}{\textit{et al.}}
\newcommand{\cis}{\textit{cis}}
\newcommand{\trans}{\textit{trans}}

% Referencing
\newcommand{\reffig}[1]{Figure~\ref{#1}}
\newcommand{\refsubfig}[2]{\reffig{#1}#2}
\newcommand{\reftable}[1]{Table~\ref{#1}}
\newcommand{\refeq}[1]{Eq.~\ref{#1}}

% Units
\DeclareSIUnit{\molar}{\mole\per\cubic\deci\metre}
\DeclareSIUnit{\Molar}{\textsc{M}}
\newcommand{\mM}{\milli\Molar}
\newcommand{\mV}{\milli\volt}

% Vectors and math stuff
\renewcommand{\vec}[1]{\boldsymbol{#1}}
\newcommand{\rpos}{\vec{r}} % positional vector
\newcommand{\normvec}{\vec{\hat{n}}} % normal vector
\newcommand{\identity}{\vec{\rm I}} % Identity vector
\newcommand{\stdev}{\sigma}

% Physical constants
\newcommand{\boltzmann}{k_{\rm B}}
\newcommand{\avogadro}{N_{\rm A}}
\newcommand{\temp}{T}
\newcommand{\faraday}{\mathcal{F}}
\newcommand{\echarge}{e}

% Field variables
\newcommand{\potential}{\varphi}			% Electrostatic potential
\newcommand{\varpotential}{V}				% Electrostatic potential
\newcommand{\concentration}{c}				% Ion concentration
\newcommand{\velocity}{\vec{u}}				% Fluid velocity
\newcommand{\pressure}{p}					% Fluid pressure
\newcommand{\efield}{\vec{E}}				% Electrical field
\newcommand{\walldistance}{d}				% Wall distance

% Dimensionless variables
\newcommand{\dconc}{\bar{\concentration}}		% Dimensionless concentration
\newcommand{\dwall}{\bar{\walldistance}}		% Dimensionless wall distance

% Material and ion parameters
\newcommand{\permittivity}{\varepsilon}
\newcommand{\absperm}{\permittivity_0}				% Permittivity of vacuum
\newcommand{\relperm}{\varepsilon_r}				% Relative permittivity
\newcommand{\dielectric}{\relperm}				% Relative permittivity
\newcommand{\diffusion}{\mathcal{D}}			% Diffusion coefficient
\newcommand{\mobility}{\mu}						% Electrophoretic mobility
\newcommand{\transportn}{t}						% Transport number
\newcommand{\chargen}{z}						% Ion charge number
\newcommand{\ionsize}{a}						% Ion size for smPNP
\newcommand{\density}{\varrho}						% Fluid density
\newcommand{\viscosity}{\eta}					% Fluid viscosity

\newcommand{\iondiffusion}[1]{\diffusion_{#1}}	% Ion Diffusion coefficient
\newcommand{\ionmobility}[1]{\mobility_{#1}}	% Ion electrophoretic mobility
\newcommand{\iontransportn}[1]{\transportn{#1}}	% Ion transport number
\newcommand{\avionconc}{\langle\concentration_{i}\rangle}

\newcommand{\tna}{\transportn_{\ce{Na+}}}

\newcommand{\molarconductivity}{\Lambda}
\newcommand{\specmolarconductivity}{\lambda}

% Derived properties
\newcommand{\scd}{\rho}							% Total charge density
\newcommand{\scdpore}{\scd_{\rm pore}^f}		% Fixed charge density
\newcommand{\scdion}{\scd_{\rm ion}}			% Ionic charge density
\newcommand{\flux}{J} 							% Ion flux
\newcommand{\volumeforce}{\vec{F}_{\rm ion}}	% Fluid volume force

\newcommand{\current}{I}
\newcommand{\currentsim}{\current_{\rm sim}}
\newcommand{\currentexp}{\current_{\rm exp}}
\newcommand{\conductance}{G}
\newcommand{\icr}{\alpha}

% Shorthands
\newcommand{\ci}{\concentration_{i}}
\newcommand{\cbulk}{\concentration_\text{s}}
\newcommand{\vbias}{\varpotential_\text{b}}
\newcommand{\Na}{\ce{Na+}}
\newcommand{\Cl}{\ce{Cl-}}
\newcommand{\Qion}{Q_{\rm ion}}
\newcommand{\radpot}{\left<\potential\right>_\text{rad}}

% Atom properties
\newcommand{\partialcharge}{\delta}
\newcommand{\atomradius}{R}

% Chemistry 
\newcommand{\pH}[1]{pH~\num{#1}}


% Colors
\definecolor{graphgreen}  {rgb}{0.30196078, 0.68627451, 0.29019608}
\definecolor{graphpurple} {rgb}{0.59607843, 0.30588235, 0.63921569}
\definecolor{graphblue}   {rgb}{0.29803921, 0.57254902, 0.98823529}
\definecolor{graphred}    {rgb}{0.91372549, 0.15686275, 0.18823529}

% Line and marker styles
\newcommand{\graphline}[2]{\raisebox{2pt}{\tikz{\draw[-,color=#2,#1,line width=1.5pt](0,0) -- (5mm,0);}}}
\newcommand{\graphmarker}[2]{\raisebox{0.5pt}{\tikz{\node[draw,scale=0.5,#1,fill=none, color=#2](){};}}}

%\newcommand{\graphlinemarker}[3]{\raisebox{0pt}{\tikz{\draw[-,#3,#1,line width = 1.0pt](2.mm,0) #2 (3.5mm,1.5mm);\draw[-,#3,#1,line width = 1.0pt](0.,0.8mm) -- (5.5mm,0.8mm)}}}
%\newcommand{\rectangle}{\raisebox{0pt}{\tikz{\draw[-,black,dotted,line width = 1.0pt](0.,0.8mm) -- (5.5mm,0.8mm);\draw[black,solid,line width = 1.0pt](2.mm,0) circle (3.5mm,1.5mm)}}}
\newcommand{\rectangle}[1]{\raisebox{0pt}{\tikz{\draw[-,black,solid,line width = 1.0pt](0.,0.8mm) -- (5.5mm,0.8mm);\draw[black,solid,line width = 1.0pt](2.mm,0) #1 (3.5mm,1.5mm)}}}


%-------------------------------------------------------------------------------
% TITLE AND AUTHOR LIST
%-------------------------------------------------------------------------------
\title{Modelling of Ion and Water Transport in the Biological Nanopore ClyA}
\newcommand{\afimec}{imec, Kapeldreef 75, B-3001 Leuven, Belgium}
\newcommand{\afkulchem}{KU Leuven, Department of Chemistry, Celestijnenlaan 200F, B-3001 Leuven, Belgium}
\newcommand{\afkulphys}{KU Leuven, Department of Physics and Astronomy, Celestijnenlaan 200D, B-3001 Leuven, Belgium}
\newcommand{\afrug}{University of Groningen, Groningen Biomolecular Sciences \& Biotechnology Institute, 9747 AG, Groningen, The Netherlands}

\author{Kherim Willems}
\affiliation{\afkulchem}
\alsoaffiliation{\afimec}

\author{Dino Rui\'{c}}
\affiliation{\afkulphys}
\alsoaffiliation{\afimec}

\author{Florian Lucas}
\affiliation{\afrug}

\author{Ujjal Barman}
\affiliation{\afimec}

%\author{Chang Chen}
%\affiliation{\afimec}

\author{Johan Hofkens}
\affiliation{\afkulchem}

\author{Giovanni Maglia}
\email{g.maglia@rug.nl}
\affiliation{\afrug}

\author{Pol Van Dorpe}
\email{Pol.VanDorpe@imec.be}
\affiliation{\afkulchem}
\alsoaffiliation{\afimec}



%===============================================================================
%
% MAIN DOCUMENT BEGINS
%
%===============================================================================
\begin{document}

% Title and authors
\maketitle
% Abstract
\newpage
\begin{abstract}
Biological nanopores have become valuable tools for the detection and analysis of molecules and the 
sequencing of nucleic acids at the single-molecule level. However, the physical understanding of the 
transport of ions and water across small nanopores and the scale the electrophoretic forces inside the 
nanopore is still superficial. From this perspective, we have built a detailed, 2D-axisymmetric continuum 
model of Cytolysin A (ClyA), a biological nanopore utilized for DNA and protein analysis. We used the 
steady-state Poison-Nernst-Planck (PNP) and Navier-Stokes (NS) equations to simulate the ionic fluxes and 
water flow through ClyA for a wide range of ionic strengths and bias voltages. To improve the accuracy and 
validity of our model at the nanoscale, we adapted the classical equations into a framework coined 
extended-PNP-NS (ePNP-NS). Our modifications include finite ion-size effects, ion-protein and water-protein 
interactions, self-consistent concentration dependencies for the ion diffusion coefficients, ion 
electrophoretic mobilities, solvent viscosity, solvent density and solvent relative permittivity. Our results 
show that, only when these are corrections enabled, our simplified ClyA model is able to accurately predict 
the ionic currents over a wide range of ionic strengths. We further report on the influence of both the bias 
voltage and the bulk ionic strength on the average ion concentrations inside the pore, the magnitude and 
direction of the electro-osmotic flow, and the shape of the electrostatic potential landscape inside the 
nanopore.
\end{abstract}
\newpage
% Introduction
\section{Introduction}
Biosensors have been an intense field of research in the recent decades.\cite{zhan2015}
Several breatkthroughs have enabled them to play an integral role in the life sciences with
applications ranging from the monitoring of blood glucose levels\cite{chen2013} to ??. 

A particular challenge for state-of-the-art biosensors is that they need to reach molecular
sensitivities.  Nanopore based sensors solve this problem elegantly by funneling a solution
containing molecules of interest through a pore in a membrane. If the pore is approximately the
same size as the molecules, there can only ever be a single molecule inside the pore and therefore the 
response of the ion current due to the blockage of the pore gives detailed information about the molecule. 
(see Fig.~\ref{} for an illustration).

However, as the size of molecules is usually on the order of nanometer, it is not yet possible to reproducibly
build solid state nanopores\cite{} since both shape and size affect the resulting signal ion currents through
the nanopore significantly. An alternative design which has proven to be very accurate and which yields
reproducible results is to use self-assembled protein nanopores suspended on lipid bilayers\cite{deam2016}.
In fact their properties as biosensors are so desirable that they have already been commercialized for DNA
sequencing applications\cite{}. 

Despite the successes of these biological nanopores, the modeling of the sensor operation is still
in its infancy due to the multi-scale nature of the problem. On the one hand, it is necessary to
resolve the charge distribution of the protein on the atomic level requiring molecular dynamics
(MD).\cite{}  On the other hand, the size of the solvent reservoirs and thus the number of
particles is too large for a comprehensive MD simulation. More suited to the problem of the
flow of the solvent as well as the electromigration and diffusion of the ions in the solvent is a
continuum approach based on the Navier-Stokes equation (NSE) and the Poisson-Nernst-Planck equations
(PNPE).\cite{}

Thus the simulation of the nanopore sits uncomfortably at the intersection of the macroscopic and the
microscopic. In this work, we want to show that it is still possible -- with remarkable accuracy --
to simulate the operation of a biological nanopore using a continuum approach based on the NSE and
PNPE. However, we will also see the limits of this approach which hint at experimentally unexplored states of
solutions within nanopores which are unlike anything that can be found in bulk solutions.

Throughout this work, we will focus on ClyA-AS type I (ClyA), a dodecameric protein nanopore evolved from
Cytolysin A of \textit{S. typhii} (\cref{fig:clya_atomistic}).\cite{soskine2013} However, the lessons learned
form the modeling can be readily transferred to other types of nanopores.

This work is organized as follows. In the section \emph{\nameref{sect:modeling_approach}} we will explain the
details of the simulation, while in the section \emph{\nameref{sect:experiment}} the experimental setup will
be described. The verification of the model and a subsequent in-depth analysis of the ion transport physics
through the pore will be conducted in the section \emph{\nameref{sect:results}}. Lastly, the discussion and
the following conclusions are left for the remaining sections.



\section{Modeling Approach}\label{sect:modeling_approach}
Our goal is to model the protein nanopore ClyA  suspended on a lipid bilyer in a 2D-axisymmetric system in an
aqueous solution of NaCl. The lipid bilayer separates two reservoirs which can be electrically biased so that
an ion current flows through the nanopore in between the reservoirs. With no loss of generality we will assume
that the upper cis-reservoir (positive $z$-direction) is grounded, while the lower trans-reservoir is biased
with an applied potential of $\Vbias$. Figure~\ref{fig:model_geometry} illustrates the geometry of the
simulation domain.

The construction of a viable modeling approach then decomposes into two separate tasks. First, finding an
averaging scheme for the angular dependence of the charge distribution of the full 3D ClyA structure shown in
Figs.~\ref{fig:clya_atomistic_side} and \ref{fig:clya_atomistic_top} in order to accurately represent the
nanopore in a 2D-axisymemtric simulation. Second, defining a viable transport model for the ion species and
the flow of the solution through the nanopore. This includes the definition of the set of parameters
comprising mobilities, diffusion coefficients, and effective sizes of the ion species as well as the
viscosity, density, and permittivity of the solution. In particular, it is necessary to take in consideration
the nanofluidics within the pore since none of the parameters retain their bulk values in nanoscale
constrictions.



\subsection{Axially Symmetric Model of ClyA}
ClyA is a multimeric protein complex that consists of 12 (or more) identical subunits which form a 
\SI{14}{\nano\meter} long hydrophilic channel through a lipid bilayer (cf.~Figs.~\ref{fig:clya_atomistic_side}
and \ref{fig:clya_atomistic_top}). The pore cavity is lined with predominantly negatively charged residues and
can be divided in roughly two cylindrical compartments: the \cis{} lumen (\SI{5.5}{\nano\meter} diameter,
\SI{10}{\nano\meter} height) and the \trans{} constriction (\SI{3.3}{\nano\meter} diameter,
\SI{4}{\nano\meter} height). Due to this high degree of axial symmetry, we modelled ClyA as a solid,
cylindrically symmetric dielectric block, embedded within a lipid bilayer represented as a
\SI{2.8}{\nano\meter}-thick\cite{kucerka2011} dielectric disk  and surrounded by a spherical,
electrolyte-filled reservoir with a radius of \SI{250}{\nano\meter} 
(cf.~Figs.~\ref{fig:model_geometry} and \ref{fig:model_geometry_zoom}).\cite{lu2012,pederson2015} This approach
greatly reduces the computational cost compared to a full 3D model, while remaining sufficiently accurate.

The nanopore geometry (\cref{fig:model_concept}b) was derived directly from a full atom model of ClyA-AS,
an improved variant of ClyA evolved from the wild-type Cytolysin A from \textit{S. typhii} using directed 
evolution.\cite{soskine2013} Briefly, a homology model of ClyA-AS was constructed from the wild-type crystal 
structure (PDB: 2WCD)\cite{mueller2009} by introducing the necessary mutations in all 12 subunits 
(\cref{suppinfo:tab:clya_as_mutations}), followed by a brief energy optimization of the mutated side-chains 
with the conjugate gradient method (MODELLER, version 9.18).\cite{sali1993} We then computed the protonation 
state of titratable residue at \pH{7.5} and assigned the correct radius and partial charge to each atom of 
the model (PROPKA 3.1\cite{olsson2011} and PBD2PQR 2.1.1\cite{jurrus2018} software packages) as parametrized 
by the CHARMM36 forcefield.\cite{best2012} The atom radii were then used to compute a 3D-dimensional map of 
the nanopore's molecular surface with a probe radius of \SI{1.4}{\angstrom}, which was radially averaged 
along the z-direction with a custom-built MATLAB script to generate an approximate axisymmetric outline of 
the nanopore. To improve the quality of the final computational mesh, the resulting polygon was then 
simplified manually by removing overlapping and superfluous vertices.

Similar to the geometry, the two-dimensional charge distribution used in this model was also derived directly
from the full atom model (cf.~\cref{fig:model_concept}c). Inspired by how charges are represented in the
particle mesh Ewald (PME) method,\cite{aksimentiev2005} we computed the fixed charge distribution of the
nanopore $\scdpore(r,z)$ by assuming that an atom $i$ of partial charge $\partialcharge_{i}$ at the location
$(x_i, y_i, z_i)$ in the full 3D atomistic pore model will contribute the partial charge
$\partialcharge_{i}/2\pi\sqrt{x_i^2 + y_i^2}$ to a point $(r_i,z_i)$ in the averaged 2D-axisymmetric model.
That means we effectively spread the charge over all angles to achieve axial symmetry. 

Following that, we assume a Gaussian distribution of the space charge density of each atom $i$ around its
respective 2D-axisymmetric coordinates $(r_i,z_i)$. With the atom radius $\atomradius_{i}$ and the sharpness
factor $\sigma = \num{0.5}$, we find for the 2D space charge density
\begin{align}
\label{eq:scdpore}
\scdpore(r,z) = \displaystyle\sum_{i} \dfrac{\echarge\partialcharge_{i}}{\pi(\stdev\atomradius_{i})^2} 
\exp{\left(-\dfrac{(r-r_i)^2 + (z-z_i)^2}{(\stdev\atomradius_{i})^2}\right)}
\end{align}
where $\echarge$ is the elementary charge. To embed $\scdpore$ with sufficient detail yet efficiently
into a numeric solver, the spatial coordinates are discretized with a grid spacing of \SI{0.005}{\nano\meter}
in the domain of $\scdpore$ and precomputed values are used.

%Instead of explicitly selecting a few amino acids as \emph{key} charged
%residues, our method includes all partial charges and is hence more comprehensive. Furthermore, since it can
%be applied as a single boundary condition over the entire model, it does not require the definition of
%additional geometric features to hold the charges.

% Grid: r from 0 to 70 A, res = 0.05 A 
% 		z from -35 to 135 A, res = 0.05 A
% Sigma = 0.5



\subsection{Poisson-Nernst-Planck-Navier-Stokes (PNP-NS) equations}
To describe the ion and water transport mechanisms within ClyA, we created a computational model of the fluid
based on a set of well-known partial-differential equations (PDEs) called the coupled
Poisson-Nernst-Planck-Navier-Stokes (PNP-NS) equations. 

While these equations have proven to be excellent tool for studying ion and water transport in both 
solid-state\cite{daiguji2004,lu2012,chaudhry2014,rempfer2016,lin2016} and biological 
nanopores,\cite{eisenberg1996,simakov2010,pederson2015} the underlying continuum assumption does not hold at 
the nanoscale under all circumstances, particularly at material interfaces\cite{vo2016} and in regions of 
high charge density.\cite{corry2000} To address these concerns and to improve the accuracy of our model, we 
propose a modified version of these equations, that we will refer to as extended PNP-NS (ePNP-NS), which takes
into account the finite size of the ions\cite{borukhov1997,lu2011} and the reduction of ion and water mobility
near protein interfaces.\cite{makarov1998, pronk2013} Because the diffusion coefficient, mobility, viscosity,
density\cite{} and relative permittivity\cite{gavish2016} often deviate significantly from the traditional
`infinite dilution' values at experimentally relevant concentrations, we implemented a self-consistent
dependency on the average ion concentration for all these parameters.

In this section we will start by describing the classical PNP-NS equations and their boundary conditions,
followed by a description of ePNP-NS framework that includes a several additions and corrections which enhance
their physical accuracy at nanoscale and their ability to predict ionic currents at high ionic strengths.

Due to the complexity of the set of self-consistent equations, we chose to solve the system using the finite
element method (FEM) as provided by the commercial software COMSOL Multiphysics (version 5.3,
\href{www.comsol.com}{www.comsol.com}).

\subsubsection{Poisson Equation} 
The Poisson equation (PE) relates the electrostatic potential to the charges present in a system 
and is given by 
\begin{align} 
\label{eq:poisson}
\nabla\cdot\left(\absperm\relperm \nabla\potential\right) = - \scdpore - \scdion,
\end{align}
with $\potential$ the electric potential, $\absperm=\SI{8.85419e-12}{\farad\per\meter}$ the vacuum
permittivity, and $\relperm$ the local relative permittivity of the medium listed in
\cref{tab:pnpns_parameters}. The total charge density on the right hand side of \cref{eq:poisson} consists of
a fixed contribution $\scdpore$ due to charges of the nanopore of \cref{eq:scdpore}, and an ion concentration
dependent component $\scdion$ caused by local non-electroneutralities in the fluid (cf.~\cref{eq:scdion}).

Dirichlet boundary conditions are applied to the boundaries of the \cis{} and \trans{} reservoirs. The
potential at the \cis{} boundary is set to $\potential = 0$ whereas at the \trans{} boundary it is set to
$\potential = \Vbias$. Neumann boundary conditions apply where at the bilayer-boundary interface.



\begin{table}[t]
\begin{center}
\begin{tabular}{c|c|l}
    Parameter & Value & Description \\\hline 
    $T$                    & \SI{298.15}{\kelvin} & System temperature\\
    $\permittivity_p$      & \num{20} & Relative permittivity of pore\cite{li2013}\\
    $\permittivity_m$      & \num{3.2}& Relative permittivity of bilayer\cite{gramse2013} \\
    $\permittivity_w$      & \num{78.15}& Relative permittivity of fluid \\
    $\density$             & \SI{997}{\kilogram\per\cubic\meter}& Density of fluid at infinite dilution\\
    $\viscosity$           & \SI{8.904}{\pascal\second}& Dynamic viscosity\\
    $\chargen_{\ce{Na+}}$  & \num{+1} & charge number of \ce{Na+} \\
    $\chargen_{\ce{Cl-}}$  & \num{-1} & charge number of \ce{Cl-} \\
    $\diffusion_{\ce{Na+}}$& \SI{1.334e-9}{\square\meter\per\second} & Diffusion coefficient of \ce{Na+} \\
    $\diffusion_{\ce{Cl-}}$& \SI{2.032e-9}{\square\meter\per\second} & Diffusion coefficient of \ce{Cl-} \\
    $\mobility_{\ce{Na+}}$ & \SI{5.192e-4}{\square\meter\per\second\per\volt} & Mobility of \ce{Na+}\\
    $\mobility_{\ce{Cl-}}$ & \SI{7.909e-4}{\square\meter\per\second\per\volt} & Mobility of \ce{Na+}
\end{tabular}
\caption{Parameters for the PNP-NS system-of-equations for an \ce{NaCl} solution.}
\label{tab:pnpns_parameters}
\end{center}
\end{table}


\subsubsection{Nernst-Planck Equation}
The Nernst-Plack equation (NPE) describes the flux $\flux_{i}$ of each ion species $i$ under the influence of
a chemical gradient (diffusion), a non-zero electrical field (electromigration) and a laminar fluid flow
(convection) and is given by
\begin{align}
\label{eq:nernst-planck}
\nabla\cdot\flux_{i} = \nabla\cdot\left( -\diffusion_{i}\nabla\concentration_{i} - 
\chargen_{i}\mobility_{i}\concentration_{i}\nabla\potential \right)
+\velocity\cdot\nabla\concentration_{i} = 0
\end{align}
with $\concentration_{i}$ the concentration, $\diffusion_{i}$ the diffusion coefficient, $\chargen_{i}$ the 
charge number, $\mobility_{i}$ the electrophoretic mobility, and $\velocity$ the fluid velocity. The relevant
parameters for \ce{NaCl} are listed in \cref{tab:pnpns_parameters}.
The NPE is coupled to the PE by the electric potential $\potential$ and the ion space charge density
\begin{align} 
\label{eq:scdion}
\scdion = \faraday\displaystyle\sum_{i}\chargen_{i}\concentration_{i},
\end{align}
where $\faraday= \SI{96485.33}{\coulomb\per\mole}$ is Faraday's constant.

The ion concentration at both reservoir boundaries was fixed using a Dirichlet boundary condition 
($\concentration_i = \concentration_\textrm{bulk}$). At the fluid-bilayer and fluid-pore interface Neumann
boundary conditions apply, i.e.~no current can flow in or out the pore or the bilayer.



\subsubsection{Navier-Stokes Equation} 
The Navier-Stokes equation (NSE) for laminar flow of an incompressible fluid will be used to describe the
fluid velocity field $\velocity$. It is given by
\begin{align}
\label{eq:navier-stokes}
\density\left(\velocity\cdot\nabla\right)\velocity = \nabla\cdot\left[-\pressure\identity + 
\viscosity\left(\nabla\velocity + \left(\nabla\velocity\right)^{\rm T}\right)\right]+\volumeforce,
\end{align}
together with the continuity equation
\begin{align}
\label{eq:navier-stokes-contin}
\density\nabla\cdot\left(\velocity\right) = 0,
\end{align}
where $\density$ and $\viscosity$ are the fluid's density and dynamic viscosity, respectively. For \ce{NaCl}
the parameters are given in \cref{tab:pnpns_parameters}. The driving force that generates the electro-osmotic
flow 
\begin{align}
\volumeforce = \echarge\avogadro\scdion\efield,
\end{align}
acts on all fluid elements with a non-zero net space charge density $\scdion$ and electrical field $\efield$.

At the boundaries of the \cis{} and \trans{} reservoirs, we use open boundary conditions with no normal stress
($\left[-\pressure\identity + \viscosity\left(\nabla\velocity + \left(\nabla\velocity\right)^{\rm
T}\right)\right]\normvec = 0$), effectively mimicking an endless reservoir. On the nanopore and bilayer
surface we applied a no-slip ($\vec{\velocity} = 0$) boundary condition.



\subsection{Extended PNP-NS Equations}
The traditional PNP-NS equations as described above cannot capture the transport accurately within fluids of
strongly varying concentrations or when the concentration is close to saturation or in nanoscale constrictions
when wall effects are non-negligible. However, the simulation of transport of ions through protein nanopores
happens precisely under these conditions and therefore we include the three following effects fully
self-consistently:
\begin{itemize}
    \item concentration dependent diffusion, mobility, viscosity, density an relative permittivity;
    \item wall-distance dependent diffusion, mobility and viscosity; and
    \item inclusion of a finite ion size term to the NPE. 
\end{itemize}
In the remainder of this work, we will refer to the model containing these effects as extended PNP-NS
(ePNP-NS).

\subsubsection{Concentration dependent properties.}
The concentration dependent ion self-diffusion coefficients $\diffusion_{i}^1(\dconc)$, cation transport 
number $\transportn_{i}(\dconc)$, viscosity $\viscosity^1(\dconc)$, density $\density(\dconc)$ and relative 
permittivity $\permittivity_{w}(\dconc)$ could be fitted directly the experimental data with the empirical 
equations given in \cref{tab:corrections_equations}. Here $\dconc = \avionconc/\SI{1}{\Molar}$ is a 
dimensionless value that represents the average ion concentration. The concentration-dependent ionic mobility 
$\mobility^1_{i}$ must first be derived from the salt's molar conductivity $\molarconductivity$ and the ion's 
transport number $\transportn_{i}$ before it can be fitted\cite{aburto2013I}
\begin{align}
\label{eq:conductivity-to-mobility}
\mobility_{i}(\concentration) = \frac{\specmolarconductivity_{i}(\concentration)}{\chargen_{i}\faraday} 
\quad\text{with}\quad \specmolarconductivity_{i}(\concentration) = \molarconductivity(\concentration) 
\transportn_{i}(\concentration),
\end{align}
where $\specmolarconductivity_{i}(\concentration)$ is the specific molar conductivity of ion $i$.

% self-diffusion coefficients \cite{mills1989}
% cation transport number\cite{panopoulos1986,currie1960,smits1966,schonert2014,dellamonica1979}
% molar conductivity \cite{bianchi1989, currie1960, goldsack1976, dellamonica1979}
% viscosity and density \cite{hai-lang1996}
% relative permittivity \cite{gavish2016}

All fitting functions can be found in \cref{tab:corrections_equations}, their fitting coefficients in 
\cref{suppinfo:tab:corrections_parameters} and a plot of the fits to the experimental data in 
\cref{suppinfo:fig:corrections}.

\subsubsection{Wall-distance corrections.}
The diffusivity of small molecules such ions or water depends heavily on their proximity, i.e. within 
\SI{1.0}{\nano\meter} distance, to large molecules such as proteins or DNA.\cite{makarov1998} In small 
nanopores (\SI{\le 5}{\nano\meter} radius), these effects encompass a significant fraction of the total 
nanopore radius, and hence cannot be neglected.\cite{simakov2010,pederson2015} For the wall-distance 
dependent diffusion coefficients $\diffusion_{i}^2(\dwall)$ and mobilities 
$\mobility_{i}^2(\dwall)$ we used Eq.~11 from ref.~\citenum{simakov2010} 
and the fitting parameters obtained by the authors. For the wall-distance dependent viscosity 
$\viscosity^2(\dwall)$, we first offset the inverse viscosity data from Figure~4 of ref.~\citenum{pronk2013} 
by each proteins' respective hydrodynamic radius, which we assume to be a good approximation of the average 
protein radius, and fitted all the resulting data points with a logistic function.

\paragraph{Finite ion-size effects.}
Because the PNP equations do not take the size of the ions into account, they tend to overscreen locations 
with high charge densities resulting in unrealistically high ion concentrations.\cite{corry2000} This problem 
can be addressed by the addition of a nonlinear flux term to the NPE that poses an upper limit to the 
concentration. We opted for the size-modified PNP (SMPNP) framework developed by Lu and Zhou,\cite{lu2011}
\begin{align}
\nabla\cdot\flux_{i} = \nabla\cdot\left(
	- \diffusion_{i}\nabla\concentration_{i}
	- \chargen_{i}\mobility_{i}\concentration_{i}\nabla\potential 
	- \frac{k_{i}\concentration_{i} \displaystyle\sum_{l} \ionsize_{l}^3 \nabla \concentration_{l}}{1 - 
	\displaystyle\sum_{l} \avogadro \ionsize_{l}^3 \concentration_{l}}\right)
  +\velocity\cdot\nabla\concentration_{i} = 0
	\quad\text{with}\quad k_{i}=\ionsize_{i}^3/\ionsize_{0}^3
\end{align}
where $\ionsize_{i}$ and $\ionsize_{0}$ represent the maximum packing radius in a cubic lattice for ions and 
solvent molecules, respectively. The radii were set to \SI{0.5}{\nano\meter} for both \ce{Na+} and \ce{Cl-} 
and to \SI{0.311}{\nano\meter} for water, corresponding to maximum concentrations of \SI{13.3}{\Molar} and 
\SI{55.2}{\Molar}, respectively. 

\paragraph{The final ePNP-NS model.}
We obtain the final ePNP-NS model by substituting the following, traditionally constant, parameters of the 
PNP-NS equations by the concentration- and distance-dependent functions
\begin{align}
\diffusion_{i}		&= \left[ \diffusion_{i}^1(\dconc) \times \diffusion_{i}^2(\dwall) \right] \times \SI{e-9}{\square\meter\per\second}, \\
\mobility_{i}  		&= \left[ \mobility_{i}^1(\dconc) \times \mobility_{i}^2(\dwall) \right] \times \SI{e-8}{\square\meter\per\second\per\volt}, \\
\viscosity     		&= \left[ \viscosity^1(\dconc) \times \viscosity^2(\dwall)\right] \times \SI{e-4}{\pascal\second}, \\
\density 	   		&= \density(\dconc) \times\SI{e3}{\kg\per\cubic\meter} \quad\text{and}, \\
\permittivity_{w} 	&= \permittivity_{w}(\dconc)
\end{align}
where $\dconc = \avionconc/\SI{1}{\Molar}$ and $\dwall = \walldistance/\SI{1}{\nano\meter}$ represent the 
dimensionless average ion concentration and dimensionless distance from the nanopore wall, respectively.
The detailed equations behind these functions are listed in \cref{tab:corrections_equations}, their fits to 
the experimental data are shown in \cref{suppinfo:fig:corrections} and the resulting fitting parameters are 
listed in \cref{suppinfo:tab:corrections_parameters}.

\begin{table*}[h]

\footnotesize
\renewcommand{\arraystretch}{1.2}
\caption{Summary of the equations used for the concentration and wall distance dependent model properties.}
\centering
\label{tab:corrections_equations}

\begin{tabularx}{16.5cm}{>{\raggedright\hsize=3.5cm}X >{\hsize=1cm}c >{\hsize=1cm}c >{\hsize=7cm}X >{\hsize=3cm}l}
	\toprule
	Property					& Dep.\textsuperscript{\emph{a}}	& Function 						& Fitting function\textsuperscript{\emph{b}}	& Source \\
	\midrule
	\multirow{2}{3.5cm}{Ion self-diffusion coefficient}	& $\dconc$	& $\diffusion_{i}^c(\dconc)$	& $P_0/\left( 1 + P_1\dconc^{0.5} + P_2\dconc + P_3\dconc^{1.5} + P_4\dconc^2 \right)$ & This work \\
														& $\dwall$	& $\diffusion_{i}^w(\dwall)$	& 
														$1-\exp{\left(-P_1(\dwall+P_2)\right)}$	& \citenum{Makarov-1998,Simakov-2010} 
														\vspace{0.25cm} \\
	
    \multirow{2}{3cm}{Ion electrophoretic mobility}		& $\dconc$	& $\mobility_{i}^c(\dconc)$		& $P_0/\left( 1 + P_1\dconc^{0.5} + P_2\dconc + P_3\dconc^{1.5} + P_4\dconc^2 \right)$	& This work \\
														& $\dwall$	& $\mobility_{i}^w(\dwall)$		& 
														$1-\exp{\left(-P_1(\dwall+P_2)\right)}$	& \citenum{Makarov-1998,Simakov-2010}	
														\vspace{0.25cm} \\
	
	Ion transport number								& $\dconc$	& $\transportn_{i}(\dconc)$		& $P_0/\left( 1 + P_1\dconc^{0.5} + P_2\dconc + P_3\dconc^{1.5} + P_4\dconc^2 \right)$& This work \vspace{0.25cm} \\
	
    \multirow{2}{*}{Dynamic viscosity}					& $\dconc$	& $\viscosity^c(\dconc)$		& $P_0 \left( 1 + P_1 \dconc^{0.5} + P_2 \dconc + P_3 \dconc^2 + P_4 \dconc^{3.5} \right)$	& This work \\
														& $\dwall$	& $\viscosity^w(\dwall)$		& $1+\exp{\left(-P_1(\dwall-P_2) 
														\right)}$	& \citenum{Pronk-2014}	\vspace{0.25cm} \\
	
	Fluid density										& $\dconc$	& $\density(\dconc)$			& $P_0 \left( 1 + P_1 \dconc + P_2 \dconc^2 \right)$	& This work \vspace{0.25cm} \\	
	
	Fluid relative permittivity							& $\dconc$	& $\permittivity_{w}(\dconc)$	& $P_0 - \left(P_0 - 
	P_1\right) L \left( \dfrac{3P_2}{P_0 - P_1} \dconc \right)$	& \citenum{Gavish-2016} \\
	\bottomrule
\end{tabularx}
\begin{flushleft}
	\textsuperscript{\emph{a}}Dependencies on either $\dconc = \avionconc/1$~M (dimensionless average ion 
	concentration) and $\dwall = \walldistance/1$~nm (dimensionless distance from the nanopore wall);
	\textsuperscript{\emph{b}}These functions are empirical and hence have no physical meaning.
	The values of the fitting parameters $P_x$ of each property can be found in Table~\ref{suppinfo:tab:corrections_parameters}.
\end{flushleft}



%\begin{tabularx}{16.5cm}{>{\raggedright\hsize=3.5cm}X >{\hsize=9cm}X >{\hsize=4cm}l}
%	\toprule
%	Property	&  Equation\textsuperscript{\emph{a}}\textsuperscript{\emph{b}}	& Equation source \\
%	\midrule
%	\multirow[t]{3}{3.5cm}{Ion self-diffusion coefficient}
%		& $\diffusion_{i}(\concentration,d) = \left[ \diffusion_{i}^1(\concentration) \times \diffusion_{i}^2(d) \right]$ $\times\SI{e-9}{\square\meter\per\second}$
%			& This work (empirical) \\
%		&  \hspace*{0.25cm} $\diffusion_{i}^1(\concentration) = P_0/(1 + P_1 c^{0.5} + P_2 c + P_3 c^{1.5} +P_4 c^{2})$
%			& This work (empirical) \\
%		&  \hspace*{0.25cm} $\diffusion_{i}^2(d) = 1-\exp(-P_1(d+P_2))$
%			& \citenum{makarov1998, simakov2010} \vspace{0.25cm} \\
%	
%	\multirow[t]{3}{3cm}{Ion electrophoretic mobility}
%		& $\mobility_{i}(\concentration,d) = \faraday \left[ \mobility_{i,r} \times (\concentration)\mobility_{i,r}(d) \right]$ $\times\SI{e-4}{\square\meter\per\second\per\volt}$
%			& This work (empirical) \\
%		&  \hspace*{0.25cm} $\mobility_{i,r}(\concentration) = P_0/(1 + P_1 c^{0.5} + P_2 c + P_3 c^{1.5} +P_4 c^{2})$
%			& This work (empirical) \\
%		&  \hspace*{0.25cm} $\mobility_{i,r}(d) = 1-\exp(-P_1(d+P_2))$
%			& \citenum{makarov1998, simakov2010}	\vspace{0.25cm} \\
%	
%	Ion transfer number			
%		& $\transportn_{i}(\concentration) = P_0/(1 + P_1 c^{0.5} + P_2 c + P_3 c^{1.5} +P_4 c^{2})$
%			& This work	(empirical) \vspace{0.25cm}\\
%	
%	\multirow[t]{3}{*}{Dynamic viscosity}
%		& $\viscosity(\concentration,d) = \left[ \viscosity^1(\concentration) \times \viscosity^2(d)\right]$ $\times\SI{e-4}{\pascal\second}$ 
%			& This work (empirical) \\
%		&  \hspace*{0.25cm} $\viscosity^1(\concentration) = P_0 \left( 1 + P_1 c^{0.5} + P_2 c + P_3 c^{2} + P_4 c^{3.5} \right)$ 
%			& \textbf{TODO} \\
%		&  \hspace*{0.25cm} $\viscosity^2(d) = 1+e^{-P_1(d-P_2)}$
%			& \citenum{pronk2013}	\vspace{0.25cm} \\
%	
%	Fluid density
%		&  $\density(\concentration) = P_0 \left( 1 + P_1 c + P_2 c^{2} \right)$ $\times\SI{e3}{\kg\per\cubic\meter}$
%			& \textbf{TODO} \vspace{0.25cm} \\	
%	
%	Fluid relative permittivity	
%		&  $\permittivity(\concentration) = P_0 - \left(P_0 - P_1\right) L \left( \dfrac{3P_2}{P_0 - P_1} c \right)$
%			& \citenum{gavish2016} \\
%	\bottomrule
%\end{tabularx}
%\begin{flushleft}
%\textsuperscript{\emph{a}}Here $\concentration = \avionconc$ (average ion concentration) and $\walldistance$ is the distance from the nanopore wall;
%\textsuperscript{\emph{b}}The values of the fitting parameters $P_x$ of each property can be found in Table~\ref{tab:corrections_parameters}.
%\end{flushleft}

\end{table*}


\section{Experimental}\label{sect:experiment}
\paragraph{Single-channel ionic current measurements.}
\textbf{For Florian}

\newpage
\section{Results}\label{sect:results}
In this section, we start by our describing experimental data on the ionic transport through ClyA in terms of 
conductance, rectification and ion selectivity, and compare these directly with our continuum simulations. 
Next, we discuss the influence of the applied bias voltage and the bulk ionic strength on 3 distinct nanopore 
properties: 1) the distribution of cations and anions inside the pore, 2) the total electrostatic potential 
and 3) the electro-osmotic flow. 



\subsection{Transport of ions through ClyA}
% Key message: The extensions made to the PNP-NS equations are necessary to accurately predict the ionic current flowing through ClyA over a wide range of salt concentrations and bias voltages

\paragraph{Ionic current and conductance.}
The ionic current flowing through a nanopore is determined by both characteristics of the pore itself (i.e., 
its shape, size and electrostatic field), and by the properties of the electrolyte that surrounds it (i.e., 
ion diffusion coefficients, ion electrophoretic mobilities, solvent viscosity, density and relative 
permittivity). In order to reliably predict the ionic current, our nanopore simulation should thus not only 
contain an accurate geometry and charge distribution, it must also model all important transport processes 
(i.e., diffusion, electromigration and convection), which in turn should be parametrized with values relevant 
to the local conditions in the nanopore and the bulk electrolyte.

To validate the ionic currents predicted by our computational model of the biological nanopore ClyA-AS Type I 
(ClyA)\citep{soskine2013}, we used single-channel electrophysiology measure the current-voltage 
characteristics for \SIlist[list-units=single]{50;150;500;1000;3000}{\milli\Molar} \ce{NaCl} and between 
\SIlist[list-units=single]{-150;+150}{\milli\volt} with \SI{5}{\milli\volt} steps (\cref{fig:conductance}). 
The ionic current $\currentsim$ in our simulation was calculated by integration of the ionic flux 
$\vec{\flux_{i}}$ of each ion $i$ across the \textit{cis} reservoir boundary $S$:
\begin{equation}
	\currentsim = \faraday\int_{S}\left(\displaystyle\sum_{i} \chargen_{i}\normvec\cdot\vec{\flux_{i}}\right)dS  
\end{equation}
with $\faraday$ the Faraday constant (\SI{96.485}{\coulomb\per\mole}), $\chargen_{i}$ the ion charge number 
and $\normvec$ the normal vector. Simulations were performed at 
\SIlist[list-units=single]{1;5;10;25;50;75;100;125;150;200;250;300;400;500;750;1000;1500;2000;2500;3000;4000;5000}{\milli\Molar}.


In \cref{fig:conductance}a we plotted the experimental and simulated (PNP-NS and ePNP-NS) IV curves for 
several concentrations (\SIlist[list-units=single]{50;150;500;3000}{\milli\Molar}) which we deemed 
representative of the experimentally relevant range of ionic strengths. The IV curves of ClyA exhibit the 
typical rectification of a negatively charged conical nanopore, with higher currents at positive compared to 
negative bias.\textbf{REF} The magnitude of the rectification depends heavily on the salt concentration 

together with the simulated values obtained with the PNP-NS and the ePNP-NS equations.
While both models show the same qualitative trends as the experimental data, i.e.  in terms of ionic current rectification

From the IV curves we were able to calculate the ionic conductance ($\conductance = \current/\varpotential$) for bias voltages between \SIlist[list-units=single]{-150;+150}{mV} and
for bulk salt concentrations from \SIrange{50}{3000}{\milli\Molar} for the experimental (\cref{fig:conductance}b) and the simulated data (\cref{fig:conductance}c).
Experimentally, the conductance of ClyA rises from approximately \SIrange{0.5}{1}{\nano\siemens} at \SI{50}{\milli\Molar}

To better quantitatively judge the difference between the experimental and simulated conductances,
we calculated the their relative errors ($(\conductance_\textrm{sim}-\conductance_\textrm{exp})/\conductance_\textrm{exp}\times 100\%$)
can be found in \cref{fig:conductance}d for both PNP-NS (left) and ePNP-NS (right).
he PNP-NS equations overestimate the real ionic current under all tested conditions (\SIrange{40}{120}{\percent} deviation),
the ePNP-NS model shows significantly improved match with the experimental data (\SIrange{5}{10}{\percent} deviation).

For PNP-NS, the relative error increases with increasing ionic strength, 
In constrast, the error for ePNP-NS is decreases with increasing salt concentrations and po
In contrast,, with re By introducing the concentration dependent properties (ePNP-NS) results in a relative errors smaller that .



% Difference in conductance mechanism between high and low salt


\paragraph{Ionic current rectification.}
A nanopore is said to exhibit ionic current rectification (ICR) when the conductance for a given bias voltage magnitude is higher when applied to one side of the nanopore compared to the other.
Here, we define the rectification $\icr$ as $\conductance_+/\conductance_-$, or the ratio between the absolute pore conductances for at positive and negative bias voltages.
ICR is a phenomenon often observed in nanopores that are both charged and contain a degree of geometrical asymmetry along the central axis of the pore.
With it's high negative charge (\SI{-72}{\elementarycharge} at \pH{7.5}) and different \textit{cis} (\SI{\approx 3.3}{\nano\meter}) and \textit{trans} (\SI{\approx 5.5}{\nano\meter}) entry diameters,
ClyA fulfils both conditions and hence exhibits a strong degree of rectification, which has been used to determine the orientation of the pore in the l
it exhibits a asymmetric at cis (\SI{5.5}{\nano\meter})




\cref{fig:rectification_contour} gives an overview experimentally determined and simulated (ePNP-NS) rectification magnitudes in terms of voltage and salt concentration.
A cross-section of these plots at \SI{100}{\milli\volt} is given in \cref{fig:rectification_section}.
 
% Experiments

While the ICR rises linearly with the applied bias voltage, the concentration dependency exhibits a maximum value of 1.28 for the experiment
The concentration dependence is not as trivial, and hence we plotted its values at \SI{100}{\milli\volt} vs. the bulk concentration in \cref{fig:rectification_section}
Due to the high degree of electrostatic screening, the ICR is close to unity (1.02) , but rises rapidly when ,The rectification rises concentration dependency is not so trivial, however, as it appears to show a maximum value of 1.28 at \SI{150}{\milli\Molar},
and decreasing for both lower and higher salt concentrations.
The simulation shows the 

% Simulations


\paragraph{Ion selectivity.}


\subsection{Ion concentration distribution}
To improve our insight into the origin of the current rectification and ion selectivity, we investigated the ion concentrations inside the nanopore.

\paragraph{Average ion concentration inside the pore.}
To show the influence of the ionic strength and the applied bias voltage on the ion concentration inside the pore,
we computed the average ion concentration inside the pore, relative to the bulk value (\cref{fig:concentration_pore_average}).
In the case of \ce{Na+}, a clear enhancement is observed for 

The mean ion concentrations inside the nanopore, normalized by their corresponding bulk values, were plotted in \cref{fig:concentration_pore_average} against the bias voltage for range of bulk concentrations (\SIlist{5;50;150;500;1000;5000}{\mM}).
for \ce{Na+} (top) and \ce{Cl-} (bottom)





with the latter showing a strong bias dependency 
with almost full depletion at negative voltages.
\paragraph{Ion concentration in the pore lumen.}
Inside the lumen of the pore ($1.6<z<\SI{12.25}{\nm}$), 
\paragraph{Ion concentration in the pore constriction.}




%The show the influence of the bulk reservoir concentration average ion concentrations relative to the bulk reservoir concentration were plotted
%against the applied bias voltages in \cref{fig:concentration_pore_average}
%
%At higher salt concentrations (\SI{>500}{\mM}) both ions are close to their bulk values ($\left<c_{i}/c_\text{bulk}\right>_\text{pore} \approx 1$).
%This is no longer the case at lower ionic strengths (\SI{<500}{\mM}), where we observe a strong enhancement for \ce{Na+} and a depletion for \ce{Cl-}.
%While the former only depeSodium ions  both ions exhibit a dependency on voltage, the one for \ce{Cl-} is The latter also depends on the  o
%The concentration depends weakly on voltage for \ce{Na+}, but strongly for \ce{Cl-}, particularly at low salt concentrations
%with increased enrichment for positive biases A weak voltage dependency can be observed for cations (stronger enhancement at positive bias voltages) and a 
%For higher concentrations,
%Above bulk concentrations of , both  and \ce{Cl-} are close to bulk under all conditions.
%Below these concentrations \ce{Na+} is enriched while \ce{Cl-} depletes from the pore.

\subsection{Electrostatic potential}

\paragraph{Potential profiles at positive bias voltage.}
\paragraph{Potential profiles at negative bias voltage.}




\subsection{Electro-osmotic flow}


\paragraph{Electro-osmotic flow profile, magnitude and direction.}
\paragraph{Electro-osmotic flow at low, medium and high ionic strengths.}
\paragraph{Hydrostatic pressure.}



\section{Discussion}\label{sect:discussion}
% Inter-figure observations and discussion

\paragraph{Rectification at negative biases is caused by depletion of anions in the nanopore lumen.}

\section{Conclusions}\label{sect:conclusions}

\begin{acknowledgement}
The authors thank ...
\end{acknowledgement}


%-------------------------------------------------------------------------------
% SUPPORTING INFORMATION
%-------------------------------------------------------------------------------
\begin{suppinfo}
	Extended materials and methods.
\end{suppinfo}


\begin{figure*}[!bt]

	\centering
	\begin{minipage}[t]{5cm}
		\begin{subfigure}[t]{5cm}
			\centering
			\caption{}\label{fig:clya_side}
      \vspace{-5mm}
			\includegraphics[scale=1]{figures/concept/clya_side}
		\end{subfigure}
   	\begin{subfigure}[t]{5cm}
      \centering
      \caption{}\label{fig:clya_top}
      \vspace{-5mm}
      \includegraphics[scale=1]{figures/concept/clya_top}
    \end{subfigure}
	\end{minipage}
  \hspace{0.5cm}
  \begin{minipage}[t]{11.5cm}
    \begin{minipage}[t]{5.5cm}
      \begin{minipage}[t]{5.5cm}
        \begin{subfigure}[t]{1.6cm}
          \centering
          \caption{}\label{fig:model_geometry_vs_wedge}
          \vspace{-3mm}
          \includegraphics[scale=1]{figures/concept/model_geometry_vs_wedge}
        \end{subfigure}
        \begin{subfigure}[t]{2.5cm}
          \centering
          \caption{}\label{fig:model_charge_density}
          \vspace{-3mm}
          \includegraphics[scale=1]{figures/concept/model_charge_density}
        \end{subfigure}
      \end{minipage}
      \begin{subfigure}[t]{5.5cm}
        \centering
        \vspace{0.5cm}
        \caption{}\label{fig:model_geometry_zoom}
        \vspace{-1cm}
        \includegraphics[scale=1]{figures/concept/model_geometry_zoom}
        %\vspace{0.25cm}
      \end{subfigure}
    \end{minipage}
    \hspace{-0.8cm}
    \begin{subfigure}[t]{5.5cm}
      \centering
      \caption{}\label{fig:model_geometry}
      \includegraphics[scale=1]{figures/concept/model_geometry}
    \end{subfigure}
  \end{minipage}

\caption[All-atom and 2D-axisymmetric models of ClyA.]
{
\textbf{All-atom and 2D-axisymmetric models of ClyA.}
(\subref{fig:clya_side}) Axial cross-sectional and (\subref{fig:clya_top}) top views of the dodecameric
nanopore ClyA-AS\cite{Soskine-2013}, derived through homology modelling from the \textit{E. coli} Cytolysin A
crystal structure (PDBID: 2WCD\cite{Mueller-2009}). Figures were rendered with
VMD.\cite{Humphrey-1996,Stone-1998}
(\subref{fig:model_geometry_vs_wedge}) The 2D-axisymmetric geometry was derived directly from the all-atom
model by computing the average inner and outer radii along the longitudinal axis of the pore, and hence
closely follows the outline of a \ang{30} wedge out of the homology model.
(\subref{fig:model_charge_density}) The fixed space charge density ($\scdpore$) map of ClyA-AS, obtained by
Gaussian projection of each atom's partial charge onto a 2D plane (see methods for details).
(\subref{fig:model_geometry_zoom}+\subref{fig:model_geometry}) The 2D-axisymmetric simulation geometry of ClyA
(grey) embedded in a lipid bilayer (green) and surrounded by a spherical water reservoir (blue). Note that all
electrolyte parameters depend on the local average ion concentration
$\avionconc=\frac{1}{n}\sum_{i}^{n}\concentration_{i}$ and that some are also influenced by the distance from
the nanopore wall $\walldistance$.
}\label{fig:model_concept}
\end{figure*}

\begin{figure*}[!htb]
  \centering
  \begin{minipage}[h]{15cm}
  \begin{minipage}[h]{5cm}
    \begin{subfigure}[t]{4.5cm}
      \centering
      \caption{}\vspace{-3mm}\label{fig:current-voltage_curves}
      \includegraphics[scale=1]{figures/conductance/current_vs_voltage_all_multiplot}
    \end{subfigure}
  \end{minipage}
  \begin{minipage}[h]{8cm}
    \begin{subfigure}[t]{8cm}
      \centering
      \caption{}\vspace{-5mm}\label{fig:conductance_contourmap_epnp}
      \includegraphics[scale=1]{figures/conductance/conductance_contourmap_epnp}
    \end{subfigure}
    \begin{subfigure}[h]{8cm}
      \vspace{3mm}
      \centering
      \caption{}\vspace{-5mm}\label{fig:conductance_rectification_contourmap_epnp}
      \includegraphics[scale=1]{figures/conductance/conductance_rectification_contourmap_epnp}
    \end{subfigure}
    \begin{subfigure}[h]{8cm}
      \vspace{3mm}
      \centering
      \caption{}\vspace{-5mm}\label{fig:transport_number_contourmap_epnp}
      \includegraphics[scale=1]{figures/conductance/transport_number_contourmap_epnp}
    \end{subfigure}
  \end{minipage}
\end{minipage}

% caption
\caption
[\textbf{Measured and simulated ionic conductance, rectification and cation selectivity of single ClyA
nanopores.}]
{
\textbf{Measured and simulated ionic conductance, rectification and cation selectivity of single ClyA
nanopores.}
(\subref{fig:current-voltage_curves}) Comparison of the simulated (PNP-NS and ePNP-NS) and experimentally
measured current-voltage (IV) curves of ClyA-AS at \SI{25\pm1}{\dC} between $\vbias=\text{
\SIrange{-200}{+200}{\mV}}$, and for $\cbulk=\text{\SIlist{0.05;0.15;0.5;1;3}{\Molar}}$ \ce{NaCl}.
Experimental errors ($n=3$) were smaller than the symbol size and hence not shown. We used the simulated ionic
currents (ePNP-NS) to generate concentration-voltage heatmaps (left) and cross-sections at characteristic
voltages (right) of
(\subref{fig:conductance_contourmap_epnp}) the ionic conductance $\conductance = \current /\vbias$,
(\subref{fig:conductance_rectification_contourmap_epnp}) the ionic conductance rectification $\icr(\vbias) = \conductance(+\vbias) / \conductance(-\vbias)$ and
(\subref{fig:transport_number_contourmap_epnp}) the \Na\ transport number
$\transportn_{\Na}=\conductance_{\Na}/\conductance$. While the $\conductance$ simply represents ClyA's ability
to transport ions as a function of the bulk reservoir concentration, $\icr$ describes the magnitude of the
ionic flux at positive vs. negative bias voltages. Finally, $\transportn_{\Na}$ reveals how much of the
current is carrier by \Na\ ions, i.e. the cation-selectivity of the pore.
}\label{fig:conductance}
\end{figure*}

% Slopes of log-log plot
%    0.005 to 0.05 M, -150 mV:
%    a=0.495±0.005, b=2.873±0.054
%    0.005 to 0.05 M, +150 mV:
%    a=0.600±0.005, b=5.894±0.104
%    0.3 to 1.5 M, -150 mV:
%    a=1.000±0.016, b=7.734±0.052
%    0.3 to 1.5 M, +150 mV:
%    a=0.787±0.001, b=8.935±0.006

\begin{figure*}[htbp]
\centering
\begin{minipage}[t]{8.2cm}
\begin{subfigure}[t]{8.2cm}
	\centering
	\caption{}\vspace{-3mm}\label{fig:concentration_contours}
	\includegraphics[scale=1]{figures/concentration/concentration_contours}
\end{subfigure}
\begin{subfigure}[t]{8.2cm}
  \centering
  \caption{}\vspace{-3mm}\label{fig:concentration_radial_profiles}
  \includegraphics[scale=1]{figures/concentration/concentration_radial_profiles}
\end{subfigure}
\begin{subfigure}[t]{8.2cm}
	\centering
	\caption{}\vspace{-3mm}\label{fig:concentration_pore_average_vs_concentration}
	\includegraphics[scale=1]{figures/concentration/concentration_pore_average_vs_concentration}
\end{subfigure}
\end{minipage}

% caption
\caption
[\textbf{Ion concentration distribution inside ClyA.}]
{
\textbf{Ion concentration distribution inside ClyA.}
(\subref{fig:concentration_contours})
Cross-section contour plots of the \ce{Na+} and \ce{Cl-} concentrations inside the pore relative to the bulk 
value of $0.15$~M at bias voltages $-100$ and $+100$~mV. These plots reveal the local concentration changes 
near charged residues, particularly in the highly negatively charged constriction. While conditions inside 
the lumen of the pore are generally close to bulk, a strong depletion of \ce{Cl-} can occur under high 
negative bias voltages.
(\subref{fig:concentration_radial_profiles})
Relative \ce{Na+} and \ce{Cl-} concentration profiles (to $\cbulk=0.15$~M) along the radius of the pore at 
the center of the constriction ($z=-0.3$~nm) and the lumen ($z=5$~nm) for $-100$ and $+100$~mV, showing the 
formation of the electrical double layer.
(\subref{fig:concentration_pore_average_vs_concentration})
Relative \ce{Na+} and \ce{Cl-} concentrations averaged over the entire pore in function of bulk salt 
concentration. ClyA's negatively charged interior results in the enhancement and the depletion of 
respectively \ce{Na+} and \ce{Cl-} concentrations inside the pore, particularly at lower ionic strengths. 
Both effects diminish with increasing concentrations and bulk-like conditions are observed for both ions at 
$\approx0.5$~M. Interestingly, the \ce{Cl-} concentration increases rapidly at higher positive bias voltages, 
resulting in bulk-like conditions at $+150$~mV for $\cbulk>0.05$~M.
}

\label{fig:concentration}

\end{figure*}
\begin{figure*}[!htb]
  \centering
  \hspace{-2cm}
  \begin{minipage}[t]{5.5cm}
    \begin{subfigure}[t]{5.5cm}
      \centering
      \caption{}\vspace{-3mm}\label{fig:flow_contour}
      \includegraphics[scale=1]{figures/flow/flow_contour_500mM}
    \end{subfigure}
    \begin{subfigure}[t]{5.5cm}
      \addtocounter{subfigure}{1}
      \vspace{3mm}
      \centering
      \caption{}\vspace{-3mm}\label{fig:flow_constriction_profiles}
      \includegraphics[scale=1]{figures/flow/flow_constriction_profiles}
    \end{subfigure}
  \end{minipage}
  \begin{subfigure}[t]{4cm}
    \addtocounter{subfigure}{-2}
    \centering
    \caption{}\vspace{2mm}\label{fig:flow_constriction_contour}
    \includegraphics[scale=1]{figures/flow/flow_constriction_contour}
  \end{subfigure}
  \begin{minipage}[t]{4cm}
    \begin{subfigure}[t]{4cm}
      \addtocounter{subfigure}{1}
      \centering
      \caption{}\vspace{-5mm}\label{fig:flow_conductance_vs_voltage}
      \includegraphics[scale=1]{figures/flow/flow_conductance_vs_voltage}
    \end{subfigure}
    \begin{subfigure}[t]{4cm}
      \vspace{2mm}
      \centering
      \caption{}\vspace{-5mm}\label{fig:flow_conductance_vs_concentration}
      \includegraphics[scale=1]{figures/flow/flow_conductance_vs_concentration}
    \end{subfigure}
    \begin{subfigure}[t]{4cm}
      \vspace{2mm}
      \centering
      \caption{}\vspace{-5mm}\label{fig:flow_conductance_rectification_vs_concentration}
      \includegraphics[scale=1]{figures/flow/flow_conductance_rectification_vs_concentration}
    \end{subfigure}
  \end{minipage}
\centering

% caption
\caption
[\textbf{Concentration and voltage dependency of the electro-osmotic flow inside ClyA.}]
{
(\subref{fig:flow_contour}) Contour plot of the electro-osmotic flow (EOF) velocity $\velocity$ at
\SI{0.5}{\Molar} and \SI{-100}{\mV} bias voltage. The arrows  on the streamlines indicate the direction of the
flow. As observed experimentally\cite{Soskine-2013} and  expected from a negatively charged conical nanopore,
the EOF follows the direction of the cation, i.e. from  \cis\ to \trans\ under negative bias voltages and vice
versa for positive ones.
(\subref{fig:flow_constriction_contour}) Contour plots of the EOF field in the trans constriction for various
salt concentrations at \SI{-100}{\mV} and
(\subref{fig:flow_constriction_profiles}) cross-sections of the absolute value of the water velocity
$\left|U_z\right|$ at $z=\SI{-1}{\nm}$. Notice that at high salt concentrations (\SI{>1}{\Molar}), the
velocity profile exhibits two `lobes' close to the nanopore walls and hence deviates from the parabolic shape
observed at lower ionic strengths.
(\subref{fig:flow_conductance_vs_voltage}) and (\subref{fig:flow_conductance_vs_concentration}) the
electro-osmotic conductance $G_{\text{eo}} = Q_{\text{eo}}/V$, with $Q_{\text{eo}}$ the total flow rate
through the pore, plotted against bias voltage and bulk salt concentration, respectively. In the low
concentration regime, $G_{\text{eo}}$ increases rapidly between \SIlist{0.005;0.5}{\Molar} after which it
decreases logarithmically for higher concentrations.
(\subref{fig:flow_conductance_rectification_vs_concentration}) The rectification of the electro-osmotic flow
rate ($\alpha_{\text{eo}} = G_{\text{eo}(}+V)/Q_{\text{eo}}(-V)$) plotted against the concentration.
$\alpha_{\text{eo}}$ shows a maximum between \SIlist{0.04;0.05}{\Molar}, after which it falls rapidly to reach
unity at approximately \SI{\approx0.45}{\Molar}, regardless of the applied bias. A minimum is then reached at
\SI{\approx1}{\Molar}, followed by a gradual approach towards unity.
}\label{fig:flow}

\end{figure*}



%-------------------------------------------------------------------------------
% REFERENCES
%-------------------------------------------------------------------------------
\newpage % new page
\bibliography{modeling1}

%\includepdf{supporting_information}
\end{document}

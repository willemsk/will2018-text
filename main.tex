%%%%%%%%%%%%%%%%%%%%%%%%%%%%%%%%%%%%%%%%%%%%%%%%%%%%%%%%%%%%%%%%%%%%%
%% This is a (brief) model paper using the achemso class
%% The document class accepts keyval options, which should include
%% the target journal and optionally the manuscript type.
%%%%%%%%%%%%%%%%%%%%%%%%%%%%%%%%%%%%%%%%%%%%%%%%%%%%%%%%%%%%%%%%%%%%%
\documentclass[journal=ancac3,manuscript=article,etalmode=truncate,maxauthors=0,layout=onecolumn]{achemso}
	\setkeys{acs}{etalmode=truncate,maxauthors=0}

%%%%%%%%%%%%%%%%%%%%%%%%%%%%%%%%%%%%%%%%%%%%%%%%%%%%%%%%%%%%%%%%%%%%%
%% Place any additional packages needed here.  Only include packages
%% which are essential, to avoid problems later. Do NOT use any
%% packages which require e-TeX (for example etoolbox): the e-TeX
%% extensions are not currently available on the ACS conversion
%% servers.
%%%%%%%%%%%%%%%%%%%%%%%%%%%%%%%%%%%%%%%%%%%%%%%%%%%%%%%%%%%%%%%%%%%%%
\usepackage[utf8]{inputenc}
\usepackage[english]{babel}
\usepackage{csquotes}
\usepackage{amsmath}
\usepackage{amsfonts}
\usepackage{amssymb}
\usepackage{mathtools}
\usepackage{textcomp}
\usepackage{textgreek}
\usepackage{gensymb}

\usepackage[version=4]{mhchem}
\usepackage[alsoload=synchem]{siunitx}
	\sisetup{separate-uncertainty=true}
	\sisetup{multi-part-units=single}
	\sisetup{tight-spacing=true}
	\sisetup{inter-unit-product=\ensuremath{{}\cdot{}}}
	\sisetup{list-units=single}
  \sisetup{range-units=single}
  \sisetup{list-final-separator={ and }}
  \sisetup{retain-explicit-plus=true}

\usepackage{float}
\usepackage{graphicx}
\usepackage{xcolor}
\usepackage{tikz}
	\usetikzlibrary{shapes,arrows.meta}
\usepackage{pgf}
\usepackage{pgfplots}
\pgfplotsset{compat=1.15}
\usepackage{array, booktabs, tabularx, multirow}
\usepackage{pdfpages}


\usepackage[font=scriptsize,labelfont=bf,labelsep=period]{caption}
	\captionsetup[table]{singlelinecheck=false,font=footnotesize,labelfont=bf}
\usepackage[font=scriptsize,labelfont=bf,labelsep=period]{subcaption}
	\captionsetup[subfigure]{labelfont=bf,textfont=normalfont,labelformat=simple,singlelinecheck=false}

\usepackage{xr-hyper}
\usepackage{hyperref}
\usepackage{cleveref}
\creflabelformat{equation}{#2#1#3}
\crefname{figure}{Fig.}{Figs.}
\Crefname{figure}{Figure}{Figures}
\crefname{table}{Tab.}{Tabs.}
\Crefname{table}{Table}{Tables}
\crefname{equation}{Eq.}{Eqs.}
\Crefname{equation}{Equation}{Equations}
\crefname{section}{Sec.}{Secs.}
\Crefname{section}{Section}{Sections}

\pdfsuppresswarningpagegroup=1

% SUPPORTING INFO
\usepackage{xr}

% Source:
% https://www.overleaf.com/learn/how-to/Cross_referencing_with_the_xr_package_in_Overleafmak

\makeatletter
\newcommand*{\addFileDependency}[1]{% argument=file name and extension
  \typeout{(#1)}
  \@addtofilelist{#1}
  \IfFileExists{#1}{}{\typeout{No file #1.}}
}
\makeatother
 
\newcommand*{\myexternaldocument}[2]{%
    \externaldocument[#1]{#2}%
    \addFileDependency{#2.tex}%
    \addFileDependency{#2.aux}%
}

\myexternaldocument{suppinfo:}{suppinfo}

%%%%%%%%%%%%%%%%%%%%%%%%%%%%%%%%%%%%%%%%%%%%%%%%%%%%%%%%%%%%%%%%%%%%%
%% If issues arise when submitting your manuscript, you may want to
%% un-comment the next line.  This provides information on the
%% version of every file you have used.
%%%%%%%%%%%%%%%%%%%%%%%%%%%%%%%%%%%%%%%%%%%%%%%%%%%%%%%%%%%%%%%%%%%%%
%%\listfiles


%%%%%%%%%%%%%%%%%%%%%%%%%%%%%%%%%%%%%%%%%%%%%%%%%%%%%%%%%%%%%%%%%%%%%
%% Place any additional macros here.  Please use \newcommand* where
%% possible, and avoid layout-changing macros (which are not used
%% when typesetting).
%%%%%%%%%%%%%%%%%%%%%%%%%%%%%%%%%%%%%%%%%%%%%%%%%%%%%%%%%%%%%%%%%%%%%

\renewcommand*{\bibfont}{\normalfont\small}

% Shorthands
\newcommand{\todo}[1]{\textbf{\textcolor{orange}{#1}}}
\newcommand{\ahl}{\textalpha HL}
\newcommand{\etal}{\textit{et al.}}
\newcommand{\cis}{\textit{cis}}
\newcommand{\trans}{\textit{trans}}

% Units
\DeclareSIUnit{\molar}{\mole\per\cubic\deci\metre}
\DeclareSIUnit{\Molar}{\textsc{M}}
\DeclareSIUnit{\atm}{\textsc{atm}}
\newcommand{\mM}{\milli\Molar}
\newcommand{\mV}{\milli\volt}
\newcommand{\um}{\micro\meter}
\newcommand{\nm}{\nano\meter}
\newcommand{\ns}{\nano\second}
\newcommand{\ps}{\pico\second}
\newcommand{\fs}{\femto\second}
\newcommand{\pN}{\pico\newton}
\newcommand{\ec}{\elementarycharge}
\newcommand{\dC}{\degreeCelsius}
\newcommand{\mps}{\meter\per\second}
\newcommand{\cnmpnspv}{\cubic\nano\meter\per\nano\second\per\volt}

% Vectors and math stuff
\renewcommand{\vec}[1]{\boldsymbol{#1}}
\newcommand{\rpos}{\vec{r}} % positional vector
\newcommand{\normvec}{\vec{\hat{n}}} % normal vector
\newcommand{\identity}{\vec{\rm I}} % Identity vector
\newcommand{\stdev}{\sigma}
\newcommand{\pav}[2]{\left< #1 \right>_{\text{#2}}} % Pore average
\newcommand{\vc}[2]{#1_{#2}}
\newcommand{\pd}[2]{\displaystyle\frac{\partial #1}{\partial #2}}
\newcommand{\hydrostresstensor}{\sigma_{ij}}
\def\dsum{\displaystyle\sum}

% Physical constants
\newcommand{\boltzmann}{k_{\rm B}}
\newcommand{\avogadro}{N_{\rm A}}
\newcommand{\temp}{T}
\newcommand{\faraday}{\mathcal{F}}
\newcommand{\echarge}{e}

% Field variables
\newcommand{\vel}{u}                  % Fluid velocity
\newcommand{\force}{F}                % A force
\newcommand{\potential}{\varphi}			% Electrostatic potential
\newcommand{\varpotential}{V}				  % Electrostatic potential
\newcommand{\concentration}{c}				% Ion concentration
\newcommand{\velocity}{\vec{u}}				% Fluid velocity
\newcommand{\pressure}{p}					    % Fluid pressure
\newcommand{\efield}{\vec{E}}         % Electrical field
\newcommand{\displacement}{\vec{D}}   % Electrical displacement field
\newcommand{\walldistance}{d}				  % Wall distance

% Dimensionless variables
\newcommand{\dconc}{\bar{\concentration}}		% Dimensionless concentration
\newcommand{\dwall}{\bar{\walldistance}}		% Dimensionless wall distance

% Material and ion parameters
\newcommand{\permittivity}{\varepsilon}
\newcommand{\absperm}{\permittivity_0}				% Permittivity of vacuum
\newcommand{\relperm}{\permittivity_r}				% Relative permittivity
\newcommand{\dielectric}{\relperm}				    % Relative permittivity
\newcommand{\diffusion}{\mathcal{D}}		     	% Diffusion coefficient
\newcommand{\mobility}{\mu}				         		% Electrophoretic mobility
\newcommand{\transportn}{t}				         		% Transport number
\newcommand{\chargen}{z}				          		% Ion charge number
\newcommand{\ionsize}{a}					           	% Ion size for smPNP
\newcommand{\density}{\varrho}			     			% Fluid density
\newcommand{\viscosity}{\eta}				         	% Fluid viscosity

\newcommand{\iondiffusion}[1]{\diffusion_{#1}}	        % Ion Diffusion coefficient
\newcommand{\ionmobility}[1]{\mobility_{#1}}	          % Ion electrophoretic mobility
\newcommand{\iontransportn}[1]{\transportn{#1}}	        % Ion transport number
\newcommand{\avionconc}{\langle\concentration\rangle}   % Average ion concentration

\newcommand{\tna}{\transportn_{\ce{Na+}}}

\newcommand{\molarconductivity}{\Lambda}
\newcommand{\specmolarconductivity}{\lambda}

% Derived properties
\newcommand{\scd}{\rho}							% Total charge density
\newcommand{\scdpore}{\scd_{\text{pore}}^f}		% Fixed charge density
\newcommand{\scdion}{\scd_{\text{ion}}}			% Ionic charge density
\newcommand{\flux}{\vec{J}} 							% Ion flux
\newcommand{\volumeforce}{\vec{\force}_{\rm ion}}	% Fluid volume force

\newcommand{\current}{I}
\newcommand{\currentsim}{\current_{\rm sim}}
\newcommand{\currentexp}{\current_{\rm exp}}
\newcommand{\conductance}{G}
\newcommand{\icr}{\alpha}

% Shorthands
\newcommand{\ci}{\concentration_{i}}
\newcommand{\cbulk}{\concentration_\text{s}}
\newcommand{\vbias}{\varpotential_\text{b}}
\newcommand{\Na}{\ce{Na+}}
\newcommand{\Cl}{\ce{Cl-}}
\newcommand{\Qion}{Q_{\rm ion}}
\newcommand{\radpot}{\left<\potential\right>_\text{rad}}
\newcommand{\radenergy}{\left< U_{\text{E},i} \right>_{\text{rad}}}
\newcommand{\deltaEt}{\Delta E_{\text{B},i}}
\newcommand{\pH}[1]{pH~\num{#1}}
\newcommand{\kT}{\boltzmann\temp}
\newcommand{\kTe}{\kT / \echarge}

% Atom properties
\newcommand{\partialcharge}{\delta}
\newcommand{\atomradius}{R}


% Colors
\definecolor{graphgreen}  {rgb}{0.30196078, 0.68627451, 0.29019608}
\definecolor{graphpurple} {rgb}{0.59607843, 0.30588235, 0.63921569}
\definecolor{graphblue}   {rgb}{0.29803921, 0.57254902, 0.98823529}
\definecolor{graphred}    {rgb}{0.91372549, 0.15686275, 0.18823529}

% Line and marker styles
\newcommand{\graphline}[2]{\raisebox{2pt}{\tikz{\draw[-,color=#2,#1,line width=1.5pt](0,0) -- (5mm,0);}}}
\newcommand{\graphmarker}[2]{\raisebox{0.5pt}{\tikz{\node[draw,scale=0.5,#1,fill=none, color=#2](){};}}}

%\newcommand{\graphlinemarker}[3]{\raisebox{0pt}{\tikz{\draw[-,#3,#1,line width = 1.0pt](2.mm,0) #2 (3.5mm,1.5mm);\draw[-,#3,#1,line width = 1.0pt](0.,0.8mm) -- (5.5mm,0.8mm)}}}
%\newcommand{\rectangle}{\raisebox{0pt}{\tikz{\draw[-,black,dotted,line width = 1.0pt](0.,0.8mm) -- (5.5mm,0.8mm);\draw[black,solid,line width = 1.0pt](2.mm,0) circle (3.5mm,1.5mm)}}}
\newcommand{\rectangle}[1]{\raisebox{0pt}{\tikz{\draw[-,black,solid,line width = 1.0pt](0.,0.8mm) -- (5.5mm,0.8mm);\draw[black,solid,line width = 1.0pt](2.mm,0) #1 (3.5mm,1.5mm)}}}


\newcommand{\afimec}{imec, Kapeldreef 75, B-3001 Leuven, Belgium}
\newcommand{\afkulchem}{KU Leuven, Department of Chemistry, Celestijnenlaan 200F, B-3001 Leuven, Belgium}
\newcommand{\afkulphys}{KU Leuven, Department of Physics and Astronomy, Celestijnenlaan 200D, B-3001 Leuven, Belgium}
\newcommand{\afrug}{University of Groningen, Groningen Biomolecular Sciences \& Biotechnology Institute, 9747 AG, Groningen, The Netherlands}

\author{Kherim Willems}
\affiliation{\afkulchem}
\alsoaffiliation{\afimec}

\author{Dino Rui\'{c}}
\affiliation{\afkulphys}
\alsoaffiliation{\afimec}

\author{Florian Lucas}
\affiliation{\afrug}

\author{Ujjal Barman}
\affiliation{\afimec}

%\author{Chang Chen}
%\affiliation{\afimec}

\author{Johan Hofkens}
\affiliation{\afkulchem}

\author{Giovanni Maglia}
\email{g.maglia@rug.nl}
\affiliation{\afrug}

\author{Pol Van Dorpe}
\email{Pol.VanDorpe@imec.be}
\affiliation{\afkulchem}
\alsoaffiliation{\afimec}



\title{Modeling of Ion and Water Transport in the Biological Nanopore ClyA}

\keywords{biological nanopore; cytolysin A; continuum simulation;
Poisson-Nernst-Planck and Navier-Stokes equations; single molecule}



\begin{document}


\begin{tocentry}
  %\centering
  %  \includegraphics[width=8.4cm]{TOC.PDF}
\end{tocentry}



\begin{abstract}
  \footnotesize
  In recent years, the protein nanopore cytolysin A (ClyA) has become a valuable tool for the detection,
  characterization and quantification of biomarkers, proteins and nucleic acids at the single-molecule level.
  Despite this extensive experimental utilization, a comprehensive computational study of ion and water
  transport through ClyA is currently lacking. Such a study yields a wealth of information on the electrolytic
  conditions inside the pore and on the scale the electrophoretic forces that drive molecular transport. To
  this end we have built a computationally efficient continuum model of ClyA which, together with an extended
  version of Poison-Nernst-Planck-Navier-Stokes (ePNP-NS) equations, faithfully reproduces its ionic
  conductance over a wide range of salt concentrations. These ePNP-NS equations aim to tackle the shortcomings
  of the traditional PNP-NS models by self-consistently taking into account the influence of both the ionic
  strength and the nanoscopic scale of the pore on all relevant electrolyte properties. In this study, we give
  both a detailed description of our ePNP-NS model and apply it to the ClyA nanopore. This enabled us to gain
  a deeper insight into the influence of ionic strength and applied voltage on the ionic conductance through
  ClyA and a plethora of quantities difficult to assess experimentally. The latter includes the cation and
  anion concentrations inside the pore, the shape of the electrostatic potential landscape and the magnitude
  of the electroosmotic flow. Our work shows that continuum models of biological nanopores---if the
  appropriate corrections are applied---can make both qualitatively and quantitatively meaningful predictions
  that could be valuable tool to aid in both the design and interpretation of nanopore experiments.
\end{abstract}


%%%%%%%%%%%%%%%%%%%%%%%%%%%%%%%%%%%%%%%%%%%%%%%%%%%%%%%%%%%%%%%%%%%%%
%% Start the main part of the manuscript here.
%%%%%%%%%%%%%%%%%%%%%%%%%%%%%%%%%%%%%%%%%%%%%%%%%%%%%%%%%%%%%%%%%%%%%
\section{Introduction}

The transport of ions and molecules through nanoscale geometries is a field of intense study that used both
experimental, theoretical and computational methods.\cite{Sparreboom-2010,Bocquet-2010,Maffeo-2012,
Thomas-2014,Wang-2014,Kim-2015} One of the primary driving forces behind this research is the development of
nanopores as label-free, stochastic sensors at the ultimate analytical limit (\ie{}~single molecule).
\cite{Bayley-2001,Dekker-2007,Venkatesan-2011,Zhang-2016} Such detectors have applications ranging from
the analysis of biopolymers,
\ie~DNA\cite{Deamer-2016,Kasianowicz-1996,Meller-2000,Maglia-2008,Butler-2008,Stoddart-2009,Franceschini-2013,Jain-2018}
or proteins,\cite{Restrepo-Perez-2018,Talaga-2009,Rodriguez-Larrea-2013, Nivala-2013,Kennedy-2016} to the
detection and quantification of
biomarkers,\cite{Chen-2013,Soskine-2012,Niedzwiecki-2013,VanMeervelt-2014,Huang-2017,Liu-2018,Galenkamp-2018}
to the fundamental study of chemical or enzymatic reactions at the single molecular
level.\cite{Willems-VanMeervelt-2017,Lieberman-2010, Nivala-2013,Ho-2015,Laszlo-2017}

Nanopores are typically operated in the resistive-pulse mode, \ie~by monitoring the fluctuations of their
ionic conductance over time.\cite{Bayley-2001,Dekker-2007,Maglia-2010,Venkatesan-2011} Experimentally, this is
achieved by placing the nanopore between two electrolyte compartments and applying a constant DC (or AC)
voltage across them. Due to the high resistance of the nanopore, virtually the full potential change occurs
within (and around) the pore, resulting in a strong electrical field (\SIrange{1e7}{1e8}{\mV\per\nm}) that
electrophoretically drives ions and water molecules through
it.\cite{Wong-2007,Mao-2014,Haywood-2014,Laohakunakorn-2015} Hence, analyte molecules such as DNA or proteins
are driven towards, and often \emph{through}, the nanopore by a combination of Coulombic (electrophoretic) and
hydrodynamic (electroosmotic) forces.\cite{Wong-2007,Grosberg-2010,Muthukumar-2010, Muthukumar-2014} If
successful, a translocation event is observed as a temporal fluctuation in the ionic conductance of the pore
that serves as a unique molecular `fingerprint' with which the molecule can be identified.\cite{Yusko-2017}
Because the frequency, magnitude, duration and even noise levels of these events depend on the properties of
both the analyte molecule and the nanopore itself, they are notoriously difficult to interpret unambiguously
without a full understanding of the nanofluidic phenomena that underlie them.

The computational approaches most widely used to study nanofluidic transport in ion channels or biological
nanopores comprise \emph{discrete} methods such as molecular dynamics
(MD)\cite{Lynden-Bell-1996,Allen-1999,Aksimentiev-2005,Luan-2008,Bhattacharya-2011,Zhang-2014,DiMarino-2015,Belkin-2016}
and Brownian dynamics
(BD),\cite{Schirmer-1999,Im-2002,Noskov-2004,Millar-2008,Egwolf-2010,DeBiase-2015,Pederson-2015} and
\emph{mean-field} (continuum) methods based on solving the Poisson-Boltzmann (PB)
equations\cite{Grochowski-2008, Baldessari-2008-1} and Poisson-Nernst-Planck (PNP)
equations,\cite{Eisenberg-1996,Gillespie-2002, Simakov-2010}. The latter can be coupled with the Navier-Stokes
(NS) equation to include the electroosmotic flow.\cite{Lu-2012,Pederson-2015} Due to their explicit atomic or
particle nature, MD and BD simulations are considered to yield the most accurate results. However, the large
computational cost of simulating a complete biological nanopore system (100K--1M atoms) for hundreds of
nanoseconds still necessitates the use of supercomputers.\cite{Aksimentiev-2005,Bhattacharya-2011} The
PNP(-NS) equations, on the other hand, are of particular interest due to their low computational cost and
analytical tractability. In a continuum approach, all materials are represented by structureless media whose
behavior is parameterized by material properties such as relative permittivity, diffusion coefficient,
electrophoretic mobility, viscosity and density. Because these properties can only emerge from the collective
behavior or interactions between small groups of atoms (\ie~the mean-field approximation), great care must
be taken when using them to compute fluxes and fields at the nanoscale.\cite{Corry-2000,Collins-2012}
Nevertheless, the PNP equations have been used extensively for the simulation of ion
channels,\cite{Im-2002,Furini-2006,Liu-2015} biological
nanopores\cite{Simakov-2010,Pederson-2015,Aguilella-Arzo-2017,Simakov-2018} and their solid-state
counterparts\cite{Cervera-2005,White-2008,Chaudhry-2014,Laohakunakorn-2015}--- often with excellent
qualitative, if not quantitative results.\cite{Maffeo-2012,Thomas-2014,Kim-2015}

To remedy the shortcomings of PNP and NS theory, a number of modifications have been proposed over the years.
These include, among others, 1) steric ion-ion interactions, 2) the effect of protein-ion/water interactions
on their motility (\ie~diffusivity and electrophoretic mobility), 3) the concentration dependencies of ion
motility, and solvent relative permittivity, viscosity and density.

The steric ion-ion interactions can be accounted for by computing the excess in chemical potential
($\mu_{i}^\text{ex}$) resulting from the finite size of the ions.\cite{Eisenberg-1996,Bazant-2009,
Daiguji-2010} Gillespie \etal{} combined PNP and density functional theory---where $\mu_{i}^\text{ex}$ was
split up in ideal, hard-sphere and electrostatic components---to successfully predict the selectivity and
current of ion channels.\cite{Gillespie-2002} In another approach, Kili\'{c} \etal{} derived a set of modified
PNP equations based on the free energy functional of the Borukhov's modified PB model\cite{Borukhov-1997}and
observed significantly more realistic concentrations for high surface potentials compared to the classical PNP
equations.\cite{Kilic-2007} To allow for non-identical ion sizes and more than two ion species, this model was
later extended by Lu \etal{}, who used it to probe the effect of finite ion size on the rate coefficients of
enzymes.\cite{Lu-2011}

The interaction of ions or small molecules like water with the heavy atoms of proteins or DNA results in a
strong reduction of their motility, as observed in MD simulations.\cite{Makarov-1998,Pronk-2014} Since these
effects happen only for distances \SI{\le1}{\nm}, they can usually be neglected for macroscopic simulations.
However, in small nanopores ($\le10$~nm radius), they comprise a significant fraction of the total nanopore
radius and hence must be taken into account.\cite{Noskov-2004,Simakov-2010,Pederson-2015,McMullen-2017} In
continuum simulations, this can be taken into account by using positional-dependent ion diffusion
coefficients; as was implemented in the so-called `soft-repulsion PNP' developed by Simakov and
Kurnikova\cite{Simakov-2010, Simakov-2018} to predict the ionic conductance of the $\alpha$-hemolysin
nanopore. Similar reductions in ion diffusion coefficients have been proposed to improve PNP theory's
estimations of the ionic conductance of ion channels.\cite{Furini-2006,Liu-2015} The motility of water
molecules is expressed by the NS equations as the fluid's viscosity. Hence, as also observed in MD simulations
for water molecules near proteins\cite{Pronk-2014} and confined in hydrophilic
nanopores,\cite{Qiao_Aluru-2003,Vo-2016,Hsu-2017} the water-solid interaction leads to a viscosity several
times higher compared to the bulk values. Note that this is valid for hydrophilic interfaces only, as the lack
of interaction with hydrophobic interfaces, such as carbon nanotubes, leads to a lower
viscosity.\cite{Ye-2011}

It is well known that the self-diffusion coefficient $\diffusion_{i}$ and electrophoretic mobility
$\mobility_{i}$ of an ion $i$ depends on the local concentrations of all the ions in the
electrolyte.\cite{ContrerasAburto-2013-1} Their values typically decrease with increasing salt concentration,
and hence should not be treated as constants. Moreover, even though the well-known Nernst-Einstein (NE)
relation $\mobility_{i}=\diffusion_{i}/\kT$ is strictly speaking only valid at infinite dilution and a good
approximation at low concentrations (\SI{<10}{\mM}), it significantly overestimates the ionic mobility at
higher salt concentrations.\cite{Mills-1989,Panopoulos-1986,ContrerasAburto-2013-1,ContrerasAburto-2013-2} In
an empirical approach, Baldessari and Santiago formulated an ionic-strength dependency of the ionic mobility
based on their activity coefficients\cite{Baldessari-2008-1} and showed excellent correspondence between the
experimental and simulated ionic conductance of long nanochannels over a wide concentration
range.\cite{Baldessari-2008-2} Alternatively, Burger \etal{} used a microscopic lattice-based model to derive
a set of PNP equations with non-linear, ion density-dependent mobilities and diffusion coefficients that
provided significantly more realistic results for ion channels.\cite{Burger-2012} Note that other electrolyte
properties, such as its viscosity,\cite{Hai-Lang-1996} density\cite{Hai-Lang-1996} and relative
permittivity,\cite{Gavish-2016} also significantly affect the ion and water flux. To better compute the charge
flux in ion channels, Chen derived a new PNP framework that includes water-ion interactions in the form of a
concentration-dependent relative permittivity and an additional ion-water interaction energy
term.\cite{Chen-2016}
% Axelsson \etal{} derived a
% set of NS equations that allowed for an incompressible fluid with a variable density and
% viscosity.\cite{Axelsson-2015}

To the best of our knowledge, no attempt has been made to consolidate the corrections discussed above into a
single framework. Hence, we propose an extended set of PNP-NS (ePNP-NS) equations, which improves the
predictive power of the PNP-NS equations at the nanoscale and beyond infinite dilution. Our ePNP-NS framework
takes into account the finite size of the ions using a size-modified PNP theory,\cite{Lu-2011} and implements
spatial-dependencies for the solvent viscosity,\cite{Pronk-2014,Hsu-2017} the ion diffusion coefficients and
their mobilities.\cite{Makarov-1998,Noskov-2004} It also includes self-consistent concentration-dependent
properties, based on empirical fits to experimental data, for both the ions, in terms of diffusion
coefficients and mobilities,\cite{Baldessari-2008-1,Mills-1989} and the solvent, in terms of density,
viscosity\cite{Hai-Lang-1996} and relative permittivity\cite{Gavish-2016}. To validate our new framework, we
applied it directly to a 2D-axisymmetric model of Cytolysin A (ClyA), a large protein nanopore that typically
contains 12 subunits\cite{Mueller-2009} or more\cite{Soskine-2013} and has been extensively used in
experimental studies of both proteins\cite{Soskine-2013,VanMeervelt-2014,Soskine-Biesemans-2015,
Biesemans-Soskine-2015,Wloka-2017,VanMeervelt-2017,Galenkamp-2018} and DNA.\cite{Franceschini-2013,
Franceschini-2016}. This allowed us to gauge the qualitative and quantitative performance of the ePNP-NS
equations and simultaneously elucidate previously unaccessible details about the environment inside the pore.

The remainder of this paper is organized as follows. In \emph{\nameref{sec:model}} we describe the equations
governing our ePNP-NS framework and detail the construction of the 2D-axisymmetric ClyA model. Next, in
\emph{\nameref{sec:results}}, we validate our model by direct comparison of simulated ionic conductance with
experimentally measured values. We then proceed to characterize the influence of the bulk ionic strength and
applied bias voltage on cation and anion concentrations inside the pore, the electrostatic potential
distribution and magnitude of the electroosmotic flow. Finally, we touch upon our key finding and their impact
in \emph{\nameref{sec:conclusions}} and describe our protocols in more detail in \emph{\nameref{sec:methods}}.


\begin{figure*}[!p]

	\centering
	\begin{minipage}[t]{5cm}
		\begin{subfigure}[t]{5cm}
			\centering
			\caption{}\label{fig:clya_side}
      \vspace{-5mm}
			\includegraphics[scale=1]{figures/concept/clya_side}
		\end{subfigure}
   	\begin{subfigure}[t]{5cm}
      \centering
      \caption{}\label{fig:clya_top}
      \vspace{-5mm}
      \includegraphics[scale=1]{figures/concept/clya_top}
    \end{subfigure}
	\end{minipage}
  \hspace{-0.25cm}
  \begin{minipage}[t]{11.5cm}
    \begin{minipage}[t]{5.5cm}
      \begin{minipage}[t]{5.5cm}
        \begin{subfigure}[t]{1.6cm}
          \centering
          \caption{}\label{fig:model_geometry_vs_wedge}
          \vspace{-3mm}
          \includegraphics[scale=1]{figures/concept/model_geometry_vs_wedge}
        \end{subfigure}
        \begin{subfigure}[t]{2.5cm}
          \centering
          \caption{}\label{fig:model_charge_density}
          \vspace{-3mm}
          \includegraphics[scale=1]{figures/concept/model_charge_density}
        \end{subfigure}
      \end{minipage}
      \begin{subfigure}[t]{5.5cm}
        \centering
        \vspace{0.5cm}
        \caption{}\label{fig:model_geometry_zoom}
        \vspace{-1cm}
        \includegraphics[scale=1]{figures/concept/model_geometry_zoom}
        %\vspace{0.25cm}
      \end{subfigure}
    \end{minipage}
    \hspace{-0.8cm}
    \begin{subfigure}[t]{5.5cm}
      \centering
      \caption{}\label{fig:model_geometry}
      \includegraphics[scale=1]{figures/concept/model_geometry}
    \end{subfigure}
  \end{minipage}

  \caption%
    [\textbf{All-atom and 2D-axisymmetric models of ClyA.}]
    {%
      \textbf{All-atom and 2D-axisymmetric models of ClyA.}
      (\subref{fig:clya_side})
      %
      Axial cross-sectional and (\subref{fig:clya_top}) top views of the dodecameric nanopore
      ClyA-AS\cite{Soskine-2013}, derived through homology modelling from the \textit{E. coli} Cytolysin A
      crystal structure (PDBID: 2WCD\cite{Mueller-2009}). Figures were rendered with
      VMD.\cite{Humphrey-1996,Stone-1998}
      %
      (\subref{fig:model_geometry_vs_wedge})
      %
      The 2D-axisymmetric geometry was derived directly from the all-atom model by computing the average inner
      and outer radii along the longitudinal axis of the pore, and hence closely follows the outline of a
      \ang{30} wedge out of the homology model.
      %
      (\subref{fig:model_charge_density})
      %
      The fixed space charge density ($\scdpore$) map of ClyA-AS, obtained by Gaussian projection of each
      atom's partial charge onto a 2D plane (see methods for details).
      %
      (\subref{fig:model_geometry_zoom}+\subref{fig:model_geometry})
      %
      The 2D-axisymmetric simulation geometry of ClyA (grey) embedded in a lipid bilayer (green) and
      surrounded by a spherical water reservoir (blue). Note that all electrolyte parameters depend on the
      local average ion concentration $\avionconc=\frac{1}{n}\sum_{i}^{n}\concentration_{i}$ and that some are
      also influenced by the distance from the nanopore wall $\walldistance$.
    %
    }\label{fig:model_concept}
\end{figure*}


\section{Mathematical model}\label{sec:model}

The use of \emph{continuum} or \emph{mean-field} representations for both the nanopore and the electrolyte
enables us to efficiently compute the steady-state ion and water fluxes under almost any condition. The
dynamic behavior of our complete system is described by the coupled Poisson, Nernst-Planck and Navier-Stokes
(PNP-NS) equations, a set of well-known partial differential equations that describe the electrostatic field,
the total ionic flux and the fluid flow, respectively.\cite{Eisenberg-1996,Cervera-2005,Lu-2012} In this
section we will describe all the components of the simulation, \ie~the 2D-axisymmetric nanopore geometry and
the system of governing equations.

\subsection{Model geometry}\label{sec:geom}

\paragraph{2D-axisymmetric model of ClyA}
%
ClyA is a relatively large protein nanopore that self-assembles on lipid bilayers to form \SI{14}{\nm} long
hydrophilic channels. The interior of the pore can be divided into roughly two cylindrical compartments
(\cref{fig:clya_side}): the \cis{} lumen (\SI{\approx6}{\nm} diameter, \SI{\approx10}{\nm} height), and the
\trans{} constriction (\SI{\approx3.3}{\nm} diameter, \SI{\approx4}{\nm} height). Because ClyA consists of 12
identical subunits (\cref{fig:clya_top}), it exhibits a high degree of radial symmetry, a geometrical feature
that can be exploited to obtain meaningful results at a much lower computational
cost.\cite{Cervera-2005,Lu-2012, Pederson-2015} However, this requires the reduction of the full 3D atomic
structure and charge distribution into a realistic 2D-axisymmetric model. To this end, we constructed a
full-atom homology model of ClyA-AS type I---a dodecameric variant of the wild type ClyA from \textit{S.
Typhii} artificially evolved for improved stability\cite{Soskine-2013}---and equilibrated it at
\SI{298.15}{\kelvin} for \SI{30}{\ns} in an explicit solvent with harmonic contraints on the protein backbone
atoms (see~\cref{sec:methods} for details). From the final \SI{5}{\ns} of this trajectory, we extracted 50
sets of atomic coordinates (\ie~every \SI{100}{\ps}) for ClyA. For each of these structures, we computed a
2D atomic density\cite{Li-2013} and charge\cite{Aksimentiev-2005} map (see~\cref{sec:methods} for details).
Both the density and the charge maps were averaged to capture the movements of the side chains. The geometry
of the nanopore was then defined as the \SI{25}{\percent} contour line of the density map, which closely
follows the outline of a \SI{30}{\degree} `wedge' of the full atom structure
(\cref{fig:model_geometry_vs_wedge}). The equilibrium charge map (\cref{fig:model_charge_density}) was loaded
directly into our solver as an interpolated function ($\scdpore$) and applied to the entire computational
domain.

\paragraph{Global geometry.}
%
The complete system (\cref{fig:model_geometry_zoom,fig:model_geometry}) consists of a large hemispherical
electrolyte reservoir ($R=\SI{250}{\nm}$), split through the middle into a \cis{} and a \trans{} compartment
by a lipid bilayer ($h=\SI{2.8}{\nm}$), which contains the nanopore at its center. Both the bilayer and the
nanopore are represented by dielectric blocks (see~\cref{tab:corrections_equations} for parameters) that are
impermeable to ions and water.


\subsection{Governing equations}\label{sec:goveq}

In an attempt to improve upon the quantitative accuracy of the PNP-NS equations for nanopore simulations, we
developed an \emph{extended} version of these equations (ePNP-NS) using the commercial finite element solver
COMSOL Multiphysics (v5.4, www.comsol.com) that self-consistently takes into account 1) the finite size of the
ions,\cite{Borukhov-1997,Lu-2011} 2) the reduction of ion and water motility close to the nanopore
walls,\cite{Makarov-1998,Noskov-2004,Pronk-2014,Pederson-2015,Vo-2016} and 3) the concentration dependency of
ion diffusion coefficients and electrophoretic mobilities, as well as electrolyte viscosity, density and
relative permittivity.\cite{Mills-1989,Hai-Lang-1996,Gavish-2016} Most of these corrections were implemented
using empirical functions fitted to experimental data, which can be found in \cref{tab:corrections_equations}
and \cref{suppinfo:tab:corrections_parameters}, respectively.

\paragraph{Electrostatic field.}
% 
As mentioned above, the electrostatic potential is evaluated using Poisson's equation
%
\begin{align}
  \label{eq:poisson}
  \nabla \cdot \left(\absperm \relperm \nabla \potential \right) = -\left( \scdpore + \scdion \right)
  \text{\,,}
\end{align}
%
with electric potential $\potential$, vacuum permittivity $\absperm=\SI{8.85419e-12}{\farad\per\meter}$ and
local relative permittivity $\relperm$.

The pore's \emph{fixed} charge distribution $\scdpore$ (see~\cref{eq:scdpore}) was derived directly from the
full atom model of ClyA-AS and the \emph{ionic} charge density in the fluid is given by
%
\begin{align}\label{eq:scdion}
  \scdion = \faraday\sum_{i}\chargen_{i}\concentration_{i}
  \text{\,,}
\end{align}
%
with Faraday's constant $\faraday= \SI{96485.33}{\coulomb\per\mole}$, ion concentration $\concentration_{i}$
and ion charge number $\chargen_{i}$ ($+1$ and $-1$ for \Na{} and \Cl{}, respectively).

To account for the concentration dependence of the electrolyte's relative permittivity, we replaced its
constant value with the following expression
%
\begin{align}
  \relperm(\avionconc) = \relperm^0 \relperm^c(\avionconc)
  \text{\,,}
\end{align}
%
with $\avionconc=\frac{1}{n}\sum_{i}^{n}\concentration_{i}$ the average ion concentration, $\relperm^0$ the
relative permittivity at infinite dilution and $\relperm^c(\avionconc)$ a concentration dependent empirical
function parameterized with experimental data.

%
\begin{table*}[p]
  \footnotesize
  \renewcommand{\arraystretch}{1.2}
  \caption{Summary of the parameters and fitting equations used in the ePNP-NS equations.}
  \centering
  \label{tab:corrections_equations}
  
  \begin{tabularx}{16.5cm}{>{\raggedright\hsize=3cm}X >{\hsize=1cm}l >{\hsize=6.5cm}X >{\hsize=2cm}l}
    \toprule
  
    Property
      & Parameter$^\text{\emph{a}}$
        & Infinite dilution value$^\text{\emph{b}}$/Fitting function$^\text{\emph{c}}$
        & Reference \\
  
    \midrule
  
    \multirow{4}{*}{Relative permittivity}
      & $\permittivity_{r,\text{p}}$
        & \num{20}
        & \citenum{Li-2013} \\
      & $\permittivity_{r,\text{m}}$
        & \num{3.2}
        & \citenum{Gramse-2013} \\
      & $\permittivity_{r,\text{f}}^0$
        & \num{78.15}
        & \citenum{Gavish-2016} \\
      & $\permittivity_{w}(\dconc)$
        & $1 - \left(1 -	\dfrac{P_1}{P_0}\right) L \left( \dfrac{3P_2}{P_0 - P_1} \dconc \right)$
        & \citenum{Gavish-2016} \\
    \multirow{4}{3cm}{Ion self-diffusion coefficient}
      & $\diffusion_{\Na}^0$
        & \SI{1.334e-9}{\square\meter\per\second}
        & \citenum{Mills-1989} \\
      & $\diffusion_{\Cl}^0$
        & \SI{2.032e-9}{\square\meter\per\second}
        & \citenum{Mills-1989} \\
      & $\diffusion_{i}^c(\dconc)$
        & $\left( 1 + P_1\dconc^{0.5} + P_2\dconc + P_3\dconc^{1.5} + P_4\dconc^2 \right)^{-1}$
        & This work \\
      & $\diffusion_{i}^w(\dwall)$
        & $1-\exp{\left(-P_1(\dwall+P_2)\right)}$
        & \citenum{Makarov-1998,Simakov-2010} \vspace{0.25cm} \\
  
    \multirow{4}{3cm}{Ion electrophoretic mobility}
      & $\mobility_{\Na}^0$
        & \SI{5.192e-4}{\square\meter\per\second\per\volt}
        & \citenum{Bianchi-1989} \\
      & $\mobility_{\Cl}^0$
        & \SI{7.909e-4}{\square\meter\per\second\per\volt}
        & \citenum{Bianchi-1989} \\
      & $\mobility_{i}^c(\dconc)$
        & $\left( 1 + P_1\dconc^{0.5} + P_2\dconc + P_3\dconc^{1.5} + P_4\dconc^2 \right)^{-1}$
        & This work \\
      & $\mobility_{i}^w(\dwall)$
        & $1-\exp{\left(-P_1(\dwall+P_2)\right)}$
        & \citenum{Makarov-1998,Simakov-2010} \vspace{0.25cm} \\
  
    \multirow{2}{3cm}{Ion transport number}
      & $\transportn_{\Na}^0$
        & 0.396
        & \citenum{Bianchi-1989} \\
      & $\transportn_{\Na}(\dconc)$
        & $\left( 1 + P_1\dconc^{0.5} + P_2\dconc + P_3\dconc^{1.5} + P_4\dconc^2 \right)^{-1}$
        & This work \vspace{0.25cm} \\
  
    \multirow{3}{*}{Dynamic viscosity}
      & $\viscosity^0$
        & \SI{8.904}{\pascal\second}
        & \citenum{Hai-Lang-1996} \\
      & $\viscosity^c(\dconc)$
        & $ 1 + P_1 \dconc^{0.5} + P_2 \dconc + P_3 \dconc^2 + P_4 \dconc^{3.5}$
        & This work \\
      & $\viscosity^w(\dwall)$
        & $1 + \exp{ \left( -P_1(\dwall-P_2) \right) }$
        & \citenum{Pronk-2014} \vspace{0.25cm} \\
  
    \multirow{2}{*}{Fluid density}
      & $\density^0$
        & \SI{997}{\kilogram\per\cubic\meter}
        & \citenum{Hai-Lang-1996} \\
      & $\density(\dconc)$
        & $1 + P_1 \dconc + P_2 \dconc^2$
        & This work \vspace{0.25cm} \\
    
    \bottomrule
  \end{tabularx}
  \begin{flushleft}
    $^\text{\emph{a}}$Dependencies on either $\dconc = \avionconc/1$~M (dimensionless average ion
    concentration) and $\dwall = \walldistance/1$~nm (dimensionless distance from the nanopore wall);
    $^\text{\emph{b}}$Values at infinite dilution for a system temperature of \SI{291.15}{\kelvin};
    $^\text{\emph{c}}$These functions are empirical and hence have no physical meaning.
    The values of the fitting parameters $P_x$ of each property can be found in \cref{suppinfo:tab:corrections_parameters} and graphs of the fits in \cref{suppinfo:fig:corrections}.
  \end{flushleft}
\end{table*}
%

\paragraph{Ionic flux.}
%
The total ionic flux $\flux_{i}$ of each ion $i$ is given by the size-modified Nernst-Planck
equation,\cite{Lu-2011} and can be expressed as the sum of diffusive, electrophoretic, convective and steric
fluxes
%
\begin{align}
  \label{eq:sm-nernst-planck}
  \flux_{i} = -\left[
    \diffusion_{i} \nabla \concentration_{i}
    + \chargen_{i} \mobility_{i} \concentration_{i} \nabla \potential
    - \velocity \concentration_{i}
    + \vec{\beta_{i}} \concentration_{i} \right]
  \text{\,,}
\end{align}
%
where
%
\begin{align}
  \vec{\beta_{i}} =
      \frac{ \ionsize_{i}^3 / \ionsize_{0}^3 \dsum_{j} \ionsize_{j}^3 \nabla \concentration_{j} }
          { 1 - \dsum_{j} \avogadro \ionsize_{j}^3 \concentration_{j} }
  \text{\,,}
\end{align}
%
and at steady state
%
\begin{align}
  \dfrac{\delta \concentration_{i}}{\delta t} ={}& - \nabla \cdot \flux_{i} = 0
  \text{\,,}
\end{align}
%
with ion diffusion coefficient $\diffusion_{i}$, concentration $\concentration_{i}$, charge number
$\chargen_{i}$, electrophoretic mobility $\mobility_{i}$, electrostatic potential $\potential$, fluid velocity
$\velocity$ and Avogadro's constant $\avogadro = \SI{6.022e23}{\per\mole}$. $\ionsize_{i}$ and $\ionsize_{0}$
are \emph{steric} cubic diameters of respectively ions and water molecules. Because currently there are no
experimentally verified values available for $\ionsize_{i}$ and $\ionsize_{0}$, we set them to \SI{0.5}{\nm}
and \SI{0.311}{\nm}, such that their limiting concentrations correspond to \SI{13.3}{\Molar} and
\SI{55.2}{\Molar}, respectively.\cite{Bazant-2009}

The reduction of the ionic motility with increasing salt concentration and in proximity to the nanopore walls
was implemented self-consistently by replacing the constant values of $\diffusion_{i}$ and $\mobility_{i}$
with the following expressions
%
\begin{align}
  \diffusion_{i}(\avionconc,\walldistance) ={}&
      \diffusion_{i}^0 \diffusion_{i}^c(\avionconc) \diffusion_{i}^w(\walldistance)  \\
  \mobility_{i}(\avionconc,\walldistance) ={}&
      \mobility_{i}^0 \mobility_{i}^c(\avionconc) \mobility_{i}^w(\walldistance)
\end{align}
%
where $\diffusion_{i}^0$ and $\mobility_{i}^0$ represent the values at infinite dilution. The concentration
dependent factors $\diffusion_{i}^c(\avionconc)$ and $\mobility_{i}^c(\avionconc)$ are empirical functions
fitted to experimental data of respectively the ion self-diffusion coefficients\cite{Mills-1989} and the
electrophoretic mobilities\cite{Bianchi-1989,Currie-1960,Goldsack-1976,DellaMonica-1979} for solutions
between \SIrange{0}{5}{\Molar} \ce{NaCl}. Likewise, the factors $\diffusion_{i}^w(\walldistance)$ and
$\mobility_{i}^w(\walldistance)$ are empirical functions that introduce a spatial dependency on the distance
from the nanopore wall $\walldistance$, and were parameterized by fitting to molecular dynamics
data.\cite{Noskov-2004,Simakov-2010,Makarov-1998}

Based on the observation that the diffusivity of nanometer- to micrometer-sized particles reduces
significantly when confined in pores and slits of comparable dimensions,\cite{Renkin-1954,Deen-1987,
Dechadilok-2006,Muthukumar-2014,Kannam-2017} Simakov \etal{}\cite{Simakov-2010} and Pederson
\etal{}\cite{Pederson-2015} reduced the ion motilities inside the pore as a function of the ratio between the
ion and the nanopore radii. We chose not to include this correction into our model, as extrapolating its
applicability for ions with a hydrodynamic radii comparable to size of the solvent molecules is
questionable.\cite{Anderson-1972,Deen-1987}

\paragraph{Fluid flow.}
%
The fluid flow and pressure are given by the Navier-Stokes equations for incompressible
fluids\cite{Axelsson-2015}
%
\begin{align}
  \label{eq:navier-stokes-variable}
  \left( \velocity \cdot \nabla \right) \left( \density\velocity \right)
  + \nabla \cdot \hydrostresstensor = \vec{\force}
  \text{\,,}
\end{align}
%
where
%
\begin{align}
  \hydrostresstensor =
  \pressure\identity - \viscosity\left[\nabla\velocity+\left(\nabla\velocity \right)^\mathsf{T}\right]
  \text{\,,}
\end{align}
%
together with the continuity equations for the fluid density
%
\begin{align}
  \velocity \cdot \nabla \density  = 0
  \text{\,,}
\end{align}
%
and the fluid velocity
%
\begin{align}
  \left( \velocity \cdot \nabla \right) \left( \density\velocity \right)
  \nabla \cdot \left( \density\velocity \right) - \velocity \cdot \nabla \density ={}& 0
  \text{\,,}
\end{align}
%
with fluid velocity $\velocity$, density $\density$, hydrodynamic stress tensor $\hydrostresstensor$,
viscosity $\viscosity$ and pressure $\pressure$. The external body force density $\vec{\force}$ that acts on
the fluid is given by
%
\begin{align}
\vec{\force} = \echarge\avogadro\scdion\efield
\text{\,,}
\end{align}
%
with $\efield = - \nabla \potential$ the electrical field vector.

As with the previous equations, we introduced a concentration dependency and wall distance dependencies for
$\viscosity$ and a concentration dependency for the $\density$ by replacing their constant values by
%
\begin{align}
  \viscosity(\avionconc,\walldistance) ={}&
    \viscosity^0 \viscosity^c(\avionconc) \viscosity^w(\walldistance) \\
  \density(\avionconc) ={}&
    \density^0 \density^c(\avionconc)
\end{align}
%
where $\viscosity^0$ and $\density^0$ are the values at infinite dilution (\ie~pure water). The empirical
functions $\viscosity^c(\avionconc)$, $\density^c(\avionconc)$ and $\viscosity^w(\walldistance)$ were
parameterized \textit{via} fitting to experimental\cite{Hai-Lang-1996} and molecular dynamics\cite{Pronk-2014}
data obtained from literature.

\subsection{Boundary conditions}
%
The reservoir boundaries were used to mimic electrodes by grounding the \cis{} side ($\potential = 0$) and
setting a fixed bias voltage at \trans{} ($\potential = \vbias$) as Dirichlet boundary conditions. To mimic an
endless reservoir, the ion concentration at both external boundaries was set to a fixed value
($\concentration_i = \cbulk$) and a no normal stress condition was used ($\hydrostresstensor\vec{n} = 0$) to
allow for the unconstrained flow in and out of the computational domain. The boundary conditions on the edges
of the reservoir shared with the nanopore and bilayer were set to no-flux ($-\vec{n}\cdot\vec{\flux}_{i} = 0$)
and no-slip ($\vec{\velocity} = 0$) to prevent the flux of ions through them and to mimic a sticky hydrophilic
surface, respectively. Finally, a Neumann boundary condition was applied at the bilayer external boundary
($-\vec{n}\cdot\left(\absperm \relperm \nabla \potential \right) = 0$).

\subsection{Note on the concentration dependencies.}
%
All concentration dependent parameters use the local ionic strength rather than their individual ion
concentrations. The main reasons for nevertheless making this simplification are the lack of non-bulk
experimental data and the absence of a tractable analytical model. Though valid for \emph{electroneutral} bulk
solutions, this approximation no longer holds inside the electrical double layer (\ie~near charged surfaces or
inside small nanopores), where local electroneutrality is violated. Nevertheless, we will see that the current
concentration dependent functions will lead to an excellent agreement with the experimental data in all but
the most extreme cases, justifying our choice \textit{a posteriori}. 

\subsection{PNP-NS vs. ePNP-NS}
%
The ePNP-NS equations revert into the regular PNP-NS equations disabling the steric flux
(\ie~$\vec{\beta}=0$) and by setting all concentration and wall distance functions to unity ($\relperm =
\relperm^0$, $\diffusion_{i} = \diffusion_{i}^0$, $\mobility_{i} = \mobility_{i}^0$, $\viscosity =
\viscosity^0$ and $\density = \density^0$).

\section{Results and discussion}\label{sec:results}

The current--voltage (IV) relationships of many nanopores, ClyA included, often deviate significantly from
Ohm's law. This is because the ionic flux results from a complex interplay between the pore's geometry
(\eg~size, shape and charge distribution), the properties of the surrounding electrolyte (\eg~salt
concentration viscosity and relative permittivity) and the externally applied conditions (\eg~bias voltage,
temperature and pressure). The ability of a computational model to quantitatively predict the ionic current of
a nanopore over a wide range of bias voltages and salt concentrations strongly indicates that it captures the
essential physics governing the nanofluidic transport. Hence, to validate our model, we experimentally
measured the single channel ionic conductance of ClyA at a wide range of experimentally relevant salt
concentrations and bias voltages and compared them with the simulated ionic transport properties in terms of
current, conductance, rectification and ion selectivity, of both the classical PNP-NS and the newly developed
ePNP-NS equations. After validation, we proceed with describing the influence of bias voltage and bulk ionic
strengths on the local ion concentrations inside the pore and the resulting electrical double layer. This is
followed by a characterization of the electrostatic potential and the electrostatic energy landscape within
ClyA for both cations and anions. We will conclude this section by discussing the properties of the
electroosmotic flow and a brief reflection of the key findings in this work.


\begin{figure*}[!p]
  \centering
  \begin{minipage}[l]{16cm}
    \begin{minipage}[t]{5cm}
      \begin{subfigure}[t]{4.5cm}
        \centering
        \caption{}\vspace{-3mm}\label{fig:current-voltage_curves}
        \includegraphics[scale=1]{figures/conductance/current_vs_voltage_all_multiplot}
      \end{subfigure}
    \end{minipage}
    %
    \begin{minipage}[t]{5cm}
      \begin{subfigure}[t]{5cm}
        \centering
        \caption{}\vspace{0mm}\label{fig:conductance_contourmap_epnp}
        \includegraphics[scale=1]{figures/conductance/conductance_contourmap_epnp}
      \end{subfigure}
      \\
      \begin{subfigure}[t]{5cm}
        \centering
        \caption{}\vspace{0mm}\label{fig:conductance_loglog_exp_pnp_epnp}
        \includegraphics[scale=1]{figures/conductance/conductance_loglog_exp_pnp_epnp}
      \end{subfigure}
    \end{minipage}
    %
    \begin{minipage}[t]{5cm}
      \begin{subfigure}[t]{5cm}
        \centering
        \caption{}\vspace{0mm}\label{fig:transport_number_contourmap_epnp}
        \includegraphics[scale=1]{figures/conductance/transport_number_contourmap_epnp}
      \end{subfigure}
      \\
      \begin{subfigure}[t]{5cm}
        \centering
        \caption{}\vspace{0mm}\label{fig:transport_number_vs_concentration_bulk_pnp_epnp}
        \includegraphics[scale=1]{figures/conductance/transport_number_vs_concentration_bulk_pnp_epnp}
      \end{subfigure}
    \end{minipage}
  \end{minipage}

  \caption%
  %
  [\textbf{Measured and simulated ionic conductance and cation selectivity of single ClyA nanopores.}]
  %
  {%
    %
    \textbf{Measured and simulated ionic conductance and cation selectivity of single ClyA nanopores.}
    %
    (\subref{fig:current-voltage_curves})
    %
    Comparison of the simulated (PNP-NS and ePNP-NS) and experimentally (expt.) measured current-voltage (IV)
    curves of ClyA-AS at \SI{25\pm1}{\dC} between $\vbias=\text{\SIrange{-200}{+200}{\mV}}$, and for
    $\cbulk=\text{\SIlist{0.05;0.15;0.5;1;3}{\Molar}}$ \ce{NaCl}. Experimental errors ($n=3$) were smaller
    than the symbol size and hence not shown.
    %
    (\subref{fig:conductance_contourmap_epnp})
    %
    Contour plot of the simulated (ePNP-NS) ionic conductance $\conductance = \current /\vbias$ as a function
    of $\vbias$ and $\cbulk$.
    %
    (\subref{fig:conductance_loglog_exp_pnp_epnp})
    %
    Log-log plots of $\cbulk$ as a function of $\cbulk$ at \SI{+150}{\mV} (top) and \SI{-150}{\mV}
    (bottom)---comparing experimental (expt.), simulated (PNP-NS and ePNP-NS), and bulk pore conductances. The
    latter was calculated by modelling the nanopore as two ionic resistors in series
    (\cref{suppinfo:eq:bulk_nanopore_conductance}).\cite{Soskine-2013,Kowalczyk-2011}
    %
    (\subref{fig:transport_number_contourmap_epnp})
    %
    Contour plot of the \Na{} transport number $\tna = \gna / \conductance$, computed from the individual
    ionic conductances in the ePNP-NS simulation, as a function of $\vbias$ and $\cbulk$. $\tna$ expresses the
    fraction of the ionic current is carried by \Na{} ions, i.e. the cation selectivity.
    %
    (\subref{fig:transport_number_vs_concentration_bulk_pnp_epnp})
    %
    Simulated (PNP-NS and ePNP-NS) values of $\tna$ as a function of $\cbulk$ for \SI{+150}{\mV} (top) and
    \SI{-150}{\mV} (bottom). Here, the `bulk' line indicates the bulk \ce{NaCl} cation transport number,
    represented by its empirical function $\tna(\cbulk)$ (see \cref{tab:corrections_equations} and
    \cref{suppinfo:tab:corrections_parameters}). The solid grey line represents $\tna = 0.5$.
    %
  }\label{fig:conductance}
\end{figure*}


\subsection{Transport of ions through ClyA}\label{sec:iont}

\paragraph{Ionic current and conductance.}
%
The ability of our model to reproduce the ionic current of a biological nanopore over a wide range of
experimentally relevant conditions (between \SIrange{-150}{+150}{\mV} and for
\SIlist{0.05;0.15;0.5;1;3}{\Molar} \ce{NaCl}) can be seen in \cref{fig:current-voltage_curves}. In this figure
we compare IV relationships of ClyA-AS as measured experimentally (`expt.'), simulated using our
2D-axisymmetric model (`PNP-NS' and `ePNP-NS') and naively estimated using an analytic
model\cite{Soskine-2013,Kowalczyk-2011} based on the bulk electrolyte conductivity and the simplified geometry
of the pore (`bulk', \cref{suppinfo:eq:bulk_nanopore_conductance}). Whereas the classical PNP-NS equations
consistently overestimated the ionic current, particularly at high salt concentrations, the predictions of the
ePNP-NS equations corresponded closely to the measured values, \emph{especially} at high ionic strengths
(\SI{>0.5}{\Molar}). The failure of the classical PNP-NS equations to correctly estimate the current is
expected however, as in this regime the model parameters (\eg~diffusivity, mobility and viscosity, ...) begin
to deviate significantly from their `infinite dilution' values (see~\cref{suppinfo:fig:corrections}). At lower
salt concentrations the ePNP-NS equations tended to minorly overestimate the ionic current, but the
discrepancies were much smaller than those observed for PNP-NS. Finally, the bulk model managed to capture the
currents suprisingly well at high salt concentrations and positive bias voltages, but faltered in the negative
voltage regime. 

The ability of a nanopore to conduct ions can be best expressed by its conductance: $\conductance = \current /
\vbias$. We computed ClyA's conductance with the ePNP-NS equations as a function of bias voltage
($\vbias=\text{\SIrange{-200}{+200}{\mV}}$) and bulk \ce{NaCl} concentration
($\cbulk=\text{\SIrange{0.005}{5}{\Molar}}$), of which a contour plot can be found in
\cref{fig:conductance_contourmap_epnp}. The near horizontal contour lines in the upper part of the plot
($\cbulk>\SI{1}{\Molar}$) show that, at high ionic strengths, ClyA maintains the same conductance regardless
of the applied bias voltage. This behavior changes at intermediate ionic strengths
($\SI{0.1}{\Molar}<\cbulk<\SI{1}{\Molar}$), where maintaining the same conductance level with increasing
negative bias amplitudes requires increasing salt concentrations. Finally, at low salt concentrations
($\cbulk<\SI{0.1}{\Molar}$), the ionic conductance increases when reducing the negative voltage amplitude but
subsequently reaches a plateau at positive bias voltages.

The cross-sections of the ionic conductance as a function of concentration at high positive and negative bias
voltages (\cref{fig:conductance_loglog_exp_pnp_epnp}), serve to demonstrate the differences between these
respective regimes. The slope of the log-log plot at \SI{+150}{\mV} is highly linear, suggesting that only a
single mechanism of ionic conduction is at work. As touched upon above, both the bulk model and the ePNP-NS
equations manage to capture the conductance at high ionic strengths, while only the latter performs well at
low ionic strengths. The predictions made with the PNP-NS equations overestimate the conductance over the
entire concentration range, but they do converge at very low ionic strenghts. At \SI{-150}{\mV}, the log-log
plot consists of two linear segments with a transition zone at $\cbulk \approx \text{\SI{0.15}{\Molar}}$. This
behavior is captured qualitatively by both simulation methods, but a perfect quantitative match is only found
for the ePNP-NS equations. The bulk model exhibits the same single-sloped trend as seen at \SI{+150}{\mV}.

The difference in ionic conduction at opposing bias voltages is also known as ionic current rectification
(ICR): $\icr(\vbias) = \conductance(+\vbias) / \conductance(-\vbias)$. ICR is a phenomenon often observed in
nanopores that are both charged and contain a degree of geometrical asymmetry along the central axis of the
pore.\cite{Constantin-2007,White-2008,Wang-2014} With its high negative charge (\SI{-72}{\ec} at pH~7.5) and
different \cis{} (\SI{\approx3.3}{\nm}) and \trans{} (\SI{\approx6}{\nm}) entry diameters
(\cref{fig:clya_side}), ClyA fulfills both conditions and hence exhibits a strong degree of rectification
(\cref{suppinfo:fig:icr}). We found $\icr$ to increase monotoneously with the bias voltage magnitude, at least
over the investigated range. The depedence of $\icr$ on the ionic strength was not monotonous, but rather
rises rapidly to a peak value at \SI{\approx0.15}{\Molar}, followed by a gradual decline towards unity at
saturating salt concentrations.

The results and comparisons discussed above indicate that ClyA's conductivity is dominated by the bulk
electrolyte conductivity above physiological salt concentrations ($\cbulk > \text{\SI{0.15}{\Molar}}$). The
breakdown of this simple dependency at lower ionic strengths is particularly evident at negative bias voltages
and is likely caused by the overlapping of the electrical double layer (EDL) inside the pore. This effectively
excludes the co-ions---\Cl{} in the case---from the interior of the pore, preventing them from contributing to
total ionic conductance and resulting in a conductance dominated by the surface charge.\cite{Uematsu-2018} The
presence of only a single ion type inside the pore may also offer an explanation as to why the ePNP-NS
equations falter at low ionic strengths. Because our ionic mobilities are derived from \emph{bulk} ionic
conductances, \ie~for unconfined ions in a locally \emph{electroneutral} environment, it is likely that our
mobility model begins to break down under these conditions.\cite{Duan-2010} Another cause of the discrepancies
could be a change in the shape or diameter of the nanopore at low salt concentrations, which cannot be
captured by our simulation due to the static nature of its geometry and charge distributions. 

Nevertheless, our simplified 2D-axisymmetric model, in conjunction with the ePNP-NS equations, is able to
accurately predict the ionic current flowing through ClyA for a wide range of experimentally relevant ionic
strengths and bias voltages. This suggests that we managed to capture the essential physical phenomena that
drive the ion and water transport through the nanopore both qualitatively and quantitatively. Hence, we expect
the distribution of the resulting properties (\eg~ion concentrations, ion fluxes, electric field and water
velocity) to closely match their real values.

\paragraph{Cation selectivity.}
%
The ion selectivity of a nanopore determines the preference with which it transports one ion type over the
other. Experimentally, it is often determined by placing the pore in a salt gradient (\ie~different salt
concentrations in the \cisi{} and \transi{} reservoirs) and measuring the potential at which the nanopore
current is zero (reversal potential, $\revpot$).\cite{Soskine-2013,Franceschini-2016} The Goldman-Hodkin-Katz
(GHK) equation can then be used to convert $\revpot$ into the permeability ratio $\pna = \gna / \gcl$. Here,
we represent the ClyA's ion selectivity by the fraction of the total current that is carried by \Na{} ions:
the apparent \Na{} transport number $\tna = \gna / (\gna + \gcl) = \pna / (\pna + 1)$
(\cref{fig:transport_number_contourmap_epnp,fig:transport_number_vs_concentration_bulk_pnp_epnp}). 

Due its negatively charged interior, we dound ClyA to be cation selective (\ie~$\tna > 0.5$) for all
investigated voltages up to a bulk salt concentration of $\cbulk \approx \text{\SI{2}{\Molar}}$ (0.5 contour
line in \cref{fig:transport_number_contourmap_epnp}). Above \SI{2}{\Molar} \ce{NaCl}, $\tna$ falls to a
minimum of value of 0.45 at $\cbulk \approx \text{\SI{5}{\Molar}}$, which is still \num{\approx1.27} times its
value in the bulk electrolyte (0.35) at that concentration. This indicates that---even at saturation---ClyA
remains preferential towards cations. Below \SI{2}{\Molar}, the ion selectivity increases logarithmically with
decreasing salt concentrations, but also becomes more sensitive to the direction and magnitude of the electric
field, with higher ion selectivities at negative bias voltages
(\cref{fig:transport_number_vs_concentration_bulk_pnp_epnp}). For example, to reach a selectivity of $\tna
\approx 0.9$, the salt concentration must fall to \SI{0.05}{\Molar} at \SI{+150}{\mV} and to only
\SI{0.125}{\Molar} at \SI{-150}{\mV}.

Using the reversal potential method, Franceschini \etal{}\cite{Franceschini-2016} found ClyA's ion selectivity
to be $\tna = 0.66$ ($\pna = 1.9$). This value corresponds the selectivity at \SI{0.5}{\Molar}, as computed
using our ePNP-NS simulation at $\vbias = \text{\SI{0}{\mV}}$, and lies in between the \cisi{}
(\SI{1}{\Molar}, $\tna = 0.57$, $\pna = 1.3$) and \transi{} (\SI{0.15}{\Molar}, $\tna = 0.84$, $\pna = 5.4$))
concentrations used in their experiment. Even though measuring the reversal potential can give valuable
insights into the selectivity ion channels and small nanopores, it should be used with caution for larger
nanopores such as ClyA. The GHK equation does not consider the ionic flux due to the electro-osmotic flow and
assumes that the Nernst-Einstein relation holds for all used concentrations. These two effects should be
ignored as they contribute significantly to the nanopore's total conductance. Furthermore, because the ion
selectivity depends strongly on the ionic strength, the measured reversal potential will necessarily be
influenced by the chosen gradient and represent the selectivity at an undetermined intermediate concentration.

% 0.15 M, 0 mV
% Tna = 0.84299
% Pna = 5.36904

% 1.0 M, 0 mV
% Tna = 0.57429
% Pna = 1.34899

% 0.05 M, +150 mV
% Tna = 0.90583
% Pna = 9.61941

% 0.125 M, -150 mV
% Tna = 0.90359
% Pna = 9.37234

\subsection{Ion Concentration Distribution}\label{sec:ionc}
%

\begin{figure}[!htb]
  \centering
  \begin{subfigure}[t]{8.2cm}
    \centering
    \caption{}\vspace{-3mm}\label{fig:concentration_pore_average_vs_concentration}
    \includegraphics[scale=1]{figures/concentration/concentration_pore_average_vs_concentration}
  \end{subfigure}
  \begin{minipage}[t]{8.2cm}
  \begin{subfigure}[t]{8.2cm}
    \centering
    \caption{}\vspace{-3mm}\label{fig:concentration_contours}
    \includegraphics[scale=1]{figures/concentration/concentration_contours}
  \end{subfigure}
  \begin{subfigure}[t]{8.2cm}
    \centering
    \caption{}\vspace{-3mm}\label{fig:concentration_radial_profiles}
    \includegraphics[scale=1]{figures/concentration/concentration_radial_profiles}
  \end{subfigure}
  \end{minipage}
  
  \caption%
  %
  [\textbf{Ion concentration distribution inside ClyA.}]
  %
  {%
    %
    \textbf{Ion concentration distribution inside ClyA.}
    %
    (\subref{fig:concentration_pore_average_vs_concentration})
    %
    Relative \Na{} and \Cl{} concentrations averaged over the entire pore volume ($\pav{\ci/\cbulk}{p}$) as a
    function of the reservoir salt concentration ($\cbulk=\text{\SIrange{0.005}{5}{\Molar}}$) and bias voltage
    ($\vbias=\text{\SIrange{-200}{+200}{\mV}}$).
    %
    (\subref{fig:concentration_contours})
    %
    Contour plots of the relative ion concentration ($\ci/\cbulk$) for both \Na{} and \Cl{} for
    $\cbulk=\SI{0.15}{\Molar}$ and at $\vbias=\text{\SIlist{-100;+100}{\mV}}$.
    %
    (\subref{fig:concentration_radial_profiles})
    %
    The relative \Na{} and \Cl{} concentration profiles along the radius of the pore, through the middle of
    the constriction ($z=\SI{-0.3}{\nm}$) and the lumen ($z=\SI{5}{\nm}$), as indicated by the arrows in
    (\subref{fig:concentration_contours}).
    %
  }\label{fig:concentration}
  \end{figure}

The ionic current results from the flux of ions through pore, whose magnitudes are proportional to the local
concentrations inside the pore. To further elucidate the origin of the current rectification and ion
selectivity, we evaluated the distribution of both cations and anions inside ClyA the effect of reservoir salt
concentration and the bias voltage on the distribution of cation and anion densities inside ClyA
(\cref{fig:concentration}) and the accumulation of mobile charges as a result of their asymmetry
(\cref{fig:ion_charge_density}).

\paragraph{Relative cation and anion concentrations.}
%
At high reservoir concentrations ($\cbulk > 1$~M), the lumen of the pore becomes fully screened from the
negative charges that line it, leading to bulk-like conditions ($\pav{\ci/\cbulk}{p}\approx1$) for both
positive and negative ions. At lower concentrations however, the reduced screening gives rise to an
overlapping electrical double layer, which results in the enhancement of cations and the depletion of anions.
The magnitude of these effects appears to follow a power law at low concentrations
($\cbulk=\SIrange{0.005}{5}{\Molar}$), with exponents of $-0.910\pm0.010$ and $0.778\pm0.015$ for \Na{} and
\Cl{}, respectively. The \Na{} concentrations exhibit at most a 2 to 3-fold increase when $\vbias$ changes
from $\vbias=-150$ to $+150$~mV. In contrast, the number of \Cl{} ions inside pore between those 2 bias
voltages can differ more than 10-fold at reservoir concentrations below $0.1$~M, with a high degree of
depletion at negative voltages.

The contour plots of the relative ion concentrations ($\ci/\cbulk$) at $\cbulk=0.15$~M
(\cref{fig:concentration_contours}) reveal that the \trans{} constriction ($1.85\text{~nm}<z<1.60\text{~nm}$)
remains depleted of anions and enhanced in cations for both $\vbias=-100$ and $+100$~mV. This is not the case
in the lumen ($1.60\text{~nm}<z<12.25\text{~nm}$), where the \Na{} concentration is bulk-like and enhanced for
$\vbias<0$ and $\vbias>0$, respectively. Conversely, the number of \Cl{} ions becomes more and more depleted
in the lumen for increasing negative bias magnitudes, and it is virtually bulk-like at higher positive bias
voltages. This is further exemplified by the radial profiles through the middle of the constriction
($z=\SI{-0.3}{\nm}$) and the lumen ($z=\SI{5}{\nm}$) (\cref{fig:concentration_radial_profiles}) which also
clearly show the formation of the electrical double layer.

Note that the figures given above were obtained from a nanoscale continuum steady-state simulation, and hence
represent a time-averaged situation (typically on the order of \SIrange{10}{100}{\ns}).\todo{REF}


\paragraph{Ion Charge Density.}
%

\begin{figure}[!htb]
  \centering
  \begin{minipage}[t]{8.2cm}
  \begin{subfigure}[t]{8.2cm}
    \centering
    \caption{}\vspace{-3mm}\label{fig:ion_charge_density_contours}
    \includegraphics[scale=1]{figures/charge_density/ion_charge_density_contours}
  \end{subfigure}
  \begin{subfigure}[t]{8.2cm}
    \centering
    \caption{}\vspace{-3mm}\label{fig:ion_charge_density_radial_profiles}
    \includegraphics[scale=1]{figures/charge_density/ion_charge_density_radial_profiles}
  \end{subfigure}
  \begin{subfigure}[t]{8.2cm}
    \centering
    \caption{}\vspace{-3mm}\label{fig:ion_charge_pore_bulk_surface_total_vs_concentration}
    \includegraphics[scale=1]{figures/charge_density/ion_charge_pore_bulk_surface_total_vs_concentration}
  \end{subfigure}
  \end{minipage}
  
  \caption%
  %
  [\textbf{Ion space charge density distribution inside ClyA.}]
  %
  {%
    %
    \textbf{Ion space charge density distribution inside ClyA.}
    %
    (\subref{fig:ion_charge_density_contours})
    %
    Cross-section contour plots of the ion space charge density ($\scdion$), expressed as number of elementary
    charges per \si{\cubic\nano\meter}, at \SI{0}{\mV} applied bias voltage and for salt concentrations
    \SIlist{0.005;0.05;0.5;5}{\Molar}.
    %
    (\subref{fig:ion_charge_density_radial_profiles})
    %
    Radial cross-sections of the $\scdion$ at the center of the constriction ($z=\SI{-0.3}{\nm}$) and the
    lumen ($z=\SI{5}{\nm}$) of ClyA. The vertical line represents the the division between ions in the `bulk'
    ($d>\SI{0.5}{\nm}$) of the pore and those located near its surface ($d\le\SI{0.5}{\nm}$).
    %
    (\subref{fig:ion_charge_pore_bulk_surface_total_vs_concentration})
    %
    The average number of ionic charges inside the pore $\pav{\Qion}{p}$, is distributed between the those
    close to the pore's surface $\pav{\Qion}{s}$, \ie~within \SI{0.5}{\nm} of the wall, and those in the
    `bulk' of the pore's interior $\pav{\Qion}{b}$.
    %
  }\label{fig:ion_charge_density}
  
\end{figure}  


The formation of an electrical double layer inside the pore and the resulting asymmetry in the cation and
anion concentrations results in a net charge density inside the pore ($\scdion$, \cref{eq:scdion}). At
reservoir concentrations \SI{\le0.5}{\Molar}, the electrical double layer (EDL) inside the pore is very
diffuse and overlaps significantly (\cref{fig:ion_charge_density_contours}, 3 leftmost panels). Moreover, the
absence of anions prevents the formation of any significant negative charge densities next to the few
positively charged residues lining the pore walls. The situation at high salt concentrations
(\eg~\SI{5}{\Molar}) is very different, with almost no charge density within the `bulk' of the pore lumen
($\walldistance\ge\SI{0.5}{\nm}$), but with pockets of highly charged and alternating pockets of positive and
negative charge densities close to the nanopore wall (\cref{fig:ion_charge_density_contours}, rightmost
panel). This sharp confinement is shown clearly by the radial density profiles
(\cref{fig:ion_charge_density_radial_profiles}) through the constriction ($z=-\SI{0.3}{\nm}$) and the lumen
($z=\SI{5}{\nm}$).

Integration of $\scdion$ over `bulk' ($d\ge\SI{0.5}{\nm}$) and surface ($d<\SI{0.5}{\nm}$) volumes inside the
pore yields respectively $\pav{\Qion}{b}$ and $\pav{\Qion}{s}$, \ie~the average number of mobile charges
present inside those locations (\cref{fig:ion_charge_pore_bulk_surface_total_vs_concentration}). Even though
the total number of charges inside the pore $\pav{\Qion}{p} = \pav{\Qion}{b} + \pav{\Qion}{s}$ rises
appreciatively with increasing reservoir concentration, the majority of these additional charges are confined
to the surface of the pore. Up until \SI{0.1}{\Molar}, $\pav{\Qion}{p}$ is distributed equally between the
surface and bulk layers (\SI{\approx+27}{\ec} and \SI{\approx+22}{\ec} , respectively). At higher
concentrations, the number of charges in the surface layer rises (to \SI{+58}{\ec} at \SI{5}{\Molar}), and
those in the bulk pore diminish (\SI{+0}{\ec}  at \SI{5}{\Molar}). $\pav{\Qion}{p}$ also depends on the
applied bias voltage, as it is \SIrange{+10}{+15}{\ec} higher at \SI{+150}{\mV} compared to \SI{-150}{\mV}.


\subsection{Equilibrium Electrostatic Potential}\label{sect:esp}

\begin{figure*}[!htb]
  \centering
  \begin{minipage}[t]{18.25cm}
    \begin{subfigure}[t]{2.5cm}
      \centering
      \caption{}\vspace{-3mm}\label{fig:potential_clya_charges}
      \includegraphics[scale=1]{figures/potential/potential_clya_charges}
    \end{subfigure}
    \hspace{-0.6cm}
    \begin{subfigure}[t]{11.5cm}
      \centering
      \caption{}\vspace{-3mm}\label{fig:potential_contours}
      \includegraphics[scale=1]{figures/potential/potential_contours_0mV}
    \end{subfigure}
    \hspace{-0.4cm}
    \begin{subfigure}[t]{4cm}
      \centering
      \caption{}\vspace{-3mm}\label{fig:potential_radial_averages}
      \includegraphics[scale=1]{figures/potential/potential_radial_averages_0mV}
    \end{subfigure}
  \end{minipage}
\centering

  \caption%
  %
  [\textbf{Electrostatic potential inside ClyA.}]
  %
  {%
    %
    \textbf{Electrostatic potential inside ClyA.}
    %
    (\subref{fig:potential_clya_charges})
    %
    A single subunit of ClyA in which all amino acids with a net charge and whose side chains face the inside
    of the pore, i.e. that contribute the most to the electrostatic potential, are highlighted. Negatively
    (Asp+Glu) and positively and positively (Lys+Arg) charged residues are colored in red and blue,
    respectively.
    %
    (\subref{fig:potential_contours})
    %
    Electrostatic potential landscape inside ClyA due to its fixed charges (i.e. at $\vbias=\SI{0}{\mV}$) at
    several key  concentrations ($\cbulk=\text{\SIlist{0.005;0.05;0.15;0.5;5}{\Molar}}$). Note that even at
    physiological salt  concentrations ($\cbulk=\SI{0.15}{\Molar}$), the negative electrostatic potential
    extends significantly inside the lumen ($1.60<z<\SI{12.25}{\nm}$), and even more so inside the \trans\
    constriction ($1.85<z<\SI{1.60}{\nm}$). For the former, localized influential negative `hotspots' can be
    found in the middle ($4<z<\SI{6}{\nm}$) and at the \cis\ entry ($10<z<\SI{12}{\nm}$).
    %
    (\subref{fig:potential_radial_averages})
    %
    Radial average of the electrostatic potential along the length of the pore ($\radpot$) for the same
    concentrations as in \subref{fig:potential_contours}. Even though the lumen of the pore becomes almost
    fully screened for $\cbulk>\SI{0.5}{\Molar}$, the constriction still retains some of its negative
    influence at \SI{5}{\Molar}.
    %
  }\label{fig:potential}
\end{figure*}


In nanometer sized pores, the modification of the electrostatic potential distribution by the charged residues
embedded in the interior walls of the protein (\cref{fig:potential_clya_charges}), significantly influences
the transport of ions and water molecules.\cite{Bhattacharya-2011}\todo{more ref} In the following section we
aim to describe the most salient features of this modulated potential and its relative importance over the
entire investigated concentration range.

\paragraph{Global electrostatic potential.}
%
The electrostatic potential distribution at $\vbias=\SI{0}{\mV}$ clearly shows the influence of the nanopore's
fixed charge distribution (\cref{fig:potential_contours}). At low ionic strengths ($\cbulk <
\SI{0.15}{\Molar}$), the lack of sufficient ionic screening results in relatively uniform negative potentials
in both the lumen and the constriction of the pore (\SI{-90}{\mV} and \SI{-53}{\mV} at respectively
\SI{0.005}{\Molar} and \SI{0.05}{\Molar}). These values significantly exceed the single ion thermal voltage
$\kTe=\SI{25.7}{\mV}$ and hence effectively prohibit anions from entering the pore. At intermediary
concentrations, ($0.15 \le \cbulk < \SI{1}{\Molar}$) the influence of the negative charges increasingly
confines itself to several `hotspots' near the nanopore walls, most notably at entry of the pore
($10<z<\SI{12}{\nm}$), in the middle of the lumen ($4<z<\SI{6}{\nm}$). Even though the potential at the center
of the pore is close to \SI{\approx0}{\mV} for the lumen, it remains uniformly negative for the constriction.
Finally, at high concentrations ($\cbulk \ge \SI{1}{\Molar}$) the potential drops to \SI{\approx0}{\mV} over
the entire lumen of the pore, with only a small negative potential inside the constriction.

\paragraph{Radially averaged potential profile.}
%
We further quantified the electrostatic potential at \SI{0}{\mV} applied bias voltage along the nanopore's
length by computing its radial average
%
\begin{align}
\radpot=\dfrac{1}{\pi R(z)^2}\int_{0}^{R(z)}\potential(r,z) \;2 \pi r \; dr \text{,}
\end{align}
%
where $R(z)$ is taken as the pore radius for $-1.85\le z \le \SI{12.25}{\nm}$, \SI{2}{\nm} for
$z<\SI{-1.85}{\nm}$ and \SI{4}{\nm} for $z>\SI{12.25}{\nm}$ (\cref{fig:potential_radial_averages}). The
electrostatic potential at the \cis{} entry ($z \approx \SI{10}{\nm}$) is dominated by the acidic residues
D114, D121 and D122, resulting in a rapid reduction of $\radpot$ upon entering the pore. Next, $\radpot$
remains approximately constant up until the middle of the lumen ($z \approx \SI{5}{\nm}$), where the next set
of negative residues, namely E53, E57 and D64, reduce the potential even further. After a brief rise,
$\radpot$ then attains its maximum amplitude inside the \trans{} constriction ($z \approx \SI{0}{\nm}$) due to
the close proximity of the amino acids E7, E11, E18, D21 and D25. For example, at physiological salt
concentrations ($0.15$~M), $\radpot$ has values of \SIlist{-19;-29;-57}{\mV}
(\numlist{-0.74;-1.1;-2.2}~$\kTe$) at the \cis{} entry, lumen center and \trans{} constriction, respectively.
Their magnitude increases approximately three-fold at \SI{0.005}{\Molar}, and drops to
\SI{\approx10}{\percent} at \SI{5}{\Molar}. A summary of these values can be found in
\cref{suppinfo:tab:radial_potential}.

\subsection{Non-equilibrium electrostatic potential}\label{sec:ese}

and to link it back to the observed ionic conductance properties through the average electrostatic energy of
cations and anions (\cref{fig:potential_energy}).

Due to it's small size, the majority of an applied bias potential drops along the length of the nanopore,
giving rise to a `tilted' version of the electrostatic potential at \SI{0}{\mV}. ClyA's geometric asymmetry
results in different electrostatic landscapes for positive and negative bias voltages. Since electromigration
is the primary contributor to the ionic current, a proper understanding of the electrostatic energy barriers
the ions must overcome when traversing the pore should provide a more quantitative explanation for the origin
of ClyA's rectification, ion selectivity and asymmetric ion concentrations. To this end, we computed the
radially averaged electrostatic energy for monovalent ions  $\radenergy = \chargen_{i} \echarge \radpot$
(\cref{fig:potential_energy_radial_averages}) and the magnitude of the energy barriers at the \trans{}
constriction ($\deltaEt$, \cref{fig:potential_energy_trans_barrier}).

%
\begin{figure}[!ht]
  \centering
  \begin{subfigure}[t]{8.25cm}
    \centering
    \caption{}\vspace{-5mm}\label{fig:potential_energy_radial_averages}
    \includegraphics[scale=1]{figures/potential_energy/potential_energy_radial_averages}
  \end{subfigure}
  \begin{subfigure}[t]{8.25cm}
    \centering
    \caption{}\vspace{-3mm}\label{fig:potential_energy_trans_barrier}
    \includegraphics[scale=1]{figures/potential_energy/potential_energy_trans_barrier}
  \end{subfigure}

  \caption%
  %
  [\textbf{Radially averaged electrostatic energy for single ions.}]
  %
  {%
    %
    \textbf{Radially averaged non-equilibrium electrostatic energy for single ions.}
    %
    (\subref{fig:potential_energy_radial_averages})
    %
    Approximate non-equilibrium electrostatic energy landscape for single ions $\radenergy = \chargen_{i}
    \echarge \radpot$ as calculated directly from the radial electrostatic potential at
    \SIlist{+150;-150}{\mV} applied bias voltages for monovalent cations and anions. The grey arrows indicate
    the direction in which the ions must travel in order  to contribute positively to the ionic current.
    %
    (\subref{fig:potential_energy_trans_barrier})
    %
    Height of the electrostatic energy barrier ($\Delta E_{\text{B},i}$) at the \trans\ constriction. Note
    that $\Delta E_{\text{B},i}$ is much higher for negative voltages and rises logarithmically at lower
    concentrations. The divergence between \SI{+0}{\mV} and \SI{-0}{\mV} for $\cbulk<\SI{0.3}{\Molar}$
    highlights the difference in barrier height when traversing the pore from \cis\ to \trans\ or vice versa.
    %
  }\label{fig:potential_energy}
\end{figure}
%

\paragraph{Energy landscape at $\mathbf{+150}$~mV.}
%
At positive bias voltages, cations traverse the pore from \trans{} to \cis{}
(\cref{fig:potential_energy_radial_averages}, top left). Upon entering the \trans{} constriction, the
electrostatic energy drops dramatically, followed by a relatively flat energy landscape with a small barrier
for entry in the lumen at $z\approx\SI{1.6}{\nm}$. At very low ionic strengths ($\cbulk<\SI{0.025}{\Molar}$),
the energy at \trans{} is significantly lower than that the final energy in the \cis{} compartment
(\eg~$\Delta\radenergy > \SI{2}{\kT}$ at \SI{0.005}{\Molar}), forcing the ions to accumulate inside the
pore. At higher concentrations ($\cbulk > \SI{0.05}{\Molar}$), the increased screening smooths out the
potential drop inside the pore, allowing the cations to migrate unhindered across the entire length of the
pore.

Anions traveling from \cis{} to \trans{} must overcome energy barriers when both entering and exiting the pore
(\cref{fig:potential_energy_radial_averages}, bottom right). The former prevents \Cl{} ions from entering the
pore, but is only relevant at lower ionic strengths ($\cbulk<0.05$~M), since its magnitude is attenuated
strongly with increasing salt concentration (from \SI{\approx1.7}{\kT} at \SI{0.005}{\Molar} to
\SI{\approx0.5}{\kT} at \SI{0.05}{\Molar}). Once inside the lumen, anions can move relatively unencumbered to
the \trans{} constriction, where they face the second, more significant energy barrier, which prevents them
from fully translocating and causes them to accumulate inside the lumen. Ions that do pass the final barrier
are rewarded with a large reduction of their electrostatic energy, as the bulk of the voltage drop occurs only
after the \trans{} constriction. As with the cations, an increase in the ionic strength significantly reduces
the

\paragraph{Energy landscape at $\mathbf{-150}$~mV.}
%
The electrostatic energy of cations traversing the pore at negative voltages (\cis{} to \trans{}) drops
gradually throughout the lumen of the pore up until the \trans{} constriction
(\cref{fig:potential_energy_radial_averages}, top right). This their efficient removal of cations from the
pore lumen, and results in a lower concentration compared to positive voltages (\cref{fig:concentration}). To
fully exit from the pore, however, cations must overcome a large energy barrier, which reduces the nanopore's
ability to conduct cations compared to positive potentials and hence contributes to the ion current
rectification.

The situation for anions at negative bias voltages (\trans{} to \cis{}) is very different as they must travel
from (\cref{fig:potential_energy_radial_averages}, bottom left), where they must overcoming the large energy
barrier at the constriction (\cref{fig:potential_energy_trans_barrier}, red curves). This effectively prevents
them from entering the pore, explaining why ClyA is more ion selective at negative bias voltages. Once across
the barrier, the continuous drop of electrostatic energy towards the \cis{} entry serves as a strong driving
force to deplete the entire lumen of anions, resulting in much lower concentrations compared to positive
voltages.

\paragraph{Concentration and voltage dependencies of the \trans{} energy barrier.}
%
Many biological nanopores contain constrictions that play crucial roles in shaping their ionic conductance
properties.\cite{Maglia-2008,Franceschini-2016,Huang-2017} The reason for this is two-fold, 1) the narrowest
part dominates the overall resistance of the pore and 2) confinement of charged residues results in much large
electrostatic energy barriers. With its highly negatively charged \trans{} constriction, ClyA's affinity for
transport of anions is diminished and that for cations in enhanced, even at high ionic strengths
(\cref{fig:transport_number_contourmap_epnp}).\cite{Soskine-2013} To further elucidate the significance of the
\trans{} electrostatic barrier ($\deltaEt$), we quantified its height of at positive and negative voltages as
a function of the salt concentration (\cref{fig:potential_energy_trans_barrier}).

We estimated the magnitude of the For concentrations between \SIlist{0.01}{1}{\Molar}, however, there is a
clear difference in ion selectivity between positive and negative bias voltages. At \SI{0.1}{\Molar}, for
example, the ion selectivity $\transportn_{\Na}$ increases $3.3$-fold from \num{5.7} at \SI{+150}{\mV} to
\num{19} at \SI{-150}{\mV}. The energy barriers under these conditions are This observation can be explained
by the difference in electrostatic energy barrier height in the constriction at positive and negative voltages
(\cref{fig:potential_energy_trans_barrier}). Again, at \SI{0.1}{\Molar} the


cation selectivities ($S_{i}=\transportn_{i}/(1-\transportn_{i})$).

%0.1M, -150mV: 1.8943409
%0.1M, +150mV: 0.90132562

% at concentrations below $0.1$~M
%($\transportn_{\Na}$.  The magnitude of th
%is  , contain a constriction with an electrostatic energy barrier that dominates the
%pore's ionic transport properties. At negative voltages,
%represents the electrostatic barrier that cations and anions must overcome to respectively exit and enter
%the
%pore, and vice versa at positive voltages. In all cases, the additional `tilting' of the energy landscape at
%increased bias magnitudes reduces the barrier height.
%. It falls logarithmically from $\approx5$~$\kT$ at $0.005$~M to $\approx0$~$\kT$ at $1$~M, wit In
%contrast, at positive voltages $\deltaEt$ starts from a value of $\approx2$~$\kT$ at $0.005$~M the remains
%where it becomes $\approx0$.


% cis
% 0.05 : -1.33 = -33.8 mV
% 0.15 : -0.75 = -19.1 mV
% 0.50 : -0.39 = -9.9 mV
% 5.00 : -0.07 = -1.8 mV
%

\subsection{electroosmotic flow}\label{sec:eof}

% electroosmotic flow
% * Introduction to EOF.
%   - What is electroosmotic flow?
%   - Why is it important?
%   - What would we like to know about it from simulations?
% * EOF in ClyA.
%   - Description of the EOF
%     + In which direction does it flow?
%     + What does the velocity profile look like?
%   - What is the effect of ...
%     + ... salt concentration?
%     + ... bias voltage?
% * Conclusions on EOF in ClyA.
%   - Key findings.
%   - Link back to published experimental data.


\begin{figure*}[!htb]
  \centering
  \hspace{-2cm}
  \begin{minipage}[t]{5.5cm}
    \begin{subfigure}[t]{5.5cm}
      \centering
      \caption{}\vspace{-3mm}\label{fig:flow_contour}
      \includegraphics[scale=1]{figures/flow/flow_contour_500mM}
    \end{subfigure}
    \begin{subfigure}[t]{5.5cm}
      \addtocounter{subfigure}{1}
      \vspace{3mm}
      \centering
      \caption{}\vspace{-3mm}\label{fig:flow_constriction_profiles}
      \includegraphics[scale=1]{figures/flow/flow_constriction_profiles}
    \end{subfigure}
  \end{minipage}
  \begin{subfigure}[t]{4cm}
    \addtocounter{subfigure}{-2}
    \centering
    \caption{}\vspace{2mm}\label{fig:flow_constriction_contour}
    \includegraphics[scale=1]{figures/flow/flow_constriction_contour}
  \end{subfigure}
  \begin{minipage}[t]{4cm}
    \begin{subfigure}[t]{4cm}
      \addtocounter{subfigure}{1}
      \centering
      \caption{}\vspace{-5mm}\label{fig:flow_conductance_vs_voltage}
      \includegraphics[scale=1]{figures/flow/flow_conductance_vs_voltage}
    \end{subfigure}
    \begin{subfigure}[t]{4cm}
      \vspace{2mm}
      \centering
      \caption{}\vspace{-5mm}\label{fig:flow_conductance_vs_concentration}
      \includegraphics[scale=1]{figures/flow/flow_conductance_vs_concentration}
    \end{subfigure}
    \begin{subfigure}[t]{4cm}
      \vspace{2mm}
      \centering
      \caption{}\vspace{-5mm}\label{fig:flow_conductance_rectification_vs_concentration}
      \includegraphics[scale=1]{figures/flow/flow_conductance_rectification_vs_concentration}
    \end{subfigure}
  \end{minipage}
\centering

  \caption%
  %
  [\textbf{Concentration and voltage dependency of the electro-osmotic flow inside ClyA.}]
  %
  {%
    %
    \textbf{Concentration and voltage dependency of the electro-osmotic flow inside ClyA.}
    %
    (\subref{fig:flow_contour})
    %
    Contour plot of the electroosmotic flow (EOF) velocity $\velocity$ at \SI{0.5}{\Molar} and \SI{-100}{\mV}
    bias voltage. The arrows  on the streamlines indicate the direction of the flow. As observed
    experimentally\cite{Soskine-2013} and  expected from a negatively charged conical nanopore, the EOF
    follows the direction of the cation, i.e. from  \cis\ to \trans\ under negative bias voltages and vice
    versa for positive ones.
    %
    (\subref{fig:flow_constriction_contour})
    %
    Contour plots of the EOF field in the trans constriction for various salt concentrations at \SI{-100}{\mV}
    and
    %
    (\subref{fig:flow_constriction_profiles})
    %
    cross-sections of the absolute value of the water velocity $\left|U_z\right|$ at $z=\SI{-1}{\nm}$. Notice
    that at high salt concentrations (\SI{>1}{\Molar}), the velocity profile exhibits two `lobes' close to the
    nanopore walls and hence deviates from the parabolic shape observed at lower ionic strengths.
    %
    (\subref{fig:flow_conductance_vs_voltage})
    %
    and
    %
    (\subref{fig:flow_conductance_vs_concentration})
    %
    the electroosmotic conductance $G_{\text{eo}} = Q_{\text{eo}}/V$, with $Q_{\text{eo}}$ the total flow
    rate through the pore, plotted against bias voltage and bulk salt concentration, respectively. In the low
    concentration regime, $G_{\text{eo}}$ increases rapidly between \SIlist{0.005;0.5}{\Molar} after which it
    decreases logarithmically for higher concentrations.
    %
    (\subref{fig:flow_conductance_rectification_vs_concentration})
    %
    The rectification of the electroosmotic flow rate
    ($\alpha_{\text{eo}} = G_{\text{eo}(}+V)/Q_{\text{eo}}(-V)$) plotted against the concentration.
    $\alpha_{\text{eo}}$ shows a maximum between \SIlist{0.04;0.05}{\Molar}, after which it falls rapidly to
    reach unity at approximately \SI{\approx0.45}{\Molar}, regardless of the applied bias. A minimum is then
    reached at \SI{\approx1}{\Molar}, followed by a gradual approach towards unity.
    %
  }\label{fig:flow}

\end{figure*}



The charged nature of many nanopores results in formation of an electroosmotic flow (EOF), \ie~a net flux of
water through the pore. This flow does not only influence the transport of ions and When investigating
macromolecules with nanopores, the EOF cannot be ignored as it exerts The EOF allows for the capture of
nucleic acids\cite{Wong-2007}, peptides\cite{Huang-2017} and proteins
\cite{Soskine-2012,Soskine-2013,VanMeervelt-2014,Soskine-Biesemans-2015,Biesemans-Soskine-2015,Wloka-2017}
irrespective of their charge.

The EOF is caused by two closely related mechanisms: 1) the excess transport of hydration water due to the
pore's ion selectivity, and 2) the viscous drag exerted by the unidirectional movement of the electrical
double layer inside the pore.


The first mechanism likely dominates in pores with a diameter close to that of the hydrated ions
(\SI{\le1}{\nm}) such as \ahl\cite{} or FraC,\cite{Huang-2017} while the second is expected to be stronger for
larger pores (\SI{>1}{\nm}), such as ClyA\cite{Soskine-2012} or most solid-state nanopores.


When investigating macromolecules with nanopores, the EOF cannot be ignored as
it exerts The EOF allows for the capture of nucleic acids\cite{Wong-2007},
peptides\cite{Huang-2017} and proteins
\cite{Soskine-2012,Soskine-2013,VanMeervelt-2014,Soskine-Biesemans-2015,Biesemans-Soskine-2015,Wloka-2017}
irrespective of their charge.


In the following section we aim to quantitatively and a qualitatively describe the influence of bias voltage
and salt concentration on the EOF inside ClyA.


\paragraph{Direction and magnitude of the EOF.}
%
The predominantly negative charges lining the interior walls of ClyA result in net water flow flow \cis{} to
\trans{} at negative bias voltages and \textit{vice versa} at positive applied potentials.  Due to the
conservation of mass, the velocity $\velocity$ is predominantly a function of the nanopore radius. Hence, it
is lowest in the lumen of the pore and highest inside the \trans{} constriction (\cref{fig:flow_contour}). At
the center of the pore, $\velocity$ reaches values of \SI{\approx0.07}{\mps} in the lumen and
\SI{\approx0.21}{\mps} in the constriction (for $\cbulk=\SI{0.5}{\Molar}$ and $\vbias=\SI{-100}{\mV}$).

\paragraph{Influence of bulk ionic strength.}
%
At low bulk ionic strengths ($\cbulk<0.5$~M), the velocity profile along the radius of the pore has a
parabolic shape (\cref{fig:flow_constriction_contour,fig:flow_constriction_profiles}), resulting from the
overlapping electrical double layer in the \trans{} constriction at the these concentrations (\cref{fig:}).
The primary reason for this is the presence of a net charge density along the entire considerable overlap of
the electrical double layer inside the nanopore, particularly in the constriction,


resulting in a nat these inside the pore result a parabolic radial velocity, similarly to a pressure driven
flow \todo{ref} . As the ionic charge density $\scdion$ becomes increasingly confined to nanopore walls
($0.5\le\cbulk<1$~M, \cref{fig:ion_charge_density}), the central maximum flattens out, resulting in the `plug'
flow profile typically observed for electroosmotically driven flows \todo{ref}. Interestingly, at high salt
concentrations ($\cbulk\ge1$~M) the velocity at the center of the pore becomes lower than that at the walls.
\todo{why? lack of charge density in the lumen, complete confinement of charges to the wall?}

\paragraph{Water conductance and rectification.}
%
To more easily compare the total amount of water transported by ClyA between conditions, we computed the
electroosmotic conductance $G_{\text{eo}} = Q_{\text{eo}}/V$ with $Q_{\text{eo}}$ the volumetric flow rate
obtained by integrating the water velocity over the reservoir boundary. \todo{voltage dependency}
$G_{\text{eo}}$ does not decrease monotonically with increasing salt concentrations, but rather exhibits a
maximum of \SI{\approx11.5}{\cnmpnspv} at $\approx0.5$~M. Tshow the expected monotonical depends

\cite{Mao-2014,Laohakunakorn-2015}



\cref{fig:flow_conductance_vs_voltage}
\cref{fig:flow_conductance_vs_concentration}


\cref{fig:flow_conductance_rectification_vs_concentration}

\paragraph{electroosmotic pressure.}
\cref{suppinfo:fig:pressure}
\cite{Hoogerheide-2014}

% Inter-figure observations and discussion

\subsection{The extended PNP-NS equations significantly improve the accuracy of continuum simulations at the
nanoscale.}

\begin{itemize}
  \item Most important `corrections':
  \subitem concentration dependent ion diffusion coefficient and mobilities
  \subitem wall distance
\end{itemize}

\subsection{Ionic current rectification is caused by depletion of anions in the nanopore lumen at negative
bias voltages.}

\subsection{ClyA's ion selectivity is highly concentration and voltage dependent.}

While the ion selectivity can be estimated experimentally using the Goldman-Hodgkin-Katz equation
(GHKe),\cite{Franceschini-2016,Huang-2017} the asymmetric salt concentrations used to determine the nanopore's
reversal potential raises the question for which salt concentration this ion selectivity is valid. Moreover,
the many assumptions and simplifications used to derive the GHKe (Nernst-Einstein relation, no convective
transport, uniform electrical field over the pore) suggest that the ion selectivity calculated for larger
pores should be considered as a rough estimation only.


\subsection{The electroosmotic flow exhibits a maximum at 0.5~M due to two
competing concentration-dependent effects.}

\begin{itemize}
  \item Reduced double layer overlap and the appearance of alternating positive and negative space charge
  densities at the nanopore walls results in reduced electroosmotic flow at higher salt concentrations.
  \item Poor electro-static screening of the highly negatively charged \trans{} constriction
  results in an body force that opposes the bulk flow of the nanopore and hence a diminishes the
  electroosmotic flow magnitude at low salt concentrations.
  \item Both effect are weakest at $\approx0.5$~M, resulting in maximum at that concentration.
\end{itemize}

\section{Conclusions}\label{sec:conclusions}

We have developed an extended version of the Poison-Nernst-Planck-Navier-Stokes (ePNP-NS) equations and used
them to model the transport of ions and water through the biological nanopore ClyA. Our ePNP-NS equations take
into account the finite size of the ions and include a self-consistent, concentration- and
positional-dependent parametrization of the ionic transport coefficients (diffusion coefficient and mobility)
and the electrolyte properties (density, viscosity and relative permittivity).

We have verified our approach by matching experimental results to a very high degree of accuracy and the
model parameters where gauged on other experiments, ultimately leaving no degrees of freedom. This shows that
a continuum approach to modeling biological nanopores is not only feasible but to a very high degree
predictive.

Inaccuracies only arise in the low salt limit, leading to an experimentally unexplored fluid configuration
within the pore which almost only contains one ion species. It is likely that if proper model parameters for
this domain can be extracted from other sources, that the ePNP-NS equations will also be able to reproduce
experimental values under these conditions.

In the analysis of the operation of the biological nanopore with our model we found that the ionic currents
very much depend on the details of the electric field structure within the pore which define potential
barriers for the ion species. This of course means that the charge configuration of the protein needs to be
implemented accurately to obtain meaningful and predictive results and any simplifications regarding the pore
structure would likely deteriorate the results.

Following our analysis of ClyA, it stands to reason that other biological nanopores of comparable size can be
treated analogously and their properties could likely be mapped out systematically using our ePNP-NS model.


\section{Materials and Methods}\label{sec:methods}

\paragraph{ClyA-AS homology model.}
%
A full atom model of ClyA-AS\cite{Soskine-2013} was built and optimized (MODELLER v9.18\cite{Sali-1993}) by
introduction of the following point mutations in each of the 12 chains of the wild-type ClyA crystal structure
(PDBID: 2WCD\cite{Mueller-2009}): K8Q, N15S, Q38K, A57G, T67V, C87A, A90V, A95S, L99Q, E103G, K118R, L119I,
I124V, T125K, V136T, F166Y, K172R, V185I, K212N, K214R, S217T, T224S, N227A, T244A, E276G, C285S, K290Q. Next,
the conformation of all mutated side chains was optimized with an double annealing protocol (heating: 150,
250, 400, 700 and \SI{1000}{\kelvin}, cooling: 1000, 800, 600, 500, 400 and \SI{300}{\kelvin}) where at each
temperature the energy was minimized for 200 iterations with a conjugate gradients algorithm (\SI{4}{\fs}
timestep).\cite{Shanno-1980} The first anneal was performed solely on the mutated residues themselves, and
the second run also took the non-bonded interactions with the neighboring atoms into account. The refined
nanopore structure was then embedded in the center of an \SI{18x18}{\nm} equilibrated DPhPC lipid bilayer
patch by manual removal of all overlapping lipids, resulting in 463 lipid molecules. The bilayer was created
with the CHARMM-GUI\cite{Jo-2008} membrane builder\cite{Lee-2016} and equilibrated with
NAMD\cite{Phillips-2005}, as described in detail in ref. [\citenum{Wu-2014}]. The system was then solvated in
a box of \SI{18x18x32}{\nm} by addition of 214640 TIP3 water molecules (VMD solvate plugin). The global charge
was neutralized by replacing 1276 random water molecules with 674 \Na{} and 602 \Cl{} ions (VMD autoionize
plugin).\cite{Humphrey-1996}

\paragraph{Molecular dynamics simulations.}
%
Using molecular dynamics (MD) with NAMD 2.12 (\SI{2}{\fs} timestep, CHARMM36 forcefield\cite{Best-2012}), the
final system was minimized for \SI{5}{\ps}, heated from 0 to \SI{298.15}{\kelvin} in \SI{4}{\ps} and
equilibrated for \SI{4}{\ns} as NpT ensemble, .\cite{Aksimentiev-2005} Finally a \SI{30}{\ns} production run
was performed using a NVT ensemble at \SI{298.15}{\kelvin} and the atomic coordinates saved every \SI{5}{\ps}.
Note that structural deterioration was prevented by harmonically restraining the protein's C$_\alpha$ atoms to
their original positions (spring constant of \SI{695}{\pN\per\nm}) during all MD runs.\cite{Bhattacharya-2011}

\paragraph{Axially symmetric geometry.}
%
The 2D-axisymmetric geometry of the ClyA-AS nanopore (\cref{fig:model_geometry_vs_wedge}) was derived directly
from its full atom model by radially averaging the molecular density. Briefly, 50 sets of atomic coordinates
were extracted from the final \SI{5}{\ns} of the coordinates of the \SI{30}{\ns} MD production run
(\ie~every \SI{100}{\fs}) and aligned by minimizing the RMSD between their backbone atoms (VMD RMSD tool).
Next, we computed and averaged the 3D-dimensional molecular density maps of all 50 structures on a
\SI{0.5}{\angstrom} resolution grid using the Gaussian function\cite{Li-2013}
%
\begin{align}\label{eq:denspore}
  \rho_\text{mol} = 1 - \displaystyle\prod_{i} \left[ 1 -
        \exp{\left(-\dfrac{-d_i^2}{(\stdev\atomradius_{i})^2}\right)} \right]
\end{align} where for each atom $i$,
%
$R_i$ is its Van der Waals radius, $d_i=\sqrt{(x-x_i)^2 + (y-y_i)^2 + (z-z_i)^2}$ is the distance of grid
coordinates $(x, y, z)$ from the atom center $(x_i, y_i, z_i)$ and $\stdev = 0.93$ is a width factor. The
resulting 3D density map was then radially averaged along the z-axis, relative to the center of the pore to
obtain a 2D-axisymmetric density map. The contourline at \SI{25}{\percent} density was used as the nanopore
simulation geometry, after manual removal of overlapping and superfluous vertices to improve the quality of
the final computational mesh.

\paragraph{Axially symmetric charge density.}
%
The 2D-axially symmetric charge distribution (\cref{fig:model_charge_density} was also derived directly
from the 50 sets of aligned nanopore coordinates ) that were used for the geometry. Inspired by how charges
are represented in the particle mesh Ewald (PME) method,\cite{Aksimentiev-2005} we computed the fixed charge
distribution of the nanopore $\scdpore(r,z)$ by assuming that an atom $i$ of partial charge
$\partialcharge_{i}$ at the location $(x_i, y_i, z_i)$ in the full 3D atomistic pore model will contribute the
partial charge $\partialcharge_{i}/2\pi r_i$ to a point $(r_i,z_i)$ with $r_i = \sqrt{x_i^2 + y_i^2}$ in the
averaged 2D-axisymmetric model. That means we effectively spread the charge over all angles to achieve axial
symmetry. Following that, we assume a Gaussian distribution of the space charge density of each atom $i$
around its respective 2D-axisymmetric coordinates $(r_i,z_i)$ is
%
\begin{multline}
\label{eq:scdpore}
  \scdpore(r,z) = \\ \dsum_{i} \dfrac{\echarge\partialcharge_{i}}{\pi(\stdev\atomradius_{i})^2}
            \exp{\left(-\dfrac{(r-r_i)^2 + (z-z_i)^2}{(\stdev\atomradius_{i})^2}\right)}
\end{multline}
%
where $\atomradius_{i}$ is the atom radius, $\sigma = \num{0.5}$ is the sharpness factor and $\echarge$ is the
elementary charge. To embed $\scdpore$ with sufficient detail yet efficiently into a numeric solver, the
spatial coordinates are discretized with a grid spacing of $0.005$~nm in the domain of $\scdpore$ and
precomputed values are used during the solver runtime. All partial charges (at pH~7.5) and radii were taken
from the CHARMM36 forcefield\cite{Best-2012} and assigned using PROPKA\cite{Olsson-2011} and
PBD2PQR.\cite{Jurrus-2018}

\paragraph{Computing electrophoretic mobilities.}
%
To obtain the concentration-dependent ionic mobility  $\mobility^c_{i}$ from fitted functions, it must first
be derived from the salt's molar conductivity $\molarconductivity$ and the ion's transport number
$\transportn_{i}$ before it can be fitted\cite{ContrerasAburto-2013-1}
%
\begin{align}
\label{eq:conductivity-to-mobility}
\mobility_{i}(\concentration) = \frac{\specmolarconductivity_{i}(\concentration)}{\chargen_{i}\faraday}
\quad\text{with}\quad \specmolarconductivity_{i}(\concentration) = \molarconductivity(\concentration)
\transportn_{i}(\concentration),
\end{align}
%
where $\specmolarconductivity_{i}(\concentration)$ is the specific molar conductivity of ion $i$.

\paragraph{Computing the simulated ionic current.}
%
The simulated ionic current $\currentsim$ at steady-state was computed by
%
\begin{equation}
  \currentsim = \faraday\int_{S}\left(\dsum_{i}\chargen_{i}\normvec\cdot\vec{\flux}_{i}\right)dS
\end{equation}
%
with $\chargen_{i}$ the charge number and $\vec{\flux}_{i}$ the total flux of each ion $i$ across \cis{}
reservoir boundary $S$, $\faraday$ the Faraday constant (\SI{96485}{\coulomb\per\mole}) and $\normvec$ the
unit vector normal to $S$.


\paragraph{ClyA expression and purification.}
%
ClyA-AS monomers were expressed, purified and oligomerized using methods described in detail
elsewhere.\cite{Soskine-2012,Soskine-2013} Briefly, \textit{E. cloni} EXPRESS BL21 (DE3) cells (Lucigen
Corporation, Middleton, USA) were transformed with a pT7 plasmid containing the ClyA-AS gene, followed by
overexpression after induction with $1$~mM isopropyl \textbeta-D-1-thiogalactopyranoside (Sigma-Aldrich,
Zwijndrecht, The Netherlands). The ClyA monomers were purified using \ce{Ni+}-NTA affinity chromatography and
oligomerized by incubation in $0.2$~\%\ D-maltoside n-dodecyl-\textbeta-D-maltopyranoside (Sigma-Aldrich,
Zwijndrecht, The Netherlands) for 30~minutes at 37\textdegree C. Pure ClyA-AS type-I (12-mer) nanopores were
obtained using native PAGE on a $4$-$15$\%\ gradient gel (Bio-Rad, Veenendaal, The Netherlands) and subsequent
excision of the correct oligomer band.

\paragraph{Recording of single-channel current-voltage curves.}
%
Experimental current-voltage curves where measured using single-channel electrophysiology, as detailed
elsewhere.\cite{Maglia-2010,Soskine-2012,Soskine-2013} Briefly, a black lipid bilayer was inside a
\SI{\approx100}{\um} diameter aperture in a thin teflon film separating two buffered electrolyte compartments
by painting with 1,2-diphytanoyl-snglycero-3-phosphocholine (DPhPC, Avanti Polar Lipids, Alabaster, USA).
Minute amounts ($\approx$) of the purified ClyA-AS type I oligimer were then added to the grounded \cis{}
reservoir and allowed to insert into the lipid bilayer. Single-channel current-voltage curves were recorded
using a custom pulse protocol of the Clampex 10.4 software package connected to AxoPatch 200B patch-clamp
amplifier via a Digidata 1440A digitizer (all from Molecular Devices, San Jose, USA). Data was acquired at
$10$~kHz and filtered using a $2$-kHz low bandpass filter. Measurements at different ionic strengths were
performed at $\approx25$\textdegree C in aqueous \ce{NaCl} solutions, buffered at pH~$7.5$ using $10$~mM MOPS
(Sigma-Aldrich, Zwijndrecht,The Netherlands).

%%%%%%%%%%%%%%%%%%%%%%%%%%%%%%%%%%%%%%%%%%%%%%%%%%%%%%%%%%%%%%%%%%%%%
%% The "Acknowledgement" section can be given in all manuscript
%% classes.  This should be given within the "acknowledgement"
%% environment, which will make the correct section or running title.
%%%%%%%%%%%%%%%%%%%%%%%%%%%%%%%%%%%%%%%%%%%%%%%%%%%%%%%%%%%%%%%%%%%%%
\begin{acknowledgement}
K.W. thanks the Research Foundation Flanders (FWO) for the doctoral fellowship and the project grant
(\todo{number}). The work of G.M. was supported by an ERC consolidator grant (DeE-Nano, \todo{number}). The
authors thank Ujjal Barman and Chang Chen for their valuable feedback during discussions.
\end{acknowledgement}

%%%%%%%%%%%%%%%%%%%%%%%%%%%%%%%%%%%%%%%%%%%%%%%%%%%%%%%%%%%%%%%%%%%%%
%% The same is true for Supporting Information, which should use the
%% suppinfo environment.
%%%%%%%%%%%%%%%%%%%%%%%%%%%%%%%%%%%%%%%%%%%%%%%%%%%%%%%%%%%%%%%%%%%%%
\begin{suppinfo}
	The supplementary info contains the Extended materials and methods, with details on the fitting of the
	electrolyte properties and the calculation of the pore averaged values, and the weak forms of the ePNP-NS
	equations. It also contains additional results, including a figures about the voltage dependency of the
	power-law exponent of the conductance fits and the electroosmotic pressure distribution inside the pore, and
	a table detailing the peak values of the radial electrostatic potential inside ClyA.
\end{suppinfo}

\bibliography{shared/bibliography}

%\includepdf{suppinfo}
\end{document}

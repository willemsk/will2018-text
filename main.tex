%-------------------------------------------------------------------------------
% PREAMBLE AND DOCUMENT FORMATTING
%-------------------------------------------------------------------------------
\documentclass[journal=ancac3,manuscript=article,etalmode=truncate,maxauthors=0,layout=twocolumn]{achemso}
	\setkeys{acs}{etalmode=truncate,maxauthors=0}

% PACKAGES
\usepackage[utf8]{inputenc}
\usepackage[english]{babel}
\usepackage{csquotes}
\usepackage{amsmath}
\usepackage{amsfonts}
\usepackage{amssymb}
\usepackage{mathtools}
\usepackage{textcomp}
\usepackage{textgreek}
\usepackage{gensymb}

\usepackage[version=4]{mhchem}
\usepackage[alsoload=synchem]{siunitx}
	\sisetup{separate-uncertainty=true}
	\sisetup{multi-part-units=single}
	\sisetup{tight-spacing=true}
	\sisetup{inter-unit-product=\ensuremath{{}\cdot{}}}
	\sisetup{list-units=single}
	\sisetup{range-units=single}

\usepackage{float}
\usepackage{graphicx}
\usepackage{xcolor}
\usepackage{tikz}
	\usetikzlibrary{shapes,arrows.meta}
\usepackage{pgf}
\usepackage{pgfplots}
\pgfplotsset{compat=1.15}
\usepackage{array, booktabs, tabularx, multirow}
\usepackage{pdfpages}

\pdfsuppresswarningpagegroup=1

% CAPTION FORMATTING
\usepackage[font=scriptsize,labelfont=bf,labelsep=period]{caption} 	% format single-image 
%captions and table titles
	\captionsetup[table]{singlelinecheck=false,font=footnotesize,labelfont=bf}
\usepackage[font=scriptsize,labelfont=bf,labelsep=period]{subcaption} 						% format 
%subfigure captions
	%\DeclareCaptionSubType*[alph]{figure}
	%\renewcommand\thesubfigure{\thefigure\alph{subfigure}}
	\captionsetup[subfigure]{labelfont=bf,textfont=normalfont,labelformat=simple,singlelinecheck=false} 	

% CROSS-REFERENCE FORMATTING
% For use with the cleveref package
% Define the format of Figure, Table, Equation, and Section cross-references in the text
\usepackage{xr-hyper}
\usepackage{hyperref}
\usepackage{cleveref}
\creflabelformat{equation}{#2#1#3}
\crefname{figure}{Fig.}{Figs.}
\Crefname{figure}{Figure}{Figures}
\crefname{table}{Tab.}{Tabs.}
\Crefname{table}{Table}{Tables}
\crefname{equation}{Eq.}{Eqs.}
\Crefname{equation}{Equation}{Equations}
\crefname{section}{Sec.}{Secs.}
\Crefname{section}{Section}{Sections}


% TITLE AND ABSTRACT
%\let\oldmaketitle\maketitle
%\let\maketitle\relax

% REFERENCES
\renewcommand*{\bibfont}{\normalfont\small}

% SUPPORTING INFO
\usepackage{xr}
\externaldocument[suppinfo:]{supporting_information}

%-------------------------------------------------------------------------------
% CUSTOM COMMANDS
%-------------------------------------------------------------------------------

% Shorthands
\newcommand{\todo}[1]{\textbf{\textcolor{orange}{#1}}}
\newcommand{\ahl}{\textalpha HL}
\newcommand{\etal}{\textit{et al.}}
\newcommand{\cis}{\textit{cis}}
\newcommand{\trans}{\textit{trans}}

% Units
\DeclareSIUnit{\molar}{\mole\per\cubic\deci\metre}
\DeclareSIUnit{\Molar}{\textsc{M}}
\newcommand{\mM}{\milli\Molar}
\newcommand{\mV}{\milli\volt}
\newcommand{\mps}{\meter\per\second}
\newcommand{\cnmpnspv}{\cubic\nano\meter\per\nano\second\per\volt}

% Vectors and math stuff
\renewcommand{\vec}[1]{\boldsymbol{#1}}
\newcommand{\rpos}{\vec{r}} % positional vector
\newcommand{\normvec}{\vec{\hat{n}}} % normal vector
\newcommand{\identity}{\vec{\rm I}} % Identity vector
\newcommand{\stdev}{\sigma}
\newcommand{\pav}[2]{\left< #1 \right>_{\text{#2}}} % Pore average
\newcommand{\vc}[2]{#1_{#2}} 
\newcommand{\pd}[2]{\displaystyle\frac{\partial #1}{\partial #2}}
\newcommand{\hydrostresstensor}{\sigma_{ij}}

% Physical constants
\newcommand{\boltzmann}{k_{\rm B}}
\newcommand{\avogadro}{N_{\rm A}}
\newcommand{\temp}{T}
\newcommand{\faraday}{\mathcal{F}}
\newcommand{\echarge}{e}

% Field variables
\newcommand{\vel}{u}
\newcommand{\force}{F}
\newcommand{\potential}{\varphi}			% Electrostatic potential
\newcommand{\varpotential}{V}				% Electrostatic potential
\newcommand{\concentration}{c}				% Ion concentration
\newcommand{\velocity}{\vec{u}}				% Fluid velocity
\newcommand{\pressure}{p}					% Fluid pressure
\newcommand{\efield}{\vec{E}}				% Electrical field
\newcommand{\displacement}{\vec{D}} % Electrical displacement fiedl
\newcommand{\walldistance}{d}				% Wall distance

% Dimensionless variables
\newcommand{\dconc}{\bar{\concentration}}		% Dimensionless concentration
\newcommand{\dwall}{\bar{\walldistance}}		% Dimensionless wall distance

% Material and ion parameters
\newcommand{\permittivity}{\varepsilon}
\newcommand{\absperm}{\permittivity_0}				% Permittivity of vacuum
\newcommand{\relperm}{\varepsilon_r}				% Relative permittivity
\newcommand{\dielectric}{\relperm}				% Relative permittivity
\newcommand{\diffusion}{\mathcal{D}}			% Diffusion coefficient
\newcommand{\mobility}{\mu}						% Electrophoretic mobility
\newcommand{\transportn}{t}						% Transport number
\newcommand{\chargen}{z}						% Ion charge number
\newcommand{\ionsize}{a}						% Ion size for smPNP
\newcommand{\density}{\varrho}						% Fluid density
\newcommand{\viscosity}{\eta}					% Fluid viscosity

\newcommand{\iondiffusion}[1]{\diffusion_{#1}}	% Ion Diffusion coefficient
\newcommand{\ionmobility}[1]{\mobility_{#1}}	% Ion electrophoretic mobility
\newcommand{\iontransportn}[1]{\transportn{#1}}	% Ion transport number
\newcommand{\avionconc}{\langle\concentration\rangle}

\newcommand{\tna}{\transportn_{\ce{Na+}}}

\newcommand{\molarconductivity}{\Lambda}
\newcommand{\specmolarconductivity}{\lambda}

% Derived properties
\newcommand{\scd}{\rho}							% Total charge density
\newcommand{\scdpore}{\scd_{\text{pore}}^f}		% Fixed charge density
\newcommand{\scdion}{\scd_{\text{ion}}}			% Ionic charge density
\newcommand{\flux}{\vec{J}} 							% Ion flux
\newcommand{\volumeforce}{\vec{\force}_{\rm ion}}	% Fluid volume force

\newcommand{\current}{I}
\newcommand{\currentsim}{\current_{\rm sim}}
\newcommand{\currentexp}{\current_{\rm exp}}
\newcommand{\conductance}{G}
\newcommand{\icr}{\alpha}

% Shorthands
\newcommand{\ci}{\concentration_{i}}
\newcommand{\cbulk}{\concentration_\text{s}}
\newcommand{\vbias}{\varpotential_\text{b}}
\newcommand{\Na}{\ce{Na+}}
\newcommand{\Cl}{\ce{Cl-}}
\newcommand{\Qion}{Q_{\rm ion}}
\newcommand{\radpot}{\left<\potential\right>_\text{rad}}
\newcommand{\radenergy}{\left< U_{\text{E},i} \right>_{\text{rad}}}
\newcommand{\deltaEt}{\Delta E_{\text{B},i}}
\newcommand{\pH}[1]{pH~\num{#1}}
\newcommand{\kT}{\boltzmann\temp}
\newcommand{\kTe}{\kT / \echarge}

% Atom properties
\newcommand{\partialcharge}{\delta}
\newcommand{\atomradius}{R}


% Colors
\definecolor{graphgreen}  {rgb}{0.30196078, 0.68627451, 0.29019608}
\definecolor{graphpurple} {rgb}{0.59607843, 0.30588235, 0.63921569}
\definecolor{graphblue}   {rgb}{0.29803921, 0.57254902, 0.98823529}
\definecolor{graphred}    {rgb}{0.91372549, 0.15686275, 0.18823529}

% Line and marker styles
\newcommand{\graphline}[2]{\raisebox{2pt}{\tikz{\draw[-,color=#2,#1,line width=1.5pt](0,0) -- (5mm,0);}}}
\newcommand{\graphmarker}[2]{\raisebox{0.5pt}{\tikz{\node[draw,scale=0.5,#1,fill=none, color=#2](){};}}}

%\newcommand{\graphlinemarker}[3]{\raisebox{0pt}{\tikz{\draw[-,#3,#1,line width = 1.0pt](2.mm,0) #2 (3.5mm,1.5mm);\draw[-,#3,#1,line width = 1.0pt](0.,0.8mm) -- (5.5mm,0.8mm)}}}
%\newcommand{\rectangle}{\raisebox{0pt}{\tikz{\draw[-,black,dotted,line width = 1.0pt](0.,0.8mm) -- (5.5mm,0.8mm);\draw[black,solid,line width = 1.0pt](2.mm,0) circle (3.5mm,1.5mm)}}}
\newcommand{\rectangle}[1]{\raisebox{0pt}{\tikz{\draw[-,black,solid,line width = 1.0pt](0.,0.8mm) -- (5.5mm,0.8mm);\draw[black,solid,line width = 1.0pt](2.mm,0) #1 (3.5mm,1.5mm)}}}


%-------------------------------------------------------------------------------
% TITLE, KEYWORDS AND AUTHOR LIST
%-------------------------------------------------------------------------------
\title{Modelling of Ion and Water Transport in the Biological Nanopore ClyA}

\keywords{biological nanopore; cytolysin A; continuum simulation; PNP-NS; single molecule}

\newcommand{\afimec}{imec, Kapeldreef 75, B-3001 Leuven, Belgium}
\newcommand{\afkulchem}{KU Leuven, Department of Chemistry, Celestijnenlaan 200F, B-3001 Leuven, Belgium}
\newcommand{\afkulphys}{KU Leuven, Department of Physics and Astronomy, Celestijnenlaan 200D, B-3001 Leuven, Belgium}
\newcommand{\afrug}{University of Groningen, Groningen Biomolecular Sciences \& Biotechnology Institute, 9747 AG, Groningen, The Netherlands}

\author{Kherim Willems}
\affiliation{\afkulchem}
\alsoaffiliation{\afimec}

\author{Dino Rui\'{c}}
\affiliation{\afkulphys}
\alsoaffiliation{\afimec}

\author{Florian Lucas}
\affiliation{\afrug}

%\author{Ujjal Barman}
%\affiliation{\afimec}

%\author{Chang Chen}
%\affiliation{\afimec}

\author{Johan Hofkens}
\affiliation{\afkulchem}

\author{Giovanni Maglia}
\email{g.maglia@rug.nl}
\affiliation{\afrug}

\author{Pol Van Dorpe}
\email{Pol.VanDorpe@imec.be}
\affiliation{\afkulchem}
\alsoaffiliation{\afimec}

% One column abstract tweak
%\let\oldmaketitle\maketitle
%\let\maketitle\relax

%===============================================================================
%
% MAIN DOCUMENT BEGINS
%
%===============================================================================
\begin{document}

% Title and authors
%\maketitle

%-------------------------------------------------------------------------------
% ABSTRACT
%-------------------------------------------------------------------------------
%\twocolumn[
%\begin{@twocolumnfalse}
%\oldmaketitle
\begin{abstract}
\footnotesize
Cytolysin A (ClyA) is a protein nanopore that has become a valuable tool for the detection, characterization 
and analysis of bio-markers, proteins and nucleic acids at the single-molecule level.  While ClyA has been 
engineered extensively towards these applications, a full understanding of its transport properties and the 
scale of the electrophoretic forces inside it, is currently lacking. From this perspective, we have built a 
2D-axisymmetric model of ClyA that accurately captures its geometry and charge distribution, and numerically 
computed the flux of ions and water through the pore. To achieve the latter, we developed an extended version 
of the steady-state Poison-Nernst-Planck-Navier-Stokes (ePNP-NS) equations that makes use of self-consistent 
concentration- and positional-dependencies for the ion diffusion coefficients and mobilities, and the solvent 
density, viscosity and relative permittivity. In this study we show that our model, in conjunction with the 
ePNP-NS equations, is able to reproduce the experimentally observed ionic currents over a wide range of ionic 
strengths and applied voltages. We further describe the influence of these latter two parameters on various 
quantities of interest, including the cation and anion concentrations inside the pore, the shape of the 
electrostatic potential landscape and the magnitude and direction of the electro-osmotic flow.
\todo{needs a final punchline}
\end{abstract}
%\end{@twocolumnfalse}
%]
%-------------------------------------------------------------------------------
% INTRODUCTION
%-------------------------------------------------------------------------------

\section{Introduction}

Over the past two decades, nanopores have more than proven their worth as stochastic sensors at the ultimate 
analytical limit.\cite{Bayley-2001,Zhang-2016} Generally, nanopores are used in `resistive pulse' mode, were 
the interaction or translocation of individual molecules with or through the pore result in the a temporary 
modulation of the ionic conductance. While nanopore research was initially focuses towards single-molecule 
DNA sequencing, in recent years the field of interest has broadened to numerous other applications, such as 
the quantification of biomarkers,\cite{Huang-2017} protein sequencing\cite{Restrepo-Perez-2018} and 
single-molecule enzymology.\cite{Willems-VanMeervelt-2017,Laszlo-2017}

% Continuum modelling nanopores
%   Mostly used for solid state nanopores
%   Modelling of biological nanopores derived from work on ion channels \cite{Maffeo-2012}
%   focused on alpha-hemolysin
% Molecular dynamics: \cite{Aksimentiev-2005,Bhattacharya-2011,Wong-Ekkabut-2016}
% Brownian dynamics:  \cite{Noskov-2004,Simakov-2010,Pederson-2015}
  
 
yields reproducible results is to use self-assembled protein nanopores suspended on lipid 
bilayers\cite{Deamer-2016}.
In fact their properties as biosensors are so desirable that they have already been commercialized for DNA
sequencing applications\todo{ref}.

Despite the successes of these biological nanopores, the modeling of the sensor operation is still in its 
infancy due to the multi-scale nature of the problem. On the one hand, it is necessary to resolve the charge 
distribution of the protein on the atomic level requiring molecular dynamics (MD)\todo{ref}.  On the other 
hand, the size of the solvent reservoirs and thus the number of particles is too large for a comprehensive MD 
simulation. More suited to the problem of the flow of the solvent as well as the electromigration and 
diffusion of the ions in the solvent is a continuum approach based on the Navier-Stokes equation (NSE) and 
the Poisson-Nernst-Planck equations (PNPE)\todo{ref}.

Thus the simulation of the nanopore sits uncomfortably at the intersection of the macroscopic and the
microscopic. In this work, we want to show that it is still possible -- with remarkable accuracy -- to 
simulate the operation of a biological nanopore using a continuum approach based on the NSE and PNPE. 
However, we will also see the limits of this approach which hint at experimentally unexplored states of
solutions within nanopores which are unlike anything that can be found in bulk solutions.


ClyA-AS was artificially evolved from of the wild type \textit{S. typhi} Cytolysin A to exhibit improved 
stability during electrical recordings in lipid bilayers.\cite{Soskine-2013} While it has been extensively 
used in experimental studies of both proteins\cite{Soskine-2013,VanMeervelt-2014,Soskine-Biesemans-2015,
Biesemans-Soskine-2015,Wloka-2017,VanMeervelt-2017} and DNA,\cite{Franceschini-2013,Franceschini-2016} 
several quantities of interest, such as the precise ionic conditions inside the pore and the magnitude of the 
electro-osmotic flow, remain unclear. Because these properties are very challenging to determine 
experimentally over a broad range of conditions, we turned to numerical modeling to shed light on these 
properties and to further our general understanding of the pore.
To this end, we implemented a modified version of the Poisson-Nernst-Planck-Navier-Stokes (PNP-NS) equations 
and solved it for a 2D-axisymmetric continuum model of ClyA-AS. Our modifications attempt to empirically 
extend the validity of the PNP-NS equations 1) at the nanoscale and 2) beyond infinite dilution. For the 
former, we  implemented spatially dependent ion transport coefficients,\cite{Makarov-1998, Simakov-2010} and 
an solvent viscosity\cite{Pronk-2014,Hsu-2017}, which respectively decrease and increase significantly in 
close proximity to the nanopore wall. For the latter, we took for the finite size of the ions into account by 
inclusion of a steric flux term,\cite{Lu-2011} and implemented self-consistent, empirical concentration 
dependencies of the ionic diffusion coefficients and mobilities,\cite{Baldessari-2008-1} and the solvent 
density,\cite{} viscosity and relative permittivity.\cite{Gavish-2016}

We begin our study of ClyA's by describing its conductance, rectification and ion selectivity over a wide 
range of experimentally accessible conditions. Next we investigate the concentration and voltage dependencies 
of the ion transport through ClyA in 
terms of conductance, rectification and selectivity. The values of the former two parameters were also 
measured experimentally and employed as a benchmark to gauge the accuracy of our model. This is followed by a 
detailed description of several parameters that are of great interest, but cannot be readily determined 
experimentally, namely the concentration of cations and anions inside the pore, the ionic charge density 
inside the pore, the distribution of the electrostatic potential and the properties of the electro-osmotic 
flow. While this work solely describes the ClyA nanopore, the lessons learned form the modeling can be 
readily transferred to other types of nanopores.

%-------------------------------------------------------------------------------
% RESULTS
%------------------------------------------------------------------------------
\section{Results and discussion}\label{sect:results}

\subsection{Model and equations}

\begin{figure*}[!t]
  
	\centering
	\begin{minipage}[t]{5cm}
		\begin{subfigure}[t]{5cm}
			\centering
			\caption{}\label{fig:clya_side}
			\includegraphics[width=5cm]{figures/concept/clya_side}
		\end{subfigure}
		\begin{subfigure}[t]{5cm}
			\centering
			\caption{}\label{fig:clya_top}
			\includegraphics[width=5cm]{figures/concept/clya_top}
		\end{subfigure}
	\end{minipage}
  %\hspace{1cm}
  \begin{minipage}[t]{11.5cm}
    \begin{subfigure}[t]{6cm}
      \centering
      \caption{}\label{fig:model_geometry}
      \includegraphics[width=6cm]{figures/concept/model_geometry}
    \end{subfigure}
    \hspace{-0.8cm}
    \begin{minipage}[t]{5.5cm}
      \begin{subfigure}[t]{5.5cm}
        \centering
        \caption{}\label{fig:model_geometry_zoom}
        \includegraphics[width=5.5cm]{figures/concept/model_geometry_zoom}
        %\vspace{0.25cm}
      \end{subfigure}
      \begin{minipage}[t]{5.5cm}
        \hspace{1cm}
        \begin{subfigure}[t]{1.6cm}
          \centering
          \caption{}\label{fig:model_geometry_vs_wedge}
          \vspace{-3mm}
          \includegraphics[width=1.6cm]{figures/concept/model_geometry_vs_wedge}
        \end{subfigure}
        \begin{subfigure}[t]{2.5cm}
          \centering
          \caption{}\label{fig:model_charge_density}
          \vspace{-3mm}
          \includegraphics[width=2.5cm]{figures/concept/model_charge_density}
        \end{subfigure}
      \end{minipage}
    \end{minipage}
  \end{minipage}

\caption[All-atom and 2D-axisymmetric models of ClyA.]
{
\textbf{All-atom and 2D-axisymmetric models of ClyA.}
(\subref{fig:clya_side}) Cross-sectional and (\subref{fig:clya_top}) top views of the dodecameric nanopore 
ClyA-AS\cite{Soskine-2013} embedded in a lipid bilayer, derived through homology modelling from the 
\textit{E. coli} Cytolysin A crystal structure (PDBID: 2WCD)\cite{Mueller-2009}.
(\subref{fig:model_geometry}) The full 2D-axisymmetric simulation geometry and 
(\subref{fig:model_geometry_zoom}) a close-up of the nanopore itself. ClyA (grey) is embedded in a lipid 
bilayer (green) and surrounded by a spherical water reservoir (blue). The relative permittivity of the 
latter, $\permittivity_w$, is dependent on the local average ion concentration $\avionconc$, while the former 
two have a fixed permittivity of $\permittivity_p=20$ and $\permittivity_m=3.2$, respectively. A bias voltage 
is applied at the \trans\ reservoir boundary ($\potential = \vbias$), red), while the \cis\ side is kept 
grounded ($\potential = 0$, blue). The fixed concentration ($\concentration_i = \concentration_{\text{s},i}$) 
and no normal stress ($\sigma_f \vec{n} = 0$) boundary conditions serve to mimic an infinite reservoir. Note 
that the in the ePNP-NS model, the viscosity $\viscosity$, the ion diffusion coefficients $\diffusion_{i}$ 
and the ion mobilities $\mobility_{i}$ are also a function of both $\avionconc$ and $\walldistance$, the 
distance from the nanopore wall. The fluid density is solely dependent on $\avionconc$. The walls of the 
nanopore and lipid bilayer in contact with the reservoir have Dirichlet boundary conditions, i.e. no-slip 
($\vec{u} = 0$) for fluids and and no-flux ($-\vec{n} \cdot \vec{\flux_i} = 0$) for ions.
(\subref{fig:model_geometry_vs_wedge})
The 2D-axisymmetric geometry was derived directly from the all-atom model by computing the average inner and 
outer radii along the longitudinal axis of the pore, and hence closely follows the outline of a \ang{30} 
wedge out of the homology model. 
(\subref{fig:model_charge_density})
The fixed space charge density ($\scdpore$) map of ClyA-AS, obtained by Gaussian projection of each atom's 
partial charge onto a 2D plane (see methods for details).
}\label{fig:model_concept}
\end{figure*}

\paragraph{A 2D-axisymmetric continuum model of ClyA-AS.}
ClyA is a relatively large protein nanopore that self-assembles on lipid bilayers to form $14$~nm long 
hydrophilic channels. The interior of the pore can be divided into roughly two cylindrical compartments 
(\cref{fig:clya_side}): the \cis\ lumen ($\approx6$~nm diameter, $\approx10$~nm height), and the \trans\ 
constriction ($\approx3.3$~nm diameter, $\approx4$~nm height). Because ClyA consists of $>10$ identical 
subunits (\cref{fig:clya_top}), it exhibits a high degree of radial symmetry --- a geometrical feature allows 
to obtain meaningful results at low computational cost by reduction of the full 3D structure to a much more 
computationally efficient 2D-axisymmetric model.\cite{Pederson-2015,Lu-2012}  

The nanopore's 2D-axisymmetric geometry (\cref{fig:model_geometry_vs_wedge}) was obtained by radially 
averaging the molecular surface definition of a full-atom homology model of ClyA-AS. The 2D charge density 
distribution (\cref{fig:model_charge_density}) was constructed by projecting the partial charges of all 
atoms, represented by 2D-Gaussian functions, onto a plane. To ensure that our model accurately portraits the 
steady-state behavior of the ClyA-AS, we averaged the geometry and charge maps over $\approx50$ full-atom 
structures, extracted from the last $5$~ns (i.e. every $100$~ps) of a $10$~ns explicit solvent molecular 
dynamics simulation (cfr. M\&M for details). In the simulation geometry 
(\cref{fig:model_geometry_zoom,fig:model_geometry}), 
the nanopore is located at the center of a large spherical electrolyte reservoir ($R=250$~nm), which is split 
into two hemispherical compartments by a lipid bilayer ($h=2.8$~nm). Because the boundaries of the nanopore 
and the bilayer are impermeable, transport of water and ions must occur through the nanopore.


\paragraph{Extended Poisson-Nernst-Planck Navier-Stokes equations.}
The coupled Poisson-Nernst-Planck Navier-Stokes equations (PNPNSE) have been exploited extensively for the 
study of transport phenomena in both solid-state\cite{Daiguji-2004,Lu-2012} and biological 
nanopores.\cite{Eisenberg-1996,Simakov-2010, Pederson-2015} While the PNPNSE yield good qualitative insights, 
their ability to quantitatively predict experimental data, particularly at non-dilute ($>1$~mM).

To address these shortcomings, we propose to modify the PNPNSE to include steric effects due to ion crowding, 
Because the PNP equations do not take the size of the ions into account, they tend to overscreen locations 
with high charge densities resulting in unrealistically high ion concentrations\cite{Corry-2000}. This 
problem can be addressed by the addition of a nonlinear flux term to the NPE that poses an upper limit to the 
concentration. We opted for the size-modified PNP (SMPNP) framework developed by Lu and Zhou,\cite{Lu-2011}
\begin{multline}
\nabla\cdot\flux_{i} = \nabla \cdot \Big[
- \diffusion_{i}\nabla\concentration_{i}
- \chargen_{i}\mobility_{i}\concentration_{i}\nabla\potential
+ \velocity\concentration_{i} \\
- \beta_{i} \concentration_{i} \Big] = 0
\end{multline}
with steric factor
\begin{equation}
\beta_{i} =
\frac{
  \ionsize_{i}^3/\ionsize_{0}^3 \displaystyle\sum_{l} \ionsize_{l}^3 \nabla \concentration_{l}
}{
  1 - \displaystyle\sum_{l} \avogadro \ionsize_{l}^3 \concentration_{l}
}
\end{equation}
where $\ionsize_{i}$ and $\ionsize_{0}$ represent the maximum packing radius in a cubic lattice for ions and 
solvent molecules, respectively. The radii were set to $0.5$~nm for both \Na\ and \Cl\ and to $0.311$~nm for 
water, corresponding to maximum concentrations of $13.3$~M and $55.2$~M, respectively. 

\subsection{Transport of ions through ClyA}\label{sect:ion_transport}

\begin{figure*}[htbp]
	
	\centering
	
	% figure path
	\includegraphics[scale=1]{figures/pdf/fig_conductance}
	
	% caption
	\caption
	[\textbf{Conductance characteristics of single ClyA nanopores at different ionic strengths.}]
	{
		\textbf{Conductance characteristics of single ClyA nanopores at different ionic strengths.}
		(a) Subset of the IV curves that were measured experimentally (\graphmarker{circle}{black}) and simulated using the classical PNP-NS (\graphline{solid}{graphgreen}) and extended PNP-NS (\graphline{solid}{graphpurple}) equations  at \SIlist[list-units=single]{50;150;500;3000}{\milli\Molar}.
		Contour plots showing the conductance ($G = I/V$) landscape of ClyA-AS vs. bias voltage and salt concentration (\ce{NaCl}) (b) as measured experimentally with single channel recordings at \SI{298(1)}{\kelvin} and 
		(c) as calculated by our simulation (left: PNP-NS, right: ePNP-NS).
		(d) Contour plots showing the of percentage relative error of the simulated conductance landscape compared to the experimental one (left: PNP-NS, right: ePNP-NS).
		The classical PNP-NS equations significantly overestimate the conductance under all conditions, particularly at high salt concentrations (\SI{>100}{\percent}).
		In contrast, the errors of the ePNP-NS model are generally below \SIrange{5}{10}{\percent}.
		The error is highest in the low concentration regime at positive bias voltages.
	}
	
	% label
	\label{fig:conductance}
	
\end{figure*}

The ionic current flowing through a nanopore is determined by the geometrical features the pore and the 
physical properties of the electrolyte that surrounds it. To accurately predict the ionic current through a 
nanopore, a simulation should thus contain an adequate representation of its most salient geometrical 
features, and accurately portrait all relevant transport phenomena. The parameters utilized to describe these 
processes depend on the local conditions inside the electrolyte, such as the ion concentrations and proximity 
to the nanopore, and hence require position dependent values. We implemented such a self-consistent 
parametrization in our ePNP-NS model with dependencies on both the average ion concentration $\avionconc$ and 
distance from the pore wall $\walldistance$. We compared the simulated ionic transport properties -- in terms 
of current, conductance, rectification and ion selectivity -- with those obtained experimentally through 
single channel recordings.

%(i.e. shape and charge distribution),
%(i.e. diffusion, electromigration and convection).
%(i.e. ion diffusion coefficients and mobilities; solvent viscosity, density and relative permittivity)

\paragraph{Ionic current.}
Using both PNP-NS and ePNP-NS, we simulated the current-voltage (IV) curves of ClyA and compared it with the 
experimentally measured values for applied bias voltages between $-150$ and $+150$~mV and for reservoir salt 
concentrations of $0.05$, $0.15$, $0.5$, $1$ and $3$~M (\cref{fig:current-voltage_curves}). Over the entire 
investigated concentration range, the IVs from the ePNP-NS model closely correspond to the measured values 
($<5\%$ error). In contrast, the PNP-NS equations significantly overestimate the ionic current for all 
measured concentrations. The PNP-NS model performs the worst at high salt concentrations, as is to be 
expected since it uses parameters that are only valid at `infinite dilution'. For ePNP-NS the error is 
largest at lower concentrations, with slightly smaller and larger currents at negative and positive bias 
voltages, respectively. Their cause is likely to be found in the under- and overestimation of the cation 
concentration at these respective voltages.\todo{ref} Another possibility could also be that our mobility 
model, which makes use of electroneutral `bulk' values for unconfined ions, begins to break down under these 
conditions.\cite{Duan-2010}

Regardless of these minor discrepancies, it is clear that our simplified, 2D-axisymmetric model, in 
conjunction with the ePNP-NS equations, is able to accurately compute the ionic current through ClyA. Hence, 
this methodology should be applicable to other small nanopores ($r_{\text{pore}}\approx1.5$~nm) with complex 
geometries and charge distributions, over the entire experimentally accessible concentration and voltage 
ranges.

% Difference in conductance mechanism between high and low salt
\paragraph{Ionic conductance.}
To further investigate ClyA's ionic transport mechanism, we computed the the ionic conductance 
$\conductance=\current / \vbias$ and the resulting concentration-voltage landscape can be found in 
\cref{fig:conductance_contourmap_epnp}. The straight slopes of the concentration-conductivity log-log plot at 
$-150$~mV exhibit a clear `kink' at $\approx0.15$~M, suggesting the presence of two different conductance 
regimes. At low ionic strengths ($0.005\le\cbulk\le0.15$~M) and high negative bias voltages ($>-100$~mV) 
fitting with the power law function $\conductance(\cbulk)=a\cbulk^{\gamma}$ yields an exponent that 
asymptotically approaches $\gamma=1/2$. This suggests that the transport mechanism is dominated by the charge 
density inside the pore, which is determined by the electrical double layer which as well has a square-root 
dependence on the bulk concentration $\cbulk$.\cite{Uematsu-2018} Under all other conditions, $\gamma$ yields 
values between $0.5$ and $?$



($0.3\le\cbulk\le1.5$~M) 
 yielded exponents $\gamma$ of 
$0.4984$~$\text{M}^{-1}$ and $1.000$~$\text{M}^{-1}$, respectively. while the linear dependency on $\cbulk$ 
is commonly observed at high salt concentrations\todo{ref}, the square-root dependency at low salt suggests a 
transport mode dominated by the surface charge density due to the overlapping of the electrical double 
layers. These mechanisms are less well defined at high positive bias voltages, however, as 
fitting yielded $\gamma$ value of $0.614$~$\text{M}^{-1}$ and $0.787$~$\text{M}^{-1}$ for low and high 
concentrations, respectively. Hence, while there are clearly two distinct ion transport modes at high 
negative bias voltages, the converging exponents at high positive voltages suggest a `hybrid' mode with 
characteristics of both regimes.

\paragraph{Ionic current rectification.}
A nanopore exhibits ionic current rectification (ICR) when the conductance for a given bias voltage magnitude 
is higher when it is applied to one side of the nanopore compared to the other. Here, we define the ICR as
\begin{equation}
  \icr(\vbias) = \displaystyle\frac{\conductance(+\vbias)}{\conductance(-\vbias)}
\end{equation}
i.e. the ratio between the conductance at positive and negative bias voltages of equal magnitude. 
ICR is a phenomenon often observed in nanopores that are both charged and contain a degree of geometrical 
asymmetry along the central axis of the pore. With it's high negative charge ($-72~\elementarycharge$ at 
pH~7.5) and different \cis\ ($\approx3.3$~nm) and \trans\ ($\approx6$~nm) entry diameters, ClyA fulfills 
both conditions and hence exhibits a strong degree of rectification. The simulated concentration-voltage 
landscape of $\icr$ was plotted in \cref{fig:conductance_rectification_contourmap_epnp} (left), together with 
cross-sections at $50$, $100$ and $150$~mV (right) for comparison with the experimentally measured values. 
Starting at $0.005$~M, $\icr$ rises rapidly with concentration until a maximum is reached at $\approx0.15$~M, 
after which it falls towards unity at $5$~M. While a good qualitative agreement is found with the 
experimental data over the entire range, $\icr$ matches quantitatively only for $\cbulk > 0.5$~M. In general, 
the simulation overestimates the rectification, particularly at the maximum and at high bias voltages, with 
$\icr$ values of $1.20$, $1.43$ and $1.69$ for the simulation and $1.14$, $1.28$ and $1.46$ for the 
experiment at $50$, $100$ and $150$~mV, respectively.

\paragraph{Ion selectivity.}
The ion selectivity of a nanopore determines the preference with which it transports one ion type over the 
other. Here, we represent the ClyA's ion selectivity by the apparent \ce{Na+} transport number
\begin{align*}
  \transportn_{\ce{Na+}} =
  \displaystyle\frac{\conductance_{\ce{Na+}}}{\conductance} =
  \displaystyle\frac{\conductance_{\ce{Na+}}}{\conductance_{\ce{Na+}} + \conductance_{\ce{Cl-}}},
\end{align*}
i.e. the fraction of the total current that is carried by \ce{Na+} ions 
(\cref{fig:transport_number_contourmap_epnp}). As expected from its negatively charged interior, ClyA is 
cation selective ($\tna > 0.5$) for all investigated voltages up to a reservoir salt concentration of 
$\approx2$~M, after which it drops below $0.5$ to a minimum value of $0.45$ at $5$~M. The latter is 
$\approx1.27\times$ its value in bulk, indicating that even at saturation ClyA is preferential towards 
cations. Between $0.05 < \cbulk < 1$~M, $\tna$ decays logarithmically ($\tna(\cbulk)=a+b\log\cbulk$) with 
slopes $b=-0.40$ and $-0.29$ for $-150$ and $+150$~mV, respectively.
\todo{What do these different slopes mean?}
At low concentrations and high applied bias voltages (below $10$~mM and $50$~mM for $+150$ and $-150$~mV, 
respectively), $\tna$ is observed to exceed unity. Under these conditions, the convective flux of \Cl\ 
becomes slightly larger than its opposing electromigratory flux, resulting in an overall negative 
contribution of \Cl\ to the ionic current.

\subsection{Ion concentration distribution}\label{sect:ion_concentration}
\begin{figure*}[htbp]
\centering
\begin{minipage}[t]{8.2cm}
\begin{subfigure}[t]{8.2cm}
	\centering
	\caption{}\vspace{-3mm}\label{fig:concentration_contours}
	\includegraphics[scale=1]{figures/concentration/concentration_contours}
\end{subfigure}
\begin{subfigure}[t]{8.2cm}
  \centering
  \caption{}\vspace{-3mm}\label{fig:concentration_radial_profiles}
  \includegraphics[scale=1]{figures/concentration/concentration_radial_profiles}
\end{subfigure}
\begin{subfigure}[t]{8.2cm}
	\centering
	\caption{}\vspace{-3mm}\label{fig:concentration_pore_average_vs_concentration}
	\includegraphics[scale=1]{figures/concentration/concentration_pore_average_vs_concentration}
\end{subfigure}
\end{minipage}

% caption
\caption
[\textbf{Ion concentration distribution inside ClyA.}]
{
\textbf{Ion concentration distribution inside ClyA.}
(\subref{fig:concentration_contours})
Cross-section contour plots of the \ce{Na+} and \ce{Cl-} concentrations inside the pore relative to the bulk 
value of $0.15$~M at bias voltages $-100$ and $+100$~mV. These plots reveal the local concentration changes 
near charged residues, particularly in the highly negatively charged constriction. While conditions inside 
the lumen of the pore are generally close to bulk, a strong depletion of \ce{Cl-} can occur under high 
negative bias voltages.
(\subref{fig:concentration_radial_profiles})
Relative \ce{Na+} and \ce{Cl-} concentration profiles (to $\concentration_\text{bulk}=0.15$~M) along the 
radius of the pore at the center of the constriction ($z=-0.3$~nm) and the lumen ($z=5$~nm) for $-100$ and 
$+100$~mV, showing the formation of the electrical double layer.
(\subref{fig:concentration_pore_average_vs_concentration})
Relative \ce{Na+} and \ce{Cl-} concentrations averaged over the entire pore in function of bulk salt 
concentration. ClyA's negatively charged interior results in the enhancement and the depletion of 
respectively \ce{Na+} and \ce{Cl-} concentrations inside the pore, particularly at lower ionic strengths. 
Both effects diminish with increasing concentrations and bulk-like conditions are observed for both ions at 
$\approx0.5$~M. Interestingly, the \ce{Cl-} concentration increases rapidly at higher positive bias voltages, 
resulting in bulk-like conditions at $+150$~mV for $\concentration_\text{bulk}>0.05$~M.
}

\label{fig:concentration}

\end{figure*}
To further elucidate the origin of the current rectification and ion selectivity, we evaluated  the effect of 
reservoir salt concentration and the bias voltage on the distribution of cation and anion densities inside 
ClyA (\cref{fig:concentration}) and the accumulation of mobile charges as a result of their asymmetry 
(\cref{fig:ion_charge_density}).

\paragraph{Relative cation and anion concentrations.}
At high reservoir concentrations ($\cbulk > 1$~M), the lumen of the pore becomes fully screened from the 
negative charges that line it, leading to bulk-like conditions ($\pav{\ci/\cbulk}{p}\approx1$) for both 
positive and negative ions. At lower concentrations however, the reduced screening gives rise to an 
overlapping electrical double layer, which results in the enhancement of cations and the depletion of anions. 
The magnitude of these effects appears to follow a power law at low concentrations ($\cbulk=0.005$ to 
$0.05$~M), with exponents of $-0.910\pm0.010$ and $0.778\pm0.015$ for \Na\ and \Cl\, respectively. The \Na\ 
concentrations exhibit at most a 2 to 3-fold increase when $\vbias$ changes from $\vbias=-150$ to $+150$~mV. 
In contrast, the number of \Cl\ ions inside pore between those 2 bias voltages can differ more than 10-fold 
at reservoir concentrations below $0.1$~M, with a high degree of depletion at negative voltages.

The contourplots of the relative ion concentrations ($\ci/\cbulk$) at $\cbulk=0.15$~M 
(\cref{fig:concentration_contours}) reveal that the \trans\ constriction ($1.85\text{~nm}<z<1.60\text{~nm}$) 
remains depleted of anions and enhanced in cations for both $\vbias=-100$ and $+100$~mV. This is not the 
case in the lumen ($1.60\text{~nm}<z<12.25\text{~nm}$), where the \Na\ concentration is bulk-like and 
enhanced for $\vbias<0$ and $\vbias>0$, respectively. Conversely, the number of \Cl\ ions becomes more and 
more depleted in the lumen for increasing negative bias magnitudes, while it is virtually bulk-like at higher 
positive bias voltages. This is further exemplified by the radial profiles through the middle of the 
constriction ($z=-0.3$~nm) and the lumen ($z=5$~nm) (\cref{fig:concentration_radial_profiles}) which also 
clearly show the formation of the electrical double layer.

Note that the figures given above were obtained from a nanoscale continuum steady-state simulation, and hence 
represent a time-averaged situation (typically on the order of $10$ to $100$~ns).\todo{REF}

% power law slopes at 0 mV (0.005 to 0.05M)
% cpos
% a=0.276±0.014, b=-0.910±0.010
% cneg
% a=3.236±0.166, b=0.778±0.015

\paragraph{Ion charge density.}
\begin{figure}[!htb]
\centering
\begin{minipage}[t]{8.2cm}
\begin{subfigure}[t]{8.2cm}
	\centering
	\caption{}\vspace{-3mm}\label{fig:ion_charge_density_contours}
	\includegraphics[scale=1]{../figures/charge_density/ion_charge_density_contours}
\end{subfigure}
\begin{subfigure}[t]{8.2cm}
  \centering
  \caption{}\vspace{-3mm}\label{fig:ion_charge_density_radial_profiles}
  \includegraphics[scale=1]{../figures/charge_density/ion_charge_density_radial_profiles}
\end{subfigure}
\begin{subfigure}[t]{8.2cm}
	\centering
	\caption{}\vspace{-3mm}\label{fig:ion_charge_pore_bulk_surface_total_vs_concentration}
	\includegraphics[scale=1]{../figures/charge_density/ion_charge_pore_bulk_surface_total_vs_concentration}
\end{subfigure}
\end{minipage}

% caption
\caption
[\textbf{Ion space charge density distribution inside ClyA.}]
{
\textbf{Ion space charge density distribution inside ClyA.}
(\subref{fig:ion_charge_density_contours}) Cross-section contour plots of the ion space charge density
($\scdion$), expressed as number of elementary charges per \si{\cubic\nano\meter}, at \SI{0}{\mV} applied bias
voltage and for salt concentrations \SIlist{0.005;0.05;0.5;5}{\Molar}.
(\subref{fig:ion_charge_density_radial_profiles}) Radial cross-sections of the $\scdion$ at the center of the
constriction ($z=\SI{-0.3}{\nm}$) and the lumen ($z=\SI{5}{\nm}$) of ClyA. The vertical line represents the
the division between ions in the `bulk' ($d>\SI{0.5}{\nm}$) of the pore and those located near its surface
($d\le\SI{0.5}{\nm}$).
(\subref{fig:ion_charge_pore_bulk_surface_total_vs_concentration}) The average number of ionic charges inside
the pore $\pav{\Qion}{p}$, is distributed between the those close to the pore's surface $\pav{\Qion}{s}$, i.e.
within \SI{0.5}{\nm} of the wall, and those in the `bulk' of the pore's interior $\pav{\Qion}{b}$.
}\label{fig:ion_charge_density}

\end{figure}

The formation of an electrical double layer inside the pore and the resulting asymmetry in the cation and 
anion concentrations results in a net charge density inside the pore ($\scdion$, \cref{eq:scdion}). At 
reservoir concentrations $\le0.5$~M, the electrical double layer (EDL) inside the pore is very diffuse and 
overlaps significantly (\cref{fig:ion_charge_density_contours}, 3 leftmost panels). Moreover, the 
absence of anions prevents the formation of any significant negative charge densities next to the few 
positively charged residues lining the pore walls. The situation at high salt concentrations (e.g. $5$~M) is 
very different, with almost no charge density within the `bulk' of the pore lumen ($\walldistance\ge0.5$~nm), 
but 
with 
pockets of highly charged and alternating pockets of positive and negative charge densities close to the 
nanopore wall (\cref{fig:ion_charge_density_contours}, rightmost panel). This sharp confinement is shown 
clearly by the radial density profiles (\cref{fig:ion_charge_density_radial_profiles}) through the 
constriction ($z=-0.3$~nm) and the lumen ($z=5$~nm).

Integration of $\scdion$ over `bulk' ($d\ge0.5$~nm) and surface ($d<0.5$~nm) volumes inside the pore yields 
respectively $\pav{\Qion}{b}$ and $\pav{\Qion}{s}$, i.e. the average number of mobile charges present inside 
those locations (\cref{fig:ion_charge_pore_bulk_surface_total_vs_concentration}). While total number of 
charges inside the pore $\pav{\Qion}{p} = \pav{\Qion}{b} + \pav{\Qion}{s}$ rises appreciatively 
with increasing reservoir concentration, the majority of these additional charges are confined to the surface 
of the pore. Up until $0.1$~M, $\pav{\Qion}{p}$ is distributed equally between the surface and bulk layers 
($\approx+27$ and $\approx+22$~$\echarge$, respectively). At higher concentrations, the number of ions in the 
surface layer rises (to $+58$~$\echarge$ at $5$~M), while that in the bulk pore diminishes ($+0$~$\echarge$ 
at $5$~M). $\pav{\Qion}{p}$ also depends on the applied bias voltage, as it is $+10$ to $+15$~$\echarge$ 
higher at $+150$~mV compared to $-150$~mV.

\subsection{Electrostatic potential}\label{sect:electrostatic_potential}

\begin{figure*}[!htb]
  \centering
  \begin{minipage}[t]{18.25cm}
    \begin{subfigure}[t]{2.5cm}
      \centering
      \caption{}\vspace{-3mm}\label{fig:potential_clya_charges}
      \includegraphics[scale=1]{../figures/potential/potential_clya_charges}
    \end{subfigure}
    \hspace{-0.6cm}
    \begin{subfigure}[t]{11.5cm}
      \centering
      \caption{}\vspace{-3mm}\label{fig:potential_contours}
      \includegraphics[scale=1]{../figures/potential/potential_contours_0mV}
    \end{subfigure}
    \hspace{-0.4cm}
    \begin{subfigure}[t]{4cm}
      \centering
      \caption{}\vspace{-3mm}\label{fig:potential_radial_averages}
      \includegraphics[scale=1]{../figures/potential/potential_radial_averages_0mV}
    \end{subfigure}
  \end{minipage}
\centering

% caption
\caption
[\textbf{Electrostatic potential inside ClyA.}]
{
\textbf{Electrostatic potential inside ClyA.}
(\subref{fig:potential_clya_charges}) A single subunit of ClyA in which all amino acids with a net charge and
whose side chains face the inside of the pore, i.e. that contribute the most to the electrostatic potential,
are highlighted. Negatively (Asp+Glu) and positively and positively (Lys+Arg) charged residues are colored in
red and blue, respectively.
(\subref{fig:potential_contours}) Electrostatic potential landscape inside ClyA due to its fixed charges (i.e.
at $\vbias=\SI{0}{\mV}$) at several key  concentrations
($\cbulk=\text{\SIlist{0.005;0.05;0.15;0.5;5}{\Molar}}$). Note that even at physiological salt  concentrations
($\cbulk=\SI{0.15}{\Molar}$), the negative electrostatic potential extends significantly inside the lumen
($1.60<z<\SI{12.25}{\nm}$), and even more so inside the \trans\ constriction ($1.85<z<\SI{1.60}{\nm}$). For
the former, localized influential negative `hotspots' can be found in the middle ($4<z<\SI{6}{\nm}$) and at
the \cis\ entry ($10<z<\SI{12}{\nm}$).
(\subref{fig:potential_radial_averages}) Radial average of the electrostatic potential along the length of the
pore ($\radpot$) for the same concentrations as in \subref{fig:potential_contours}. While the lumen of the
pore becomes almost fully screened for $\cbulk>\SI{0.5}{\Molar}$, the constriction still retains some of its
negative influence at \SI{5}{\Molar}.
}\label{fig:potential}
\end{figure*}


In nanometer sized pores, the modification of the electrostatic potential distribution by the charged 
residues embedded in the interior walls of the protein (\cref{fig:potential_clya_charges}), significantly 
influences the transport of ions and water molecules.\cite{Bhattacharya-2011}\todo{more ref}
In the following section we aim to describe the most salient features of this modulated potential and its 
relative importance over the entire investigated concentration range.

\paragraph{Global electrostatic potential.}
The electrostatic potential distribution at $\vbias=0$~mV clearly shows the influence of the nanopore's fixed 
charge distribution (\cref{fig:potential_contours}).
At low ionic strengths ($\cbulk < 0.15$~M), the lack of sufficient ionic screening results in relatively 
uniform negative potentials in both the lumen and the constriction of the pore ($-90$~mV and $-53$~mV at 
respectively $0.005$ and $0.05$~M). These values significantly exceed the single ion thermal voltage 
$\kTe=25.7$~mV and hence effectively prohibit anions from entering the pore.
At intermediary concentrations, ($0.15 \le \cbulk < 1$~M) the influence of the negative charges increasingly 
confines itself to several `hotspots` near the nanopore walls, most notably at entry of the pore 
($10<z<12$~nm), in the middle of the lumen ($4<z<6$~nm). While the potential at the center of the pore is 
close to $\approx0$~mV for the lumen, it remains uniformly negative for the constriction.
Finally, at high concentrations ($\cbulk \ge 1$~M) the potential drops to $\approx0$~mV over the entire lumen 
of the pore, with only a small negative potential inside the constriction.

\paragraph{Radially averaged potential profile.}
We further quantified the electrostatic potential at $0$~mV applied bias voltage along the nanopore's length 
by computing its radial average 
\begin{align}
\radpot=\displaystyle\frac{1}{\pi \cdot R(z)^2}\int_{0}^{R(z)}\potential(r,z) \cdot 2 \pi \cdot r dr \text{,}
\end{align}
where $R(z)$ is taken as the pore radius for $-1.85\le z \le12.25$~nm, $2$~nm for $z<-1.85$~nm and $4$~nm for 
$z>12.25$~nm (\cref{fig:potential_radial_averages}).
The electrostatic potential at the \cis\ entry ($z \approx 10$~nm) is dominated by the acidic residues D114, 
D121 and D122, resulting in a rapid reduction of $\radpot$ upon entering the pore. Next, $\radpot$ remains 
approximately constant up until the middle of the lumen ($z \approx 5$~nm), where the next set of negative 
residues, namely E53, E57 and D64, reduce the potential even further. After a brief rise, $\radpot$ then 
attains its maximum amplitude inside the \trans\ constriction ($z \approx 0$~nm) due to the close proximity 
of the amino acids E7, E11, E18, D21 and D25. For example, at physiological salt concentrations ($0.15$~M), 
$\radpot$ has values of $-19$, $-29$ and $-57$~mV ($-0.74$, $-1.1$ and $-2.2$~$\kTe$) at the \cis\ entry, 
lumen center and \trans\ constriction, respectively. Their magnitude increases approximately three-fold at 
$0.005$~M, and drops to $\approx10$~\% at $5$~M. A summary of these values can be found in 
\cref{suppinfo:tab:radial_potential}.

\subsection{Electrostatic energy}

and to link it back to the observed ionic conductance properties through the average electrostatic energy of 
cations and anions (\cref{fig:potential_energy}).

Due to it's small size, the majority of an applied bias potential drops along the length of the nanopore, 
giving rise to a `tilted' version of the electrostatic potential at $0$~mV. ClyA's geometric asymmetry 
results in different electrostatic landscapes for positive and negative bias voltages. Since electromigration 
is the primary contributor to the ionic current, a proper understanding of the electrostatic energy barriers 
the ions must overcome when traversing the pore should provide a more quantitative explanation for the origin 
of ClyA's rectification, ion selectivity and asymmetric ion concentrations. To this end, we computed the 
radially averaged electrostatic energy for monovalent ions  $\radenergy = \chargen_{i} \echarge \radpot$ 
(\cref{fig:potential_energy_radial_averages}) and the magnitude of the energy barriers at the \trans\ 
constriction ($\deltaEt$, \cref{fig:potential_energy_trans_barrier}).

\begin{figure}[!ht]
  \centering
  \begin{subfigure}[t]{8.25cm}
    \centering
    \caption{}\vspace{-5mm}\label{fig:potential_energy_radial_averages}
    \includegraphics[scale=1]{figures/potential_energy/potential_energy_radial_averages}
  \end{subfigure}
  \begin{subfigure}[t]{8.25cm}
    \centering
    \caption{}\vspace{-3mm}\label{fig:potential_energy_trans_barrier}
    \includegraphics[scale=1]{figures/potential_energy/potential_energy_trans_barrier}
  \end{subfigure}

\caption
[\textbf{Radially averaged electrostatic energy for single ions.}]
{
\textbf{Radially averaged electrostatic energy for single ions.}
(\subref{fig:potential_energy_radial_averages}) Approximate electrostatic energy landscape for single ions
$\radenergy = \chargen_{i} \echarge \radpot$ as calculated directly from the radial electrostatic potential at
\SIlist{+150;-150}{\mV} applied bias voltages for monovalent cations and anions. The grey arrows indicate the
direction in which the ions must travel in order  to contribute positively to the ionic current.
(\subref{fig:potential_energy_trans_barrier}) Height of the electrostatic energy barrier ($\Delta
E_{\text{B},i}$) at the \trans\ constriction. Note that  $\Delta E_{\text{B},i}$ is much higher for negative
voltages and rises logarithmically at lower  concentrations. The divergence between \SI{+0}{\mV} and
\SI{-0}{\mV} for $\cbulk<\SI{0.3}{\Molar}$ highlights the difference in barrier height when traversing the
pore from \cis\ to \trans\ or vice versa.
}\label{fig:potential_energy}

\end{figure}


\paragraph{Energy landscape at $\mathbf{+150}$~mV.}
At positive bias voltages, cations traverse the pore from \trans\ to \cis\ 
(\cref{fig:potential_energy_radial_averages}, top left). Upon entering the \trans\ constriction, the 
electrostatic energy drops dramatically, followed by a relatively flat energy landscape with a small barrier 
for entry in the lumen at $z\approx1.6$~nm . At very low ionic strengths ($\cbulk<0.025$~M), the energy at 
\trans\ is significantly lower than that the final energy in the \cis\ compartment (e.g. 
$\Delta\radenergy>2$~$\kT$ at $0.005$~M), forcing the ions to accumulate inside the pore. At higher 
concentrations ($\cbulk>0.05$~M), the increased screening smooths out the potential drop inside the pore, 
allowing the cations to migrate unhindered across the entire length of the pore.

Anions travelling from \cis\ to \trans\ must overcome energy barriers when both entering and exiting the pore 
(\cref{fig:potential_energy_radial_averages}, bottom right). The former prevents \Cl\ ions from entering the 
pore, but is only relevant at lower ionic strengths ($\cbulk<0.05$~M), since its magnitude is attenuated 
strongly with increasing salt concentration (from $\approx1.7$~$\kT$ at $0.005$~M to $\approx0.5$~$kT$ at 
$0.05$~M). Once inside the lumen, anions can move relatively unencumbered to the \trans\ constriction, where 
they face the second, more significant energy barrier, which prevents them from fully translocating and 
causes them to accumulate inside the lumen. Ions that do pass the final barrier are rewarded with a large 
reduction of their electrostatic energy, as the bulk of the voltage drop occurs only after the \trans\ 
constriction. As with the cations, an increase in the ionic strength significantly reduces the 

\paragraph{Energy landscape at $\mathbf{-150}$~mV.}
The electrostatic energy of cations traversing the pore at negative voltages (\cis\ to \trans) drops 
gradually throughout the lumen of the pore up until the \trans\ constriction
(\cref{fig:potential_energy_radial_averages}, top right). This their efficient removal of cations 
from the pore lumen, and results in a lower concentration compared to positive voltages 
(\cref{fig:concentration}). To fully exit from the pore, however, cations must overcome a large energy 
barrier, which reduces the nanopore's ability to conduct cations compared to positive potentials and hence 
contributes to the ion current rectification.
 
The situation for anions at negative bias voltages (\trans\ to \cis)is very different as they must travel 
from  (\cref{fig:potential_energy_radial_averages}, bottom left), where they must overcoming the 
large energy barrier at the constriction (\cref{fig:potential_energy_trans_barrier}, red curves). This 
effectively prevents them from entering the pore, explaining why ClyA is more ion selective at negative 
bias voltages. Once across the barrier, the continuous drop of electrostatic energy towards the \cis\ entry 
serves as a strong driving force to deplete the entire lumen of anions, resulting in much lower 
concentrations compared to positive voltages.

\paragraph{Concentration and voltage dependencies of the \trans\ energy barrier.}
Many biological nanopores contain constrictions that play crucial roles in shaping their ionic 
conductance properties.\cite{Maglia-2008,Franceschini-2016,Huang-2017} The reason for this is two-fold, 1) 
the narrowest part dominates the overall resistance of the pore and 2) confinement of charged residues 
results in much large electrostatic energy barriers. With its highly negatively charged \trans\ 
constriction, ClyA  affinity for transport of anions is diminished while that for cations in enhanced, even 
at high ionic strengths (\cref{fig:transport_number_contourmap_epnp}).\cite{Soskine-2013}
To further elucidate the significance of the \trans\ electrostatic barrier ($\deltaEt$), we quantified its 
height of at positive and negative voltages as a function of the salt concentration 
(\cref{fig:potential_energy_trans_barrier})

T the effect thisresulting in high 


We estimated the magnitude of the 
For concentrations between $0.01$ and $1$~M, however, there is a clear difference in ion selectivity between 
positive and negative bias voltages. At $0.1$~M, for example, the ion selectivity 
$\transportn_{\Na}$ increases $3.3$-fold from $5.7$ at $+150$~mV 
to $19$ at $-150$~mV. The energy barriers under these conditions are This observation can be explained by the 
difference in electrostatic energy barrier height in the constriction at positive and negative voltages 
(\cref{fig:potential_energy_trans_barrier}). Again, at $0.1$~M the 


cation selectivities ($S_{i}=\transportn_{i}/(1-\transportn_{i})$).  

%0.1M, -150mV: 1.8943409
%0.1M, +150mV: 0.90132562

% at concentrations below $0.1$~M 
%($\transportn_{\Na}$.  The magnitude of th
%is  , contain a constriction with an electrostatic energy barrier that dominates the 
%pore's ionic transport properties. At negative voltages, 
%represents the electrostatic barrier that cations and anions must overcome to respectively exit and enter 
%the 
%pore, and vice versa at positive voltages. In all cases, the additional `tilting' of the energy landscape at 
%increased bias magnitudes reduces the barrier height. 
%. It falls logarithmically from $\approx5$~$\kT$ at $0.005$~M to $\approx0$~$\kT$ at $1$~M, wit In 
%contrast, at positive voltages $\deltaEt$ starts from a value of $\approx2$~$\kT$ at $0.005$~M the remains 
%where it becomes $\approx0$. 


% cis
% 0.05 : -1.33 = -33.8 mV
% 0.15 : -0.75 = -19.1 mV
% 0.50 : -0.39 = -9.9 mV
% 5.00 : -0.07 = -1.8 mV
% 

\subsection{Electro-osmotic flow}
\begin{figure*}[htbp]
  \centering
  \hspace{-2cm}
  \begin{minipage}[t]{5.5cm}
    \begin{subfigure}[t]{5.5cm}
      \centering
      \caption{}\vspace{-3mm}\label{fig:flow_contour}
      \includegraphics[scale=1]{figures/flow/flow_contour_500mM}
    \end{subfigure}
    \begin{subfigure}[t]{5.5cm}
      \addtocounter{subfigure}{1}
      \vspace{3mm}
      \centering
      \caption{}\vspace{-3mm}\label{fig:flow_constriction_profiles}
      \includegraphics[scale=1]{figures/flow/flow_constriction_profiles}
    \end{subfigure}
  \end{minipage}
  \begin{subfigure}[t]{4cm}
    \addtocounter{subfigure}{-2}
    \centering
    \caption{}\vspace{2mm}\label{fig:flow_constriction_contour}
    \includegraphics[scale=1]{figures/flow/flow_constriction_contour}
  \end{subfigure}
  \begin{minipage}[t]{4cm}
    \begin{subfigure}[t]{4cm}
      \addtocounter{subfigure}{1}
      \centering
      \caption{}\vspace{-5mm}\label{fig:flow_conductance_vs_voltage}
      \includegraphics[scale=1]{figures/flow/flow_conductance_vs_voltage}
    \end{subfigure}
    \begin{subfigure}[t]{4cm}
      \vspace{2mm}
      \centering
      \caption{}\vspace{-5mm}\label{fig:flow_conductance_vs_concentration}
      \includegraphics[scale=1]{figures/flow/flow_conductance_vs_concentration}
    \end{subfigure}
    \begin{subfigure}[t]{4cm}
      \vspace{2mm}
      \centering
      \caption{}\vspace{-5mm}\label{fig:flow_conductance_rectification_vs_concentration}
      \includegraphics[scale=1]{figures/flow/flow_conductance_rectification_vs_concentration}
    \end{subfigure} 
  \end{minipage}
\centering

% caption
\caption
[\textbf{Concentration and voltage dependency of the electro-osmotic flow inside ClyA.}]
{
(a) Representative contour plots of the electro-osmotic flow velocity under low (\SI{50}{\milli\Molar}) and
high (\SI{500}{\milli\Molar}) salt concentrations at negative (\SI{-150}{\milli\volt}) and positive (\SI{+150}{\milli\volt}) bias potentials.
The arrows indicate the direction of the fluid while the lines show the shape of the velocity field.
(b) Velocity profiles at the centre of the pore for bulk concentrations of \SI{50}{\milli\Molar}
(\SI{+150}{\milli\volt}: \graphline{solid}{graphgreen} and\SI{-150}{\milli\volt}: \graphline{dotted}{graphgreen}) 
and \SI{500}{\milli\Molar} (\SI{+150}{\milli\volt}: \graphline{solid}{graphpurple} and 
\SI{-150}{\milli\volt}: \graphline{dotted}{graphpurple}) at negative and positive bias potentials.
Negative values indicate flow from \textit{cis} to \textit{trans} (i.e. top to bottom) and vice versa.
The velocity from the \textit{cis} side rises gradually to a plateau value in the lumen.
Close to the trans constriction, the fluid velocity increases rapidly to a maximum value,
after which it falls again at the same rate upon exit from the pore.
(c) Total electro-osmotic flow rate ($Q_\textrm{eo}$, \SI{}{\cubic\nano\metre\per\nano\second}) vs. bulk salt concentration for positive (top) and negative (bottom) bias potentials.
In the low concentration regime, $Q_\textrm{eo}$ increases rapidly with concentration to a maximum at approximately \SI{500}{\milli\Molar}, followed by by a gradual decline.
(d) The rectification of the electro-osmotic flow rate ($R^+=Q_\textrm{eo}^+/Q_\textrm{eo}^-$) plotted against the concentration.
$R^+$ shows a maximum at \SI{50}{\milli\Molar}, after which it falls to reach unity at approximately \SI{500}{\milli\Molar}, regardless of the applied bias.
A minimum is then reached at \SI{1000}{\milli\Molar}, followed by again by a gradual approach to unity at higher concentrations.
}

% label
\label{fig:electro-osmotic_flow}

\end{figure*}
The electro-osmotic flow (EOF) is one of the most important properties of biological nanopores, as it allows 
for the capture of nucleic acids\cite{Wong-2007}, peptides\cite{Huang-2017} and 
proteins
\cite{Soskine-2012,Soskine-2013,VanMeervelt-2014,Soskine-Biesemans-2015,Biesemans-Soskine-2015,Wloka-2017}
irrespective of their charge. In the following section we aim to quantitatively and a qualitatively describe 
the influence of bias voltage and salt concentration on the EOF inside ClyA.

\paragraph{Direction, magnitude and radial velocity profile.}
The direction and magnitude of the EOF in a nanopore is generally determined by respectively the type of 
charges (i.e. negative or positive) and their amount on its walls \todo{ref}. In the case of the 
predominantly negatively charged ClyA, the EOF follows the direction of the cation, i.e. from \cis\ to 
\trans\ for negative, and vice versa for positive bias voltages. Due to the conservation of mass, the EOF 
velocity $U$ is predominantly a function of the nanopore radius. Hence, it is lowest in the lumen of the pore 
and highest inside the \trans\ constriction (\cref{fig:flow_contour}). At the center of the pore, $U$ reaches 
values of \SI{\approx0.07}{\mps} in the lumen and \SI{\approx0.21}{\mps} in the constriction (for 
$\cbulk=0.5$~M and $\vbias=-100$~mV).

At low ionic strengths ($\cbulk<0.5$~M), the no-slip boundary condition on the nanopore walls and relatively 
uniform positive charge density inside the pore result a parabolic radial velocity, similarly to a pressure 
driven flow \todo{ref} (\cref{fig:flow_constriction_contour,fig:flow_constriction_profiles}). As the 
ionic charge density $\scdion$ becomes increasingly confined to nanopore walls ($0.5\le\cbulk<1$~M, 
\cref{fig:ion_charge_density}), the central maximum flattens out, resulting in the `plug' flow profile 
typically observed for electro-osmotically driven flows \todo{ref}. Interestingly, at high salt 
concentrations ($\cbulk\ge1$~M) the velocity at the center of the pore becomes lower than that at the walls.
\todo{why? lack of charge density in the lumen, complete confinement of charges to the wall?}

\paragraph{Water conductance and rectification.}
To more easily compare the total amount of water transported by ClyA between conditions, we computed the 
electro-osmotic conductance $G_{\text{eo}} = Q_{\text{eo}}/V$ with $Q_{\text{eo}}$ the volumetric flow rate 
obtained by integrating the water velocity over the reservoir boundary. \todo{voltage dependency} 
$G_{\text{eo}}$ does not decrease monotonically with increasing salt concentrations, but rather exhibits a 
maximum of \SI{\approx11.5}{\cnmpnspv} at $\approx0.5$~M. Tshow the expected monotonical depends

\cite{Mao-2014,Laohakunakorn-2015}



\cref{fig:flow_conductance_vs_voltage}
\cref{fig:flow_conductance_vs_concentration}


\cref{fig:flow_conductance_rectification_vs_concentration}

\paragraph{Electro-osmotic pressure.}
\cref{suppinfo:fig:pressure}
\cite{Hoogerheide-2014}


%-------------------------------------------------------------------------------
% DISCUSSION
%------------------------------------------------------------------------------
\section{Discussion}\label{sect:discussion}
% Inter-figure observations and discussion

\subsection{The extended PNP-NS equations significantly improve the accuracy of continuum simulations at the 
nanoscale.}

\begin{itemize}
  \item Most important `corrections':
  \subitem concentration dependent ion diffusion coefficient and mobilities
  \subitem wall distance 
\end{itemize}

\subsection{Ionic current rectification is caused by depletion of anions in the nanopore lumen at negative 
bias voltages.}

\subsection{ClyA's ion selectivity is highly concentration and voltage dependent.}
While the ion selectivity can be estimated experimentally using the Goldman-Hodgkin-Katz 
equation (GHKe),\cite{Franceschini-2016,Huang-2017} the asymmetric salt concentrations used to determine the 
nanopore's reversal potential raises the question for which salt concentration this ion selectivity is valid.
Moreover, the many assumptions and simplifications used to derive the GHKe (Nernst-Einstein relation, no 
convective transport, uniform electrical field over the pore) suggest that the ion selectivity calculated for 
larger pores should be considered as a rough estimation only.

\subsection{The electro-osmotic flow exhibits a maximum at 0.5~M due to two competing concentration-dependent 
effects.}

\begin{itemize}
  \item Reduced double layer overlap and the appearance of alternating positive and negative space charge 
  densities at the nanopore walls results in reduced electro-osmotic flow at higher salt concentrations.
  \item Poor electro-static screening of the the highly negatively charged \trans\ constriction 
  results in an body force that opposes the bulk flow of the nanopore and hence a diminishes the 
  electro-osmotic flow magnitude at low salt concentrations.
  \item Both effect are weakest at $\approx0.5$~M, resulting in maximum at that concentration.
\end{itemize}

\section{Conclusions}\label{sect:conclusions}
\todo{conclusions}


\section{Materials and methods}

% was derived an representative axisymmetric outline of the pore, we radially averaged its 3D molecular 
%surface (MSMS\cite{Sanner-1995}) of the equilibrated homology model pore 
%and radially 
%averaged its boundary to obtain a 2D outline.
%
%and fixed charge distribution 
%were derived directly from an equilibrated, full-atom homology model of the 
%pore (MODELLER 9.17,\cite{Sali-1993}, VMD 1.9.4\cite{Humphrey-1996} and NAMD 2.13\cite{Phillips-2005}).
%
%The charge distribution was computed by projecting the partial 
%charge of each atom (obtained with PDB2PQR 2.1.1\cite{Jurrus-2018}), represented by a 2D-Gaussian function, 
%on a 2D grid.


\paragraph{Modeling approach}
Our goal is to model the protein nanopore ClyA  suspended on a lipid bilyer in a 2D-axisymmetric system in an
aqueous solution of NaCl. The lipid bilayer separates two reservoirs which can be electrically biased so that
an ion current flows through the nanopore in between the reservoirs. With no loss of generality we will assume
that the upper cis-reservoir (positive $z$-direction) is grounded, while the lower trans-reservoir is biased
with an applied potential of $\vbias$. \cref{fig:model_geometry} illustrates the geometry of the simulation 
domain.

The construction of a viable modeling approach then decomposes into two separate tasks. First, finding an
averaging scheme for the angular dependence of the charge distribution of the full 3D ClyA structure shown in
\cref{fig:clya_side,fig:clya_top} in order to accurately represent the nanopore in a 2D-axisymmetric 
simulation. Second, defining a viable transport model for the ion species and the flow of the solution 
through the nanopore. This includes the definition of the set of parameters comprising mobilities, diffusion 
coefficients, and effective sizes of the ion species as well as the viscosity, density, and permittivity of 
the solution. In particular, it is necessary to take in consideration the nanofluidics within the pore since 
none of the parameters retain their bulk values in nanoscale constrictions.

\paragraph{Axially Symmetric Model of ClyA}
ClyA is a multimeric protein complex that consists of 12 identical subunits which form a $14$~nm long 
hydrophilic channel through a lipid bilayer (cf.~\cref{fig:clya_side,fig:clya_top}). The pore cavity is lined 
with predominantly negatively charged residues and can be divided in roughly two cylindrical compartments: 
the \cis\ lumen ($\approx5.5$~nm diameter, $\approx10$~nm height) and the \trans\ constriction 
($\approx3.3$~nm diameter, $\approx4$~nm height). Due to this high degree of axial symmetry, we modelled ClyA 
as a solid, cylindrically symmetric dielectric block, embedded within a lipid bilayer which is represented as 
a $2.8$~nm-thick\cite{Kucerka-2011} dielectric disk  and surrounded by a spherical, electrolyte-filled 
reservoir with a radius of $250$~nm 
(cf.~\cref{fig:model_geometry,fig:model_geometry_zoom})\cite{Lu-2012,Pederson-2015}. This approach greatly 
reduces the computational cost compared to a full 3D model, while retaining sufficient molecular details as 
to produce meaningful results.

The nanopore geometry (\cref{fig:model_geometry_vs_wedge}) was derived directly from a full atom model of 
ClyA-AS -- an improved variant of ClyA evolved from the wild-type Cytolysin A from \textit{S. typhii} using 
directed evolution.\cite{Soskine-2013} Briefly, a homology model of ClyA-AS was constructed from the 
wild-type crystal structure (PDB: 2WCD)\cite{Mueller-2009} by introducing the necessary mutations in all 12 
subunits an minimizing the resulting structure (see SI for details). We then computed the protonation state 
of titratable residue at pH~$7.5$ (PROPKA 3.1\cite{Olsson-2011}) and assigned the correct radius and partial 
charge to each atom of the model (PBD2PQR 2.1.1\cite{Jurrus-2018}) as parametrized by the CHARMM36 
forcefield.\cite{Best-2012} The atom radii were then used to compute a 3D-dimensional map of the 
nanopore's molecular surface with a probe radius of $1.4$~\AA, which was radially averaged along the 
z-direction with a custom-built MATLAB script to generate an approximate axisymmetric outline of the 
nanopore. To improve the quality of the final computational mesh, the resulting polygon was then simplified 
manually by removing overlapping and superfluous vertices.

Similar to the geometry, the two-dimensional charge distribution used in this model was also derived directly
from the full atom model (cf.~\cref{fig:model_charge_density}). Inspired by how charges are represented in the
particle mesh Ewald (PME) method,\cite{Aksimentiev-2005} we computed the fixed charge distribution of the
nanopore $\scdpore(r,z)$ by assuming that an atom $i$ of partial charge $\partialcharge_{i}$ at the location
$(x_i, y_i, z_i)$ in the full 3D atomistic pore model will contribute the partial charge
$\partialcharge_{i}/2\pi\sqrt{x_i^2 + y_i^2}$ to a point $(r_i,z_i)$ in the averaged 2D-axisymmetric model.
That means we effectively spread the charge over all angles to achieve axial symmetry. 

Following that, we assume a Gaussian distribution of the space charge density of each atom $i$ around its
respective 2D-axisymmetric coordinates $(r_i,z_i)$. With the atom radius $\atomradius_{i}$ and the sharpness
factor $\sigma = \num{0.5}$, we find for the 2D space charge density
\begin{align}
\label{eq:scdpore}
\scdpore(r,z) = \displaystyle\sum_{i} \dfrac{\echarge\partialcharge_{i}}{\pi(\stdev\atomradius_{i})^2} 
\exp{\left(-\dfrac{(r-r_i)^2 + (z-z_i)^2}{(\stdev\atomradius_{i})^2}\right)}
\end{align}
where $\echarge$ is the elementary charge. To embed $\scdpore$ with sufficient detail yet efficiently
into a numeric solver, the spatial coordinates are discretized with a grid spacing of $0.005$~nm in the 
domain of $\scdpore$ and precomputed values are used during the solver runtime.

\paragraph{Poisson-Nernst-Planck-Navier-Stokes equations}
To describe the ion and water transport mechanisms within ClyA, we created a computational model of the fluid
based on a set of well-known partial-differential equations (PDEs) called the coupled
Poisson-Nernst-Planck-Navier-Stokes (PNP-NS) equations. 

While these equations have proven to be excellent tool for studying ion and water transport in both 
solid-state\cite{Daiguji-2004,Lu-2012,Chaudhry-2014,Rempfer-2016,Lin-2016} and biological 
nanopores\cite{Eisenberg-1996,Simakov-2010,Pederson-2015}, the underlying continuum assumption does not hold 
at the nanoscale ($r_{\text{pore}}<1$~nm) under all circumstances, particularly at material 
interfaces\cite{Vo-2016} and in regions of high charge density\cite{Corry-2000}. To address these concerns 
and to improve the accuracy of our model, we propose a modified version of these equations, that we will 
refer to as extended PNP-NS (ePNP-NS), which takes into account the finite size of the 
ions\cite{Borukhov-1997,Lu-2011} and the reduction of ion and water mobility near protein 
interfaces.\cite{Makarov-1998, Pronk-2014} Because the diffusion coefficients\cite{Mills-1989}, 
mobilities\cite{Baldessari-2008-2}, viscosity\cite{Hai-Lang-1996}, density\cite{Hai-Lang-1996} and relative 
permittivity\cite{Gavish-2016} often deviate significantly from the traditional `infinite dilution' values at 
experimentally relevant concentrations, we implemented a self-consistent dependency on the average ion 
concentration for all these parameters.

In this section we will start by describing the classical PNP-NS equations and their boundary conditions,
followed by a description of ePNP-NS framework that includes several additions and corrections which enhance
their physical accuracy at nanoscale and their ability to predict ionic currents at high ionic strengths.

Due to the complexity of the set of self-consistent equations, we chose to solve the system using the finite
element method (FEM) as provided by the commercial software COMSOL Multiphysics (version 5.3,
\href{www.comsol.com}{www.comsol.com}).

\paragraph{Poisson Equation} 
The Poisson equation (PE) relates the electrostatic potential to the charges present in a system 
and is given by 
\begin{align} 
\label{eq:poisson}
\nabla\cdot\left(\absperm\relperm \nabla\potential\right) = - \scdpore - \scdion,
\end{align}
with $\potential$ the electric potential, $\absperm$ the vacuum permittivity 
(\SI{8.85419e-12}{\farad\per\meter}), and $\relperm$ the local relative permittivity of the medium listed in
\cref{tab:pnpns_parameters}. The total charge density on the right hand side of \cref{eq:poisson}
consists of a fixed contribution $\scdpore$ due to charges of the nanopore of \cref{eq:scdpore}, and an ion
concentration dependent component $\scdion$ caused by local non-electroneutralities in the fluid
(cf.\cref{eq:scdion}).

Dirichlet boundary conditions are applied to the boundaries of the \cis\ and \trans\ reservoirs. The
potential at the \cis\ boundary is set to $\potential = 0$ whereas at the \trans\ boundary it is set to
$\potential = \vbias$. Neumann boundary conditions apply at the bilayer-boundary interface.

\begin{table*}[t]
  \footnotesize
\begin{center}
  \caption{Parameters for the PNP-NS system-of-equations for an \ce{NaCl} solution.}
  \label{tab:pnpns_parameters}
\begin{tabular}{ccl}
    \toprule
    Parameter & Value & Description \\
    \midrule 
    $T$                 & \SI{298.15}{\kelvin} & System temperature\\
    $\permittivity_p$   & \num{20} & Relative permittivity of pore\cite{Li-2013}\\
    $\permittivity_m$   & \num{3.2} & Relative permittivity of bilayer\cite{Gramse-2013} \\
    $\permittivity_f$   & \num{78.15} & Relative permittivity of fluid \\
    $\density$          & \SI{997}{\kilogram\per\cubic\meter} & Density of fluid at infinite dilution\\
    $\viscosity$        & \SI{8.904}{\pascal\second} & Dynamic viscosity\\
    $\chargen_{\Na}$    & \num{+1} & charge number of \Na \\
    $\chargen_{\Cl}$   & \num{-1} & charge number of \Cl \\
    $\diffusion_{\Na}$  & \SI{1.334e-9}{\square\meter\per\second} & Diffusion coefficient of \Na \\
    $\diffusion_{\Cl}$  & \SI{2.032e-9}{\square\meter\per\second} & Diffusion coefficient of \Cl \\
    $\mobility_{\Na}$   & \SI{5.192e-4}{\square\meter\per\second\per\volt} & Mobility of \Na \\
    $\mobility_{\Cl}$   & \SI{7.909e-4}{\square\meter\per\second\per\volt} & Mobility of \Cl \\
    \bottomrule
\end{tabular}
\end{center}
\end{table*}


\paragraph{Nernst-Planck Equation}
The Nernst-Plack equation (NPE) describes the flux $\flux_{i}$ of each ion species $i$ under the influence of
a chemical gradient (diffusion), a non-zero electrical field (electromigration) and a laminar fluid flow
(convection) and is given by
\begin{multline}\label{eq:nernst-planck}
\nabla\cdot\flux_{i} = \nabla \cdot \left[ -\diffusion_{i}\nabla\concentration_{i} - 
\chargen_{i}\mobility_{i}\concentration_{i}\nabla\potential + \velocity\concentration_{i}\right] = 0
\end{multline}
with $\concentration_{i}$ the concentration, $\diffusion_{i}$ the diffusion coefficient, $\chargen_{i}$ the 
charge number, $\mobility_{i}$ the electrophoretic mobility, and $\velocity$ the fluid velocity. The relevant
parameters for \ce{NaCl} are listed in \cref{tab:pnpns_parameters}.
The NPE is coupled to the PE by the electric potential $\potential$ and the ion space charge density
\begin{align} 
\label{eq:scdion}
\scdion = \faraday\displaystyle\sum_{i}\chargen_{i}\concentration_{i},
\end{align}
where $\faraday= \SI{96485.33}{\coulomb\per\mole}$ is Faraday's constant.

The ion concentration at both reservoir boundaries was fixed using a Dirichlet boundary condition 
($\concentration_i = \cbulk$). At the fluid-protein and fluid-bilayer interface Neumann
boundary conditions apply, i.e.~no current can flow in or out of the protein or the bilayer.


\paragraph{Navier-Stokes Equation} 
The Navier-Stokes equation (NSE) for laminar flow of an incompressible fluid with variable density and 
variable viscosity used to describe the fluid velocity field $\velocity$ is given by\cite{Axelsson-2015}
\begin{multline}\label{eq:navier-stokes}
\density\left(\velocity\cdot\nabla\right)\velocity =
\nabla\cdot\left[-\pressure\identity + 
\viscosity\left(\nabla\velocity + \left(\nabla\velocity\right)^{\rm T}\right)\right]+\volumeforce,
\end{multline}
together with the continuity equation
\begin{align}
\label{eq:navier-stokes-contin}
\density\nabla\cdot\left(\velocity\right) = 0,
\end{align}
where $\density$ and $\viscosity$ are the fluid's density and dynamic viscosity, respectively. For \ce{NaCl}
the parameters are given in \cref{tab:pnpns_parameters}. The driving force that generates the electro-osmotic
flow 
\begin{align}
\volumeforce = \echarge\avogadro\scdion\efield,
\end{align}
acts on all fluid elements with a non-zero net space charge density $\scdion$ and electrical field $\efield$.

At the boundaries of the \cis\ and \trans\ reservoirs, we use open boundary conditions with no normal stress
($\left[-\pressure\identity + \viscosity\left(\nabla\velocity + \left(\nabla\velocity\right)^{\rm
  T}\right)\right]\normvec = 0$), effectively mimicking an endless reservoir. On the nanopore and bilayer
surface we applied a no-slip ($\vec{\velocity} = 0$) boundary condition.


\paragraph{Extended PNP-NS Equations}
The traditional PNP-NS equations as described above cannot capture the transport accurately within fluids of
strongly varying concentrations or when the concentration is close to saturation or in nanoscale constrictions
when wall effects are non-negligible. However, the simulation of transport of ions through protein nanopores
happens precisely under these conditions and therefore we include the three following effects fully
self-consistently: (1) concentration dependent diffusion, mobility, viscosity, density an relative 
permittivity; (2) wall-distance dependent diffusion, mobility and viscosity; and (3) inclusion of a finite 
ion size term in the NPE. In the remainder of this work, we will refer to the model containing these effects 
as extended PNP-NP (ePNP-NS). 

Concentration and wall-distance dependent parameters can be included by substitution of experimentally
determined functions as
\begin{align}
\diffusion_{i}		&= \left[ \diffusion_{i}^c(\dconc) \times \diffusion_{i}^w\left(\dwall\right) \right] 
\times \SI{e-9}{\square\meter\per\second}, \label{eq:diffusion} \\
\mobility_{i}  		&= \left[ \mobility_{i}^c(\dconc) \times \mobility_{i}^w\left(\dwall\right) \right] 
\times \SI{e-8}{\square\meter\per\second\per\volt}, \label{eq:mobility}\\
\viscosity     		&= \left[ \viscosity^c(\dconc) \times \viscosity^w\left(\dwall\right)\right] 
\times \SI{e-4}{\pascal\second}, \label{eq:viscosity}\\
\density 	   		&= \density(\dconc) \times\SI{e3}{\kg\per\cubic\meter}, \label{eq:density}\\
\permittivity_{f} 	&= \permittivity_{f}(\dconc),
\label{eq:permittivity}
\end{align}
where the superscripts $c$ and $w$ indicate the concentration and wall-distance dependent functions,
respectively, $\dconc = \avionconc/1$~M and $\dwall = \walldistance/1$~nm represent the dimensionless 
average ion concentration and dimensionless distance from the nanopore wall, respectively. For the present 
work, the definition of these functions is given in \cref{tab:corrections_equations}, their fits to the 
experimental data are shown in \cref{suppinfo:fig:corrections}, and the resulting fitting parameters are 
listed in \cref{suppinfo:tab:corrections_parameters}.

\begin{table*}[h]

\footnotesize
\renewcommand{\arraystretch}{1.2}
\caption{Summary of the parameters and fitting equations used in the ePNP-NS equations.}
\centering
\label{tab:corrections_equations}

\begin{tabularx}{16.5cm}{>{\raggedright\hsize=3cm}X >{\hsize=1cm}l >{\hsize=6.5cm}X >{\hsize=2cm}l}
	\toprule

	Property
    & Parameter$^\text{\emph{a}}$
      & Infinite dilution value$^\text{\emph{b}}$/Fitting function$^\text{\emph{c}}$
      & Reference \\

	\midrule

	\multirow{4}{*}{Relative permittivity}
    & $\permittivity_{r,\text{p}}$
      & \num{20}
      & \citenum{Li-2013} \\
    & $\permittivity_{r,\text{m}}$
      & \num{3.2}
      & \citenum{Gramse-2013} \\
    & $\permittivity_{r,\text{f}}^0$
      & \num{78.15}
      & \citenum{Gavish-2016} \\
    & $\permittivity_{w}(\dconc)$
      & $1 - \left(1 -	\dfrac{P_1}{P_0}\right) L \left( \dfrac{3P_2}{P_0 - P_1} \dconc \right)$
      & \citenum{Gavish-2016} \\
	\multirow{4}{3cm}{Ion self-diffusion coefficient}
    & $\diffusion_{\Na}^0$
      & \SI{1.334e-9}{\square\meter\per\second}
      & \citenum{Mills-1989} \\
    & $\diffusion_{\Cl}^0$
      & \SI{2.032e-9}{\square\meter\per\second}
      & \citenum{Mills-1989} \\
    & $\diffusion_{i}^c(\dconc)$
      & $\left( 1 + P_1\dconc^{0.5} + P_2\dconc + P_3\dconc^{1.5} + P_4\dconc^2 \right)^{-1}$
      & This work \\
    & $\diffusion_{i}^w(\dwall)$
      & $1-\exp{\left(-P_1(\dwall+P_2)\right)}$
      & \citenum{Makarov-1998,Simakov-2010} \vspace{0.25cm} \\

  \multirow{4}{3cm}{Ion electrophoretic mobility}
    & $\mobility_{\Na}^0$
      & \SI{5.192e-4}{\square\meter\per\second\per\volt}
      & \citenum{Bianchi-1989} \\
    & $\mobility_{\Cl}^0$
      & \SI{7.909e-4}{\square\meter\per\second\per\volt}
      & \citenum{Bianchi-1989} \\
    & $\mobility_{i}^c(\dconc)$
      & $\left( 1 + P_1\dconc^{0.5} + P_2\dconc + P_3\dconc^{1.5} + P_4\dconc^2 \right)^{-1}$
      & This work \\
    & $\mobility_{i}^w(\dwall)$
      & $1-\exp{\left(-P_1(\dwall+P_2)\right)}$
      & \citenum{Makarov-1998,Simakov-2010} \vspace{0.25cm} \\

	\multirow{2}{3cm}{Ion transport number}
    & $\transportn_{\Na}^0$
      & 0.396
      & \citenum{Bianchi-1989} \\
    & $\transportn_{\Na}(\dconc)$
      & $\left( 1 + P_1\dconc^{0.5} + P_2\dconc + P_3\dconc^{1.5} + P_4\dconc^2 \right)^{-1}$
      & This work \vspace{0.25cm} \\

  \multirow{3}{*}{Dynamic viscosity}
    & $\viscosity^0$
      & \SI{8.904}{\pascal\second}
      & \citenum{Hai-Lang-1996} \\
    & $\viscosity^c(\dconc)$
      & $ 1 + P_1 \dconc^{0.5} + P_2 \dconc + P_3 \dconc^2 + P_4 \dconc^{3.5}$
      & This work \\
    & $\viscosity^w(\dwall)$
      & $1 + \exp{ \left( -P_1(\dwall-P_2) \right) }$
      & \citenum{Pronk-2014} \vspace{0.25cm} \\

  \multirow{2}{*}{Fluid density}
    & $\density^0$
      & \SI{997}{\kilogram\per\cubic\meter}
      & \citenum{Hai-Lang-1996} \\
    & $\density(\dconc)$
      & $1 + P_1 \dconc + P_2 \dconc^2$
      & This work \vspace{0.25cm} \\

	\bottomrule
\end{tabularx}
\begin{flushleft}
	$^\text{\emph{a}}$Dependencies on either $\dconc = \avionconc/1$~M (dimensionless average ion
	concentration) and $\dwall = \walldistance/1$~nm (dimensionless distance from the nanopore wall);
  $^\text{\emph{b}}$Values at infinite dilution for a system temperature of \SI{291.15}{\kelvin};
	$^\text{\emph{c}}$These functions are empirical and hence have no physical meaning.
	The values of the fitting parameters $P_x$ of each property can be found in \cref{suppinfo:tab:corrections_parameters} and graphs of the fits in \cref{suppinfo:fig:corrections}.
\end{flushleft}
\end{table*}

    % $T$                        & \SI{298.15}{\kelvin} & System temperature\\
    % $\permittivity_\text{p}^0$ & \num{20} & Relative permittivity of pore\cite{Li-2013}\\
    % $\permittivity_\text{m}^0$ & \num{3.2} & Relative permittivity of bilayer\cite{Gramse-2013} \\
    % $\permittivity_\text{f}^0$ & \num{78.15} & Relative permittivity of fluid \\
    % $\density^0$               & \SI{997}{\kilogram\per\cubic\meter} & Density of fluid\\
    % $\viscosity^0$             & \SI{8.904}{\pascal\second} & Dynamic viscosity\\
    % $\chargen_{\Na}$           & \num{+1} & charge number of \Na \\
    % $\chargen_{\Cl}$           & \num{-1} & charge number of \Cl \\
    % $\diffusion_{\Na}^0$       & \SI{1.334e-9}{\square\meter\per\second} & Diffusion coefficient of \Na\ \\
    % $\diffusion_{\Cl}^0$       & \SI{2.032e-9}{\square\meter\per\second} & Diffusion coefficient of \Cl\ \\
    % $\mobility_{\Na}^0$        & \SI{5.192e-4}{\square\meter\per\second\per\volt} & Mobility of \Na\ \\
    % $\mobility_{\Cl}^0$        & \SI{7.909e-4}{\square\meter\per\second\per\volt} & Mobility of \Cl\ \\


To obtain the concentration-dependent ionic mobility  $\mobility^c_{i}$ from fitted functions, it must first
be derived from the salt's molar conductivity $\molarconductivity$ and the ion's transport number
$\transportn_{i}$ before it can be fitted\cite{ContrerasAburto-2013-1}
\begin{align}
\label{eq:conductivity-to-mobility}
\mobility_{i}(\concentration) = \frac{\specmolarconductivity_{i}(\concentration)}{\chargen_{i}\faraday} 
\quad\text{with}\quad \specmolarconductivity_{i}(\concentration) = \molarconductivity(\concentration) 
\transportn_{i}(\concentration),
\end{align}
where $\specmolarconductivity_{i}(\concentration)$ is the specific molar conductivity of ion $i$.

Note that all parameters simply depend on the average local concentration rather than on the concentration of 
each species. This is because measurements are usually performed in bulk solutions where the concentrations 
of the ions are the same. However, at interfaces and within nanopores the concentrations may differ wildly. 
But due to the lack of experimental data for these conditions, we need to resort to extrapolation -- which is 
simply the average. Nevertheless, we will see that the current concentration and dependent functions will 
lead to an excellent fit to experimental data in all but the most extreme cases, justifying our choice 
\textit{a posteriori}.

The diffusivity of ions or small molecules like water depends heavily on their proximity to large molecules
such as proteins or DNA.\cite{Makarov-1998} Since these effects happen only for distances of less than about 
a nanometer, they can usually be neglected for macroscopic simulations. However, in small nanopores 
($\le5$~nm radius), they comprise a significant fraction of the total nanopore radius and hence must be taken 
into account.\cite{Simakov-2010,Pederson-2015,McMullen-2017}  The wall-distance dependent diffusion 
coefficients $\diffusion_{i}^w(\dwall)$ and mobilities $\mobility_{i}^w(\dwall)$ and their respective fitting 
parameters were directly adapted from Ref.~\citenum{Simakov-2010}. The wall-distance dependent viscosity
$\viscosity^w(\dwall)$ was fitted as a logistic function to the average of the inverse viscosities from
Ref.~\citenum{Pronk-2014} where the hydrodynamic radius of the proteins was interpreted as the position of the
wall (cf.~\cref{suppinfo:fig:corrections}).

In their simulations of the nanopore \textalpha-hemolysin (\ahl), Simakov and Pederson also reduced the ion 
diffusion coefficients as a function of the ratio between the radii of the ion and the 
nanopore\cite{Simakov-2010,Pederson-2015}. This correction is based on the experimental observation that the 
diffusivity of micrometer-sized particles reduces significantly when they are confined in pores and slits of 
comparable dimensions\cite{Renkin-1954,Deen-1987,Dechadilok-2006}, and while this relationship certainly 
holds for nanometer-sized particles or proteins,\cite{Muthukumar-2014,Kannam-2017} its validity is 
questionable for small ions whose hydrodynamic radius is comparable to that of the 
solvent\cite{Anderson-1972,Deen-1987}.

Because the PNP equations do not take the size of the ions into account, they tend to overscreen locations 
with high charge densities resulting in unrealistically high ion concentrations\cite{Corry-2000}. This 
problem can be addressed by the addition of a nonlinear flux term to the NPE that poses an upper limit to the 
concentration. We opted for the size-modified PNP (SMPNP) framework developed by Lu and Zhou,\cite{Lu-2011}
\begin{multline}
\nabla\cdot\flux_{i} = \nabla \cdot \Big[
- \diffusion_{i}\nabla\concentration_{i}
- \chargen_{i}\mobility_{i}\concentration_{i}\nabla\potential
+ \velocity\concentration_{i} \\
- \beta_{i} \concentration_{i} \Big] = 0
\end{multline}
with steric factor
\begin{equation}
\beta_{i} =
\frac{
  \ionsize_{i}^3/\ionsize_{0}^3 \displaystyle\sum_{l} \ionsize_{l}^3 \nabla \concentration_{l}
}{
  1 - \displaystyle\sum_{l} \avogadro \ionsize_{l}^3 \concentration_{l}
}
\end{equation}
where $\ionsize_{i}$ and $\ionsize_{0}$ represent the maximum packing radius in a cubic lattice for ions and 
solvent molecules, respectively. The radii were set to $0.5$~nm for both \Na\ and \Cl\ and to $0.311$~nm for 
water, corresponding to maximum concentrations of $13.3$~M and $55.2$~M, respectively. 

\paragraph{Computation of simulated ionic current.}\todo{finish him}
The simulated ionic currents were computed from the ionic flux under steady-state conditions using
\begin{equation}
\currentsim = \faraday\int_{S}\left(\displaystyle\sum_{i}\chargen_{i}\normvec\cdot\vec{\flux}_{i}\right)dS  
\end{equation}
with $\chargen_{i}$ the charge number and $\vec{\flux}_{i}$ the total flux of each ion $i$ across the
reservoir boundary $S$, $\faraday$ the Faraday constant (\SI{96485}{\coulomb\per\mole}) and $\normvec$ the
normal vector.

\paragraph{ClyA expression and purification.}
ClyA-AS monomers were expressed, purified and oligomerized using methods described in detail 
elsewhere.\cite{Soskine-2012,Soskine-2013} Briefly, \textit{E. cloni} EXPRESS BL21 (DE3) cells (Lucigen 
Corporation, Middleton, USA) were transformed with a pT7 plasmid containing the ClyA-AS gene, followed by 
overexpression after induction with $1$~mM isopropyl \textbeta-D-1-thiogalactopyranoside (Sigma-Aldrich, 
Zwijndrecht, The Netherlands). The ClyA monomers were purified using \ce{Ni+}-NTA affinity chromatography and 
oligomerized by incubation in $0.2$~\%\ D-maltoside n-dodecyl-\textbeta-D-maltopyranoside (Sigma-Aldrich, 
Zwijndrecht, The Netherlands) for 30~minutes at 37\textdegree C. Pure ClyA-AS type-I (12-mer) nanopores were 
obtained using native PAGE on a $4$-$15$\%\ gradient gel (Bio-Rad, Veenendaal, The Netherlands) and 
subsequent excision of the correct oligomer band.

\paragraph{Recording of single-channel current-voltage curves.}
Experimental current-voltage curves where measured using single-channel electrophysiology, as detailed 
elsewhere.\cite{Maglia-2010,Soskine-2012,Soskine-2013} Briefly, a $\approx100$~\textmu m diameter 
black lipid bilayer, using 1,2-diphytanoyl-snglycero-3-phosphocholine (DPhPC, Avanti Polar Lipids, Alabaster, 
USA) was formed between two buffered electrolyte compartments. Minute amounts ($\approx$) of the purified 
ClyA-AS type I oligimer were then added to the grounded \cis\ reservoir and allowed to insert into the lipid 
bilayer. Single-channel current-voltage curves were recorded using a custom pulse protocol of the Clampex 
10.4 software package connected to AxoPatch 200B patch-clamp amplifier via a Digidata 1440A digitizer (all 
from Molecular Devices, San Jose, USA). Data was acquired at $10$~kHz and filtered using a $2$-kHz low 
bandpass filter. Measurements at different ionic strengths were performed at $\approx25$\textdegree C in 
aqueous \ce{NaCl} solutions, buffered at pH~$7.5$ using $10$~mM MOPS (Sigma-Aldrich, Zwijndrecht,The 
Netherlands).

%-------------------------------------------------------------------------------
% ACKNOWLEDGEMENT
%-------------------------------------------------------------------------------

\begin{acknowledgement}
K.W. thanks the Research Foundation Flanders (FWO) for the doctoral fellowship and the project grant 
(\todo{number}). The work of G.M. was supported by an ERC consolidator grant (DeE-Nano, \todo{number}).
The authors thank Ujjal Barman and Chang Chen for their valuable feedback during discussions.
\end{acknowledgement}


%-------------------------------------------------------------------------------
% SUPPORTING INFORMATION
%-------------------------------------------------------------------------------
\begin{suppinfo}
	The supplementary info contains the Extended materials and methods, with details on the construction of the 
	ClyA-AS homology model, the fitting of the electrolyte properties and the calculation of the pore averaged 
	values. It also contains additional results, including a figures about the voltage dependency of the 
	power-law exponent of the conductance fits and the electro-osmotic pressure distribution inside the pore, 
	and a table detailing the peak values of the radial electrostatic potential inside ClyA.
\end{suppinfo}


%-------------------------------------------------------------------------------
% REFERENCES
%-------------------------------------------------------------------------------
\bibliography{modeling1}

%\includepdf{supporting_information}
\end{document}

\documentclass[twoside,twocolumn,9pt]{article}

\usepackage{extsizes}
\usepackage[super,sort&compress,comma]{natbib} 
\usepackage[version=3]{mhchem}
\usepackage[left=1.5cm, right=1.5cm, top=1.785cm, bottom=2.0cm]{geometry}
\usepackage{balance}
\usepackage{times,mathptmx}
\usepackage{sectsty}
\usepackage{graphicx}
\usepackage{lastpage}
\usepackage[format=plain,justification=justified,singlelinecheck=false,font={stretch=1.125,small,sf},labelfont=bf,labelsep=space]{caption}
  \captionsetup[table]{singlelinecheck=false,font=footnotesize,labelfont=bf}
\usepackage[font={stretch=1.125,small,sf},labelfont=bf,labelsep=period]{subcaption}
	\captionsetup[subfigure]{labelfont=bf,textfont=normalfont,labelformat=simple,singlelinecheck=false}
\usepackage{float}
\usepackage{fancyhdr}
\usepackage{fnpos}
\usepackage[english]{babel}
\addto{\captionsenglish}{%
  \renewcommand{\refname}{Notes and references}
}
\usepackage{array}
\usepackage{droidsans}
\usepackage{charter}
\usepackage[T1]{fontenc}
\usepackage[usenames,dvipsnames]{xcolor}
\usepackage{setspace}
\usepackage[compact]{titlesec}
\usepackage{hyperref}
%%%Please don't disable any packages in the preamble, as this may cause the template to display incorrectly.%%%

\definecolor{cream}{RGB}{222,217,201}

\usepackage[utf8]{inputenc}

\usepackage{csquotes}
\usepackage{amsmath}
\usepackage{amsfonts}
\usepackage{amssymb}
\usepackage{mathtools}
\usepackage{textcomp}
\usepackage{textgreek}
\usepackage{gensymb}

\usepackage[alsoload=synchem]{siunitx}
	\sisetup{separate-uncertainty=true}
	\sisetup{multi-part-units=single}
	\sisetup{tight-spacing=true}
	\sisetup{inter-unit-product=\ensuremath{{}\cdot{}}}
	\sisetup{list-units=single}
  \sisetup{range-units=single}
  \sisetup{list-final-separator={ and }}
  \sisetup{retain-explicit-plus=true}

\usepackage{booktabs, tabularx, multirow}
\usepackage{pdfpages}
\usepackage{xr-hyper}
\usepackage{cleveref}
\creflabelformat{equation}{#2#1#3}
\crefname{figure}{Fig.}{Figs.}
\Crefname{figure}{Figure}{Figures}
\crefname{table}{Tab.}{Tabs.}
\Crefname{table}{Table}{Tables}
\crefname{equation}{Eq.}{Eqs.}
\Crefname{equation}{Equation}{Equations}
\crefname{section}{Sec.}{Secs.}
\Crefname{section}{Section}{Sections}

\pdfsuppresswarningpagegroup=1

% SUPPORTING INFO
\usepackage{xr}

% Source:
% https://www.overleaf.com/learn/how-to/Cross_referencing_with_the_xr_package_in_Overleafmak

\makeatletter
\newcommand*{\addFileDependency}[1]{% argument=file name and extension
  \typeout{(#1)}
  \@addtofilelist{#1}
  \IfFileExists{#1}{}{\typeout{No file #1.}}
}
\makeatother

\newcommand*{\myexternaldocument}[2]{%
    \externaldocument[#1]{#2}%
    \addFileDependency{#2.tex}%
    \addFileDependency{#2.aux}%
}

\myexternaldocument{suppinfo:}{suppinfo}

%%%%%%%%%%%%%%%%%%%%%%%%%%%%%%%%%%%%%%%%%%%%%%%%%%%%%%%%%%%%%%%%%%%%%
%% If issues arise when submitting your manuscript, you may want to
%% un-comment the next line.  This provides information on the
%% version of every file you have used.
%%%%%%%%%%%%%%%%%%%%%%%%%%%%%%%%%%%%%%%%%%%%%%%%%%%%%%%%%%%%%%%%%%%%%
%%\listfiles


%%%%%%%%%%%%%%%%%%%%%%%%%%%%%%%%%%%%%%%%%%%%%%%%%%%%%%%%%%%%%%%%%%%%%
%% Place any additional macros here.  Please use \newcommand* where
%% possible, and avoid layout-changing macros (which are not used
%% when typesetting).
%%%%%%%%%%%%%%%%%%%%%%%%%%%%%%%%%%%%%%%%%%%%%%%%%%%%%%%%%%%%%%%%%%%%%

\renewcommand*{\bibfont}{\normalfont\small}

% Shorthands
\newcommand{\todo}[1]{\textbf{\textcolor{orange}{#1}}}
\newcommand{\ahl}{\textalpha HL}
\newcommand{\etal}{\textit{et al.}}
\newcommand{\cis}{\textit{cis}}
\newcommand{\trans}{\textit{trans}}

% Units
\DeclareSIUnit{\molar}{\mole\per\cubic\deci\metre}
\DeclareSIUnit{\Molar}{\textsc{M}}
\DeclareSIUnit{\atm}{\textsc{atm}}
\newcommand{\mM}{\milli\Molar}
\newcommand{\mV}{\milli\volt}
\newcommand{\um}{\micro\meter}
\newcommand{\nm}{\nano\meter}
\newcommand{\ns}{\nano\second}
\newcommand{\ps}{\pico\second}
\newcommand{\fs}{\femto\second}
\newcommand{\pN}{\pico\newton}
\newcommand{\ec}{\elementarycharge}
\newcommand{\dC}{\degreeCelsius}
\newcommand{\mps}{\meter\per\second}
\newcommand{\cnmpnspv}{\cubic\nano\meter\per\nano\second\per\volt}

% Vectors and math stuff
\renewcommand{\vec}[1]{\boldsymbol{#1}}
\newcommand{\rpos}{\vec{r}} % positional vector
\newcommand{\normvec}{\vec{\hat{n}}} % normal vector
\newcommand{\identity}{\vec{\rm I}} % Identity vector
\newcommand{\stdev}{\sigma}
\newcommand{\pav}[2]{\left< #1 \right>_{\text{#2}}} % Pore average
\newcommand{\vc}[2]{#1_{#2}}
\newcommand{\pd}[2]{\displaystyle\frac{\partial #1}{\partial #2}}
\newcommand{\hydrostresstensor}{\sigma_{ij}}
\def\dsum{\displaystyle\sum}

% Physical constants
\newcommand{\boltzmann}{k_{\rm B}}
\newcommand{\avogadro}{N_{\rm A}}
\newcommand{\temp}{T}
\newcommand{\faraday}{\mathcal{F}}
\newcommand{\echarge}{e}

% Field variables
\newcommand{\vel}{u}                  % Fluid velocity
\newcommand{\force}{F}                % A force
\newcommand{\potential}{\varphi}			% Electrostatic potential
\newcommand{\varpotential}{V}				  % Electrostatic potential
\newcommand{\concentration}{c}				% Ion concentration
\newcommand{\velocity}{\vec{u}}				% Fluid velocity
\newcommand{\pressure}{p}					    % Fluid pressure
\newcommand{\efield}{\vec{E}}         % Electrical field
\newcommand{\displacement}{\vec{D}}   % Electrical displacement field
\newcommand{\walldistance}{d}				  % Wall distance

% Dimensionless variables
\newcommand{\dconc}{\bar{\concentration}}		% Dimensionless concentration
\newcommand{\dwall}{\bar{\walldistance}}		% Dimensionless wall distance

% Material and ion parameters
\newcommand{\permittivity}{\varepsilon}
\newcommand{\absperm}{\permittivity_0}				% Permittivity of vacuum
\newcommand{\relperm}{\permittivity_r}				% Relative permittivity
\newcommand{\dielectric}{\relperm}				    % Relative permittivity
\newcommand{\diffusion}{\mathcal{D}}		     	% Diffusion coefficient
\newcommand{\mobility}{\mu}				         		% Electrophoretic mobility
\newcommand{\transportn}{t}				         		% Transport number
\newcommand{\chargen}{z}				          		% Ion charge number
\newcommand{\ionsize}{a}					           	% Ion size for smPNP
\newcommand{\density}{\varrho}			     			% Fluid density
\newcommand{\viscosity}{\eta}				         	% Fluid viscosity

\newcommand{\iondiffusion}[1]{\diffusion_{#1}}	        % Ion Diffusion coefficient
\newcommand{\ionmobility}[1]{\mobility_{#1}}	          % Ion electrophoretic mobility
\newcommand{\iontransportn}[1]{\transportn{#1}}	        % Ion transport number
\newcommand{\avionconc}{\langle\concentration\rangle}   % Average ion concentration

\newcommand{\tna}{\transportn_{\ce{Na+}}}

\newcommand{\molarconductivity}{\Lambda}
\newcommand{\specmolarconductivity}{\lambda}

% Derived properties
\newcommand{\scd}{\rho}							% Total charge density
\newcommand{\scdpore}{\scd_{\text{pore}}^f}		% Fixed charge density
\newcommand{\scdion}{\scd_{\text{ion}}}			% Ionic charge density
\newcommand{\flux}{\vec{J}} 							% Ion flux
\newcommand{\volumeforce}{\vec{\force}_{\rm ion}}	% Fluid volume force

\newcommand{\current}{I}
\newcommand{\currentsim}{\current_{\rm sim}}
\newcommand{\currentexp}{\current_{\rm exp}}
\newcommand{\conductance}{G}
\newcommand{\icr}{\alpha}

% Shorthands
\newcommand{\ci}{\concentration_{i}}
\newcommand{\cbulk}{\concentration_\text{s}}
\newcommand{\vbias}{\varpotential_\text{b}}
\newcommand{\Na}{\ce{Na+}}
\newcommand{\Cl}{\ce{Cl-}}
\newcommand{\Qion}{Q_{\rm ion}}
\newcommand{\radpot}{\left<\potential\right>_\text{rad}}
\newcommand{\radenergy}{\left< U_{\text{E},i} \right>_{\text{rad}}}
\newcommand{\deltaEt}{\Delta E_{\text{B},i}}
\newcommand{\pH}[1]{pH~\num{#1}}
\newcommand{\kT}{\boltzmann\temp}
\newcommand{\kTe}{\kT / \echarge}

% Atom properties
\newcommand{\partialcharge}{\delta}
\newcommand{\atomradius}{R}


% Colors
\definecolor{graphgreen}  {rgb}{0.30196078, 0.68627451, 0.29019608}
\definecolor{graphpurple} {rgb}{0.59607843, 0.30588235, 0.63921569}
\definecolor{graphblue}   {rgb}{0.29803921, 0.57254902, 0.98823529}
\definecolor{graphred}    {rgb}{0.91372549, 0.15686275, 0.18823529}

% Line and marker styles
\newcommand{\graphline}[2]{\raisebox{2pt}{\tikz{\draw[-,color=#2,#1,line width=1.5pt](0,0) -- (5mm,0);}}}
\newcommand{\graphmarker}[2]{\raisebox{0.5pt}{\tikz{\node[draw,scale=0.5,#1,fill=none, color=#2](){};}}}

%\newcommand{\graphlinemarker}[3]{\raisebox{0pt}{\tikz{\draw[-,#3,#1,line width = 1.0pt](2.mm,0) #2 (3.5mm,1.5mm);\draw[-,#3,#1,line width = 1.0pt](0.,0.8mm) -- (5.5mm,0.8mm)}}}
%\newcommand{\rectangle}{\raisebox{0pt}{\tikz{\draw[-,black,dotted,line width = 1.0pt](0.,0.8mm) -- (5.5mm,0.8mm);\draw[black,solid,line width = 1.0pt](2.mm,0) circle (3.5mm,1.5mm)}}}
\newcommand{\rectangle}[1]{\raisebox{0pt}{\tikz{\draw[-,black,solid,line width = 1.0pt](0.,0.8mm) -- (5.5mm,0.8mm);\draw[black,solid,line width = 1.0pt](2.mm,0) #1 (3.5mm,1.5mm)}}}


\newcommand{\afimec}{imec, Kapeldreef 75, B-3001 Leuven, Belgium}
\newcommand{\afkulchem}{KU Leuven, Department of Chemistry, Celestijnenlaan 200F, B-3001 Leuven, Belgium}
\newcommand{\afkulphys}{KU Leuven, Department of Physics and Astronomy, Celestijnenlaan 200D, B-3001 Leuven, Belgium}
\newcommand{\afrug}{University of Groningen, Groningen Biomolecular Sciences \& Biotechnology Institute, 9747 AG, Groningen, The Netherlands}

\author{Kherim Willems}
\affiliation{\afkulchem}
\alsoaffiliation{\afimec}

\author{Dino Rui\'{c}}
\affiliation{\afkulphys}
\alsoaffiliation{\afimec}

\author{Florian Lucas}
\affiliation{\afrug}

\author{Ujjal Barman}
\affiliation{\afimec}

%\author{Chang Chen}
%\affiliation{\afimec}

\author{Johan Hofkens}
\affiliation{\afkulchem}

\author{Giovanni Maglia}
\email{g.maglia@rug.nl}
\affiliation{\afrug}

\author{Pol Van Dorpe}
\email{Pol.VanDorpe@imec.be}
\affiliation{\afkulchem}
\alsoaffiliation{\afimec}


% \keywords{biological nanopore; cytolysin A; continuum simulation; Poisson-Nernst-Planck and Navier-Stokes
% equations; accurate; single molecule}



\begin{document}

\pagestyle{fancy}
\thispagestyle{plain}
\fancypagestyle{plain}{
%%%HEADER%%%
\renewcommand{\headrulewidth}{0pt}
}
%%%END OF HEADER%%%

%%%PAGE SETUP - Please do not change any commands within this section%%%
\makeFNbottom
\makeatletter
\renewcommand\LARGE{\@setfontsize\LARGE{15pt}{17}}
\renewcommand\Large{\@setfontsize\Large{12pt}{14}}
\renewcommand\large{\@setfontsize\large{10pt}{12}}
\renewcommand\footnotesize{\@setfontsize\footnotesize{7pt}{10}}
\makeatother

\renewcommand{\thefootnote}{\fnsymbol{footnote}}
\renewcommand\footnoterule{\vspace*{1pt}% 
\color{cream}\hrule width 3.5in height 0.4pt \color{black}\vspace*{5pt}} 
\setcounter{secnumdepth}{5}

\makeatletter 
\renewcommand\@biblabel[1]{#1}            
\renewcommand\@makefntext[1]% 
{\noindent\makebox[0pt][r]{\@thefnmark\,}#1}
\makeatother 
\renewcommand{\figurename}{\small{Fig.}~}
\sectionfont{\sffamily\Large}
\subsectionfont{\normalsize}
\subsubsectionfont{\bf}
\setstretch{1.125} %In particular, please do not alter this line.
\setlength{\skip\footins}{0.8cm}
\setlength{\footnotesep}{0.25cm}
\setlength{\jot}{10pt}
\titlespacing*{\section}{0pt}{4pt}{4pt}
\titlespacing*{\subsection}{0pt}{15pt}{1pt}
%%%END OF PAGE SETUP%%%

%%%FOOTER%%%
\fancyfoot{}
\fancyfoot[LO,RE]{\vspace{-7.1pt}\includegraphics[height=9pt]{head_foot/LF}}
\fancyfoot[CO]{\vspace{-7.1pt}\hspace{13.2cm}\includegraphics{head_foot/RF}}
\fancyfoot[CE]{\vspace{-7.2pt}\hspace{-14.2cm}\includegraphics{head_foot/RF}}
\fancyfoot[RO]{\footnotesize{\sffamily{1--\pageref{LastPage} ~\textbar  \hspace{2pt}\thepage}}}
\fancyfoot[LE]{\footnotesize{\sffamily{\thepage~\textbar\hspace{3.45cm} 1--\pageref{LastPage}}}}
\fancyhead{}
\renewcommand{\headrulewidth}{0pt} 
\renewcommand{\footrulewidth}{0pt}
\setlength{\arrayrulewidth}{1pt}
\setlength{\columnsep}{6.5mm}
\setlength\bibsep{1pt}
%%%END OF FOOTER%%%

%%%FIGURE SETUP - please do not change any commands within this section%%%
\makeatletter 
\newlength{\figrulesep} 
\setlength{\figrulesep}{0.5\textfloatsep} 

\newcommand{\topfigrule}{\vspace*{-1pt}% 
\noindent{\color{cream}\rule[-\figrulesep]{\columnwidth}{1.5pt}} }

\newcommand{\botfigrule}{\vspace*{-2pt}% 
\noindent{\color{cream}\rule[\figrulesep]{\columnwidth}{1.5pt}} }

\newcommand{\dblfigrule}{\vspace*{-1pt}% 
\noindent{\color{cream}\rule[-\figrulesep]{\textwidth}{1.5pt}} }

\makeatother
%%%END OF FIGURE SETUP%%%

%%%TITLE, AUTHORS AND ABSTRACT%%%
\twocolumn[
  \begin{@twocolumnfalse}
{\includegraphics[height=30pt]{head_foot/journal_name}\hfill\raisebox{0pt}[0pt][0pt]{\includegraphics[height=55pt]{head_foot/RSC_LOGO_CMYK}}\\[1ex]
\includegraphics[width=18.5cm]{head_foot/header_bar}}\par
\vspace{1em}
\sffamily
\begin{tabular}{m{4.5cm} p{13.5cm} }

\includegraphics{head_foot/DOI} & \noindent\LARGE{\textbf{%
  Accurate modeling of a biological nanopore with an extended continuum framework$^\dag$
}} \\
\vspace{0.3cm} & \vspace{0.3cm} \\

 & \noindent\large{%
   Kherim Willems,\textit{$^{ab}$}
   Dino Rui\'c,\textit{$^{bc}$}
   Florian Lucas,\textit{$^{d}$}
   Ujjal Barman,\textit{$^{b}$}
   Niels Verellen,\textit{$^{b}$}
   Johan Hofkens,\textit{$^{a}$}
   Giovanni Maglia,$^{\ast}$\textit{$^{d}$} and
   Pol Van Dorpe$^{\ast}$\textit{$^{bc}$}} \\


\includegraphics{head_foot/dates} & \noindent\normalsize{%
%
Despite the broad success of biological nanopores as powerful instruments for the analysis of proteins and
nucleic acids at the single-molecule level, a fast simulation methodology to accurately model their
nanofluidic properties is currently unavailable. This limits the rational engineering of nanopore traits and
makes the unambiguous interpretation of experimental results challenging. Here, we present a continuum
approach that can faithfully reproduce the experimentally measured ionic conductance of the biological
nanopore Cytolysin A (ClyA) over a wide range of ionic strengths and bias potentials. Our model consists of
the extended Poison-Nernst-Planck and Navier-Stokes (ePNP-NS) equations and a computationally efficient
2D-axisymmetric representation for the geometry and charge distribution of the nanopore. Importantly, the
ePNP-NS equations achieve this accuracy by self-consistently considering the finite size of the ions and the
influence of both the ionic strength and the nanoscopic scale of the pore on the local properties of the
electrolyte. These comprise the mobility and diffusivity of the ions, and the density, viscosity and relative
permittivity of the solvent. Crucially, by applying our methodology to ClyA, a biological nanopore used for
single-molecule enzymology studies, we could directly quantify several nanofluidic characteristics difficult
to determine experimentally. These include the ion selectivity, the ion concentration distributions, the
electrostatic potential landscape, the magnitude of the electro-osmotic flow field, and the internal pressure
distribution. Hence, this work provides a means to obtain fundamental new insights into the nanofluidic
properties of biological nanopores and paves the way towards their rational engineering.
%
} \\

\end{tabular}

 \end{@twocolumnfalse} \vspace{0.6cm}

  ]
%%%END OF TITLE, AUTHORS AND ABSTRACT%%%

%%%FONT SETUP - please do not change any commands within this section
\renewcommand*\rmdefault{bch}\normalfont\upshape
\rmfamily
\section*{}
\vspace{-1cm}


%%%FOOTNOTES%%%

\footnotetext{\textit{$^{a}$~\afkulchem.}}
\footnotetext{\textit{$^{b}$~\afimec. E-mail: Pol.VanDorpe@imec.be}}
\footnotetext{\textit{$^{c}$~\afkulphys.}}
\footnotetext{\textit{$^{d}$~\afrug. E-mail: g.maglia@rug.nl}}


%Please use \dag to cite the ESI in the main text of the article.
%If you article does not have ESI please remove the the \dag symbol from the title and the footnotetext below.
\footnotetext{\dag~Electronic Supplementary Information (ESI) available. See DOI: 00.0000/00000000.}
%additional addresses can be cited as above using the lower-case letters, c, d, e... If all authors are from the same address, no letter is required



%%%END OF FOOTNOTES%%%

%%%MAIN TEXT%%%%


%%%%%%%%%%%%%%%%%%%%%%%%%%%%%%%%%%%%%%%%%%%%%%%%%%%%%%%%%%%%%%%%%%%%%
%% Start the main part of the manuscript here.
%%%%%%%%%%%%%%%%%%%%%%%%%%%%%%%%%%%%%%%%%%%%%%%%%%%%%%%%%%%%%%%%%%%%%
\section{Introduction}

The transport of ions and molecules through nanoscale geometries is a field of intense study that uses both
experimental, theoretical and computational methods.\cite{Sparreboom-2010,Bocquet-2010,Maffeo-2012,
Thomas-2014,Wang-2014,Kim-2015} One of the primary driving forces behind this research is the development of
nanopores as label-free, stochastic sensors at the ultimate analytical limit (\ie{}~single molecule).
\cite{Bayley-2001,Dekker-2007,Venkatesan-2011,Zhang-2016} These detectors have applications ranging from the
analysis of biopolymers such as
DNA\cite{Deamer-2016,Kasianowicz-1996,Meller-2000,Maglia-2008,Butler-2008,Stoddart-2009,Franceschini-2013,Jain-2018}
or proteins,\cite{Restrepo-Perez-2018,Talaga-2009,Rodriguez-Larrea-2013, Nivala-2013,Kennedy-2016} to the
detection and quantification of
biomarkers,\cite{Chen-2013,Soskine-2012,Niedzwiecki-2013,VanMeervelt-2014,Huang-2017,Liu-2018,Galenkamp-2018}
or the fundamental study of chemical or enzymatic reactions at the single molecular
level.\cite{Willems-VanMeervelt-2017,Lieberman-2010, Nivala-2013,Ho-2015,Laszlo-2017}

Nanopores are typically operated in the resistive-pulse mode, where the fluctuations of their ionic
conductance are monitored over time.\cite{Bayley-2001,Dekker-2007,Maglia-2010,Venkatesan-2011} Experimentally,
this is achieved by placing the nanopore between two electrolyte compartments and applying a constant DC (or
AC) voltage across them. Due to the high resistance of the nanopore, virtually the full potential change
occurs within (and around) the pore, resulting in a strong electric field ($10^6$--$10^7$~\si{\V\per\m})
that can electrophoretically drive ions and water molecules through
it.\cite{Wong-2007,Mao-2014,Haywood-2014,Laohakunakorn-2015} Analyte molecules such as DNA or proteins are
then driven towards, and often \emph{through}, the nanopore by a combination of Coulombic (electrophoretic)
and hydrodynamic (electro-osmotic) forces.\cite{Wong-2007,Grosberg-2010,Muthukumar-2010, Muthukumar-2014} If
successful, a translocation event is observed as a temporal fluctuation in the ionic conductance of the pore
that serves as a unique molecular `fingerprint' with which the molecule can be identified.\cite{Yusko-2017}
Because the frequency, magnitude, duration and even noise levels\cite{Yusko-2017,Houghtaling-2019} of these
events depend on the properties of both the analyte molecule and the nanopore itself, they are notoriously
difficult to interpret unambiguously without a full understanding of the nanofluidic phenomena that underlie
them.

The computational approaches most widely used to study nanofluidic transport in ion channels or biological
nanopores comprise \emph{discrete} methods such as molecular dynamics
(MD)\cite{Lynden-Bell-1996,Allen-1999,Aksimentiev-2005,Luan-2008,Bhattacharya-2011,Zhang-2014,DiMarino-2015,Belkin-2016}
and Brownian dynamics
(BD),\cite{Schirmer-1999,Im-2002,Noskov-2004,Millar-2008,Egwolf-2010,DeBiase-2015,Pederson-2015} and
\emph{mean-field} (continuum) methods based on solving the Poisson-Boltzmann (PB)
equations\cite{Grochowski-2008, Baldessari-2008-1} and Poisson-Nernst-Planck (PNP)
equations.\cite{Eisenberg-1996,Gillespie-2002, Simakov-2010} The latter two can be coupled with the
Navier-Stokes (NS) equation to include electro-osmotically or pressure driven fluid
flow.\cite{Lu-2012,Pederson-2015} Due to their explicit atomic or particle nature, MD and BD simulations are
considered to yield the most accurate results. However, the large computational cost of simulating a complete
biological nanopore system (100K--1M atoms) for hundreds of nanoseconds still necessitates the use of
supercomputers.\cite{Aksimentiev-2005,Bhattacharya-2011,Wilson-2019} The PNP(-NS) equations, on the other
hand, are of particular interest due to their low computational demands and analytical tractability. In a
continuum approach, the simulated system is subdivided in several `structureless' domains, the behavior of
which is parameterized by material properties such as relative permittivity, diffusion coefficient,
electrophoretic mobility, viscosity and density. Because these properties can only emerge from the collective
behavior or interactions between small groups of atoms (\ie~the mean-field approximation), great care must be
taken when using them to compute fluxes and fields at the nanoscale, where computational elements may only
contain a few molecules.\cite{Corry-2000,Collins-2012} Nevertheless, even though the PNP equations have been
used extensively for the qualitative simulation of ion channels,\cite{Im-2002,Furini-2006,Liu-2015} biological
nanopores\cite{Simakov-2010,Pederson-2015,Aguilella-Arzo-2017,Simakov-2018} and their solid-state
counterparts,\cite{Cervera-2005,White-2008,Chaudhry-2014,Laohakunakorn-2015} the extent to which they are
quantitatively accurate is often challenged.\cite{Corry-2000,Collins-2012,Maffeo-2012,Thomas-2014,Kim-2015}
To remedy the shortcomings of PNP and NS theory, a number of modifications have been proposed over the years.
These include, among others, (1) steric ion-ion interactions, (2) the effect of protein-ion/water interactions
on their motility (\ie~diffusivity and electrophoretic mobility), (3) the concentration dependencies of ion
motility, and solvent relative permittivity, viscosity and density.

The steric ion-ion interactions can be accounted for by computing the excess in chemical potential
($\mu_{i}^\text{ex}$) resulting from the finite size of the ions.\cite{Eisenberg-1996,Bazant-2009,
Daiguji-2010} Gillespie \etal{} combined PNP and density functional theory---where $\mu_{i}^\text{ex}$ was
split up in ideal, hard-sphere and electrostatic components---to successfully predict the selectivity and
current of ion channels.\cite{Gillespie-2002} In another approach, Kili\'{c} \etal{} derived a set of modified
PNP equations based on the free energy functional of the Borukhov's modified PB model\cite{Borukhov-1997}and
observed significantly more realistic concentrations for high surface potentials compared to the classical PNP
equations.\cite{Kilic-2007} To allow for non-identical ion sizes and more than two ion species, this model was
later extended by Lu \etal{}, who used it to probe the effect of finite ion size on the rate coefficients of
enzymes.\cite{Lu-2011}

The interaction of ions or small molecules such as water with the heavy atoms of proteins or DNA results in a
strong reduction of their motility, as observed in MD simulations.\cite{Makarov-1998,Pronk-2014,Wilson-2019}
Since these effects happen only at distances \SI{\le1}{\nm}, they can usually be neglected for macroscopic
simulations. However, in small nanopores (\SI{\le10}{\nm} radius), they comprise a significant fraction of the
total nanopore radius and hence must be taken into
account.\cite{Noskov-2004,Simakov-2010,Pederson-2015,McMullen-2017} In continuum simulations, this can be
achieved with the use of positional-dependent ion diffusion coefficients. An example implementation is the
`soft-repulsion PNP' developed by Simakov and Kurnikova,\cite{Simakov-2010,Simakov-2018} who used it to
predict the ionic conductance of the \textalpha-hemolysin nanopore. Similar reductions in ion diffusion
coefficients have been proposed to improve PNP theory's estimations of the ionic conductance of ion
channels.\cite{Furini-2006,Liu-2015,DeBiase-2015} The motility of water molecules is expressed by the NS
equations as the fluid's viscosity. Hence, as also observed in MD simulations for water molecules near
proteins\cite{Pronk-2014} and confined in hydrophilic nanopores,\cite{Qiao-Aluru-2003,Vo-2016,Hsu-2017} the
water-solid interaction leads to a viscosity several times higher compared to the bulk values. Note that this
is valid for hydrophilic interfaces only, as the lack of interaction with hydrophobic interfaces, such as
carbon nanotubes, leads to a lower viscosity.\cite{Ye-2011}

It is well known that the self-diffusion coefficient $\diffusion_{i}$ and electrophoretic mobility
$\mobility_{i}$ of an ion $i$ depends on the local concentrations of all the ions in the
electrolyte.\cite{ContrerasAburto-2013-1} Their values typically decrease with increasing salt concentration,
and should not be treated as constants. Moreover, even though the Nernst-Einstein (NE) relation
$\mobility_{i}=\diffusion_{i}/\kT$ is strictly speaking only valid at infinite dilution and a good
approximation at low concentrations (\SI{<10}{\mM}), it significantly overestimates the ionic mobility at
higher salt concentrations.\cite{Mills-1989,Panopoulos-1986,ContrerasAburto-2013-1,ContrerasAburto-2013-2} In
an empirical approach, Baldessari and Santiago formulated an ionic-strength dependency of the ionic mobility
based on the activity coefficient of the salt\cite{Baldessari-2008-1} and showed excellent correspondence
between the experimental and simulated ionic conductance of long nanochannels over a wide concentration
range.\cite{Baldessari-2008-2} Alternatively, Burger \etal{} used a microscopic lattice-based model to derive
a set of PNP equations with non-linear, ion density-dependent mobilities and diffusion coefficients that
provided significantly more realistic results for ion channels.\cite{Burger-2012} Note that other electrolyte
properties, such as its viscosity,\cite{Hai-Lang-1996} density\cite{Hai-Lang-1996} and relative
permittivity,\cite{Gavish-2016} also significantly affect the ion and water flux. To better compute the charge
flux in ion channels, Chen derived a new PNP framework\cite{Chen-2016} that includes water-ion interactions in
the form of a concentration-dependent relative permittivity and an additional ion-water interaction energy
term.
% Axelsson \etal{} derived a
% set of NS equations that allowed for an incompressible fluid with a variable density and
% viscosity.\cite{Axelsson-2015}

\begin{figure*}[t]

  \centering
	\begin{minipage}[t]{5.5cm}
		\begin{subfigure}[t]{5cm}
			\centering
			\caption{}\label{fig:clya_side}
      \vspace{-5mm}
			\includegraphics[scale=1]{figures/concept/clya_side}
    \end{subfigure}
    \vspace{0.5cm}
   	\begin{subfigure}[t]{5cm}
      \centering
      \caption{}\label{fig:clya_top}
      \vspace{-5mm}
      \includegraphics[scale=1]{figures/concept/clya_top}
    \end{subfigure}
	\end{minipage}
  \begin{minipage}[t]{12cm}
    \hspace{0.5cm}
    \begin{minipage}[t]{5.5cm}
      \begin{minipage}[t]{5.5cm}
        \begin{subfigure}[t]{1.6cm}
          \centering
          \caption{}\label{fig:model_geometry_vs_wedge}
          \vspace{-3mm}
          \includegraphics[scale=1]{figures/concept/model_geometry_vs_wedge}
        \end{subfigure}
        \begin{subfigure}[t]{2.5cm}
          \centering
          \caption{}\label{fig:model_charge_density}
          \vspace{-3mm}
          \includegraphics[scale=1]{figures/concept/model_charge_density}
        \end{subfigure}
      \end{minipage}
      \begin{subfigure}[t]{5.5cm}
        \hspace{1cm}
        \centering
        \vspace{0.5cm}
        \caption{}\label{fig:model_geometry_zoom}
        \vspace{-1cm}
        \includegraphics[scale=1]{figures/concept/model_geometry_zoom}
        %\vspace{0.25cm}
      \end{subfigure}
    \end{minipage}
    \begin{subfigure}[t]{5.5cm}
      \centering
      \caption{}\label{fig:model_geometry}
      \includegraphics[scale=1]{figures/concept/model_geometry}
    \end{subfigure}
  \end{minipage}

  \caption%
  %
  [All-atom and 2D-axisymmetric models of ClyA.]
  %
  {%
    %
    All-atom and 2D-axisymmetric models of ClyA.
    %
    (\subref{fig:clya_side})
    %
    Axial cross-sectional and (\subref{fig:clya_top}) top views of the dodecameric nanopore
    ClyA-AS,\cite{Soskine-2013} derived through homology modelling from the \textit{E. coli} Cytolysin A
    crystal structure (PDBID: 2WCD\cite{Mueller-2009}). Figures were rendered with
    VMD.\cite{Humphrey-1996,Stone-1998}
    %
    (\subref{fig:model_geometry_vs_wedge})
    %
    The 2D-axisymmetric geometry was derived directly from the all-atom model using a the radially averaged
    atomic density (see methods for details). Hence, it closely follows the outline of a \ang{30} wedge out of
    the homology model.
    %
    (\subref{fig:model_charge_density})
    %
    The fixed space charge density ($\scdpore$) map of ClyA-AS, obtained by Gaussian projection of each atom's
    partial charge onto a 2D plane (see methods for details).
    %
    (\subref{fig:model_geometry_zoom}+\subref{fig:model_geometry})
    %
    The 2D-axisymmetric simulation geometry of ClyA (grey) embedded in a lipid bilayer (green) and surrounded
    by a spherical water reservoir (blue). Note that all electrolyte parameters depend on the local average
    ion concentration $\avionconc=\frac{1}{n}\sum_{i}^{n}\concentration_{i}$ and that some are also influenced
    by the distance from the nanopore wall $\walldistance$.
  %
  }\label{fig:model_concept}
\end{figure*}


To the best of our knowledge, no attempt has been made to consolidate all of the corrections discussed above
into a single framework. Hence, we propose an extended set of PNP-NS (ePNP-NS) equations, which improves the
predictive power of the PNP-NS equations at the nanoscale and beyond infinite dilution. Our ePNP-NS framework
takes into account the finite size of the ions using a size-modified PNP theory,\cite{Lu-2011} and implements
spatial-dependencies for the solvent viscosity,\cite{Pronk-2014,Hsu-2017} the ion diffusion coefficients and
their mobilities.\cite{Makarov-1998,Noskov-2004} It also includes self-consistent concentration-dependent
properties---based on empirical fits to experimental data---for all ions in terms of diffusion
coefficients and mobilities,\cite{Baldessari-2008-1,Mills-1989} and for the solvent in terms of density,
viscosity\cite{Hai-Lang-1996} and relative permittivity\cite{Gavish-2016}. To validate our new framework, we
applied it directly to a 2D-axisymmetric model of Cytolysin A (ClyA), a large protein nanopore that typically
contains 12 subunits\cite{Mueller-2009} or more\cite{Soskine-2013} and has been extensively used in
experimental studies of both proteins\cite{Soskine-2013,VanMeervelt-2014,Soskine-Biesemans-2015,
Biesemans-Soskine-2015,Wloka-2017,VanMeervelt-2017,Galenkamp-2018,Willems-Ruic-Biesemans-2019} and
DNA.\cite{Franceschini-2013,Franceschini-2016,Nomidis-2018} This allowed us to gauge the qualitative and
quantitative performance of the ePNP-NS equations and simultaneously elucidate previously unaccessible details
about the environment inside the pore.

The remainder of this paper is organized as follows. In \emph{\nameref{sec:model}} we describe the equations
governing our ePNP-NS framework and detail the construction of the 2D-axisymmetric ClyA model. Next, in
\emph{\nameref{sec:results}}, we validate our model by direct comparison of simulated ionic conductance with
experimentally measured values. We then proceed to characterize the influence of the bulk ionic strength and
the applied bias voltage on cation and anion concentrations inside the pore, the electrostatic potential
distribution and magnitude of the electro-osmotic flow. Finally, we touch upon our key findings and their
impact in \emph{\nameref{sec:conclusions}} and describe our protocols in more detail in
\emph{\nameref{sec:methods}}.




\section{Mathematical model}\label{sec:model}

The use of \emph{continuum} or \emph{mean-field} representations for both the nanopore and the electrolyte
enables us to efficiently compute the steady-state ion and water fluxes under almost any condition. The
dynamic behavior of our complete system is described by the coupled Poisson, Nernst-Planck and Navier-Stokes
(PNP-NS) equations, a set of partial differential equations that describe the electrostatic field,
the total ionic flux and the fluid flow, respectively.\cite{Eisenberg-1996,Cervera-2005,Lu-2012}

\subsection{Model geometry}\label{sec:geom}

\subsubsection{2D-axisymmetric model of ClyA.}
%
ClyA is a relatively large protein nanopore that self-assembles on lipid bilayers to form \SI{14}{\nm} long
hydrophilic channels. The interior of the pore can be divided into roughly two cylindrical compartments
(\cref{fig:clya_side}): the \cisi{} \lumeni{} (\SI{\approx6}{\nm} diameter, \SI{\approx10}{\nm} height), and
the \transi{} constriction (\SI{\approx3.3}{\nm} diameter, \SI{\approx4}{\nm} height). Because ClyA consists
of 12 identical subunits (\cref{fig:clya_top}), it exhibits a high degree of radial symmetry, a geometrical
feature that can be exploited to obtain meaningful results at a much lower computational
cost.\cite{Cervera-2005,Lu-2012, Pederson-2015} However, this requires the reduction of the full 3D atomic
structure and charge distribution to a realistic 2D-axisymmetric model. To this end, we constructed a
full-atom homology model of ClyA-AS type I---a dodecameric variant of the wild type ClyA from \textit{S.
Typhii} artificially evolved for improved stability\cite{Soskine-2013}---and equilibrated it at
\SI{298.15}{\kelvin} for \SI{30}{\ns} in an explicit solvent with harmonic constraints on the protein backbone
atoms (see~\nameref{sec:methods} for details). From the final \SI{5}{\ns} of this trajectory we extracted 50
sets of atomic coordinates for ClyA (\ie~every \SI{100}{\ps}). For each of these structures, we computed a 2D
atomic density\cite{Li-2013} and charge\cite{Aksimentiev-2005} map (see~\nameref{sec:methods} for details),
which we then averaged to represent the conformational diversity of the side chains. The geometry of the
nanopore was then defined as the \SI{25}{\percent} contour line of the density map, which closely follows the
outline of a \SI{30}{\degree} `wedge' of the full atom structure (\cref{fig:model_geometry_vs_wedge}). The
equilibrium charge map (\cref{fig:model_charge_density}) was loaded directly into our solver as a linear
interpolation function ($\scdpore$) and applied across all computational domains.

\subsubsection{Global geometry.}
%
The complete system (\cref{fig:model_geometry_zoom,fig:model_geometry}) consists of a large hemispherical
electrolyte reservoir ($R=\SI{250}{\nm}$), split through the middle into a \cisi{} and a \transi{} compartment
by a lipid bilayer ($h=\SI{2.8}{\nm}$), which contains the nanopore at its center. Both the bilayer and the
nanopore are represented by dielectric blocks (see~\cref{tab:corrections_equations} for parameters) that are
impermeable to ions and water.


\subsection{Governing equations}\label{sec:goveq}

To improve upon the quantitative accuracy of the PNP-NS equations for nanopore simulations, we developed an
\emph{extended} version of these equations (ePNP-NS) and implemented it in the commercial finite element
solver COMSOL Multiphysics (v5.4, www.comsol.com). Our ePNP-NS equations self-consistently take into account
(1) the finite size of the ions,\cite{Borukhov-1997,Lu-2011} (2) the reduction of ion and water motility close
to the nanopore walls,\cite{Makarov-1998,Noskov-2004,Pronk-2014,Pederson-2015,Vo-2016} and (3) the
concentration dependency of ion diffusion coefficients and electrophoretic mobilities, as well as electrolyte
viscosity, density and relative permittivity.\cite{Mills-1989,Hai-Lang-1996,Gavish-2016} Most of these
corrections make use of empirical functions that were fitted to experimental data
(\cref{tab:corrections_equations,suppinfo:tab:corrections_parameters}),
the implementation of which will be explored next.

\subsubsection{Electrostatic field.}
% 
We make use of Poisson's equation to evaluate the electric potential
%
\begin{align}
  \label{eq:poisson}
  \nabla \cdot \left(\absperm \relperm \nabla \potential \right) = -\left( \scdpore + \scdion \right)
  \text{\,,}
\end{align}
%
with $\potential$ the electric potential, $\absperm$ the vacuum permittivity
(\SI{8.85419e-12}{\farad\per\meter}) and $\relperm$ local relative permittivity
(\cref{suppinfo:tab:corrections_parameters}).
The pore's \emph{fixed} charge distribution, $\scdpore$,  was derived directly from the full atom model of
ClyA-AS (see~\cref{eq:scdpore}). The \emph{ionic} charge density in the fluid is given by
%
\begin{align}\label{eq:scdion}
  \scdion = \faraday\sum_{i}\chargen_{i}\concentration_{i}
  \text{\,,}
\end{align}
%
with $\faraday$ Faraday's constant (\SI{96485.33}{\coulomb\per\mole}), and $\concentration_{i}$ the ion
concentration and $\chargen_{i}$ ion charge number of ion $i$.
To account for the concentration dependence of the electrolyte's relative permittivity, we replaced $\relperm$
with the expression
%
\begin{align}
  \relperm(\avionconc) = \relperm^0 \relperm^c(\avionconc)
  \text{\,,}
\end{align}
%
with $\avionconc=\frac{1}{n}\sum_{i}^{n}\concentration_{i}$ the average ion concentration, $\relperm^0$ the
relative permittivity at infinite dilution and $\relperm^c(\avionconc)$ a concentration dependent empirical
function parameterized with experimental data.

%
\begin{table}[!t]
  \footnotesize
  \renewcommand{\arraystretch}{1.2}
  \caption{\ Summary of the parameters and fitting equations used in the ePNP-NS equations}
  \centering
  \label{tab:corrections_equations}
  
  \begin{tabularx}{0.48\textwidth}{>{\raggedright\hsize=1.5cm}X >{\hsize=1cm}l >{\hsize=3.8cm}X >{\hsize=1cm}l}
    \toprule
  
    Name
      & Symbol$^\text{\emph{a}}$
        & Infinite dilution value$^\text{\emph{b}}$/Function$^\text{\emph{c}}$
        & Reference \\
  
    \midrule
  
    \multirow{4}{1.5cm}{Relative permittivity}
      & $\permittivity_{r,\text{p}}$
        & \num{20}
        & \citenum{Li-2013} \\
      & $\permittivity_{r,\text{m}}$
        & \num{3.2}
        & \citenum{Gramse-2013} \\
      & $\permittivity_{r,\text{f}}^0$
        & \num{78.15}
        & \citenum{Gavish-2016} \\
      & $\permittivity_{w}(\dconc)$
        & $1 - \left(1 -	\dfrac{P_1}{P_0}\right) L \left( \dfrac{3P_2}{P_0 - P_1} \dconc \right)$
        & \citenum{Gavish-2016} \\
    \multirow{4}{1.5cm}{Ion self-diffusion coefficient}
      & $\diffusion_{\Na}^0$
        & \SI{1.334e-9}{\square\meter\per\second}
        & \citenum{Mills-1989} \\
      & $\diffusion_{\Cl}^0$
        & \SI{2.032e-9}{\square\meter\per\second}
        & \citenum{Mills-1989} \\
      & $\diffusion_{i}^c(\dconc)$
        & $\left( 1 + P_1\dconc^{0.5} + P_2\dconc + P_3\dconc^{1.5} + P_4\dconc^2 \right)^{-1}$
        & This work \\
      & $\diffusion_{i}^w(\dwall)$
        & $1-\exp{\left(-P_1(\dwall+P_2)\right)}$
        & \citenum{Makarov-1998,Simakov-2010} \vspace{0.25cm} \\
  
    \multirow{4}{1.5cm}{Ion electrophoretic mobility}
      & $\mobility_{\Na}^0$
        & \SI{5.192e-4}{\square\meter\per\second\per\volt}
        & \citenum{Bianchi-1989} \\
      & $\mobility_{\Cl}^0$
        & \SI{7.909e-4}{\square\meter\per\second\per\volt}
        & \citenum{Bianchi-1989} \\
      & $\mobility_{i}^c(\dconc)$
        & $\left( 1 + P_1\dconc^{0.5} + P_2\dconc + P_3\dconc^{1.5} + P_4\dconc^2 \right)^{-1}$
        & This work \\
      & $\mobility_{i}^w(\dwall)$
        & $1-\exp{\left(-P_1(\dwall+P_2)\right)}$
        & \citenum{Makarov-1998,Simakov-2010} \vspace{0.25cm} \\
  
    \multirow{2}{1.5cm}{Ion transport number}
      & $\transportn_{\Na}^0$
        & 0.396
        & \citenum{Bianchi-1989} \\
      & $\transportn_{\Na}(\dconc)$
        & $\left( 1 + P_1\dconc^{0.5} + P_2\dconc + P_3\dconc^{1.5} + P_4\dconc^2 \right)^{-1}$
        & This work \vspace{0.25cm} \\
  
    \multirow{3}{1.5cm}{Dynamic viscosity}
      & $\viscosity^0$
        & \SI{8.904}{\pascal\second}
        & \citenum{Hai-Lang-1996} \\
      & $\viscosity^c(\dconc)$
        & $ 1 + P_1 \dconc^{0.5} + P_2 \dconc + P_3 \dconc^2 + P_4 \dconc^{3.5}$
        & This work \\
      & $\viscosity^w(\dwall)$
        & $1 + \exp{ \left( -P_1(\dwall-P_2) \right) }$
        & \citenum{Pronk-2014} \vspace{0.25cm} \\
  
    \multirow{2}{1.5cm}{Fluid density}
      & $\density^0$
        & \SI{997}{\kilogram\per\cubic\meter}
        & \citenum{Hai-Lang-1996} \\
      & $\density(\dconc)$
        & $1 + P_1 \dconc + P_2 \dconc^2$
        & This work \vspace{0.25cm} \\
    
    \bottomrule
  \end{tabularx}
  \begin{flushleft}
    $^\text{\emph{a}}$Dependencies on either $\dconc = \avionconc/1$~M (dimensionless average ion
    concentration) and $\dwall = \walldistance/1$~nm (dimensionless distance from the nanopore wall);
    $^\text{\emph{b}}$Values at infinite dilution for a system temperature of \SI{298.15}{\kelvin};
    $^\text{\emph{c}}$These functions are empirical and hence have no physical meaning.
    The values of the fitting parameters $P_x$ of each property can be found in \cref{suppinfo:tab:corrections_parameters} and graphs of the fits in \cref{suppinfo:fig:corrections}.
  \end{flushleft}
\end{table}
%

\subsubsection{Ionic flux.}
%
The total ionic flux $\flux_{i}$ of each ion $i$ is given by the size-modified Nernst-Planck
equation,\cite{Lu-2011} and can be expressed as the sum of diffusive, electrophoretic, convective and steric
fluxes
%
\begin{align}
  \label{eq:sm-nernst-planck}
  \flux_{i} = -\left[
    \diffusion_{i} \nabla \concentration_{i}
    + \chargen_{i} \mobility_{i} \concentration_{i} \nabla \potential
    - \velocity \concentration_{i}
    + \diffusion_{i} \vec{\beta_{i}} \concentration_{i} \right]
  \text{\,,}
\end{align}
%
where $\vec{\beta_{i}}$ is the steric flux vector
%
\begin{align}
  \vec{\beta_{i}} =
      \frac{ \ionsize_{i}^3 / \ionsize_{0}^3 \dsum_{j} \avogadro \ionsize_{j}^3 \nabla \concentration_{j} }
          { 1 - \dsum_{j} \avogadro \ionsize_{j}^3 \concentration_{j} }
  \text{\,,}
\end{align}
%
and at steady state
%
\begin{align}
  \dfrac{\partial \concentration_{i}}{\partial \timedim} ={}& - \nabla \cdot \flux_{i} = 0
  \text{\,,}
\end{align}
%
with $\diffusion_{i}$ the ion diffusion coefficient, $\concentration_{i}$ the ion concentration,
$\chargen_{i}$ the ion charge number, $\mobility_{i}$ the electrophoretic mobility of ion $i$. $\potential$ is
the electrostatic potential, $\velocity$ the fluid velocity and $\avogadro$ Avogadro's constant
(\SI{6.022e23}{\per\mole}). $\ionsize_{i}$ and $\ionsize_{0}$ are \emph{steric} cubic diameters of
respectively ions and water molecules. Because currently there are no experimentally verified values available
for $\ionsize_{i}$ and $\ionsize_{0}$, we set them to \SI{0.5}{\nm} (max. \SI{13.3}{\Molar}) and
\SI{0.311}{\nm} (max. \SI{55.2}{\Molar}).\cite{Bazant-2009}

The reduction of the ionic motility at increasing salt concentrations and in proximity to the nanopore walls
was implemented self-consistently by replacing $\diffusion_{i}$ and $\mobility_{i}$ with the expressions
%
\begin{align}
  \diffusion_{i}(\avionconc,\walldistance) ={}&
      \diffusion_{i}^0 \diffusion_{i}^c(\avionconc) \diffusion_{i}^w(\walldistance)  \\
  \mobility_{i}(\avionconc,\walldistance) ={}&
      \mobility_{i}^0 \mobility_{i}^c(\avionconc) \mobility_{i}^w(\walldistance)
\end{align}
%
where $\diffusion_{i}^0$ and $\mobility_{i}^0$ represent the values at infinite dilution. The concentration
dependent factors $\diffusion_{i}^c(\avionconc)$ and $\mobility_{i}^c(\avionconc)$ are empirical functions
fitted to experimental data (between \SIrange{0}{5}{\Molar} \ce{NaCl}) of respectively the ion self-diffusion
coefficients\cite{Mills-1989} and the electrophoretic
mobilities.\cite{Bianchi-1989,Currie-1960,Goldsack-1976,DellaMonica-1979} Likewise, the factors
$\diffusion_{i}^w(\walldistance)$ and $\mobility_{i}^w(\walldistance)$ are empirical functions that introduce
a spatial dependency on the distance from the nanopore wall $\walldistance$, and were parameterized by fitting
to molecular dynamics data.\cite{Noskov-2004,Simakov-2010,Makarov-1998,Wilson-2019}

Based on the observation that the diffusivity of nanometer- to micrometer-sized particles reduces
significantly when confined in pores and slits of comparable dimensions,\cite{Renkin-1954,Deen-1987,
Dechadilok-2006,Muthukumar-2014,Kannam-2017} Simakov \etal{}\cite{Simakov-2010} and Pederson
\etal{}\cite{Pederson-2015} reduced the ion motilities inside the pore as a function of the ratio between the
ion and the nanopore radii. We chose not to include this correction into our model, as extrapolating its
applicability for ions with a hydrodynamic radii comparable to size of the solvent molecules is
questionable.\cite{Anderson-1972,Deen-1987}

\subsubsection{Fluid flow.}
%
As derived by Axelsson \etal{},\cite{Axelsson-2015} the fluid flow and pressure field inside an incompressible
fluid with a variable density and variable viscosity is given by the Navier-Stokes equations:
%
\begin{align}
  \label{eq:navier-stokes-variable}
  \dfrac{\partial}{\partial \timedim} \left( \density \velocity \right) +
  \left( \velocity \cdot \nabla \right) \left( \density\velocity \right)
  + \nabla \cdot \hydrostresstensor = \volumeforce
  \text{\,,}
\end{align}
%
where
%
\begin{align}
  \hydrostresstensor =
  \pressure\identity - \viscosity\left[\nabla\velocity+\left(\nabla\velocity \right)^\mathsf{T}\right]
  \text{\,,}
\end{align}
%
together with the continuity equations for the fluid density
%
\begin{align}
  \label{eq:continuity-density}
  \dfrac{\partial \density}{\partial \timedim} + \velocity \cdot \nabla \density  = 0
  \text{\,,}
\end{align}
%
and the divergence constraint for the momentum
%
\begin{align}
  \nabla \cdot \left( \density\velocity \right) - \velocity \cdot \nabla \density ={}& 0
  \text{\,,}
\end{align}
%
with $\velocity$ the fluid velocity, $\density$ the fluid density, $\hydrostresstensor$ the hydrodynamic
stress tensor, $\viscosity$ the viscosity and $\pressure$ the pressure. The external body force density
$\volumeforce$ that acts on the fluid is given by
%
\begin{align}\label{eq:ion_force_density}
  \volumeforce = \echarge\avogadro\scdion\efield
  \text{\,,}
\end{align}
%
with $\efield = - \nabla \potential$ the electric field vector.
At steady-state, the partial derivatives w.r.t.~time in \cref{eq:navier-stokes-variable,eq:continuity-density}
become equal to zero:
\begin{align}
  \dfrac{\partial}{\partial \timedim} \left( \density \velocity \right) ={}& 0 \\
  \dfrac{\partial \density}{\partial \timedim} ={}& 0
  \text{\,.}
\end{align}

As with the previous equations, we introduced a concentration dependency and wall distance dependencies for
$\viscosity$ and a concentration dependency for the $\density$ by replacing their constant values by
%
\begin{align}
  \viscosity(\avionconc,\walldistance) ={}&
    \viscosity^0 \viscosity^c(\avionconc) \viscosity^w(\walldistance) \\
  \density(\avionconc) ={}&
    \density^0 \density^c(\avionconc)
\end{align}
%
where $\viscosity^0$ and $\density^0$ are the values at infinite dilution (\ie~pure water). The empirical
functions $\viscosity^c(\avionconc)$, $\density^c(\avionconc)$ and $\viscosity^w(\walldistance)$ were
parameterized \textit{via} fitting to experimental\cite{Hai-Lang-1996} and molecular dynamics\cite{Pronk-2014}
data obtained from literature.

\subsection{Boundary conditions and concentration dependencies}
%
The reservoir boundaries were set up, with Dirichlet conditions, to act as electrodes: the \cisi{} side was
grounded ($\potential = 0$) and a fixed bias potential was applied along the \transi{} edge
($\potential = \vbias$). To simulate the presence of an endless reservoir, the ion concentration at both
external boundaries were fixed to the bulk salt concentration ($\concentration_i = \cbulk$) and the
unconstrained flow in and out of the computational domain was enabled by means of a `no normal stress`
condition ($\hydrostresstensor\vec{n} = 0$). The boundary conditions on the edges of the reservoir shared with
the nanopore and bilayer were set to no-flux ($-\vec{n}\cdot\vec{\flux}_{i} = 0$) and no-slip
($\vec{\velocity} = 0$), preventing the flux of ions through them and mimicking a sticky hydrophilic surface,
respectively. Finally, a Neumann boundary condition was applied at the bilayer's external boundary
($-\vec{n}\cdot\left(\absperm \relperm \nabla \potential \right) = 0$).
%
All concentration dependent parameters use the local ionic strength rather than their individual ion
concentrations. Though valid for \emph{electroneutral} bulk solutions, this approximation no longer holds
inside the electrical double layer (\ie~near charged surfaces or inside small nanopores), where local
electroneutrality is violated. The main reasons for making this simplification regardless are the lack of
non-bulk experimental data and the absence of a tractable analytical model. Furthermore, we will see that the
current implementation of our concentration dependent functions will lead to an excellent agreement with the
experimental data in all but the most extreme cases, justifying our choice \textit{a posteriori}. 
%
The ePNP-NS equations revert into the regular PNP-NS equations by disabling the steric flux ($\vec{\beta}=0$)
and by setting all concentration and wall distance functions to unity ($\relperm = \relperm^0$,
$\diffusion_{i} = \diffusion_{i}^0$, $\mobility_{i} = \mobility_{i}^0$, $\viscosity = \viscosity^0$ and
$\density = \density^0$).

%
%
%
\section{Results and discussion}\label{sec:results}

The current--voltage (IV) relationships of many nanopores, ClyA included, often deviate significantly from
Ohm's law. This is because the ionic flux arises from a complex interplay between the pore's geometry
(\eg~size, shape and charge distribution), the properties of the surrounding electrolyte (\eg~salt
concentration viscosity and relative permittivity) and the externally applied conditions (\eg~bias voltage,
temperature and pressure). The ability of a computational model to quantitatively predict the ionic current of
a nanopore over a wide range of bias voltages and salt concentrations strongly indicates that it captures the
essential physics governing the nanofluidic transport. Hence, to validate our model, we experimentally
measured the single channel ionic conductance of ClyA at a wide range of experimentally relevant salt
concentrations ($\cbulk$) and bias voltages ($\vbias$). We compared these experimental data with the simulated
ionic transport properties in terms of current, conductance, rectification and ion selectivity, of both the
classical PNP-NS and the newly developed ePNP-NS equations---finding a quantitative match between the
experiment and the simulations.

% After validation, we proceed with describing the influence of the
% bias voltage ($\vbias$) and bulk salt concentration ($\cbulk$) on the local ion and charge distribution and
% inside the pore. This is followed by a characterization of the electrostatic potential and the electrostatic
% energy landscape within ClyA for both cations and anions. We will conclude this section by discussing the
% properties of the electro-osmotic flow.
% and a brief reflection of the key findings in this work.


\begin{figure*}[!t]
  \centering
  \begin{minipage}[l]{16cm}
    \begin{minipage}[t]{5cm}
      \begin{subfigure}[t]{4.5cm}
        \centering
        \caption{}\vspace{-3mm}\label{fig:current-voltage_curves}
        \includegraphics[scale=1]{figures/conductance/current_vs_voltage_all_multiplot}
      \end{subfigure}
    \end{minipage}
    %
    \begin{minipage}[t]{5cm}
      \begin{subfigure}[t]{5cm}
        \centering
        \caption{}\vspace{0mm}\label{fig:conductance_contourmap_epnp}
        \includegraphics[scale=1]{figures/conductance/conductance_contourmap_epnp}
      \end{subfigure}
      \\
      \begin{subfigure}[t]{5cm}
        \centering
        \caption{}\vspace{0mm}\label{fig:conductance_loglog_exp_pnp_epnp}
        \includegraphics[scale=1]{figures/conductance/conductance_loglog_exp_pnp_epnp}
      \end{subfigure}
    \end{minipage}
    %
    \begin{minipage}[t]{5cm}
      \begin{subfigure}[t]{5cm}
        \centering
        \caption{}\vspace{0mm}\label{fig:transport_number_contourmap_epnp}
        \includegraphics[scale=1]{figures/conductance/transport_number_contourmap_epnp}
      \end{subfigure}
      \\
      \begin{subfigure}[t]{5cm}
        \centering
        \caption{}\vspace{0mm}\label{fig:transport_number_vs_concentration_bulk_pnp_epnp}
        \includegraphics[scale=1]{figures/conductance/transport_number_vs_concentration_bulk_pnp_epnp}
      \end{subfigure}
    \end{minipage}
  \end{minipage}

  \caption%
  %
  [Measured and simulated ionic conductance and cation selectivity of single ClyA nanopores.]
  %
  {%
    %
    Measured and simulated ionic conductance and cation selectivity of single ClyA nanopores.
    %
    (\subref{fig:current-voltage_curves})
    %
    Comparison between the experimentally (expt.) measured, bulk pore model (bulk) and the simulated (PNP-NS
    and ePNP-NS) current-voltage (IV) curves of ClyA-AS at \SI{25\pm1}{\dC} between
    $\vbias=\text{\SIrange{-200}{+200}{\mV}}$, and for $\cbulk=\text{\SIlist{0.05;0.15;0.5;1;3}{\Molar}}$
    \ce{NaCl}. The bulk current was calculated by \cref{suppinfo:eq:bulk_nanopore_current} by modeling ClyA as
    two series resistors (\cref{suppinfo:eq:bulk_nanopore_conductance}),\cite{Soskine-2013,Kowalczyk-2011}
    using the bulk \ce{NaCl} conductivity at the given concentrations. Experimental errors ($n=3$) were
    smaller than the symbol size and hence not shown.
    %
    (\subref{fig:conductance_contourmap_epnp})
    %
    Contour plot of the simulated (ePNP-NS) ionic conductance $\conductance = \current /\vbias$ as a function
    of $\vbias$ and $\cbulk$.
    %
    (\subref{fig:conductance_loglog_exp_pnp_epnp})
    %
    Log-log plots of $\conductance$ as a function of $\cbulk$ at \SI{+150}{\mV} (top) and \SI{-150}{\mV}
    (bottom)---comparing results obtained through experiments, PNP-NS and ePNP-NS simulations, and the simple
    resistor pore model.
    %
    (\subref{fig:transport_number_contourmap_epnp})
    %
    Contour plot of the \Na{} transport number $\tna = \gna / \conductance$, computed from the individual
    ionic conductances in the ePNP-NS simulation, as a function of $\vbias$ and $\cbulk$. The $\tna$ expresses
    the fraction of the ionic current is carried by \Na{} ions, \ie~the cation selectivity.
    %
    (\subref{fig:transport_number_vs_concentration_bulk_pnp_epnp})
    %
    Simulated (PNP-NS and ePNP-NS) values of $\tna$ as a function of $\cbulk$ for \SI{+150}{\mV} (top) and
    \SI{-150}{\mV} (bottom). Here, the `bulk' line indicates the bulk \ce{NaCl} cation transport number,
    represented by its empirical function $\tna(\cbulk)$ (see \cref{tab:corrections_equations} and
    \cref{suppinfo:tab:corrections_parameters}). The solid grey line represents $\tna = 0.5$.
    %
  }\label{fig:conductance}
\end{figure*}


\subsection{Transport of ions through ClyA}\label{sec:iont}

\subsubsection{Ionic current and conductance.}
%
The ability of our model to reproduce the ionic current of a biological nanopore over a wide range of
experimentally relevant conditions (between $\vbias = \text{\SIrange{-150}{+150}{\mV}}$ and for $\cbulk =
\text{\SIlist{0.05;0.15;0.5;1;3}{\Molar}}$ \ce{NaCl}) can be seen in \cref{fig:current-voltage_curves}. Here,
we compare IV relationships of ClyA-AS as measured experimentally (`expt.'), simulated using our
2D-axisymmetric model (`PNP-NS' and `ePNP-NS') and naively analytically estimated (`bulk') using a resistor
model of the pore\cite{Soskine-2013,Kowalczyk-2011}
(\cref{suppinfo:eq:bulk_nanopore_current,suppinfo:eq:bulk_nanopore_conductance}). Whereas the classical PNP-NS
equations consistently overestimated the ionic current, particularly at high salt concentrations, the
predictions of the ePNP-NS equations corresponded closely to the measured values, \emph{especially} at high
ionic strengths ($\cbulk > \text{\SI{0.5}{\Molar}}$). The inability of the classical PNP-NS equations to
correctly estimate the current is expected however, as in this regime the model parameters (\eg~diffusivity,
mobility and viscosity, ...) begin to deviate significantly from their `infinite dilution' values
(see~\cref{suppinfo:fig:corrections}). At $\cbulk < \text{\SI{0.15}{\Molar}}$ the ePNP-NS equations tended to
minorly overestimate the ionic current, but the discrepancies were much smaller than those observed for
PNP-NS. Finally, the bulk model managed to capture the currents surprisingly well at high salt concentrations
and positive bias voltages, indicating that, under these conditions, the distribution of ions inside the pore
is similar the the bulk. In contrast, this simple model faltered in the negative voltage regime, suggesting
that next to the geometry of the pore, also its electrostatics are important.

The ability of a nanopore to conduct ions can be best expressed by its conductance: $\conductance = \current /
\vbias$. We computed ClyA's conductance with the ePNP-NS equations as a function of bias voltage
($\vbias=\text{\SIrange{-200}{+200}{\mV}}$) and bulk \ce{NaCl} concentration
($\cbulk=\text{\SIrange{0.005}{5}{\Molar}}$), of which a contour plot can be found in
\cref{fig:conductance_contourmap_epnp}. The near horizontal contour lines in the upper part of the plot
show that, at high ionic strengths ($\cbulk>\SI{1}{\Molar}$), ClyA maintains the same conductance regardless
of the applied bias voltage. This behavior changes at intermediate concentrations
($\SI{0.1}{\Molar}<\cbulk<\SI{1}{\Molar}$), where maintaining the same conductance level with increasing
negative bias amplitudes requires increasing salt concentrations. Finally, at low salt concentrations
($\cbulk<\SI{0.1}{\Molar}$), the ionic conductance increases when reducing the negative voltage amplitude but
subsequently levels out at positive bias voltages.

The cross-sections of the ionic conductance as a function of concentration at high positive and negative bias
voltages (\cref{fig:conductance_loglog_exp_pnp_epnp}), serve to demonstrate the differences between these
respective regimes. At high positive ($\vbias=\text{\SI{+150}{\mV}}$) and negative
($\vbias=\text{\SI{-150}{\mV}}$) bias voltages, the slopes of the conductance log-log plots with respect to
the bulk salt concentration show linear and bi-linear behavior, respectively. This could be indicative of a
different mode of ion conduction of positive and negative bias voltages, at least for low concentrations
($\cbulk<\text{\SI{0.15}{\Molar}}$). While the bulk model and the PNP-NS equations manage to capture the
conductance at respectively high and low ionic strengths, only the ePNP-NS equations perform well over the
entire concentration range. Overall, the predictions made using PNP-NS overestimate the conductance over the
entire concentration range, but they do converge with those computed with ePNP-NS when approaching infinite
dilution ($\cbulk<\text{\SI{0.01}{\Molar}}$).
% At $\vbias = \text{\SI{-150}{\mV}}$, the log-log plot consists of two linear segments with a transition zone
% at $\cbulk \approx \text{\SI{0.15}{\Molar}}$. This behavior is captured qualitatively by both simulation
% methods, but an excellent quantitative match is only found for the ePNP-NS equations. The bulk model
% exhibits the same single-sloped trend as seen at $\vbias = \text{\SI{+150}{\mV}}$.

The difference in ionic conduction at opposing bias voltages is also known as ionic current rectification
(ICR): $\icr(\vbias) = \conductance(+\vbias) / \conductance(-\vbias)$. ICR is a phenomenon often observed in
nanopores that are both charged, and contain a degree of geometrical asymmetry along the central axis of the
pore.\cite{Constantin-2007,White-2008,Wang-2014} As can be seen in \cref{suppinfo:fig:icr}, ClyA exhibits a
strong degree of rectification, which is to be expected given its predominantly negatively charged interior
and asymmetric \cisi{} (\SI{\approx3.3}{\nm}) and \transi{} (\SI{\approx6}{\nm}) entry diameters
(\cref{fig:clya_side}). We found $\icr$ to increase monotonously with the bias voltage magnitude, at least
over the investigated range. We found the dependence of $\icr$ on the ionic strength not to be monotonous, but
rather rising rapidly to a peak value at $\cbulk=\text{\SI{0.15}{\Molar}}$, followed by a gradual decline
towards unity at saturating salt concentrations. This concentration is within the transition zone observed in
the conductance at negative bias voltages (\cref{fig:conductance_loglog_exp_pnp_epnp}) and provides further
evidence for a change in the conductive properties of the pore in this regime.

The results and comparisons discussed above indicate that ClyA's conductivity is dominated by the bulk
electrolyte conductivity above physiological salt concentrations ($\cbulk > \text{\SI{0.15}{\Molar}}$). The
breakdown of this simple dependency at lower ionic strengths is particularly evident at negative bias voltages
and is likely caused by the overlapping of the electrical double layer (EDL) inside the pore (\ie~the Debye
length is \SI{\approx1.4}{\nm} at $\cbulk=\text{\SI{0.05}{\Molar}}$). This effectively excludes the co-ions
(\Cl) from the interior of the pore and attracts as many counter-ions (\Na) as needed to screen the fixed
charges of the pore. As a result, \Cl{} ions do not contribute to ionic current and the conductance is
dominated by \Na{} ions attracted by ClyA's `surface' charges.\cite{Uematsu-2018} The presence of only a
single ion type inside the pore at low ionic strength may also offer an explanation as to why the ePNP-NS
equations are more accurate at higher ionic strengths ($\cbulk\ge\text{\SI{0.15}{\Molar}}$). Because our ionic
mobilities are derived from \emph{bulk} ionic conductances, \ie~for unconfined ions in a locally
\emph{electroneutral} environment, it is likely that our mobility model begins to break down under conditions
where only a single ion type is present.\cite{Duan-2010} Another cause of the discrepancies could be a slight
narrowing of the nanopore at low salt concentrations, which cannot be captured by our simulation due to the
static nature of its geometry and charge distributions. Nevertheless, our simplified 2D-axisymmetric model, in
conjunction with the ePNP-NS equations, is able to accurately predict the ionic current flowing through ClyA
for a wide range of experimentally relevant ionic strengths and bias voltages. This suggests that our
continuum system can accurately capture the essential physical phenomena that drive the ion and water
transport through the nanopore both \emph{qualitatively} and \emph{quantitatively}. Hence, we expect the
distribution of the resulting properties (\eg~ion concentrations, ion fluxes, electric field and water
velocity) to closely correspond to their true values.

\subsubsection{Cation selectivity.}
%
The ion selectivity of a nanopore determines the preference with which it transports one ion type over the
other. Experimentally, it is often determined by placing the pore in a salt gradient (\ie~different salt
concentrations in the \cisi{} and \transi{} reservoirs) and measuring the reversal potential ($\revpot$),
\ie~the bias voltage at which the nanopore current is zero.\cite{Soskine-2013,Franceschini-2016} The
Goldman-Hodgkin-Katz (GHK) equation can then be used to convert $\revpot$ into the permeability ratio $\pna =
\gna / \gcl$. Here, we represent the ClyA's ion selectivity
%
(\cref{fig:transport_number_contourmap_epnp,fig:transport_number_vs_concentration_bulk_pnp_epnp}) 
%
by the fraction of the total current that is carried by \Na{} ions: the apparent \Na{} transport number
$\tna = \gna / (\gna + \gcl) = \pna / (\pna + 1)$. 

As expected from its negatively charged interior, we found ClyA to be cation selective (\ie~$\tna > 0.5$) for
all investigated voltages up to a bulk salt concentration of $\cbulk \approx \text{\SI{2}{\Molar}}$ \ce{NaCl}
(0.5 contour line in \cref{fig:transport_number_contourmap_epnp}). Above this concentration, $\tna$ falls to a
minimum of value of 0.45 at $\cbulk \approx \text{\SI{5}{\Molar}}$, which is still \num{\approx1.27} times its
value in the bulk electrolyte (0.35). This shows that---even at saturating concentrations where the Debye
length is \SI{<0.2}{\nm}---ClyA enhances the transport of cations. Below \SI{2}{\Molar}, the ion selectivity
increases logarithmically with decreasing salt concentrations, but it also becomes more sensitive to the
direction and magnitude of the electric field:  with negative bias voltages yielding higher ion selectivities
(\cref{fig:transport_number_vs_concentration_bulk_pnp_epnp}). For example, to reach a selectivity of $\tna
\approx 0.9$, the salt concentration must fall to \SI{0.05}{\Molar} at \SI{+150}{\mV} and to only
\SI{0.125}{\Molar} at \SI{-150}{\mV}.

Using the reversal potential method, Franceschini \etal{}\cite{Franceschini-2016} found ClyA's ion selectivity
to be $\tna = 0.66$ ($\pna = 1.9$). This corresponds well to the average between the \cisi{}
($\cbulk=\text{\SI{1}{\Molar}}$, $\tna = 0.57$, $\pna = 1.3$) and \transi{}
($\cbulk=\text{\SI{0.15}{\Molar}}$, $\tna = 0.84$, $\pna = 5.4$) reservoir concentrations used in their
experiment. Therefore, this suggests that although measuring the reversal potential gives valuable insights
into the selectivity ion channels and small nanopores, it does not describe the ion selectivity under
\emph{symmetric} conditions. In addition, the GHK equation does not consider the ionic flux due to the
electro-osmotic flow and assumes that the Nernst-Einstein relation holds for all used concentrations. These
two effects should not be ignored as they contribute significantly to the total conductance of the pore.
Furthermore, because the ion selectivity depends strongly on the ionic strength and often the applied bias
voltage, the measured reversal potential will necessarily be influenced by the chosen salt gradient and
represents the selectivity at an undetermined intermediate concentration.

% 0.15 M, 0 mV
% Tna = 0.84299
% Pna = 5.36904

% 1.0 M, 0 mV
% Tna = 0.57429
% Pna = 1.34899

% 0.05 M, +150 mV
% Tna = 0.90583
% Pna = 9.61941

% 0.125 M, -150 mV
% Tna = 0.90359
% Pna = 9.37234

\subsection{Ion concentration distribution}\label{sec:ionc}
%

\begin{figure*}[!t]
  \centering
  \begin{minipage}[t]{8cm}
    \begin{subfigure}[t]{8cm}
      \centering
      \caption{}\vspace{-3mm}\label{fig:concentration_pore_average_vs_cbulk}
      \includegraphics[scale=1]{figures/concentration/concentration_pore_average_vs_concentration}
    \end{subfigure}
    \\
    \begin{subfigure}[t]{8cm}
      \centering
      \caption{}\vspace{-3mm}\label{fig:concentration_contours}
      \includegraphics[scale=1]{figures/concentration/concentration_contours}
    \end{subfigure}
    \\
    \begin{subfigure}[t]{8cm}
      \centering
      \caption{}\vspace{-3mm}\label{fig:concentration_radial_profiles}
      \includegraphics[scale=1]{figures/concentration/concentration_radial_profiles}
    \end{subfigure}
  \end{minipage}
  \begin{minipage}[t]{8cm}
    \begin{subfigure}[t]{8cm}
      \centering
      \caption{}\vspace{-3mm}\label{fig:ion_charge_dist_vs_cbulk}
      \includegraphics[scale=1]{figures/charge_density/ion_charge_pore_bulk_surface_total_vs_concentration}
    \end{subfigure}
    \\
    \begin{subfigure}[t]{8cm}
      \centering
      \caption{}\vspace{-3mm}\label{fig:ion_scd_contours}
      \includegraphics[scale=1]{figures/charge_density/ion_charge_density_contours}
    \end{subfigure}
    \\
    \begin{subfigure}[t]{8cm}
      \centering
      \caption{}\vspace{-3mm}\label{fig:ion_scd_radial_profiles}
      \includegraphics[scale=1]{figures/charge_density/ion_charge_density_radial_profiles}
    \end{subfigure}
  \end{minipage}

  \caption%
  %
  [Ion concentration distribution inside ClyA.]
  %
  {%
    %
    Ion concentration distribution inside ClyA.
    %
    (\subref{fig:concentration_pore_average_vs_cbulk})
    %
    Relative \Na{} and \Cl{} concentrations averaged over the entire pore volume ($\pavi$) as a function of
    the reservoir salt concentration ($\cbulk = \text{\SIrange{0.005}{5}{\Molar}}$) and bias voltage ($\vbias
    = \text{\SIrange{-200}{+200}{\mV}}$).
    %
    (\subref{fig:concentration_contours})
    %
    Contour plots of the relative ion concentration ($\ci/\cbulk$) for both \Na{} and \Cl{} for $\cbulk =
    \text{\SI{0.15}{\Molar}}$ and at $\vbias = \text{\SIlist{-150;+150}{\mV}}$.
    %
    (\subref{fig:concentration_radial_profiles})
    %
    The relative \Na{} and \Cl{} concentration profiles along the radius of the pore, through the middle of
    the constriction ($z = \text{\SI{-0.3}{\nm}}$) and the \lumeni{} ($z = \text{\SI{5}{\nm}}$), as indicated 
    by the arrows in (\subref{fig:concentration_contours}).
    %
    (\subref{fig:ion_charge_dist_vs_cbulk})
    %
    The average number of ionic charges inside the pore $\pav{\Qion}{\text{PT}}$, is distributed between those
    close to the pore's surface $\pav{\Qion}{\text{PS}}$, \ie~within \SI{0.5}{\nm} of the wall, and those in
    the `bulk' of the pore's interior $\pav{\Qion}{\text{PB}}$.
    %
    (\subref{fig:ion_scd_contours})
    %
    Cross-section contour plots of the ion space charge density ($\scdion$), expressed as number of elementary
    charges per \si{\cubic\nano\meter}, at $\vbias = \text{\SI{0}{\mV}}$ and for $\cbulk =
    \text{\SIlist{0.05;0.15;0.5;5}{\Molar}}$.
    %
    (\subref{fig:ion_scd_radial_profiles})
    %
    Radial cross-sections of the $\scdion$ at the center of the constriction ($z = \text{\SI{-0.3}{\nm}}$) and
    the \lumeni{} ($z = \text{\SI{5}{\nm}}$) of ClyA. The vertical line represents the the division between
    ions in the `bulk' ($\walldistance > \text{\SI{0.5}{\nm}}$) of the pore and those located near its surface
    ($\walldistance \le \text{\SI{0.5}{\nm}}$).
    %
  }\label{fig:concentration_and_scd}
  \end{figure*}

Following the validation of the model in previous section, we now proceed by describing the local ionic
concentrations inside ClyA. Detailed knowledge of the ionic environment can be valuable to experimentalists
who seek to trap and study single enzymes with
ClyA.\cite{Soskine-Biesemans-2015,VanMeervelt-2017,Galenkamp-2018} Moreover, it gives insight into the origin
of the ion current rectification, ion selectivity and the electro-osmotic flow. Note that the figures below
were obtained from a nanoscale continuum \emph{steady-state} simulation, they represent a time-averaged
situation on the on the order of \SIrange{10}{100}{\ns}.\cite{Im-2002}

% We evaluated the densities of
% both \Na{} and \Cl{} ions inside ClyA and investigated the effect of bulk salt concentration and the bias
% voltage on 1) the distribution of their concentration
% (\cref{fig:concentration_pore_average_vs_cbulk,fig:concentration_contours,fig:concentration_radial_profiles})
% and 2) the accumulation of mobile charges inside the pore as a result of their asymmetric enhancement or
% depletion (\cref{fig:ion_charge_dist_vs_cbulk,fig:ion_scd_contours,fig:ion_scd_radial_profiles}).

\subsubsection{Relative cation and anion concentrations.}
%
We use the relative ion concentration averaged over the entire volume of the pore ($\pavi$), as a measure for
global ionic conditions inside the pore (\cref{fig:concentration_pore_average_vs_cbulk}). At low ionic
strengths ($\cbulk < \text{\SI{0.05}{\Molar}}$), our simulation predicts a strong enhancement of the \Na{}
concentration ($\pavNa$) and a clear depletion of the \Cl{} concentration ($\pavCl$) inside the pore. This
effect diminishes rapidly with increasing ionic strengths, which can be explained by the electrolytic
screening of the negative charges lining the walls of ClyA (\ie~the electrical double layer). At low reservoir
concentrations ($\cbulk < \text{\SI{0.05}{\Molar}}$) the number of ions in the bulk is sparse, leading to the
attraction and repulsion of respectively as many \Na{} and \Cl{} ions as the chemical potential allows. As the
concentration increases, the overall availability of ions improves and the extreme concentration differences
between the pore and the bulk are no longer required to offset the fixed charges lining ClyA's interior walls.
For example, increasing the reservoir concentration at equilibrium ($\vbias = \text{\SI{0}{\mV}}$) from
\SIrange{0.005}{0.05}{\Molar} causes $\pavNa$ to fall an order of magnitude (\numrange{34}{4.4}) and $\pavCl$
to rise an order of magnitude (\numrange{0.05}{0.31}). Even though their concentrations still differ
significantly from those in the reservoir ($\pavNa = 2.1$ and $\pavCl = 0.58$) at physiological salt
concentrations ($\cbulk = \text{\SI{0.15}{\Molar}}$), they do approach bulk-like values ($1.14 \ge \pavNa \ge
1$ and $0.89 \le \pavCl \le 1$) at higher concentrations ($\cbulk \ge \text{\SI{1}{\Molar}}$).


We also observed a significant difference between their sensitivities to the applied bias voltage,
particularly at low salt concentrations (\cref{fig:concentration_pore_average_vs_cbulk}, left of
\SI{<0.15}{\Molar} line). Whereas the \Na{} concentration shows only a limited response, the \Cl{}
concentration changes much more dramatically. For example, at $\cbulk = \text{\SI{0.15}{\Molar}}$ and when
changing the bias voltage from \SIrange{-150}{+150}{\mV}, $\pavNa$ rises \num{\approx 1.7}-fold
(\numrange{1.6}{2.7}) and $\pavCl$ increases \num{\approx 3.8}-fold (\numrange{0.28}{1.06}). This difference
is clearly visualized by the contour plots of the relative ion concentrations ($\ci/\cbulk$) at $\cbulk =
\text{\SI{0.15}{\Molar}}$ and for $\vbias = \text{\SIlist{-150;+150}{\mV}}$
(\cref{fig:concentration_contours}). They reveal that the \transi{} constriction ($\num{-1.85} < z <
\text{\SI{1.6}{\nm}}$) remains depleted of \Cl{} and enhanced in \Na{} for both $\vbias =
\text{\SI{-150}{\mV}}$ and $\vbias = \text{\SI{+150}{\mV}}$. This is not the case in the \lumeni{} ($\num{1.6}
< z < \text{\SI{12.25}{\nm}}$), in which the \Na{} concentration is bulk-like for $\vbias <
\text{\SI{0}{\mV}}$ and enhanced for $\vbias > \text{\SI{0}{\mV}}$. Conversely, the number of \Cl{} ions
becomes more and more depleted in the \lumeni{} for increasing negative bias magnitudes, and it is virtually
bulk-like at higher positive bias voltages. This is further exemplified by the radial profiles of the ion
concentrations (\cref{fig:concentration_radial_profiles}) through the middle of the constriction ($z =
\text{\SI{-0.3}{\nm}}$) and the \lumeni{} ($ z = \text{\SI{5}{\nm}}$), which also clearly shows the extent of
the electrical double layer.

% Note that the figures given above were obtained from a nanoscale continuum steady-state simulation, and
% hence represent a time-averaged situation (typically on the order of \SIrange{10}{100}{\ns}).


\subsubsection{Ion charge density.}
%
The formation of an electrical double layer inside the pore, and the resulting asymmetry in the cation and
anion concentrations, gives rise to a net charge density inside the pore ($\scdion$, \cref{eq:scdion}). To
investigate the distribution of these charges within ClyA, we divided the total interior volume of the pore
(PT) into a `pore surface' (PS) and a `pore bulk' (PB) region. The PB region encompasses a cylindrical volume
at the entry of the pore up until \SI{0.5}{\nm} from the wall ($\walldistance \ge \text{\SI{0.5}{\nm}}$), the
approximate distance from which the wall begins to exert a significant influence on the properties of the
electrolyte (\cref{suppinfo:fig:corrections}c). The PS region includes the remaining volume between the PB
domain and the nanopore wall ($\walldistance < \text{\SI{0.5}{\nm}}$). Integration of $\scdion$ over the PS
and PB regions yields the average number of mobile charges present inside those locations
(\cref{fig:ion_charge_dist_vs_cbulk}): $\pav{\Qion}{\text{PB}}$ and $\pav{\Qion}{\text{PS}}$, respectively.
Although the total number of mobile charges inside the pore, $\pav{\Qion}{\text{PT}} = \pav{\Qion}{\text{PB}}
+ \pav{\Qion}{\text{PS}}$, rises appreciatively with increasing reservoir concentrations, the majority of
these additional charges are confined to the walls of the pore. Up until a reservoir concentration
$\cbulk=\text{\SI{\approx0.15}{\Molar}}$, we found $\pav{\Qion}{\text{PT}}$ to be distributed equally between
the surface (\SI{\approx+27}{\ec}) and bulk (\SI{\approx+22}{\ec}) layers. At high salt concentrations
($\cbulk > \text{\SI{1}{\Molar}}$), the number of charges in the PS region more than doubles (towards
$\pav{\Qion}{\text{PS}}=\text{\SI{+58}{\ec}}$ at \SI{5}{\Molar}), and those in the PB region diminish (towards
$\pav{\Qion}{\text{PB}} \approx \text{\SI{0}{\ec}}$ at \SI{5}{\Molar}). The bias voltage also influences the
total number of mobile charges in the pore. As can be seen from our simulation results for three different
voltages (\cref{fig:ion_charge_dist_vs_cbulk}), $\pav{\Qion}{\text{PT}}$ is approximately
\SIrange{+10}{+15}{\ec} higher at $\vbias = \text{\SI{+150}{\mV}}$ as compared to $\vbias =
\text{\SI{-150}{\mV}}$ for the full range of ion concentrations. Interestingly, at reservoir concentration
\SI{>0.15}{\Molar}, $\pav{\Qion}{\text{PB}}$ becomes independent of applied voltage and the changes in
$\pav{\Qion}{\text{PT}}$ can be attributed to $\pav{\Qion}{\text{PS}}$.

The cross-section contour plots of $\scdion$ inside ClyA for four different bulk concentrations ($\cbulk =
\text{\SIlist{0.05;0.15;0.5;5}{\Molar}}$) reveal the redistribution of the mobile charges with increasing
ionic strength in more detail. Up until a bulk concentration of $\cbulk = \text{\SI{\le0.5}{\Molar}}$, the EDL
inside the pore overlaps significantly with itself, as evidenced by the net positive charge density found
throughout the interior of the pore (\cref{fig:ion_scd_contours}). Moreover, the absence of \Cl{} ions
effectively prevents the formation of a negatively charged EDL next to the few positively charged residues
lining the pore walls. The situation at high salt concentrations (\eg~\SI{5}{\Molar}) is very
different, with almost no charge density within the PB region of the pore ($\walldistance \ge
\text{\SI{0.5}{\nm}}$), but with pockets of highly charged and alternating positive and negative charge
densities close to the nanopore wall (\cref{fig:ion_scd_contours}, rightmost panel). This sharp confinement is
shown clearly by the radial density profiles (\cref{fig:ion_scd_radial_profiles}) drawn through the
constriction ($z = \text{\SI{-0.3}{\nm}}$, purple triangles) and the \lumeni{} ($z = \text{\SI{5}{\nm}}$,
green triangles).

It is well known that the activity of an enzyme depends on the composition of the electrolyte that surrounds
it.\cite{Purich-2010-7} Hence, we expect that the interpretation of kinetic data obtained from enzymes trapped
inside the nanopore\cite{VanMeervelt-2017,Galenkamp-2018} will benefit from the precise quantification of the
ionic conditions inside the pore, including the concentration difference with the reservoir but also the
significant imbalance between cations and anions.\cite{Warren-1966}


\subsection{Electrostatic potential and energy}\label{sect:esp}

\begin{figure*}[!t]
  \centering
  \begin{minipage}[t]{16cm}
    \begin{subfigure}[t]{2.5cm}
      \centering
      \caption{}\vspace{-3mm}\label{fig:potential_clya_charges}
      \includegraphics[scale=1]{figures/potential/potential_clya_charges}
    \end{subfigure}
    \hspace{-0.6cm}
    \begin{subfigure}[t]{10cm}
      \centering
      \caption{}\vspace{-3mm}\label{fig:potential_contours}
      \includegraphics[scale=1]{figures/potential/potential_contours_0mV}
    \end{subfigure}
    \hspace{-0.4cm}
    \begin{subfigure}[t]{3.5cm}
      \centering
      \caption{}\vspace{-3mm}\label{fig:potential_radial_averages}
      \includegraphics[scale=1]{figures/potential/potential_radial_averages_0mV}
    \end{subfigure}
  \end{minipage}
  \centering

  \caption%
  %
  [Equilibrium electrostatic potential inside ClyA.]
  %
  {%
    %
    Equilibrium electrostatic potential inside ClyA.
    %
    (\subref{fig:potential_clya_charges})
    %
    A single subunit of ClyA in which all amino acids with a net charge and whose side chains face the inside
    of the pore, \ie~those contribute the most to the electrostatic potential, are highlighted. Negatively
    (Asp+Glu) and positively and positively (Lys+Arg) charged residues are colored in red and blue,
    respectively.
    %
    (\subref{fig:potential_contours})
    %
    As a result of these fixed charges ClyA exhibits a complex electrostatic potential ($\potential$)
    landscape at equilibrium (\ie~at $\vbias = \text{\SI{0}{\mV}}$), and whose values inside the pore we have
    plotted for several key concentrations ($\cbulk = \text{\SIlist{0.005;0.05;0.15;0.5;5}{\Molar}}$). Note
    that even at physiological salt concentrations ($\cbulk = \text{\SI{0.15}{\Molar}}$), the negative
    electrostatic potential extends significantly inside the \lumeni{} ($\num{1.6} < z <
    \text{\SI{12.25}{\nm}}$), and even more so inside the \transi{} constriction ($\num{1.85} < z <
    \text{\SI{1.6}{\nm}}$). For the former, localized influential negative `hotspots' can be found in the
    middle ($\num{4} < z < \text{\SI{6}{\nm}}$) and at the \cisi{} entry ($\num{10} < z <
    \text{\SI{12}{\nm}}$).
    %
    (\subref{fig:potential_radial_averages})
    %
    Radial average of the equilibrium electrostatic potential along the length of the pore ($\radpot$) for the
    same concentrations as in (\subref{fig:potential_contours}). Even though the \lumeni{} of the pore is
    almost fully screened for $\cbulk > \text{\SI{0.5}{\Molar}}$, the constriction still retains some of its
    negative influence even at \SI{5}{\Molar}.
    %
  }\label{fig:potential}
\end{figure*}

The electrostatic potential, or rather the spatial change thereof in the form of an electric field, is one of
the primary driving forces within a nanopore. Typically, the potential can be split into an external,
`non-equilibrium' contribution, resulting from the bias voltage applied between the \transi{} and the \cisi{}
reservoirs, and an intrinsic, `equilibrium' component, caused by the fixed charge distribution of the
pore.\cite{Willems-Ruic-Biesemans-2019} To accurately describe and understand the nanopore transport processes
both contributions to the net electric field inside the pore are essential, as their relative magnitudes and
directions can significantly influence the transport of
ions,\cite{Aksimentiev-2005,Bhattacharya-2011,DeBiase-2015,Basdevant-2019} water
molecules\cite{Laohakunakorn-2015,Bhadauria-2017} and
biopolymers.\cite{Buchsbaum-2013,Muthukumar-2014,Willems-Ruic-Biesemans-2019}

% potential drop of \SI{100}{\mV} across a \SI{10}{\nm} pore results in an electric field of
% \SI[retain-unity-mantissa = false]{\approx10^7}{\volt\per\meter}

% In the following section we aim to describe the most salient features of this potential and the extent of
% its influence over the entire investigated concentration range (\cref{fig:potential}). Moreover, because
% electromigration is the primary contributor to the ionic current, a proper understanding of the
% electrostatic landscape---and the energy barriers faced by the ions traversing the pore---should provide a
% more quantitative explanation for the origin of ClyA's rectification, ion selectivity and asymmetric ion
% concentrations.

\subsubsection{A few important charged residues.}
%
The interior walls of the ClyA nanopore (\cref{fig:potential_clya_charges}) are riddled with negatively
charged amino acids (\ie~aspartate or glutamate), interspaced by a few positively charged residues 
(\ie~lysine or arginine). When grouping these charges by proximity, we found three clusters with significantly
more negative than positive residues: inside the \transi{} constriction ($-1.85 < z < \text{\SI{1.6}{\nm}}$;
E7, E11, K14, E18, D21, D25), in the middle of the \cisi{} \lumeni{} ($4 < z < \text{\SI{6}{\nm}}$; E53,
E57, D64, K147) and at the top of the pore ($10 < z < \text{\SI{12}{\nm}}$; D114, R118, D121, D122).
As we shall see, these clusters leave strong negative fingerprints in the global electrostatic potential.

\subsubsection{Distribution of the equilibrium electrostatic potential.}
%
The electrostatic potential at equilibrium ($\eqpot$, \ie~at $\vbias = \text{\SI{0}{\mV}}$) reveals the effect
of ClyA's fixed charges on the potential inside the pore (\cref{fig:potential_contours}). Due to electric
screening by the mobile charge carriers in the electrolyte, however, the extent of their influence strongly
depends on the bulk ionic strength. The contour plot cross-sections of $\eqpot$ for $\cbulk =
\text{\SIlist{0.005;0.05;0.15;0.5;5}{\Molar}}$ (\cref{fig:potential_contours}) and their corresponding radial
averages (\cref{fig:potential_radial_averages}) demonstrate this effect aptly. The radial average
($\radeqpot$) represents the mean value along the longitudinal axis of the pore and can be computed using
%
\begin{align}
  \radpot=\dfrac{1}{\pi R(z)^2}\int_{0}^{R(z)}\potential(r,z) \;2 \pi r \; dr \text{\,,}
\end{align}
%
where $R(z)$ is taken as the ClyA's radius inside the pore ($-1.85 \le z \le \text{\SI{12.25}{\nm}}$), and as
fixed values of \SI{4}{\nm} and \SI{2}{\nm} inside the \cisi{} ($z > \text{\SI{12.25}{\nm}}$) and \transi{}
(($z < \text{\SI{-1.85}{\nm}}$) reservoirs, respectively. Starting from the \cisi{} entry ($z \approx
\SI{10}{\nm}$), the electrostatic potential is dominated by the acidic residues D114, D121 and D122, resulting
in a rapid reduction of $\radeqpot$ upon entering the pore. Next, $\radeqpot$ slowly decreases up until the
middle of the \lumeni{} ($z \approx \text{\SI{5}{\nm}}$), where the next set of negative residues, namely E53,
E57 and D64, lower it even further. After a brief increase, $\radeqpot$ attains its maximum amplitude inside
the \transi{} constriction ($z \approx \text{\SI{0}{\nm}}$) due to the close proximity of the amino acids E7,
E11, E18, D21 and D25, and then quickly falls to \num{0} inside the \transi{} reservoir.

At low ionic strengths ($\cbulk < \text{\SI{0.05}{\Molar}}$), the lack of sufficient ionic screening results
in relatively high negative potentials throughout the entire pore. For example, at low concentrations ($\cbulk
= \text{\SIlist{0.005;0.05}{\Molar}}$), the $\radeqpot$ inside the constriction ramps up to values of
\SI{-144}{\mV} (\SI{-5.60}{\kBTe}) and \SI{-86}{\mV} (\SI{-3.35}{\kBTe}), respectively. These values
significantly exceed the single ion thermal voltage $\si{\kBTe} = \text{\SI{25.7}{\mV}}$. Hence, on the one
hand they prohibit anions such as \Cl{} from entering the pore, and on the other they attract cations such as
\Na{} and trap them inside the pore. For intermediate concentrations ($0.05 \le \cbulk <
\text{\SI{0.5}{\Molar}}$) the influence of the negative charges becomes increasingly confined to several
`hotspots' near the nanopore walls, most notably at entry of the pore ($10 < z < \text{\SI{12}{\nm}}$), in the
middle of the \lumeni{} ($4 < z < \text{\SI{6}{\nm}}$), and in the constriction ($-1.85 < z <
\text{\SI{1.6}{\nm}}$), in accordance with the charge groups discussed in the previous section. Even though
the magnitude of $\radeqpot$ at $\cbulk = \text{\SI{0.15}{\Molar}}$ drops below \SI{1}{\kBTe} inside the
\lumeni{} ($\radeqpot \approx \text{\SI{-14}{\mV}}$), it remains strongly negative inside the constriction
($\radeqpot \approx \text{\SI{-47}{\mV}}$). Finally, at high reservoir concentrations ($\cbulk \ge
\text{\SI{0.5}{\Molar}}$) the potential is close to \SI{\approx0}{\mV} over the entire \lumeni{} of the pore,
with only a small negative potential remaining inside the constriction. A summary of the most salient
$\radeqpot$ values can be found in \cref{suppinfo:tab:radial_potential}.

%
\begin{figure}[!t]
  \centering
  \begin{subfigure}[t]{8.25cm}
    \centering
    \caption{}\vspace{-5mm}\label{fig:potential_energy_radial_averages}
    \includegraphics[scale=1]{figures/potential_energy/potential_energy_radial_averages}
  \end{subfigure}
  \begin{subfigure}[t]{8.25cm}
    \centering
    \caption{}\vspace{-3mm}\label{fig:potential_energy_trans_barrier}
    \includegraphics[scale=1]{figures/potential_energy/potential_energy_trans_barrier}
  \end{subfigure}

  \caption%
  %
  [Non-equilibrium electrostatic energy landscape for single ions.]
  %
  {%
    %
    Non-equilibrium electrostatic energy landscape for single ions.
    %
    (\subref{fig:potential_energy_radial_averages})
    %
    Radially averaged non-equilibrium electrostatic energy landscape for single ions, $\radenergy =
    \chargen_{i} \echarge \radpot$, as calculated directly from the radial electrostatic potential at $\vbias
    = \text{\SIlist{+150;-150}{\mV}}$ for monovalent cations and anions. The grey arrows indicate the
    direction in which the ions must travel in order for them to contribute positively to the ionic current.
    %
    (\subref{fig:potential_energy_trans_barrier})
    %
    Height of the electrostatic energy barrier ($\Delta E_{\text{B},i}$) at the \transi{} constriction as a
    function of the bulk salt concentration. Note that $\Delta E_{\text{B},i}$ is much higher for negative
    voltages and rises logarithmically at lower concentrations. The divergence between
    \SI[explicit-sign=+]{0}{\mV} and \SI[explicit-sign=-]{0}{\mV} for $\cbulk < \text{\SI{0.3}{\Molar}}$
    highlights the difference in barrier height when traversing the pore from \cisi{} to \transi{} or
    \textit{vice versa}.
    %
  }\label{fig:potential_energy}
\end{figure}
%

\subsubsection{Non-equilibrium electrostatic energy at $\mathbf{+150}$ and $\mathbf{-150}$~mV.}
%
To link back the observed ionic conductance properties to the electrostatic potential, we computed the
radially averaged electrostatic energy for a monovalent ion, $\radenergy = \chargen_{i} \echarge \radpot$, at
$\vbias = \text{\SI{\pm150}{\mV}}$ for the entire range of simulated ionic strengths
(\cref{fig:potential_energy_radial_averages}). The resulting energy plot represents the energy
landscape---filled with barriers (hills) or traps (valleys)---that a positive or negative ion must traverse in
order to contribute positively to (\ie~increase) the ionic current.

At positive bias voltages, cations traverse the pore from \transi{} to \cisi{}
(\cref{fig:potential_energy_radial_averages}, first plot). Upon entering the negatively charged constriction,
their electrostatic energy drops dramatically, followed by a relatively flat section with a small barrier for
entry in the \lumeni{} at $z \approx \text{\SI{1.6}{\nm}}$. At very low ionic strengths ($\cbulk <
\text{\SI{0.05}{\Molar}}$), the energy at \transi{} is significantly lower than the energy of the cation in
the \cisi{} compartment (\eg~$\Delta\radenergy > \text{\SI{2}{\kT}}$ at \SI{0.005}{\Molar}), forcing the ions
to accumulate inside the pore. At higher concentrations ($\cbulk > \text{\SI{0.05}{\Molar}}$), the increased
screening smooths out the potential drop inside the pore, allowing the cations to migrate unhindered across
the entire length of the pore. Anions at $\vbias = \text{\SI{+150}{\mV}}$ travel from \cisi{} to \transi{}
(\cref{fig:potential_energy_radial_averages}, second plot) and must overcome energy barriers at both sides of
the pore. The \cisi{} barrier prevents anions from entering the pore. However, because its magnitude is
attenuated strongly with increasing salt concentration (\SIrange{1.7}{0.5}{\kT} when increasing the reservoir
salt concentration from $\cbulk = \text{\SIrange{0.005}{0.05}{\Molar}}$), it is only relevant at lower ionic
strengths ($\cbulk < \text{\SI{0.05}{\Molar}}$). Once inside the \lumeni{}, anions can move relatively
unencumbered to the \transi{} constriction, where they face the second, more significant energy barrier. This
prevents them from fully translocating and causes them to accumulate inside the lumen and explains why we
observe higher \Cl{} concentrations inside the pore at positive bias voltages
(\cref{fig:concentration_contours}). As with the cations, an increase in the ionic strength significantly
reduces these hurdles, resulting in a much smoother landscape for $\cbulk > \text{\SI{0.15}{\Molar}}$.

At negative voltages, cations move through the pore from \cisi{} to \transi{}, with a slow and continuous drop
of the electrostatic energy throughout the \lumeni{} of the pore up until the constriction
(\cref{fig:potential_energy_radial_averages}, third plot). This results in the efficient removal of cations
from the pore \lumeni{}, and explains the lower \Na{} concentration observed at positive voltages
(\cref{fig:concentration_pore_average_vs_cbulk}). To fully exit from the pore, however, cations must overcome
a large energy barrier, which reduces the nanopore's ability to conduct cations compared to positive
potentials and hence contributes to the ion current rectification. The situation for anions at negative bias
voltages (\ie~travelling from \transi{} to \cisi{}) is very different
(\cref{fig:potential_energy_radial_averages}, fourth plot). Their ability to even enter the pore is severely
hampered by an energy barrier of a few \si{\kT} at the \transi{} constriction. Any anions that do cross this
barrier, and those still present in the \lumeni{} of ClyA, will rapidly move towards the \cisi{} entry and
exit from the pore due to a continuous drop of their electrostatic energy. This effectively depletes the
entire \lumeni{} of anions, which can be observed from the much lower \Cl{} concentrations at negative
voltages (see~\cref{fig:concentration_pore_average_vs_cbulk}).

\subsubsection{Concentration and voltage dependencies of the energy barrier at the constriction.}
%
Many biological nanopores contain constrictions that play crucial roles in shaping their ionic conductance
properties.\cite{Maglia-2008,Franceschini-2016,Huang-2017} The reason for this is two-fold, (1) the narrowest
part dominates the overall resistance of the pore and (2) confinement of charged residues results in much
larger electrostatic energy barriers. With its highly negatively charged \transi{} constriction, ClyA's
affinity for transport of anions is diminished and that for cations is enhanced compared to bulk, even at high
ionic strengths (\cref{fig:transport_number_vs_concentration_bulk_pnp_epnp}).\cite{Soskine-2013} To further
elucidate the significance of the \transi{} electrostatic barrier ($\deltaEt$), we quantified its height at
positive and negative voltages as a function of the salt concentration
(\cref{fig:potential_energy_trans_barrier}).

Because the application of a bias voltage effectively tilts the energy landscape, it reduces the magnitude of
the energy barriers for both positive and negative potentials, as evidenced by the lowering of the curves with
increasing bias magnitude (\cref{fig:potential_energy_trans_barrier}, light to dark color shading). Likewise,
raising the bulk salt concentration results in a continuous decrease of $\deltaEt$ due to an increase in the
screening of the fixed charges lining the constriction. At moderate to higher reservoir concentrations
($\cbulk > \text{\SIrange{0.1}{0.5}{\Molar}}$, depending on $\vbias$), $\deltaEt$ falls below \SI{1}{\kT}
regardless of the bias voltage, and its effect on the ion transport through the pore is significantly reduced.


The barrier heights for ions under the influence of a positive bias voltage (\ie~\Na{} moving from \transi{}
to \cisi{} and \Cl{} moving from \cisi{} to \transi{}, blue lines in
\cref{fig:potential_energy_trans_barrier}), experience a $\deltaEt$ roughly half that of those under a
negative voltage (red lines in \cref{fig:potential_energy_trans_barrier}). For example, increasing the salt
concentration from \SIrange{0.005}{0.15}{\Molar}, causes $\deltaEt$ to drop from \SIrange{1.8}{0.73}{\kT} at
$\vbias = \text{\SI{+150}{\mV}}$ and from \SIrange{4.4}{1.5}{\kT} at $\vbias = \text{\SI{-150}{\mV}}$. These
differences in barrier heights are directly reflected by ClyA's higher degree of ion selectivity at negative
compared to positive bias voltages (\cref{fig:transport_number_vs_concentration_bulk_pnp_epnp}). 

% At $\cbulk = \text{\SI{0.15}{\Molar}}$, for example, the ion selectivity $\transportn_{\Na}$ increases from
% \num{0.79} at $\vbias = \text{\SI{+150}{\mV}}$ to \num{0.88} at $\vbias = \text{\SI{-150}{\mV}}$.

% csalt	vbias	
% 0.005	-75.0	4.862614
% 0.150	-75.0	1.801352
% 0.005	-150.0	4.430853
% 0.150	-150.0	1.527426

% csalt	vbias	
% 0.005	75.0	2.130737
% 0.150	75.0	1.089031
% 0.005	150.0	1.836765
% 0.150	150.0	0.733521

% 0.15M @ +150 mV
% Tna = 0.78640
% Pna = 3.68165
% 0.15M @ -150 mV
% Tna = 0.88245
% Pna = 7.50708

%0.1M, -150mV: 1.8943409
%0.1M, +150mV: 0.90132562

% cis
% 0.05 : -1.33 = -33.8 mV
% 0.15 : -0.75 = -19.1 mV
% 0.50 : -0.39 = -9.9 mV
% 5.00 : -0.07 = -1.8 mV
%

\subsection{Transport of water through ClyA}\label{sec:eof}


\begin{figure*}[!t]
  \centering
  %\hspace{-2cm}
  \begin{minipage}[t]{5.5cm}
    \begin{subfigure}[t]{5.5cm}
      \centering
      \caption{}\vspace{-3mm}\label{fig:flow_contour}
      \includegraphics[scale=1]{figures/flow/flow_contour_500mM}
    \end{subfigure}
    \begin{subfigure}[t]{5.5cm}
      \vspace{3mm}
      \centering
      \caption{}\vspace{-3mm}\label{fig:flow_constriction_profiles}
      \includegraphics[scale=1]{figures/flow/flow_constriction_profiles}
    \end{subfigure}
  \end{minipage}
  \begin{minipage}[t]{11.5cm}
    \hspace{1cm}
    \begin{subfigure}[t]{5cm}
      \centering
      \caption{}\vspace{-3mm}\label{fig:flow_conductance_vs_concentration}
      \includegraphics[scale=1]{figures/flow/flow_conductance_vs_concentration}
    \end{subfigure}
    \begin{subfigure}[t]{5cm}
      \centering
      \caption{}\vspace{-3mm}\label{fig:flow_conductance_rectification_vs_concentration}
      \includegraphics[scale=1]{figures/flow/flow_conductance_rectification_vs_concentration}
    \end{subfigure}
    \\
    \begin{minipage}[t]{10cm}
      \hspace{1cm}
      \begin{subfigure}[t]{5.5cm}
        \centering
        \caption{}\vspace{-3mm}\label{fig:pressure_contour}
        \includegraphics[scale=1]{figures/pressure/pressure_contour_150mM_+000mV}
      \end{subfigure}
      \hspace{-5mm}
      \begin{subfigure}[t]{2.5cm}
        \centering
        \caption{}\vspace{-3mm}\label{fig:pressure_radial_averages}
        \includegraphics[scale=1]{figures/pressure/pressure_radial_averages_+000mV}
      \end{subfigure}
    \end{minipage}
  \end{minipage}
\centering

  \caption%
  %
  [Concentration and voltage dependency of the electro-osmotic flow inside ClyA.]
  %
  {%
    %
    Concentration and voltage dependency of the electro-osmotic flow inside ClyA.
    %
    (\subref{fig:flow_contour})
    %
    Contour plot of the electro-osmotic flow (EOF) velocity $\velocity$ at \SI{0.5}{\Molar} and \SI{-100}{\mV}
    bias voltage. The arrows  on the streamlines indicate the direction of the flow. As observed
    experimentally\cite{Soskine-2013} and  expected from a negatively charged conical nanopore, the EOF
    follows the direction of the cation, \ie~from \cisi{} to \transi{} under negative bias voltages and
    \textit{vice versa} for positive ones.
    %
    (\subref{fig:flow_constriction_profiles})
    %
    Cross-section profiles of the absolute value of the water velocity $\left|U_z\right|$ inside \transi{}
    constriction (at $z = \text{\SI{-1}{\nm}}$) for various salt concentrations at $\vbias =
    \text{\SI{-100}{\mV}}$. Notice that at high salt concentrations ($\cbulk > \text{\SI{1}{\Molar}}$), the
    velocity profile exhibits two `lobes' close to the nanopore walls and hence deviates from the parabolic
    shape observed at lower ionic strengths.
    %
    (\subref{fig:flow_conductance_vs_concentration})
    %
    Concentration dependence of the electro-osmotic conductance $\flowcond = \flowrate / \vbias$, with
    $\flowrate$ the total flow rate through the pore (\cref{eq:flowrate}). In the low concentration regime,
    $\flowcond$ increases rapidly between \SIlist{0.005;0.5}{\Molar} after which it decreases logarithmically
    for higher concentrations.
    %
    (\subref{fig:flow_conductance_rectification_vs_concentration})
    %
    The rectification of the electro-osmotic conductance ($\eor(V) = \flowcond (+V) / \flowcond (-V)$) plotted
    against the bulk salt concentration. The $\eor$ increases with bias voltage and exhibits an inversion
    point at $\cbulk \approx \text{\SI{0.45}{\Molar}}$.
    %
    (\subref{fig:pressure_contour})
    %
    Contourmap of the hydrodynamic pressure $\pressure$ at $\cbulk=\SI{0.15}{\Molar}$ and
    $\vbias=\SI{0}{\mV}$, showing that the large variations in \Na{} concentration along the pore wall result
    in osmotic pressure `hotspots' (\SIrange{5}{30}{\atm}) inside the confined fluid.
    %
    (\subref{fig:pressure_radial_averages})
    %
    The axial pressure profile and averaged along the the entire radius of the pore at $\vbias=\SI{0}{\mV}$.
    %
  }\label{fig:flow}

\end{figure*}

The charged nature of the inner surface of many nanopores gives rise to a net flux of water through the pore,
called the electro-osmotic flow (EOF).\cite{Qiao-Aluru-2003,Thompson-2003,Mao-2014} The EOF not only
contributes significantly to the ionic current, but the magnitude of the viscous drag force it exerts on
proteins is often of the same order as the Coulombic electrophoretic force
(EPF).\cite{vanDorp-2009,Firnkes-2010,Willems-Ruic-Biesemans-2019} Hence, it strongly influences the capture
and translocation of biomolecules including nucleic acids,\cite{Wong-2007,Luan-2008,Firnkes-2010}
peptides,\cite{Huang-2017,Li-2018,Huang-2019} and
proteins.\cite{Soskine-2012,Soskine-2013,VanMeervelt-2014,Soskine-Biesemans-2015,Biesemans-Soskine-2015,Wloka-2017,Galenkamp-2018,Willems-Ruic-Biesemans-2019}
Because the drag exerted by the EOF depends primarily on the size and shape of the biomolecule of interest and
not on its charge,\cite{Willems-Ruic-Biesemans-2019} it can be employed to capture molecules even against the
electric field.\cite{Soskine-2012} The EOF is a consequence of interaction between the fixed charges on the
nanopore walls and mobile charges in the electrolyte. It can be described by two closely related mechanisms:
(1) the excess transport of the hydration shell water molecules in one direction due to the pore's ion
selectivity, and (2) the viscous drag exerted by the unidirectional movement of the electrical double layer
inside the pore. The first mechanism likely dominates in pores with a diameter close to that of the hydrated
ions (\SI{\le1}{\nm}) such as \textalpha-hemolysin (\ahl{}) or Fragaceatoxin C
(FraC),\cite{Huang-2017,Huang-2019} while the second is expected to be stronger for larger pores
(\SI{>1}{\nm}), such as ClyA\cite{Soskine-2012,Willems-Ruic-Biesemans-2019} or most solid-state
nanopores.\cite{Mao-2014,Laohakunakorn-2015} In our simulation, the EOF is generated according to the second
mechanism by coupling of the Navier-Stokes and the Poisson-Nernst-Planck equations through a volume force
($\volumeforce$, \cref{eq:ion_force_density}). This coupling dictates that the electric field exerts a net
force on the fluid if it contains a net ionic charge density---as is the case for the electrical double layer
lining the walls of ClyA (\cref{fig:ion_charge_dist_vs_cbulk}).

% In the following section we aim to quantitatively and a qualitatively describe the properties of the water
% velocity inside ClyA and how it is influenced by the bulk ionic strength and the applied bias voltage
% (\cref{fig:flow}).


\subsubsection{Direction, magnitude and distribution of the water velocity.}
%
As expected, given ClyA's negatively charged interior surface and the resulting positively charged electrical
double layer, the direction of the net water flow inside ClyA follows the electric field, \ie~from the \cisi{}
to the \transi{} at negative bias voltages (\cref{fig:flow_contour}). This corresponds to the observations and
analysis of single-molecule protein capture\cite{Soskine-2013} and
trapping\cite{Soskine-Biesemans-2015,Biesemans-Soskine-2015,Willems-Ruic-Biesemans-2019} experiments using the
ClyA-AS nanopore. Along the longitudinal axis ($z$) of the pore, the water velocity is governed by the
conservation of mass, meaning it is lowest in the wide \cisi{} \textit{lumen} and highest in the narrow
\transi{} constriction (\cref{fig:flow_contour}). For example, at $\vbias = \text{\SI{-100}{\mV}}$ and $\cbulk
= \text{\SI{0.5}{\Molar}}$ the velocity at the center of the pore is \SI{\approx0.07}{\mps} in the
\textit{lumen} and \SI{\approx0.21}{\mps} in the constriction.

Along the radial axis ($r$), $\velocity$ has a parabolic profile with the highest value at the center of the
pore and the lowest at the wall due to the no-slip boundary condition (\cref{fig:flow_constriction_profiles}).
Such a parabolic profile contrasts the expected `plug flow' for an EOF, but follows logically from the overlap
of the electrical double layer inside the pore and the resulting uniform volume force---analogous to a
gravity- or pressure-driven Stokes flow. At concentrations higher than \SI{0.5}{\Molar}, however, the
increasing degree of confinement of the double layer---and its charge---to the nanopore walls
(see~\cref{fig:ion_charge_dist_vs_cbulk}, $\pav{\Qion}{s}$) results in a flattening of the central maximum and
hence a plug flow profile. Interestingly, at very high salt concentrations ($\cbulk \ge
\text{\SI{1}{\Molar}}$) the velocity profile in the constriction exhibits a dimple at the center of the pore.
This is the result of a self-induced pressure gradient caused by the expansion of the EOF as it exits the
pore.\cite{Melnikov-2017}

\subsubsection{Influence of bulk ionic strength and bias voltage on the electro-osmotic conductance.}
%
In analogy to the ionic conductance, the amount of water transported by ClyA can be expressed by the
electro-osmotic conductance $\flowcond = \flowrate / \vbias$ (\cref{fig:flow_conductance_vs_concentration}).
Here, $\flowrate$ is the net volumetric flow rate of water through the pore and computed by integrating the
water velocity across the reservoir boundary (\cref{eq:flowrate}). The strength of the EOF depends strongly
and non-monotonically on the bulk ionic strength: $\flowcond$ rapidly increases with ionic strength until a
peak value is reached at $\cbulk \approx \text{\SI{0.5}{\Molar}}$, followed by a gradual logarithmic decline
(\cref{fig:flow_conductance_vs_concentration}). For example, at $\vbias = \text{\SI{-150}{\mV}}$, $\flowcond$
first increases from \SI{1.85}{\cnmpnspv} at \SI{0.005}{\Molar} to \SI{11.3}{\cnmpnspv} at \SI{0.5}{\Molar},
followed by a gradual decline to \SI{4.00}{\cnmpnspv} at \SI{5}{\Molar}.

The sensitivity of the EOF to the magnitude and sign of the bias voltage is given by the electro-osmotic
conductance rectification $\eor(V) = \flowcond (+V) / \flowcond (-V)$
(\cref{fig:flow_conductance_rectification_vs_concentration}). For all voltage magnitudes, $\eor$ shows a
maximum at $\cbulk \approx \text{\SI{0.045}{\Molar}}$, after which it falls rapidly to reach unity ($\eor =
1$) at approximately $\cbulk \approx \text{\SI{0.45}{\Molar}}$. A minimum is then reached at
\SI{\approx1}{\Molar}, followed by a gradual approach towards unity at $\cbulk = \text{\SI{5}{\Molar}}$.

%The magnitude of the EOF is in agreement with the \cite{Willems-Ruic-Biesemans-2019}

\subsubsection{Pressure distribution inside ClyA.}
%
The large variations of \Na{} concentration along the walls of ClyA---up to several orders of magnitude over
the course of a few nanometers (see~\cref{fig:concentration_contours})---induce regions of high `osmotic'
pressure with peak values up to \SI{30}{\atm} (\cref{fig:pressure_contour}). The largest `hotspots' are
located at \cisi{} entry of the pore ($z = \text{\SI{11}{\nm}}$), in the middle of the \lumeni{} ($z =
\text{\SI{4.5}{\nm}}$) and inside the entire constriction ($z = \text{\SI{-1}{\nm}}$)
(\cref{fig:pressure_radial_averages}), and their influence extends well towards the center of the pore. Up
until $\cbulk \approx \text{\SI{0.5}{\Molar}}$, increasing the reservoir salt concentration does not seem to
strongly influence the overall magnitude of the pressure spots. Hence, because such large pressure differences
can  exert a significant amount of force on particles translocating through nanopores,\cite{Hoogerheide-2014}
we expect them to play an important role in the detailed trapping dynamics of proteins inside
ClyA.\cite{Soskine-Biesemans-2015,Willems-Ruic-Biesemans-2019}


\section{Conclusions}\label{sec:conclusions}

We have developed an extended version of the Poison-Nernst-Planck-Navier-Stokes (ePNP-NS) equations that is
capable of accurately modelling the transport of ions and water through biological nanopores, yielding a
wealth of information that is both qualitatively and quantitatively accurate. Our ePNP-NS equations combine
many of the improvements to the PNP-NS equations available in literature, in addition to several new
corrections. These include the finite size of the ions, self-consistent concentration- and
positional-dependent parametrization of the ionic transport coefficients (diffusion coefficient and mobility)
and of the electrolyte properties (density, viscosity and relative permittivity).

The use of computationally inexpensive continuum models is pervasive in the solid-state nanopore field, but
their application to the structurally more complex biological nanopores has been limited to date. We made use
of the radial symmetry of the biological nanopore ClyA to create a 2D-axisymmetric model of the pore which, in
conjuction with the ePNP-NS equations, is able to accurately describe the ionic current of ClyA for a wide
range of experimentally relevant ionic strengths and bias voltages. Our approach shows that continuum
modelling of biological nanopores is not only feasible, but can also be predictive. Our results describe in
great detail the properties of ClyA, such its true ion selectivity, the differences between cation and anion
concentrations inside the pore, the distribution and magnitude of the electrostatic potential, the velocity of
the electro-osmotic flow and the presence of highly localized `hotspots' of osmotic pressure. 

In summary, we fully expect our model and the principles of our ePNP-NS framework to be transferable to other
biological nanopores and nanoscale transport systems. We believe our model comprises a powerful and practical
tool that can aid with (1) elucidating the link between ionic current observed during a nanopore experiment
and the actual physical phenomenon, (2) describing the electrophoretic and electro-osmotic properties of any
biological nanopore and (3) guiding the design of new variants of existing nanopores. 

%Even though some inaccuracies still arise at low salt concentrations (\SI{<50}{\mM}), they are both less
%relevant experimentally and can likely be resolved with an improved mobility model.

%
%
\section{Materials and methods}\label{sec:methods}

\subsection{Molecular modelling}

\subsubsection{ClyA-AS homology model.}
%
A full atom model of ClyA-AS\cite{Soskine-2013} was built and optimized (MODELLER v9.18\cite{Sali-1993}) by
introduction of the following point mutations in each of the 12 chains of the wild-type ClyA crystal structure
(PDBID: 2WCD\cite{Mueller-2009}): K8Q, N15S, Q38K, A57G, T67V, C87A, A90V, A95S, L99Q, E103G, K118R, L119I,
I124V, T125K, V136T, F166Y, K172R, V185I, K212N, K214R, S217T, T224S, N227A, T244A, E276G, C285S, K290Q. Next,
the conformation of all mutated side chains was optimized with an double annealing protocol (heating: 150,
250, 400, 700 and \SI{1000}{\kelvin}, cooling: 1000, 800, 600, 500, 400 and \SI{300}{\kelvin}) where at each
temperature the energy was minimized for 200 iterations with a conjugate gradients algorithm (\SI{4}{\fs}
timestep).\cite{Shanno-1980} The first anneal was performed solely on the mutated residues themselves, and the
second run also took the non-bonded interactions with the neighboring atoms into account. The refined nanopore
structure was then embedded in the center of an \SI{18x18}{\nm} equilibrated DPhPC lipid bilayer patch by
manual removal of all overlapping lipids, resulting in 463 lipid molecules. The bilayer was created with the
CHARMM-GUI\cite{Jo-2008} membrane builder\cite{Lee-2016} and equilibrated with NAMD\cite{Phillips-2005}, as
described in detail in ref.~[\citenum{Wu-2014}]. The system was then solvated in a box of \SI{18x18x32}{\nm}
by addition of 214640 TIP3 water molecules (VMD solvate plugin), and the global charge was neutralized by
replacing 1276 random water molecules with 674 \Na{} and 602 \Cl{} ions (VMD autoionize
plugin).\cite{Humphrey-1996}

\subsubsection{Molecular dynamics simulations.}
%
Using molecular dynamics (MD) with NAMD 2.12 (\SI{2}{\fs} timestep, CHARMM36 forcefield\cite{Best-2012}), the
final system was minimized for \SI{5}{\ps}, heated from 0 to \SI{298.15}{\kelvin} in \SI{4}{\ps} and
equilibrated for \SI{4}{\ns} as NpT ensemble.\cite{Aksimentiev-2005} Finally a \SI{30}{\ns} production run was
performed using a NVT ensemble at \SI{298.15}{\kelvin} and the atomic coordinates saved every \SI{5}{\ps}.
Note that structural deterioration was prevented by harmonically restraining the protein's C\textalpha{} atoms
to their original positions (spring constant of \SI{695}{\pN\per\nm}) during all MD
runs.\cite{Bhattacharya-2011}

\subsubsection{Axially symmetric geometry.}
%
The 2D-axisymmetric geometry of the ClyA-AS nanopore (\cref{fig:model_geometry_vs_wedge}) was derived directly
from its full atom model by radially averaging the molecular density. Briefly, 50 sets of atomic coordinates
were extracted from the final \SI{5}{\ns} of the coordinates of the \SI{30}{\ns} MD production run
(\ie~every \SI{100}{\fs}) and aligned by minimizing the RMSD between their backbone atoms (VMD RMSD tool).
Next, we computed and averaged the 3D-dimensional molecular density maps of all 50 structures on a
\SI{0.5}{\angstrom} resolution grid using the Gaussian function\cite{Li-2013}
%
\begin{align}\label{eq:denspore}
  \rho_\text{mol} = 1 - \displaystyle\prod_{i} \left[ 1 - 
    \exp{\left(-\dfrac{-d_i^2}{(\stdev\atomradius_{i})^2}\right)} \right]
    \text{\,,}
\end{align}
%
where for each atom $i$, $R_i$ is its Van der Waals radius, $d_i=\sqrt{(x-x_i)^2 + (y-y_i)^2 + (z-z_i)^2}$ is
the distance of grid coordinates $(x, y, z)$ from the atom center $(x_i, y_i, z_i)$ and $\stdev = 0.93$ is a
width factor. The resulting 3D density map was then radially averaged along the z-axis, relative to the center
of the pore to obtain a 2D-axisymmetric density map. The contourline at \SI{25}{\percent} density was used as
the nanopore simulation geometry, after manual removal of overlapping and superfluous vertices to improve the
quality of the final computational mesh.

\subsubsection{Axially symmetric charge density.}
%
The 2D-axially symmetric charge distribution (\cref{fig:model_charge_density} was also derived directly from
the 50 sets of aligned nanopore coordinates) that were used for the geometry. Inspired by how charges are
represented in the particle mesh Ewald (PME) method,\cite{Aksimentiev-2005} we computed the fixed charge
distribution of the nanopore $\scdpore(r,z)$ by assuming that an atom $i$ of partial charge
$\partialcharge_{i}$, at the location $(x_i, y_i, z_i)$ in the full 3D atomistic pore model, contributes an
amount $\partialcharge_{i}/2\pi r_i$ to the partial charge at a point $(r_i,z_i)$ with $r_i = \sqrt{x_i^2 +
y_i^2}$ in the averaged 2D-axisymmetric model. This effectively spreads the charge over all angles to achieve
axial symmetry. We assumed a Gaussian distribution of the space charge density of each atom $i$
around its respective 2D-axisymmetric coordinates $(r_i,z_i)$ is
%
\begin{align}
\label{eq:scdpore}
  \scdpore(r,z) = \dsum_{i} \dfrac{\echarge\partialcharge_{i}}{\pi(\stdev\atomradius_{i})^2}
            \exp{\left(-\dfrac{(r-r_i)^2 + (z-z_i)^2}{(\stdev\atomradius_{i})^2}\right)}
  \text{\,,}
\end{align}
%
where $\atomradius_{i}$ is the atom radius, $\sigma = 0.5$ is the sharpness factor and $\echarge$ is the
elementary charge. To embed $\scdpore$ with sufficient detail, yet efficiently, into a numeric solver, the
spatial coordinates were discretized with a grid spacing of \SI{0.005}{\nm} in the domain of $\scdpore$ and
precomputed values were used during the solver runtime. All partial charges (at \pH{7.5}) and radii were taken
from the CHARMM36 forcefield\cite{Best-2012} and assigned using PROPKA\cite{Olsson-2011} and
PBD2PQR.\cite{Jurrus-2018}

\subsubsection{Computing electrophoretic mobilities.}
%
To obtain the concentration-dependent ionic mobility  $\mobility^c_{i}$ from fitted functions, it must first
be derived from the salt's molar conductivity $\molarconductivity$ and the ion's transport number
$\transportn_{i}$ before it can be fitted\cite{ContrerasAburto-2013-1}
%
\begin{align}
\label{eq:conductivity-to-mobility}
  \mobility_{i}(\concentration) = \frac{\specmolarconductivity_{i}(\concentration)}{\chargen_{i}\faraday}
  \quad\text{with}\quad \specmolarconductivity_{i}(\concentration) = \molarconductivity(\concentration)
  \transportn_{i}(\concentration)
  \text{\,,}
\end{align}
%
where $\specmolarconductivity_{i}(\concentration)$ is the specific molar conductivity of ion $i$.

\paragraph{Computing the simulated ionic current and electro-osmotic flow rate.}
%
The simulated ionic current $\currentsim$ at steady-state was computed by
%
\begin{align}\label{eq:currentsim}
  \currentsim = \faraday\int_{S}\left(\dsum_{i}\chargen_{i}\normvec\cdot\vec{\flux}_{i}\right)dS
  \text{\,,}
\end{align}
%
with $\chargen_{i}$ the charge number and $\vec{\flux}_{i}$ the total flux of each ion $i$ across \cisi{}
reservoir boundary $S$, $\faraday$ the Faraday constant (\SI{96485}{\coulomb\per\mole}) and $\normvec$ the
unit vector normal to $S$. Similarly, the volumetric flow rate, \ie~the volume of water passing through the pore per unit time, is given by
%
\begin{align}\label{eq:flowrate}
  \flowrate = \int_{S}\left(\dsum_{i}\normvec\cdot\vec{\velocity}\right)dS
  \text{\,.}
\end{align}
%

%
%
\subsection{Single-channel nanopore experiments}
%

\subsubsection{ClyA expression and purification.}
%
ClyA-AS monomers were expressed, purified and oligomerized using methods described in detail
elsewhere.\cite{Soskine-2012,Soskine-2013} Briefly, \textit{E. cloni} EXPRESS BL21 (DE3) cells (Lucigen
Corporation, Middleton, USA) were transformed with a pT7-SC1 plasmid containing the ClyA-AS gene, followed by
overexpression after induction with \SI{0.5}{\mM} isopropyl \textbeta{}-D-1-thiogalactopyranoside
(IPTG, Carl Roth, Karlsruhe, Germany). The ClyA monomers were purified using \ce{Ni+}-NTA affinity
chromatography and oligomerized by incubation in \SI{0.2}{\percent} D-maltoside
n-dodecyl-\textbeta{}-D-maltopyranoside (Sigma-Aldrich, Zwijndrecht, The Netherlands) for \SI{30}{\minute} at
\SI{37}{\dC}. Pure ClyA-AS type-I (12-mer) nanopores were obtained using native PAGE on a
\SIrange[range-phrase = --]{4}{15}{\percent} gradient gel (Bio-Rad, Veenendaal, The Netherlands) and
subsequent excision of the correct oligomer band.

\subsubsection{Recording of single-channel current-voltage curves.}
%
Experimental current-voltage curves where measured using single-channel electrophysiology, as detailed
elsewhere.\cite{Maglia-2010,Soskine-2012,Soskine-2013} First, a black lipid bilayer was formed inside a
\SI{\approx100}{\um} diameter aperture in a thin teflon film separating two buffered electrolyte compartments.
This was achieved by applying a droplet of \SI{5}{\percent} hexadecane in pentane (Sigma-Aldrich, Zwijndrecht,
The Netherlands) over the aperture and leaving it to dry for \SI{1}{\minute} at \SI{25}{\dC}. The buffered
electrolyte solution was added to both compartments, topped with \SI{10}{\uL} of
\SI{6.25}{\milli\gram\per\milli\liter} 1,2-diphytanoyl-\textit{sn}-glycero-3-phosphocholine (DPhPC, Avanti
Polar Lipids, Alabaster, USA) in pentane. The pentane was left to evaporate for \SI{2}{\minute} at
\SI{25}{\dC}. A lipid bilayer was formed by lowering and raising the buffer level over the aperture. Minute
amounts (\SI{\approx0.2}{\uL}) of the purified ClyA-AS type I oligomer were then added to the grounded \cisi{}
reservoir and allowed to insert into the lipid bilayer. Single-channel current-voltage curves were recorded
using a custom pulse protocol of the Clampex 10.4 software package connected to AxoPatch 200B patch-clamp
amplifier via a Digidata 1440A digitizer (all from Molecular Devices, San Jose, USA). Data was acquired at
\SI{10}{\kHz} and filtered using a \SI{2}{\kHz} low bandpass filter. Measurements at different ionic strengths
were performed at \SI{\approx25}{\dC} in aqueous \ce{NaCl} (Carl Roth, Karlsruhe, Germany) solutions, buffered
at \pH{7.5} using \SI{10}{\mM} MOPS (Carl Roth, Karlsruhe, Germany).

%
%
\section*{Conflicts of interest}
%
There are no conflicts to declare.

%
%
\section*{Acknowledgements}
%
K.W. gratefully acknowledges the support by the IWT (grant number 3E130054). P.V.D. and J.H. gratefully
acknowledge the financial support by the FWO (grant number G.0683.15). G.M. has received funding from the
European Research Council (ERC) under the European Union's Horizon 2020 research and innovation programme
(Grant agreement No. 726151). J.H. gratefully acknowledges financial support from the Flemish government
through long term structural funding Methusalem (CASAS2, Meth/15/04). The authors thank Dr. Chang Chen and Dr.
Yi Li for their valuable feedback during discussions.

%%%END OF MAIN TEXT%%%

%The \balance command can be used to balance the columns on the final page if desired. It should be placed anywhere within the first column of the last page.

\balance

%If notes are included in your references you can change the title from 'References' to 'Notes and references' using the following command:
%\renewcommand\refname{Notes and references}

\bibliography{shared/bibliography}
\bibliographystyle{rsc}

%\includepdf{suppinfo}
\end{document}

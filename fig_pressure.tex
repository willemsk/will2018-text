\begin{figure*}[htbp]
  \centering
  \begin{minipage}[t]{10.75cm}
    \begin{subfigure}[t]{5.5cm}
      \centering
      \caption{}\vspace{-3mm}\label{fig:pressure_contour}
      \includegraphics[scale=1]{figures/pressure/pressure_contour_150mM_-100mV}
    \end{subfigure}
    \hspace{-5mm}
    \begin{subfigure}[t]{2.5cm}
      \centering
      \caption{}\vspace{-3mm}\label{fig:pressure_radial_averages}
      \includegraphics[scale=1]{figures/pressure/pressure_radial_averages_-100mV}
    \end{subfigure}
  \end{minipage}
\centering

% caption
\caption
[\textbf{Pressure distribution inside ClyA.}]
{
\textbf{Pressure distribution inside ClyA.}
(\subref{fig:pressure_contour})
Contourmap of the pressure at $\cbulk=0.15$~M and $\varpotential_\text{bias}=-100$~mV, 
showing that the \ce{Na+} concentration `hotspots' near the pore wall result in build-up of electro-osmotic 
pressure ($5$ to $30$~atm) inside the confined fluid. Pressure drops of such magnitude over the course of a 
few nanometer could potentially exert a significant force on a captured protein.\cite{hoogerheide2014}
(\subref{fig:pressure_radial_averages})
The axial pressure profile and averaged along the the entire radius of the pore at 
$\varpotential_\text{bias}=-100$~mV.
%At concentrations $\ge0.5$~M, a negative pressure develops in the `bulk' of the lumen due to the 
}

% label
\label{fig:pressure}

\end{figure*}
\begin{figure*}[!htb]
  \centering
  \begin{minipage}[t]{10.75cm}
    \begin{subfigure}[t]{5.5cm}
      \centering
      \caption{}\vspace{-3mm}\label{fig:pressure_contour}
      \includegraphics[scale=1]{figures/pressure/pressure_contour_150mM_+000mV}
    \end{subfigure}
    \hspace{-5mm}
    \begin{subfigure}[t]{2.5cm}
      \centering
      \caption{}\vspace{-3mm}\label{fig:pressure_radial_averages}
      \includegraphics[scale=1]{figures/pressure/pressure_radial_averages_+000mV}
    \end{subfigure}
  \end{minipage}
\centering

% caption
\caption
[\textbf{Pressure distribution inside ClyA.}]
{
\textbf{Pressure distribution inside ClyA.}
(\subref{fig:pressure_contour}) Contourmap of the pressure at $\cbulk=\SI{0.15}{\Molar}$ and
$\vbias=\SI{0}{\mV}$, showing that the \Na\ concentration `hotspots' near the pore wall result in build-up of
electro-osmotic pressure (\SIrange{5}{30}{\atm}) inside the  confined fluid. Pressure drops of such magnitude
over the course of a few nanometers could potentially exert a significant force on a captured
protein.\cite{Hoogerheide-2014}
(\subref{fig:pressure_radial_averages}) The axial pressure profile and averaged along the the entire radius of
the pore at $\vbias=\SI{0}{\mV}$.
%At concentrations $\ge0.5$~M, a negative pressure develops in the `bulk' of the lumen due to the
}\label{fig:pressure}

\end{figure*}

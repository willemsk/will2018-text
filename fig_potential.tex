\begin{figure*}[!htb]
  \centering
  \begin{minipage}[t]{18.25cm}
    \begin{subfigure}[t]{2.5cm}
      \centering
      \caption{}\vspace{-3mm}\label{fig:potential_clya_charges}
      \includegraphics[scale=1]{figures/potential/potential_clya_charges}
    \end{subfigure}
    \hspace{-0.6cm}
    \begin{subfigure}[t]{11.5cm}
      \centering
      \caption{}\vspace{-3mm}\label{fig:potential_contours}
      \includegraphics[scale=1]{figures/potential/potential_contours_0mV}
    \end{subfigure}
    \hspace{-0.4cm}
    \begin{subfigure}[t]{4cm}
      \centering
      \caption{}\vspace{-3mm}\label{fig:potential_radial_averages}
      \includegraphics[scale=1]{figures/potential/potential_radial_averages_0mV}
    \end{subfigure}
  \end{minipage}
\centering

% caption
\caption
[\textbf{Electrostatic potential inside ClyA.}]
{
\textbf{Electrostatic potential inside ClyA.}
(\subref{fig:potential_clya_charges}) A single subunit of ClyA in which all amino acids with a net charge and
whose side chains face the inside of the pore, i.e. that contribute the most to the electrostatic potential,
are highlighted. Negatively (Asp+Glu) and positively and positively (Lys+Arg) charged residues are colored in
red and blue, respectively.
(\subref{fig:potential_contours}) Electrostatic potential landscape inside ClyA due to its fixed charges (i.e.
at $\vbias=\SI{0}{\mV}$) at several key  concentrations
($\cbulk=\text{\SIlist{0.005;0.05;0.15;0.5;5}{\Molar}}$). Note that even at physiological salt  concentrations
($\cbulk=\SI{0.15}{\Molar}$), the negative electrostatic potential extends significantly inside the lumen
($1.60<z<\SI{12.25}{\nm}$), and even more so inside the \trans\ constriction ($1.85<z<\SI{1.60}{\nm}$). For
the former, localized influential negative `hotspots' can be found in the middle ($4<z<\SI{6}{\nm}$) and at
the \cis\ entry ($10<z<\SI{12}{\nm}$).
(\subref{fig:potential_radial_averages}) Radial average of the electrostatic potential along the length of the
pore ($\radpot$) for the same concentrations as in \subref{fig:potential_contours}. While the lumen of the
pore becomes almost fully screened for $\cbulk>\SI{0.5}{\Molar}$, the constriction still retains some of its
negative influence at \SI{5}{\Molar}.
}\label{fig:potential}
\end{figure*}

\begin{figure*}[htbp]
  \centering
  \begin{minipage}[t]{10.75cm}
    \begin{subfigure}[t]{5.5cm}
      \centering
      \caption{}\vspace{-3mm}\label{fig:potential_contour}
      \includegraphics[scale=1]{figures/potential/potential_contour_150mM_0mV}
    \end{subfigure}
    \hspace{-5mm}
    \begin{subfigure}[t]{2.5cm}
      \centering
      \caption{}\vspace{-3mm}\label{fig:potential_radial_averages}
      \includegraphics[scale=1]{figures/potential/potential_radial_averages_0mV}
    \end{subfigure}
    \hspace{-5mm}
    \begin{subfigure}[t]{2.5cm}
      \centering
      \caption{}\vspace{-3mm}\label{fig:potential_clya_charges}
      \includegraphics[scale=1]{figures/potential/potential_clya_charges}
    \end{subfigure}
  \end{minipage}
  \hspace{-8mm}
  \begin{minipage}[t]{6.25cm}
    \begin{subfigure}[t]{6.25cm}
      \centering
      \caption{}\vspace{-3mm}\label{fig:potential_energy_radial_averages}
      \includegraphics[scale=1]{figures/potential/potential_energy_radial_averages}
    \end{subfigure}
    \begin{subfigure}[t]{6.25cm}
      \vspace{-3mm}
      \centering
      \caption{}\vspace{-3mm}\label{fig:potential_trans_energy_barrier}
      \includegraphics[scale=1]{figures/potential/potential_trans_energy_barrier}
    \end{subfigure}
  \end{minipage}
\centering

% caption
\caption
[\textbf{Concentration and voltage dependency of the electro-osmotic flow inside ClyA.}]
{
(\subref{fig:potential_contour})
Even at physiological salt concentrations (0.15~M), the negative electrostatic potential extends 
significantly inside the lumen of ClyA, particularly inside the \textit{trans} constriction ($1.85\text{~nm}< 
z<1.60\text{~nm}$).
(\subref{fig:potential_radial_averages})
The average electrostatic potential along the the entire radius of the nanopore shows the effect of increased 
ionic screening Comparing the radial averages 
at enable 
(\subref{fig:potential_clya_charges})
A single subunit of ClyA highlighting all amino acids with a net charge, i.e. those that contribute the most 
to the electrostatic potential. Negatively (Asp+Glu) and positively and positively (Lys+Arg) charged residues 
are colored in red and blue, respectively.
(\subref{fig:potential_energy_radial_averages})
Approximate radial electrostatic energy landscape for single ions as calculated directly from the 
electrostatic potential at $+150$ and $-150$~mV applied bias voltages:
$\left< U_{\text{E},i} \right>_{\text{rad}} =
\chargen_{i}\elementarycharge\left< \potential \right>_{\text{rad}}$.
The arrows at top indicate the direction in which the ion must travel in order to contribute positively 
to the ionic current. While for \ce{Na+} it is electrostatically favorable to enter the pore under virtually 
all circumstances, it must over come an energetic barrier can enter the pore under all circumstances with
(\subref{fig:potential_trans_energy_barrier})
Electrostatic barrier at the trans constriction ($\Delta E_{\text{B},i}$) in function of bulkd salt 
concentration (left) and bias voltage (right).
}

% label
\label{fig:electro-osmotic_flow}

\end{figure*}
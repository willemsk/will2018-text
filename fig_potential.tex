\begin{figure*}[!htb]
  \centering
  \begin{minipage}[t]{18.25cm}
    \begin{subfigure}[t]{12cm}
      \centering
      \caption{}\vspace{-3mm}\label{fig:potential_contours}
      \includegraphics[scale=1]{figures/potential/potential_contours_0mV}
    \end{subfigure}
    \hspace{-5mm}
    \begin{subfigure}[t]{3cm}
      \centering
      \caption{}\vspace{-3mm}\label{fig:potential_radial_averages}
      \includegraphics[scale=1]{figures/potential/potential_radial_averages_0mV}
    \end{subfigure}
    \hspace{-6mm}
    \begin{subfigure}[t]{2.5cm}
      \centering
      \caption{}\vspace{-3mm}\label{fig:potential_clya_charges}
      \includegraphics[scale=1]{figures/potential/potential_clya_charges}
    \end{subfigure}
  \end{minipage}
\centering

% caption
\caption
[\textbf{Electrostatic potential inside ClyA.}]
{
\textbf{Electrostatic potential inside ClyA.}
(\subref{fig:potential_contours})
Electrostatic potential landscape inside ClyA due to its fixed charges (i.e. at $\vbias=0$~mV) at several key 
concentrations ($\cbulk=0.005$, $0.05$, $0.15$, $0.5$ and $5$~M). Note that even at physiological salt 
concentrations ($\cbulk=0.15$~M), the negative electrostatic potential extends significantly inside the lumen 
($1.60<z<12.25$~nm), and even more so inside the \trans\ constriction ($1.85<z<1.60$~nm). For the former, 
localized influential negative `hotspots' can be found in the middle ($4<z<6$~nm) and at the \cis\ entry 
($10<z<12$~nm).
(\subref{fig:potential_radial_averages})
The electrostatic potential along the length of the pore and averaged along the the entire radius of the 
nanopore ($\left<\potential\right>_{\text{rad}}$) allows for bette quantification of the effect of increased 
screening at higher reservoir concentrations. While the lumen of the pore becomes almost fully screened for 
$\cbulk>0.5$~M, the constriction still retains some of its negative influence at $5$~M. 
(\subref{fig:potential_clya_charges})
A single subunit of ClyA in which all amino acids with a net charge and whose side chains face the inside of 
the pore, i.e. that contribute the most to the electrostatic potential, are highlighted. Negatively (Asp+Glu) 
and positively and positively (Lys+Arg) charged residues are colored in red and blue, respectively.
}\label{fig:potential}
\end{figure*}
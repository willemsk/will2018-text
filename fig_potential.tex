\begin{figure*}[htbp]
  \centering
  \begin{minipage}[t]{10.75cm}
    \begin{subfigure}[t]{5.5cm}
      \centering
      \caption{}\vspace{-3mm}\label{fig:potential_contour}
      \includegraphics[scale=1]{figures/potential/potential_contour_150mM_0mV}
    \end{subfigure}
    \hspace{-5mm}
    \begin{subfigure}[t]{2.5cm}
      \centering
      \caption{}\vspace{-3mm}\label{fig:potential_radial_averages}
      \includegraphics[scale=1]{figures/potential/potential_radial_averages_0mV}
    \end{subfigure}
    \hspace{-6mm}
    \begin{subfigure}[t]{2.5cm}
      \centering
      \caption{}\vspace{-3mm}\label{fig:potential_clya_charges}
      \includegraphics[scale=1]{figures/potential/potential_clya_charges}
    \end{subfigure}
  \end{minipage}
  \hspace{-8mm}
  \begin{minipage}[t]{6.25cm}
    \begin{subfigure}[t]{6.25cm}
      \centering
      \caption{}\vspace{-5mm}\label{fig:potential_energy_radial_averages}
      \includegraphics[scale=1]{figures/potential/potential_energy_radial_averages}
    \end{subfigure}
    \begin{subfigure}[t]{6.25cm}
      \vspace{-3mm}
      \centering
      \caption{}\vspace{-3mm}\label{fig:potential_trans_energy_barrier}
      \includegraphics[scale=1]{figures/potential/potential_trans_energy_barrier}
    \end{subfigure}
  \end{minipage}
\centering

% caption
\caption
[\textbf{Electrostatic potential inside ClyA.}]
{
\textbf{Electrostatic potential inside ClyA.}
(\subref{fig:potential_contour})
At physiological salt concentrations (0.15~M), the negative electrostatic potential extends significantly
inside the lumen of ClyA, particularly inside the \textit{trans} constriction 
($1.85<z<1.60$~nm). In the lumen of the pore, influential negative `hotspots' can be found in the middle 
($4<z<6$~nm) and at the \cis\ entry ($10<z<12$~nm). 
(\subref{fig:potential_radial_averages})
The electrostatic potential along the length of the pore and averaged along the the entire radius of the 
nanopore ($\left<\potential\right>_\text{rad}$) shows the effect of increased screening at higher reservoir 
concentrations. Above $0.5$~M, the lumen of the pore becomes almost fully screened while the constriction 
remains highly negative still. 
(\subref{fig:potential_clya_charges})
A single subunit of ClyA highlighting all amino acids with a net charge that contribute the most to the 
electrostatic potential, i.e. whose side chains face the inside of the pore. Negatively (Asp+Glu) and 
positively and positively (Lys+Arg) charged residues are colored in red and blue, respectively.
(\subref{fig:potential_energy_radial_averages})
Approximate electrostatic energy landscape for single ions as calculated directly from the radial 
electrostatic potential at $+150$ and $-150$~mV applied bias voltages:
$\left< U_{\text{E},i} \right>_{\text{rad}} =
\chargen_{i}\elementarycharge\left< \potential \right>_{\text{rad}}$.
The arrows at top indicate the direction in which the ions must travel in order to contribute positively 
to the ionic current. The light and dark lines correspond to low and high bulk salt concentrations, 
respectively.
(\subref{fig:potential_trans_energy_barrier})
Electrostatic energy barrier height ($\Delta E_{\text{B},i}$) at the \trans\ constriction depends strongly on 
$\cbulk$ (left) and weakly on the $\vbias$ (right). The $\Delta E_{\text{B},i}$ values for high negative and 
positive bias voltages converge at concentrations $> 0.5$~M. Discontinuity at $0$~mV reflects the difference 
in $\Delta E_{\text{B},i}$ when traversing the pore from \cis\ to \trans\ or vice versa.
}

% label
\label{fig:potential}

\end{figure*}
\begin{figure*}[htbp]
	
	\centering
	
	% figure path
	\includegraphics[scale=1]{figures/pdf/fig_conductance}
	
	% caption
	\caption
	[\textbf{Conductance characteristics of single ClyA nanopores at different ionic strengths.}]
	{
		\textbf{Conductance characteristics of single ClyA nanopores at different ionic strengths.}
		(a) Subset of the IV curves that were measured experimentally (\graphmarker{circle}{black}) and simulated using the classical PNP-NS (\graphline{solid}{graphgreen}) and extended PNP-NS (\graphline{solid}{graphpurple}) equations  at \SIlist[list-units=single]{50;150;500;3000}{\milli\Molar}.
		Contour plots showing the conductance ($G = I/V$) landscape of ClyA-AS vs. bias voltage and salt concentration (\ce{NaCl}) (b) as measured experimentally with single channel recordings at \SI{298(1)}{\kelvin} and 
		(c) as calculated by our simulation (left: PNP-NS, right: ePNP-NS).
		(d) Contour plots showing the of percentage relative error of the simulated conductance landscape compared to the experimental one (left: PNP-NS, right: ePNP-NS).
		The classical PNP-NS equations significantly overestimate the conductance under all conditions, particularly at high salt concentrations (\SI{>100}{\percent}).
		In contrast, the errors of the ePNP-NS model are generally below \SIrange{5}{10}{\percent}.
		The error is highest in the low concentration regime at positive bias voltages.
	}
	
	% label
	\label{fig:conductance}
	
\end{figure*}
\begin{figure*}[htbp]
  \centering
  \begin{subfigure}[t]{4.5cm}
    \centering
    \caption{}\vspace{-3mm}\label{fig:current-voltage_curves}
    \includegraphics[scale=1]{figures/conductance/current_vs_voltage_all_multiplot}
  \end{subfigure}
  \begin{minipage}[t]{8cm}
    \begin{subfigure}[t]{8cm}
      \centering
      \caption{}\vspace{-5mm}\label{fig:conductance_contourmap_epnp}
      \includegraphics[scale=1]{figures/conductance/conductance_contourmap_epnp}
    \end{subfigure}
    \begin{subfigure}[t]{8cm}
      %\vspace{5mm}
      \centering
      \caption{}\vspace{-5mm}\label{fig:conductance_rectification_contourmap_epnp}
      \includegraphics[scale=1]{figures/conductance/conductance_rectification_contourmap_epnp}
    \end{subfigure} 
  \end{minipage}

% caption
\caption
[\textbf{Measured and simulated ion conductance and current rectification characteristics of single ClyA 
nanopores.}]
{
\textbf{Measured and simulated ion conductance and current rectification characteristics of single ClyA 
nanopores.}
(\subref{fig:current-voltage_curves})
Comparison of the single-pore current-voltage (IV) characteristics over a wide range of ionic strengths (from 
\SIrange{0.05}{3}{\Molar} \ce{NaCl}) reveals the inability of the classical PNP-NS equations to 
quantitatively predict the experimentally measured values, particularly at high ionic strengths. In contrast, 
the extended PNP-NS (ePNP-NS) equations exhibit excellent agreement over the entire concentration range. All 
experimental measurements were performed at \SI{25\pm1}{\degreeCelsius}. Errorbars represent the standard 
deviation of three independent measurements ($n=3$).
(\subref{fig:conductance_contourmap_epnp})
Contourmap of the simulated (ePNP-NS) ionic conductance ($G(c) = I / V$) vs. bias voltage and bulk salt 
concentration. The graphs at the bottom and on the right show the dependency of the conductance on applied 
bias voltage and on the bulk salt concentration. Solid lines and circles represent simulated and experimental 
data, respectively. The kink in the slope of the concentration-conductance log-log plot at 
\SI{\approx0.15}{\Molar} indicates the presence of 2 distinct conductance regimes. Power law fitting ($G = a 
c^b$) shows near a linear ($b\approx1$) dependency at high ionic strengths (\SIrange{0.3}{1.5}{\Molar}) and a 
near square-root ($b\approx0.5$) dependency at low salt concentrations (\SIrange{0.005}{0.075}{\Molar}).
(\subref{fig:conductance_rectification_contourmap_epnp})
Contourmap of the simulated (ePNP-NS) ionic conductance rectification ($\alpha(V) = G(+V) / G(-V)$) vs. bias 
voltage and bulk salt concentration. While $\alpha$ depends linearly on the applied bias in the investigated 
voltage regime, a clear maximum is observed in the concentration dependency at $\SI{\approx0.15}{\Molar}$. 
While the measurement and the model agree qualitatively, a good quantitative agreement is found only at 
concentrations \SI{\ge 0.5}{\Molar}.
}

\label{fig:conductance}
	
\end{figure*}

% Slopes of log-log plot
%    0.005 to 0.05 M, -150 mV:
%    a=0.495±0.005, b=2.873±0.054
%    0.005 to 0.05 M, +150 mV:
%    a=0.600±0.005, b=5.894±0.104
%    0.3 to 1.5 M, -150 mV:
%    a=1.000±0.016, b=7.734±0.052
%    0.3 to 1.5 M, +150 mV:
%    a=0.787±0.001, b=8.935±0.006
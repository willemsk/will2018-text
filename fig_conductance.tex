\begin{figure*}[htbp]
  \centering
  \begin{subfigure}[t]{4.5cm}
    \centering
    \caption{}\vspace{-3mm}\label{fig:current-voltage_curves}
    \includegraphics[scale=1]{figures/conductance/current_vs_voltage_all_multiplot}
  \end{subfigure}
  \begin{minipage}[t]{8cm}
    \begin{subfigure}[t]{8cm}
      \centering
      \caption{}\vspace{-5mm}\label{fig:conductance_contourmap_epnp}
      \includegraphics[scale=1]{figures/conductance/conductance_contourmap_epnp}
    \end{subfigure}
    \begin{subfigure}[t]{8cm}
      %\vspace{5mm}
      \centering
      \caption{}\vspace{-5mm}\label{fig:conductance_rectification_contourmap_epnp}
      \includegraphics[scale=1]{figures/conductance/conductance_rectification_contourmap_epnp}
    \end{subfigure}
    \begin{subfigure}[t]{8cm}
      %\vspace{5mm}
      \centering
      \caption{}\vspace{-5mm}\label{fig:transport_number_contourmap_epnp}
      \includegraphics[scale=1]{figures/conductance/transport_number_contourmap_epnp}
    \end{subfigure} 
  \end{minipage}

% caption
\caption
[\textbf{Measured and simulated ion conductance and current rectification characteristics of single ClyA 
nanopores.}]
{
\textbf{Measured and simulated ion conductance and current rectification characteristics of single ClyA 
nanopores.}
(\subref{fig:current-voltage_curves})
Comparison of the single-pore current-voltage (IV) characteristics over a wide range of ionic strengths (from 
\SIrange{0.05}{3}{\Molar} \ce{NaCl}) reveals the inability of the classical PNP-NS (dotted green lines) 
equations to quantitatively predict the experimentally measured values (black circles), particularly at high 
ionic strengths. In contrast, the extended PNP-NS (ePNP-NS, purple lines) equations exhibit excellent 
agreement over the entire concentration range. All experimental measurements were performed at 
$25\pm1$\textdegree C. Errorbars represent the standard deviation of three independent measurements ($n=3$).
(\subref{fig:conductance_contourmap_epnp})
Concentration-voltage heatmap (left) of the simulated ionic conductance,
$\conductance(\concentration) = \current / \varpotential$,
and the bulk salt concentration dependencies (right) at $+150$~mV (blue) and $-150$~mV (red). The simulated 
(ePNP-NS) and experimental data are represented by solid lines and open circles, respectively. The kink in 
the slopes of the log-log plots at $\approx0.15$~M suggest that the conductance mechanism is different at low 
and high bulk ionic strengths. While the low and high regimes exhibit respectively
square-root ($\conductance \propto \concentration_\text{bulk}^{0.5}$) and
linear ($\conductance \propto \concentration_\text{bulk}^1$) dependencies at negative bias voltages, the 
exponents converge at positive bias voltages (low: $\conductance \propto \concentration_\text{bulk}^{0.61}$, 
high: $\conductance \propto \concentration_\text{bulk}^{0.79}$).
(\subref{fig:conductance_rectification_contourmap_epnp})
Concentration-voltage heatmap of the simulated (ePNP-NS) ionic conductance rectification 
($\icr(\varpotential) = \conductance(+\varpotential) / \conductance(-\varpotential)$), which highlights 
ClyA's tendency for higher currents at positive bias voltages. $\icr$ is a linear function of voltage, but 
shows a maximum in the concentration dependency at $\approx0.15$~M. The measurement and the model agree 
qualitatively about the shape of the curve, a good quantitative match is found only at concentrations 
$\ge0.5$~M.
(\subref{fig:transport_number_contourmap_epnp})
Concentration-voltage heatmap of the fraction of the current carried by \ce{Na+} ions, i.e. the 
\ce{Na+} transport number ($\transportn_{\ce{Na+}} = \conductance_{\ce{Na+}}/\conductance$). ClyA remains 
cation-selective up until a concentration of $\approx2$~M, after which the electrostatic screening and the 
high \ce{Cl-} mobility of result in more anions traversing the pore than cations. At very low concentrations 
($50$~mM at $-150$~mV and $10$~mM at $+150$~mV), the net \ce{Cl-} flux follows the direction of the 
electro-osmotic flow (i.e. against the electric potential gradient), resulting in a negative contribution the 
the overall ionic current and a \ce{Na+} transport number $>1$.
}

\label{fig:conductance}
	
\end{figure*}

% Slopes of log-log plot
%    0.005 to 0.05 M, -150 mV:
%    a=0.495±0.005, b=2.873±0.054
%    0.005 to 0.05 M, +150 mV:
%    a=0.600±0.005, b=5.894±0.104
%    0.3 to 1.5 M, -150 mV:
%    a=1.000±0.016, b=7.734±0.052
%    0.3 to 1.5 M, +150 mV:
%    a=0.787±0.001, b=8.935±0.006
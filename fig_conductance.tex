\begin{figure*}[htbp]
  \centering
  \begin{subfigure}[t]{4.5cm}
    \centering
    \caption{}\vspace{-3mm}\label{fig:current-voltage_curves}
    \includegraphics[scale=1]{figures/conductance/current_vs_voltage_all_multiplot}
  \end{subfigure}
  \begin{minipage}[t]{8cm}
    \begin{subfigure}[t]{8cm}
      \centering
      \caption{}\vspace{-5mm}\label{fig:conductance_contourmap_epnp}
      \includegraphics[scale=1]{figures/conductance/conductance_contourmap_epnp}
    \end{subfigure}
    \begin{subfigure}[t]{8cm}
      %\vspace{5mm}
      \centering
      \caption{}\vspace{-5mm}\label{fig:conductance_rectification_contourmap_epnp}
      \includegraphics[scale=1]{figures/conductance/conductance_rectification_contourmap_epnp}
    \end{subfigure} 
  \end{minipage}

% caption
\caption
[\textbf{Measured and simulated ion conductance and current rectification characteristics of single ClyA 
nanopores.}]
{
\textbf{Measured and simulated ion conductance and current rectification characteristics of single ClyA 
nanopores.}
(\subref{fig:current-voltage_curves})
Comparison of the single-pore current-voltage (IV) characteristics over a wide range of ionic strengths (from 
\SIrange{0.05}{3}{\Molar} \ce{NaCl}) reveals the failure of the classical PNP-NS equations 
(\graphline{dashed}{graphgreen}) to quantitatively predict the experimentally measured values  
(\graphmarker{circle}{black}), particularly at high ionic strengths. In contrast, the extended PNP-NS  
(ePNP-NS) equations (\graphline{solid}{graphpurple}) exhibit excellent agreement over the entire 
concentration range. All measurements were performed at \SI{25\pm1}{\degreeCelsius}. Errorbars represent the 
standard deviation of three independent measurements ($n=3$).
(\subref{fig:conductance_contourmap_epnp})
Contourmap of the simulated (ePNP-NS) ionic conductance ($G = I / V$) between \SIlist{-150;150}{\milli\volt} 
and from \SIrange{0.005}{5}{\Molar}. The log-log plot on the right shows the dependency the bulk ionic 
strength for \SI{+100}{\milli\volt} (blue) and \SI{-100}{\milli\volt} (red) for Solid lines represent for 
\SI{-100}{\milli\volt} (\graphline{solid}{graphred}) and \SI{+100}{\milli\volt} 
(\graphline{solid}{graphblue}) applied bias voltages. The corresponding experimental values are plotted as 
circles. The bottom graph shows the dependency on applied bias voltage while the cross-section
(\subref{fig:conductance_rectification_contourmap_epnp})
Contourmap of the ionic conductance rectification ($\alpha = G_+ / G_-$), quantifying the increased 
conductivity of ClyA at positive bias voltages ($\alpha > 1$). As in 
(\subref{fig:conductance_contourmap_epnp}), circles and solid lines represent the measured experimental and 
the simulated ePNP-NS datasets, respectively. While the measurement and the model agree qualitatively, a 
good quantitative agreement is found only at concentrations \SI{\ge 0.5}{\Molar}.



%(\subref{fig:conductance-voltage_curves})
%The nanopore ionic conductance ($G = I/V$) for all the experimentally measured (\graphmarker{circle}{black}) 
%and matching PNP-NS (\graphline{dashed}{graphgreen}) and ePNP-NS (\graphline{dashed}{graphgreen}) 
%simulations.
% (), the simulated current-voltage (IV) characteristics  significantly from the  ( are in close agreement 
% with the simulated data using the extended PNP-NS (ePNP-NS) equations , 
%of measured and Comparison of the  curves (a) and conductance curves $G = I/V$ (b) that were measured 
%experimentally (\graphmarker{circle}{black}) and simulated using the classical PNP-NS 
%(\graphline{solid}{graphgreen}) and extended PNP-NS () equations  at 
%\SIlist[list-units=single]{50;150;500;1000;3000}{\milli\Molar}.
%(c) Contour plots showing the relative error ($\displaystyle\frac{G_\text{s}-G_\text{m}}{G_{m}}$) 
%between the simulation (left: PNP-NS, right: ePNP-NS) and the experimentally measured values in term 
%of both the bias voltage and the concentration.
%Note that the classical PNP-NS equations significantly overestimate the conductance under all 
%conditions, particularly at high salt concentrations (\SI{>100}{\percent}).
%In contrast, the errors of the ePNP-NS model are generally below \SIrange{5}{10}{\percent}, with the 
%highest error in the low concentration regime and at positive bias voltages.
%(d) Contour plots of the ion conductance rectification (ICR) vs. the bias voltage and the salt 
%concentration as determined experimentally (left) and with ePNP-NS simuation (right).
}

\label{fig:conductance}
	
\end{figure*}